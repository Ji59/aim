\section{Agent}\label{sec:agent}

\emph{Agent} značí chytré auto, které má zájem projet křižovatkou.
Při~příjezdu agenta na~křižovatku agent nahlásí křižovatce odkud jede a kam směřuje.
Křižovatka poté pomocí vybraného algoritmu vygeneruje \emph{cestu} skrze křižovatku.
\emph{Cesta} je tvořena posloupností vrcholů, přes které má agent jet.
Pokud algoritmus nenajde žádnou cestu skrze křižovatku, křižovatka zamítne vjezd \emph{agenta}.
V~případě kdy agent čeká na~vjezd příliš dlouho, \uv{agentu dojde trpělivost a vydá se jinou cestou}.
Znamená to, že je ze~simulace odstraňen.
Avšak stále je zaznamenáno, že se agent pokusil křižovatkou projet.

\emph{Agent} je reprezentován obdélníkem s~délkou $d=0.6~b$ a~šířkou $w=0.37~b$,
kde $b$ značí \emph{velikost bloku} dané křižovatky.
Zároveň se agent dokáže otáčet na~místě a okamžitě podle potřeby.
Agent se vždy otáčí ve~směru jízdy.
Agent se pohybuje rychlostí jedné hrany grafu křižovatky za~jeden krok simulace.

Pokud se nějaké dva obdélníky protnou, nastane srážka příslušných dvou agentů.
Agenti nemají žádný způsob jak informovat křižovatku o~srážce.
Z~toho důvodu po~kolizi sražení agenti zmizí ze~simulace, tudíž už s~nimi nemohou žádní další agenti kolidovat.

Simulace podporuje načítání agentů ze~souboru nebo generování agentů za~běhu simulace.

\subsection{Generování agentů}\label{subsec:generovani-agentu}

Před~startem simulace je možné navolit~si určité vlastnosti agentů a po~kolik kroků se mají agenti generovat.
Uživatel si může určit počet vygenerovaných agentů v~každém kroku simulace nastavením parametrů $na_{\min}$ a~$na_{\max}$.
Simulace vygeneruje nový počet agentů uniformně náhodně mezi $na_{\min}$ a~$na_{\max}$.

Dále je možné určit preference \emph{směrů} odkud agenti budou přijíždět a~kam budou směřovat.
Pro~každý \emph{směr} je možné určit pravděpodobnost, s~jakou se zde agent objeví, či tam bude směřovat.
Tyto pravděpodobnosti se musí sečíst na~$1$.

Poté je možné měnit samotné parametry agentů.
Z~těchto parametrů budu používat pouze \emph{odchylku}.
\emph{Odchylka} určuje o~kolik procent jsou parametry agenta odlišné vůči sděleným parametrům křižovatce,
a~nabývá hodnot mezi nulou a~sto procenty.
Ovlivněné parametry jsou šířka, délka a~rychlost agenta.
Kromě těchto parametrů ovlivňuje odchylka i~příjezd agenta.
Příjezd je možné zpozdit až~o~\emph{odchylku} kroku.
Tímto způsobem se snažím simulovat reálnější křižovatky, kde došlo k~určité chybě v~měření.
