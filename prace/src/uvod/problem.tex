\section{Problém křižovatky}\label{sec:problem}

Chytré křižovatky se již objevili v mnohých městech.
Mám na mysli světelné křižovatky, které dokáží poznat, že všechna čekající auta z daného směru už projela.
Při detekci takovéto situace křižovatka nastaví červenou z daného směru a zároveň pustí auta z dalšího směru dříve.
Plánování pořadí a délek směrů je komplexní záležitost, pokročilejší plánování popsali například \citet*{Goldstein}.
\citet*{Liang} ve svém článku popsali trénování světelných křižovatek pomocí zpětnovazebního učení a rozšířili algoritmus i na tisíce propojených křižovatek.
Tento způsob jsem ve své práci vůbec neřešil, jelikož nabízí minimální zlepšení na jedné křižovatce v hustých provozech a minimálně využívá autonomity vozidel.

Další způsob je decentralizované plánování, kdy auta mezi sebou komunikují a sdělují si své plány.
Tímto způsobem se zabývali například \citet*{Wu}.
V jejich článku je porovnán kruhový objezd se světelnou křižovatkou.
Jelikož není přítomna centrální řídící jednotka, pro implementaci do reálného světa stačí přidat určitý protokol do aut.
Avšak aplikovat komplexnější plánování je při větším počtu aut obtížné.

Výše popsané algoritmy často používají FCFS (First Come First Served) strategii na pořadí plánování.
Při použití FCFS má nejvyšší priority v plánování auto, které dorazilo ke křižovatce nejdříve.
Tímto způsobem se minimalizují čekací doby jednotlivých aut.

Já se zaměřím pouze na postupy s centrální jednotkou, která komunikuje s jednotlivými agenty.
Problém plánování lze převézt na jiné známé problémy.
Hlavní cíl této práce jsou následující dva přístupy, \nameref{subsec:individualni_planovani} a \nameref{subsec:hromadne_planovani}.

\subsection{Individuální plánování}\label{subsec:individualni_planovani}

Jak název napovídá, agenti jsou naplánováni jeden po druhém.
Každý nový plán je rozvržen tak, aby nekolidoval s žádným už naplánovaným agentem.

Jako první se problematikou zabývali \citet*{Dresner}.
Jejich přístup byl mnohem reálnější než můj.
Agenti měli plynulé zatáčení a uměli zrychlovat a zpomalovat.
Dále \citet{Dresner} v práci řešili problémy komunikace mezi agenty a křižovatkou, řešení situace při kolizi a podpora pro lidmi řízená vozidla.
Agenti v jejich práci jsou schopni jezdit pouze v předem daných pruzích.
V těchto pruzích následně mohou agenti zrychlovat či zpomalovat.
Algoritmus nejprve přiřadí agentovi maximální rychlost a zjistí, zda by na jeho cestě došlo ke kolizi.
Pokud ano, zkusí nižší rychlost před místem kolize.
Postup se opakuje, dokud nedojde k nalezení nekolizní cesty skrze křižovatku.
Agenti jsou plánováni pomocí FCFS strategie, tedy podle pořadí příjezdu ke křižovatce.
První agent je i naplánován první, apod.
Tímto algoritmem jsem se inspiroval u algoritmu \nameref{sec:safe-lines}.
U mé implementace jezdí agenti stejnou rychlostí v předem daných pruzích.
Pokud by u agenta došlo ke kolizi, je jeho příjezd odložen.

Řešení je možné rozšířit pomocí libovolného prohledávacího algoritmu.
Tento přístup jsem vyzkoušel použitím známého algoritmu \nameref{sec:a-star},
kdy prvního agenta naplánuji nejkratší cestou, poté naplánuji druhého tak, aby nekolidoval s prvním.
Takto pokračuji pro všechny přijíždějící agenty.

\subsection{Hromadné plánování}\label{subsec:hromadne_planovani}

Řešení tohoto typu jsou složitější, avšak teoreticky by měly být schopny tvořit celkově lepší plány.

Chytrá křižovatka se dá převézt na online \emph{MAPF} (Multi-Agent Path Finding) problém.

\subsubsection{Offline~MAPF}\label{subsubsec:offline-mapf}

\emph{Offline~MAPF} má na vstupu dvojici $G, A$, kde $G=(V, E)$ je graf a $A = \{a_1, \dots, a_k\}$ je množina agentů.
Každý agent $a_i$ má svojí výchozí pozici $s_i \in V$ a cílovou pozici $g_i \in V$.
Čas je rozdělen na diskrétní úseky (kroky).
Během jednoho kroku může agent přejet do sousedního vrcholu, nebo počkat v aktuálním.
Plán pro agenta $a_i$ je sled $\pi_i = s_i, v_2, \dots, v_{n-1}, g_i$ na grafu $G$, čili $v_2, \dots, v_{n-1} \in V$ a
$(s_i, v_2) \in E, (v_{n-1}, g_i) \in E), \forall_{i \in 2, \dots, n-2} (v_i, v_{i+1}) \in E$.
Délka plánu je $|\pi_i| = n$, pozice agenta $a_i$ v kroku $c$ je $\pi_i[c]$.

Agenti $a_i$ a $a_j$, $i \neq j$ jsou v \emph{kolizi} v kroku $c$ právě tehdy když $\pi_i[c] = \pi_j[c]$ nebo
$\pi_i[c] = \pi_j[c + 1] \land \pi_i[c + 1] = \pi_j[c]$.
Slovy agenti jsou v \emph{kolizi}, pokud jsou na stejném místě, nebo projíždí stejnou hranou.

Cílem \emph{offline MAPF} je nalezení plánu $\pi = \cup_{i=1}^{k} \pi_i$, které nemá žádné kolize.
Takovýto plán nazýváme validní.
Pro problém mohou existovat různé plány, tyto plány bývají často porovnány pomocí \emph{SOC} (Sum Of Costs).
Plán $\pi$ má cenu $|\pi| = \sum_{i=1}^{k} |\pi_i|$.
Alternativní způsob porovnání je objektivní funkcí pro plán $\pi$ funkce $\textrm{makespan}(\pi)$,
která značí počet kroků, než všichni agenti dorazí do svého cíle.

\paragraph{Řešení~offline~MAPF}

Nejjednodušší způsob řešení je využít A* algoritmus, kde následníci stavu jsou kartézský součin přes všechny možné tahy všech agentů.
Toto řešení má často vysoký větvící faktor.
Proto se vyvinuly vylepšení, například \emph{Independence Detection}, \emph{Conflict Avoidance Table} nebo \emph{Operator Decomposition} \citep{Standley_2010} a mnoho dalších.

\citet*{Sharon} navrhli algoritmus \emph{CBS} (Conflict-Based Search), který nalezne nejkratší cesty pro všechny agenty.
Poté hledá konflikty mezi jednotlivými plány.
Pokud nalezne konflikt, vznikne omezující podmínka pro jednoho agenta v kolizi.
Tato podmínka znemožní agentovi být na konfliktním vrcholu.
Poté se s novou podmínkou spustí nové prohledávání pro tohoto agenta.
Zároveň vznikne druhá větev výpočtu, ve které má tuto podmínku druhý agent z kolize.
Takto postupně vzniká binární strom, kde synové vrcholu mají vždy podmínky z rodiče plus pro prvního / druhého agenta novou podmínku.
Pokud se nalezena nekolizní cesta pro všechny vrcholy, algoritmus skončí.
Pořadí prohledávání listů ve stromu je určeno podle SOC listů.
Toto pořadí zaručuje optimální řešení \citep{Sharon}.
Algoritmus byl nadále rozšířen a vylepšen \citep{Boyarski}.

\emph{MAPF} problém je možné převést na SAT problém.
Nejprve se vytvoří výrokové proměnné pro každého agenta, každý vrchol a každý čas.
Následně přidáme podmínky, aby agenti nebyli v kolizi.
Tento způsob řešení je spíše vhodný pro optimalizování $\textrm{makespan}$ funkce,
avšak je možné vytvořit varianty cílené na SOC \citep{bartak}.
Blíže je toto řešení popsáno v kapitole \nameref{sec:sat-planner}.

Další způsob řešení

\subsubsection{Online~MAPF}\label{subsubsec:online-mapf}

Rozšíření \emph{offline~MAPF} problému na online variantu zkoumali ve své práci \citet*{Svancara}.
\emph{Online~MAPF} má u každého agenta $a_i = (t_i, s_i, g_i)$ kromě místa příjezdu a cíle také čas příjezdu $t_i$.
Tento čas není dopředu znám.
\emph{Online~MAPF} začíná s počátečním \emph{offline~MAPF} plánem pro agenty, kteří přijeli v čase $0$.
Tento plán budu značit $\pi^0$.
Pokaždé, když se objeví noví agenti, vytvoří se nový plán $\pi^j$.
Celkový plán je tedy $\Pi = (\pi^0, \pi^1, \dots, \pi^m)$, kde $m$ je počet unikátních kroků ($t_1, t_2, \dots, t_m$), kdy se objevili agenti.
Označím si $\pi^j[x:y]$ část plánu $\pi^j$ v krocích $x, x + 1, \dots, y - 1, y$.
Celkový plán, který budou agenti vykonávat je tedy $Ex[\Pi] = \pi^0[0:t_1] \circ \pi^1[t_1 + 1:t_2] \circ \dots \circ \pi^m[t_m + 1:\infty]$.

\citet{Svancara} našli problémy u \emph{online~MAPF}.
První problém nastane, pokud agenti zůstanou na svém místě po doražení do cíle.
Zároveň pokud by se agenti okamžitě objevili v grafu, mohli by ihned způsobit kolizi, kterou algoritmy nemohli predikovat.
Žádný z těchto problémů u mě nastat nemůže, jelikož agenti mohou být zamítnuti, pokud by došlo ke kolizi hned na vjezdu.
Taktéž agenti mizí z křižovatky po doražení do výjezdu.

Opět zavedu cenu plánu jako součet délek plánů pro jednotlivé agenty $|Ex[\Pi]| = \sum_{i=1}^{k} |Ex[\Pi]_i| = \sum_{i=1}^{k} t_{Ex[\Pi]}[g_i] - t_i$,
kde $t_{Ex[\Pi]}[g_i]$ je krok, kdy agent $a_i = (t_i, s_i, g_i)$ naposledy dorazil do cílového vrcholu $g_i$.
Z analýzy \citet{Svancara} víme, že cena $|Ex[\Pi]|$ je ekvivalentní objektivní funkci $\sum_{t=1}^{\infty} \textrm{NotAtGoal}(t)$,
kde $\textrm{NotAtGoal}(t)$ udává počet agentů, kteří ještě nedorazili do svého cíle v čase $t$.
Také objektivní funkce $\sum_{i=1}^{k} |Ex[\Pi]_i| - o_i$, kde $o_i$ je délka nejkratší cesty mezi $s_i$ a $g_i$,
je ekvivalentní $|Ex[\Pi]|$.

Každý \emph{online~MAPF} problém je možné převést na \emph{offline~MAPF} pokud dáme dopředu algoritmu vědět, kdy se agenti objeví.
Díky tomu můžeme porovnat optimalitu online řešičů.
\citet{Svancara} dokázali, že žádný online algoritmus nemůže zajistit offline optimální řešení.
\emph{Snapshot-optimální} plány jsou plány, které našli optimální plán za předpokladu, že se  žádní žádní agenti neobjeví.
\citet*{Morag} provedli rozsáhlé experimenty a zjistili, že \emph{snapshot-optimální} plány nejsou o moc horší než optimální.
Ve všech typech experimentů byly \emph{snapshot-optimální} ceny plánů alespoň v $80\%$ běhů totožné s optimálním plánem
a ve zbylých případech se plány lišily minimálně.

\paragraph{Řešení~online~MAPF}

V práci \citet{Svancara} jsou návrhy různých postupů řešení \emph{online~MAPF} problémů:
\begin{itemize}
	\item \textbf{Replan~Single} (RS) - tento přístup je totožný s~přístupem \nameref{subsec:individualni_planovani}.
	\item \textbf{Replan~Single~Grouped}\label{par:replan-single-grouped} (RSG) - v tomto přístupu se plánují pouze pro noví agenty.
	Plánování probíhá pro všechny agenty najednou.
	Zde lze použít \emph{offline~MAPF} řešič, který se musí vyhnout kolizím s již naplánovanými trasami.
	\item \textbf{Replan~All} (RA) - za použití této strategie se použije \emph{offline~MAPF} řešič na všechny agenty pokaždé, když dorazí noví agenti.
	Pokud je řešič optimální, \emph{Replan~all} vrací \emph{snapshot~optimální} řešení \citep{Svancara}.
	\item \textbf{Online~Independence~Detection} (OID)- Tento přístup se snaží minimalizovat množství přeplánovaných agentů.
	Nejprve najde cestu pro všechny nové agenty ignorujíce už naplánované.
	Poté zjistí kolize mezi starými a novými agenty.
	Pokud byly nalezeny kolize, přeplánují se trasy kolizních agentů.
	Pro zaručení \emph{snapshot~optimálního} plánu je nutné udělat dodatečné úpravy \citep{Svancara}.
	\item \textbf{Suboptimal~Independence~Detection} (SubID) - pozměňuje OID dovolováním neoptimálních cest.
	Přesněji cena plánu SubID je nejvýše $D$ krát delší než cena \emph{snapshot~optimálního} plánu.
	Avšak díky této úpravě by měl být počet přeplánování, a tedy i  čas výpočtu, nižší.
\end{itemize}

\subsection{Inteligentní křižovatka}\label{subsec:inteligentni-krizovatka}
Problém popsaný v této práci, přidává do \emph{online~MAPF} další podmínky.
U křižovatky všichni agenti mají specifikovány vrcholy, kde může být jejich start a cíl.
Agenti také nejsou pouhé body, ale mají svojí velikost.
Díky tomu je zjišťování kolizí komplexnější.

Dále je možné přiblížit se reálné křižovatce dalšími úpravami.
\begin{itemize}
	\item Pokud agentovi nezáleží na pruhu, kterým vyjede, může sdělit algoritmu pouze směr výjezdu.
	Algoritmus má za cíl najít cestu na libovolný výjezd v daném směru.
	V důsledku může agent místo jednoho koncového vrcholu mít množinu vrcholů.
	\item Agent představuje jedoucí vozidlo.
	Proto mohu po algoritmu vyžadovat, aby se agent nikdy nezastavil na místě.
	Zároveň mohu vyžadovat, aby se agentovo cesta neměnila po vjezdu do křižovatky.
	Podmínku, že křižovatka nemůže měnit individuální plány za běhu, již zmínil \citet{Dresner}.
	Kdybychom změnu plánů dovolili, mohlo by dojít k chybě v komunikaci, díky níž by agent prováděl původní plán,
	avšak křižovatka by počítala s novým plánem.
	\item % TODO
\end{itemize}
