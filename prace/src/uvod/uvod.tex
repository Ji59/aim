\chapter*{Úvod}\label{ch:uvod}
\addcontentsline{toc}{chapter}{Úvod}

Vývoj aut jde velice rychle kupředu.
Každým rokem jsme blíže k světu s~autonomními auty, které budou řídit za~nás.
Samořiditelná auta už~jezdí mezi námi, avšak stále jsou značně omezena manuálně řízenými vozy a~zákony.
Tato auta mají téměř okamžitou odezvu na~vnější vlivy, dokáží se navigovat skrze dopravu s~vysokou přesností
a~také mají výhodu jednodušší vzájemné komunikace.
Těchto vlastností se dá využít pro~zvýšení efektivity dopravní sítě.
Zlepšení na~běžných silnicích nemusí být značné, avšak u~křižovatek by~mohl být rozdíl velmi velký.

Zaměřit se na~křižovatky je důležité i~z~pohledu nehodovosti.
Ačkoliv většina silniční komunikace probíhá na~nekřížících se úsecích, na~našem území dochází kolem $21\%$ všech nehod
právě na~křižovatkách (\href{https://www.czso.cz/documents/10180/20534694/32025414a06.pdf}
{Český Statistický Úřad - https://www.czso.cz/documents/10180/20534694/32025414a06.pdf}).

Tato práce se~zabývá problémem projetí co~největšího množství autonomních aut skrze křižovatku.
K~řešení tohoto problému používá křižovatka určitou centrální skříňku, se kterou auta komunikují.
Po~příjezdu auta ke křižovatce se auto nahlásí centrální skříňce odkud přijíždí a kam by chtělo jet.
Křižovatka poté autu naplánuje nekolizní trasu skrze křižovatku.

Pro~tuto práci jsem si vytvořil simulátor prostředí, kde~je možné napodobit průjezdy aut různými křižovatkami.
Simulovaný svět obsahuje značná zjednodušení reálného světa, avšak dle~mého názoru jsou nápady určitou formou přenositelné.

Práce obsahuje různá řešení tohoto problému pomocí simulovaného prostředí na~různých křižovatkách a~množství aut.
Tato~řešení jsou následně porovnána dle~rozličných parametrů.
