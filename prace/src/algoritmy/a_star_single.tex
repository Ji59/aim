\subsection{Individuální A* (A*RS)}\label{subsec:individualni_a_star}

%Popis úpravy A* algoritmu pro řešený problém, parametry a pseudokód.

\labeltext{A*RS}{str:individualni_a_star} patří do kategorie \ref{str:rs} algoritmů.
Plánuje totiž stejně jako \nameref{sec:safe_lanes} jednoho agenta po druhém.
Akorát algoritmus dovoluje agentům \uv{opustit} svoje pruhy.

Cena cesty se počítá podobně jako při hledání \hyperref[par:pruh]{pruhu} v křižovatce.
Cena mí více kritérií, a to \hyperref[par:ars_vzdalenost]{vzdálenost},
\hyperref[par:ars_uhel_zataceni]{úhel zatáčení} a \hyperref[par:ars_pocet_zataceni]{počet zatáčení}.

\paragraph{Vzdálenost}\label{par:ars_vzdalenost} je počet hran grafu, přes které cesta vede.
Duplicitní hrany se započítávají vícekrát.

\paragraph{Úhel zatáčení}\label{par:ars_uhel_zataceni} určuje úhel, o který se musí agent za cesty otočit.
Pro každý prostřední vrchol na cestě se dopočítá úhel mezi hranami,
přes kterou se agent na vrchol dostal a kterou odjel.
\nameref{par:ars_uhel_zataceni} je součet absolutních hodnot těchto úhlů.

\paragraph{Počet zatáčení}\label{par:ars_pocet_zataceni} udává počet
nenulových \hyperref[par:ars_uhel_zataceni]{úhlů zatáčení}.

Cesty jsou nejprve porovnávány podle \hyperref[par:ars_vzdalenost]{vzdálenosti},
poté \hyperref[par:ars_uhel_zataceni]{úhlu zatáčení} a nakonec podle \hyperref[par:ars_pocet_zataceni]{počtu zatáček}.

\paragraph{Heuristika}\label{par:ars_heuristika} v tomto případě je minimální délka cesty
z aktuálního vrcholu do nejbližšího z cílových vrcholů.
Pokud žádná taková cesta neexistuje, je hodnota \hyperref[par:ars_heuristika]{heuristiky} $\infty$.

Prohledávací prostor jsou rozšířené vrcholy křižovatky.
Pro jednodušší výpočet si u stavu mimo vrcholu pamatuji též krok, ve kterém by agent na daný vrchol přijel.
Dále si ukládám předchozí stav, cenu cesty z počátku a odhad ceny zbylé cesty dané heuristikou.
Datová struktura stavu vypadá následovně:
% @formatter:off
\begin{code}[frame=none]
stav {
	vrchol
	krok
	rodic       // předchozí stav
	vzdalenost  // vzdálenost z počátečního stavu
	uhel        // úhel zatáčení na cestě z počátečního stavu
	zatacky     // počet zatáčení na cestě z počátečního stavu
	heuristika  // hodnota heuristiky v aktuálním stavu
}
\end{code}
% @formatter:on

Následující stavy daného stavu jsou všechny validní stavy dané sousedy vrcholu aktuálního stavu.
Formálně pro vrchol $u$ jsou jeho sousedi vrcholy ${v \in V | (u,v)\in E}$, 
kde $V$ je množina vrcholů a $E$ množina hran.

\subsection{Parametry}\label{subsec:parametry}
Pro reálnější pohyby agentů po křižovatce je vhodné omezit množinu sousedů vrcholu.
Avšak určení vhodného omezení je komplikované.
Proto jsem se rozhodl umožnit omezení měnit následujícími parametry.

MAXIMUM_VERTEX_VISITS_DEF = 2;

ALLOW_AGENT_STOP_DEF = false;

MAXIMUM_PATH_DELAY_DEF = Integer.MAX_VALUE;

ALLOW_AGENT_RETURN_DEF = false;
