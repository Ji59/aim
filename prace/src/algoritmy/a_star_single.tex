\subsection{Individuální A* (\ref{str:a_star_ars})}\label{subsec:individualni_a_star}\labeltext{A*RS}{str:a_star_ars}

%Popis úpravy A* algoritmu pro řešený problém, parametry a pseudokód.

\ref{str:a_star_ars} patří do~kategorie \ref{str:rs} algoritmů.
Plánuje stejně jako \nameref{sec:safe_lanes}, jednoho agenta po~druhém.
Akorát tentokrát algoritmus dovoluje agentům \uv{opustit} svoje~pruhy.

Pro~správné fungování \nameref{sec:a_star} je potřeba vhodně vybrat cenu cesty a heuristiku.
Cena cesty se počítá podobně jako při~hledání \hyperref[par:pruh]{pruhu} v~křižovatce.
Cena je složena z více kritérií, a to \hyperref[par:ars_vzdalenost]{vzdálenost},
\hyperref[par:ars_uhel_zataceni]{úhel zatáčení} a \hyperref[par:ars_pocet_zataceni]{počet zatáčení}.
Priorita kritérií je nejprve podle menší \hyperref[par:ars_vzdalenost]{vzdálenosti},
poté menšího \hyperref[par:ars_uhel_zataceni]{úhlu zatáčení}
a nakonec vyšší \hyperref[par:ars_pocet_zataceni]{počtu zatáček}.
Pokud má více cest všechny kritéria stejné, mají všechny stejnou cenu.

\paragraph{Vzdálenost}\label{par:ars_vzdalenost} je počet hran grafu, přes~které cesta vede.
Duplicitní hrany se započítávají vícekrát.
Pokud agent stojí na místě, vzdálenost se nemění.

\paragraph{Úhel zatáčení}\label{par:ars_uhel_zataceni} určuje úhel, o~který se musí agent při sledování cesty otočit.
Pro~každý prostřední vrchol se na~cestě dopočítá úhel mezi hranami,
přes~kterou se agent na~vrchol dostal a kterou odjel.
Tento úhel se dá jednoduše geometricky dopočítat, jelikož známe přesnou pozici vrcholů.
\nameref{par:ars_uhel_zataceni} je součet absolutních hodnot těchto úhlů.

\paragraph{Počet zatáčení}\label{par:ars_pocet_zataceni} udává počet
nenulových \hyperref[par:ars_uhel_zataceni]{úhlů zatáčení}.

\paragraph{Heuristika}\label{par:ars_heuristika} v tomto případě je minimální délka cesty
z~aktuálního vrcholu do~nejbližšího z~cílových vrcholů.
Při~tomto výpočtu ignoruji všechny agenty na~křižovatce.
Pokud žádná taková cesta neexistuje, je hodnota \hyperref[par:ars_heuristika]{heuristiky} $\infty$.

Prohledávací prostor jsou vrcholy křižovatky rozšířené o~krok, ve~kterém by agent na~daný vrchol přijel.
Dále si pro~jednodušší práci se stavy ukládám předchozí stav,
cenu cesty z~počátku a odhad ceny zbylé cesty dané heuristikou.
Datová struktura stavu\labeltext{stav}{str:ars_stav} vypadá následovně:
% @formatter:off
\begin{code}[frame=none]
state {
  vertex
  step
  parent      // předchozí stav
  distance    // vzdálenost z počátečního stavu
  angle       // úhel zatáčení na cestě z počátečního stavu
  turns       // počet zatáčení na cestě z počátečního stavu
  heuristics  // hodnota heuristiky v aktuálním stavu
}
\end{code}
% @formatter:on

Následující stavy daného stavu jsou všechny validní stavy dané sousedy vrcholu aktuálního stavu.
Formálně pro~vrchol $u$ jsou jeho sousedi vrcholy $\{v \in V | (u,v)\in E\}$,
kde $V$ je množina vrcholů a $E$ je množina hran.

\subsubsection{Parametry}\label{subsubsec:ars_parametry}
Pro~reálnější pohyby agentů po~křižovatce je vhodné omezit množinu sousedů vrcholu.
Avšak určení vhodného omezení je komplikované.
Proto jsem se rozhodl umožnit omezení měnit parametry.
Zároveň parametry napomáhají rychlejšímu prohledávání omezením prohledávaného prostoru.
Bohužel při~nastavení parametrů algoritmus ztrácí svojí optimalitu.

\paragraph{Maximum návštěv vrcholu (\ref{par:ars_mnv})}\labeltext{MNV}{par:ars_mnv}
udává maximální počet výskytů libovolného vrcholu na~cestě.
Proto jsou validní hodnoty kladná celá čísla.
Nejreálnější hodnota parametru by byla $1$, není pro~auto moc přirozené jezdit ve~smyčkách.
Avšak vyšší hodnoty mohou vést k~obecně lepšímu řešení.

\paragraph{Povolené zastavování (\ref{par:ars_pz})}\labeltext{PZ}{par:ars_pz}
určuje, zda-li může agent stát na~místě.
Formálněji pokud dovolím zastavování, vrchol se stane sám sobě sousedem a může se vyskytovat vícekrát na~cestě za~sebou.
Jednotlivé stání na~místě se pořád započítávají jako jednotlivé návštěvy,
tudíž maximální doma stání musí být menší než~hodnota \ref{par:ars_mnv}.

\paragraph{Maximální prodleva při cestě (\ref{par:ars_mpc})}\labeltext{MPC}{par:ars_mpc}
určuje o~kolik kroků může být nalezená cesta delší než~optimální cesta (doba jízdy v~\hyperref[par:pruh]{pruhu}).
Hodnota $0$ znamená, že agent může jet pouze cestou stejné \hyperref[par:ars_vzdalenost]{vzdálenosti}
jako je \hyperref[par:ars_vzdalenost]{vzdálenost} \hyperref[par:pruh]{pruhu}.
Proto jsou hodnoty tohoto parametru nezáporná celá čísla.
Tento parametr slouží zároveň pro~omezení prohledávání, jelikož nám zakazuje dlouhé cesty.

\paragraph{Povolené vracení (\ref{par:ars_pv})}\labeltext{PV}{par:ars_pv}
může agentovi zakázat vracet~se na~vrchol, ze~kterého přijel.
Tento parametr dovoluje agentovi pouze \uv{jízdu dopředu}, což je přirozené od~jízdy v~autě očekávat.
Formálně pokud je tento parametr nenastaven, jsou zakázány cesty obsahující $uv+u$,
kde $u, v \in V$, $u \neq v$ a $v+$ znamená posloupnost vrcholů $v$ délky alespoň jedna.

\subsubsection{Sousední stavy}\label{subsubsec:sousedni_stavy}

Nyní blíže popíšu výběr následujícího stavu.
Přesněji popíšu funkci \ref{subsubsec:sousedni_stavy}\labeltext{\textrm{neighbour\_states}}{alg:sousedni_stavy},
která dostává na~vstup stav a vrací následující množinu platných stavů.
Množina následných stavů je ovlivněna \hyperref[subsubsec:ars_parametry]{parametry}.
Avšak většinou se jedná o~podmnožinu sousedních vrcholů aktuálního vrcholu, popřípadě ještě aktuální vrchol.
Tuto~množinu budu nazývat \emph{sousedi}\labeltext{sousedi}{str:ars_sousedi}.

Pro~získání \hyperref[str:ars_sousedi]{sousedů} začnu s~množinou všech vrcholů,
do~kterých vede hrana z~vrcholu aktuálního stavu.
Budu předpokládat, že tuto množinu je jednoduše možné získat od~grafu křižovatky.
Pokud je nastavený \hyperref[subsubsec:ars_parametry]{parametr} \ref{par:ars_pz},
přidám do~množiny \hyperref[str:ars_sousedi]{sousedů} aktuální vrchol.
Následuje ukázka kódu provádějící tuto~operaci.


\labeltext{\textrm{neighbours}}{alg:ars_neighbours}
% @formatter:off
\begin{code}[fontsize=\footnotesize]
// PZ předem nastavená hodnota parametru povolení zastavení
// G graf křižovatky s funkcí N vracející sousedy vrcholu

// vstup aktuální stav
// výstup množina možných sousedů
neighbours(state)
  neighbours <- empty
  neighbours.addAll(G.N(state.vertex))
  if PZ
    neighbours.add(state.vertex)
  return neighbours
\end{code}
% @formatter:on

Agent se nesmí srazit s~žádným jiným naplánovaným agentem při~jízdě do~sousedního vrcholu.
Kolizní trasu můžeme detekovat funkcemi \ref{alg:kol_safe_vertex}, \ref{alg:kol_safe_step_to}
a \ref{alg:kol_safe_step_from} popsanými v~sekci \nameref{sec:kolize}.
Skoro všechny potřebné informace jsou obsažené ve~stavu a sousedním vrcholu.
Avšak kontroly kolizí potřebují ještě znát \nameref{par:polomer_agenta}.

Pseudokód tohoto kroku je následovný.

\labeltext{\textrm{collision\_free\_neighbours}}{alg:ars_collision_free_neighbours}
% @formatter:off
\begin{code}[fontsize=\footnotesize]
// vstup aktuální stav, množina sousedů, poloměr agenta
// výstup množina nekolizních sousedů
collision_free_neighbours(state, neighbours, da)
  for vertex in neighbours:
    if not (safe_vertex(state.step + 1, vertex, da) and
      safe_step_to(state.step, state.vertex, vertex, da) and
      safe_step_from(state.step, state.vertex, vertex, da))
        sousedi.remove(vertex)
  return neighbours
\end{code}
% @formatter:on

Pro~kontrolu \hyperref[subsubsec:ars_parametry]{parametru} \ref{par:ars_mnv}
je možné projít všechny předchozí stavy až po~počáteční
a odebrat všechny vrcholy vyskytující se alespoň \ref{par:ars_mnv} krát během tohoto průchodu.
Následující pseudokód zobrazuje přesnější postup odstraňování.

\labeltext{\textrm{control\_MNV}}{alg:ars_mnv}
% @formatter:off
\begin{code}[fontsize=\footnotesize]
// MNV předem nastavená hodnota maxima návštěv vrcholu

// vstup aktuální stav, množina sousedních vrcholů
// výstup množina sousedů navštívená nejvýše MNV - 1
control_MNV(state, neighbours)
  occurs <- empty // slovník návštěv vrcholů s implicitní hodnotou 0
  predecessor <- state
  while predecessor is not NULL
    occurs[predecessor.vertex] <- occurs[predecessor.vertex] + 1
    predecessor <- predecessor.parent
  for vertex in neighbours
    if occurs[vertex] >= MNV
      neighbours.remove(vertex)
  return neighbours
\end{code}
% @formatter:on

Pokud je hodnota \hyperref[subsubsec:ars_parametry]{parametru} \ref{par:ars_mpc} konečná,
je možné provést filtrování sousedních vrcholů podle nejkratší vzdálenosti ze~souseda do~nejbližšího cíle.
Tuto vzdálenost víme díky heuristice, která má stejnou hodnotu.
Nejprve si spočítám maximální krok cesty\labeltext{MKC}{str:ars_mkc} (\ref{str:ars_mkc}) pro~daného agenta.
Ten můžu spočítat jako součet počátečního kroku, heuristiky počátečního vrcholu a \ref{par:ars_mpc}.
Jednoduše je vidět, že agent nemůže dorazit do~cíle po~kroku \ref{str:ars_mkc}.

Následně mohu zkontrolovat, jestli se agent může dostat ze~sousedního vrcholu
do~některého cíle do~kroku \ref{str:ars_mkc}.
Vzdálenost souseda do~nejbližšího cíle zjistím z~heuristiky.
Tudíž stačí porovnat součet kroku stavu a heuristiky souseda s~hodnotou \ref{str:ars_mkc}.
Přesný vzhled filtrace je popsán v následujícím pseudokódu.

\labeltext{\textrm{control\_MPC}}{alg:ars_mpc}
% @formatter:off
\begin{code}[fontsize=\footnotesize]
// MKC předem spočtená hodnota maximálního kroku cesty
// heuristics množina vzdáleností do nejbližšího cíle indexovaná vrcholy

// vstup aktuální stav, množina sousedních vrcholů
// výstup množina sousedů dostatečně blízkých cílům
control_MPC(state, neighbours):
  for vertex in neighbours
    if state.step + heuristics[vertex] >= MKC:
      neighbours.remove(vertex)
  return neighbours
\end{code}
% @formatter:on

Poslední filtrování sousedů probíhá pouze, pokud je nastaven
\hyperref[subsubsec:ars_parametry]{parametr} \ref{par:ars_pv}.
Opět lze při~kontrole využít procházení předchozích stavů.
Přesněji nejprve naleznu nejpozději navštívený vrchol před~aktuálním.
Pokud takový vrchol neexistuje, mohu skončit.
Jinak odeberu tento vrchol ze~sousedů.
Následující pseudokód zobrazuje přesnější průběh.

\labeltext{\textrm{control\_PV}}{alg:ars_pv}
% @formatter:off
\begin{code}[fontsize=\footnotesize]
// PV předem nastavená hodnota parametru povolení vrácení

// vstup aktuální stav, množina sousedních vrcholů
// výstup množina sousedů s odebraným předchozím vrcholem
control_PV(state, neighbours)
  if PV
    return neighbours
  predecessor <- state.parent
  while predecessor is not NULL and predecessor.vertex is state.vertex
    predecessor <- predecessor.parent
  if predecessor is not NULL
    neighbours.remove(predecessor.vertex)
  return neighbours
\end{code}
% @formatter:on

Nyní již můžu definovat funkci \ref{alg:sousedni_stavy} jako funkci, která vygeneruje validní sousední stavy.
Funkce nejprve pomocí funkce \ref{alg:ars_neighbours} vygeneruje sousední vrcholy.
Poté~odstraní neplatné sousedy funkcemi \ref{alg:ars_collision_free_neighbours},
\ref{alg:ars_mnv}, \ref{alg:ars_mpc} a \ref{alg:ars_pv}.
Nakonec pro~každý zbylý vrchol vytvoří nový stav.

Pro~vytvoření následujícího stavu je potřeba znát
úhel mezi vrcholy předchozího stavu, aktuálního stavu a dotyčného sousedního vrcholu.
Přesný výpočet zde vynechám.
Budu předpokládat, že se mohu zeptat grafu křižovatky na~tento úhel.

Pseudokód celkové funkce na~generování sousedních stavů je níže.


% @formatter:off
\begin{code}[fontsize=\footnotesize]
// G graf křižovatky s funkcí angle(u, v, w) na výpočet úhlu mezi třemi vrcholy
// heuristics množina vzdáleností do nejbližšího cíle indexovaná vrcholy

// vstup aktuální stav, poloměr agenta
// výstup množina sousedních stavů
neighbour_states(state, da)
  neighbours <- neighbours(state)
  neighbours <- nekolizni_neighbours(state, neighbours, da)
  neighbours <- kontrola_MNV(state, neighbours)
  neighbours <- kontrola_MPC(state, neighbours)
  neighbours <- kontrola_PV(state, neighbours)

  current_vertex <- state.vertex
  last_vertex <- NULL
  if state.parent is not NULL
    last_vertex <- state.parent.vertex
  states <- empty

  for vertex in neighbours
    neighbour_state <- new
    neighbour_state.krok <- state.krok + 1
    if current_vertex is vertex
      neighbour_state.distance <- state.distance
    else
      neighbour_state.distance <- state.distance + 1

    angle <- 0
    if last_vertex is not NULL
      angle <- angle(last_vertex, current_vertex, vertex)
    neighbour_state.angle <- state.angle + angle
    if angle is 0
      neighbour_state.turns <- state.turns
    else
      neighbour_state.turns <- state.turns + 1
    neighbour_state.heuristics <- heuristics[vertex]
    states.add(neighbour_state)

  return states
\end{code}
% @formatter:on

\subsubsection{Výsledný algoritmus}\label{subsubsec:ars_vysledny_algoritmus}

Nakonec popíšu kompletní upravený \nameref{subsec:individualni_a_star} algoritmus.
Jelikož je využívána \ref{str:rs} strategie, stačí definovat funkci \ref{alg:plan_agent}.

Nejprve je nutné pro~daného agenta spočítat heuristiku pro~každý vrchol.
To mohu provést procházením přes~cílové vrcholy.
Pro~každý cíl spočtu vzdálenost od~každého vrcholu do~tohoto cíle.
Pokud je tato hodnota menší než zatím nejmenší nalezená vzdálenost, přepíši tuto hodnotu.
Pro~účely počítání vzdálenosti mezi vrcholy si můžu tyto vzdálenosti uložit předem ke~grafu.
Tyto hodnoty lze získat například Floyd–Warshallovo algoritmem (\citet*{Floyd-Warshall}).

Poté se spočte hodnota posledního kroku \ref{str:ars_mkc} a uloží se pro~pozdější použití.
Stejně tak se zjistí poloměr agenta.

Následně algoritmus vygeneruje počáteční stav a přidá ho do~fronty prohledávaných stavů.
Poté~algoritmus odebere z~fronty nejmenší stav podle součtu ceny cesty do~vrcholu stavu a heuristiky.
Pokud vrchol odebraného stavu je mezi cílovými, zkonstruuje se cesta a přiřadí se agentovi.
Plánování agenta je v~tomto případě hotové, už~zbývá jen přidat naplánovanou cestu
do~\hyperref[par:obsazene_pozice]{tabulky obsazených pozic} pomocí \ref{str:kol_add_planned_agent},
aby nedošlo ke~kolizi s~pozdějšími agenty.
Agent je následně vrácen zpět \hyperref[sec:simulace]{simulátoru}.
Pokud vrchol stavu není mezi cílovými,
získají se pomocí funkce \ref{str:ars_sousedi} sousední stavy a přidají se do~fronty.
Takto algoritmus pokračuje, dokud nenajde validní cestu, nebo není fronta prázdná.
V~tom případě nemůže být agent naplánován, jeho cestování je pozdržené, a simulátoru se vrátí $NULL$.

Konstrukci cesty provedu postupným procházením předků koncového stavu.
Pro~každý takový stav přidám na~začátek cesty vrchol zmíněného stavu.
Tento postup budu opakovat, dokud nenarazím na předka s rodičem $NULL$.
Tento postup umístím do funkce \ref{alg:ars_construct_path}\labeltext{\textrm{construct\_path}}{alg:ars_construct_path}.

% @formatter:off
\begin{code}[fontsize=\footnotesize]
// vstup finální stav cesty
// výstup kompletní cesta z vrcholů
construct_path(state)
  path <- empty
  while state is not NULL
    path.addFirst(state.vertex)
    state <- state.parent
\end{code}
% @formatter:on


Následující pseudokód zachycuje celý \nameref{subsec:individualni_a_star} algoritmus.

% @formatter:off
\begin{code}[fontsize=\footnotesize]
// G graf křižovatky se vzdálenostmi mezi vrcholy
// MPC parametr maximální prodlevy cesty

// vstup plánovaný krok, agent, vjezd a výjezdy
// výstup agent nebo NULL
plan_agent(step, agent, entry, exits)
  heuristics <- compute_heuristics(G)
  MKC <- step + MPC + heuristics[entry]
  da <- agent.diameter

  // prioritní fronta porovnávající prvky podle ceny cesty a heuristiky
  queue <- emptyPriorityQueue
  initial_state <- empty
  initial_state.vertex <- entry
  initial_state.step <- step
  initial_state.heuristics <- heuristics[entry]
  queue.push(initial_state)

  while not queue.empty()
    state <- queue.pop()
    if exits.contain(state.vertex)
      agent.path <- construct_path(state)
      add_planned_agent(step, agent)
      return agent
    else
      queue.addAll(neighbour_states(state, da))

  return NULL
\end{code}
% @formatter:on


\nameref{sec:a_star} algoritmus je možné zrychlit pamatováním si už~navštívených stavů (stavů odebraných z~fronty).
Z~množiny sousedních stavů je možné odebrat všechny navštívené stavy.
K~tomu je potřeba určit, kdy~jsou stavy stejné.
Obecně jsou zde stavy stejné, pokud mají stejný vrchol, stejný krok a zároveň vrcholy všech předků si odpovídají.
Pokud je hodnota \ref{alg:ars_mnv} $\infty$ a je povolené vracení, stačí jenom kontrolovat aktuální vrchol a krok.
Během testování jsem zjistil, že kontrola pouze rovnosti posledního vrcholu a stavu značně zrychluje běh algoritmu,
aniž by~se výrazně zhoršilo plánování.
Proto jsem se rozhodl udělat kompromis a určit, že dva stavy jsou si rovné právě tehdy,
když mají stejný vrchol, krok, a jejich rodiče mají stejný vrchol.
