\subsection{Conflict-Based Search (\ref{str:cbs})}\label{subsec:conflict_based_search}\labeltext{CBS}{str:cbs}

%Popis algoritmu, úprava pro můj problém.
%Parametry, pseudokód.

\nameref{subsec:conflict_based_search} algoritmus \citep*{Sharon} rozšiřuje jakýkoliv \ref{str:rs} algoritmus
na multiagentní plánování.
V mém případě budu rozšiřovat \ref{str:a_star_ars}.

\ref{str:cbs} začíná individuálním naplánováním všech agentů nezávisle na~sobě.
Čili agenti nesmějí mít kolize s~již cestujícími agenty,
avšak mohou mí kolizní trajektorii s~jinými aktuálně plánovanými agenty.
Poté se zkontroluje, zda-li nemají nějací agenti kolizní trajektorie.
Pokud ne, plánování úspěšně končí.
Jinak se prohledávání rozdělí na dva případy.
V obou případech je přeplánován jeden agent s podmínkou, že se musí vyhnout koliznímu místu.
Poté se opakuje opětovné hledání kolizí a rozdělování na případy.
Aby nedošlo k zacyklení, je nutné při plánování agenta vyhnout se nejen aktuální kolizi, ale také všem předchozím.
Výpočet postupně vytváří strom,
kde každý vrchol obsahuje cesty agentů (mohou být navzájem kolizní) a tabulku zakázaných pozic.
Algoritmus skončí v~prvním nalezeném vrcholu neobsahujícím kolizní trasy.

Mohlo by se stát, že plánování jednoho agenta selže.
V~tom případě je agent zcela odstraněn z~vrcholu.
Následně jsou nalezeni agenti, kteří byli v~historii přeplánováni kvůli odstraněnému agentovi.
Pro~tyto agenty jsou nalezeny nové cesty, jelikož pro~ně může existovat lepší cesta.

Algoritmus postupně prochází listy stromu výpočtu.
Pořadí průchodu je určeno počtem agentů.
Pokud je počet agentů u~více listů shodný, vybere se vrchol s nejmenší vzdáleností podobně jako u \ref{str:a_star_arsg}.
Algoritmus naplánuje pouze agenty, kteří mají cesty ve~vybraném listu, vjezd zbylých agentů je zamítnut.

\ref{str:cbs} najde optimální cestu pro všechny agenty \citep{Sharon}.
Avšak velikost stromu může být obrovská.
Proto jsem se rozhodl obětovat optimalitu s zjednodušit práci algoritmu.
Ve zjednodušeným režimu algoritmus přeplánuje takovým způsobem, aby neměl žádné kolize s ostatními plánovanými agenty.

\subsubsection{Parametry}\label{subsubsec:cbs_parametry}

\nameref{subsubsec:cbs_parametry} algoritmu jsou stejné jako u \ref{str:varsg} a mají podobný význam.
Hodnoty Maximum návštěv vrcholu (\ref{par:ars_mnv}), Povolené zastavování (\ref{par:ars_pz}),
Maximální prodleva při~cestě (\ref{par:ars_mpc}) a Povolené vracení (\ref{par:ars_pv})
algoritmus používá při~plánování jednoho agenta.
Tyto \hyperref[subsubsec:ars_parametry]{parametry} ovlivňují plánování stejně jako u \ref{str:a_star_ars}.
Hodnota parametru \ref{par:arsg_zvp} opět určuje po jak dlouhé prodlevě má algoritmus přejít na zjednodušené plánování.

\subsection{CBS-OID}\label{subsec:cbsoid}

\ref{str:cbs} lze podobně jako \ref{str:varsg} rozšířit na \ref{str:oid} variantu.
K plánovaným agentů se přidají agenti z předchozích kroků.
Jako počáteční cesty těchto agentů se použijí jejich již naplánované trasy, tudíž se znova nepočítají.
Výpočet je poté shodný, až na~případy, kdy pro~ně nebyla nalezena cesta.
Pokud k~takové situaci dojde, namísto odstranění agenta se odstraní celý list ze~stromu výpočtu.

Parametry jsou rozšířené stejně jako u \nameref{subsubsec:a_star_aoid} o~Maximální počet agentů (\ref{par:aoid_mpa})
a Počet přeplánovaných kroků (\ref{par:aoid_ppk}).
Význam těchto parametrů je shodný.
