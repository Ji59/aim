\subsection{Hromadný A* (\ref{str:a_star_arsg})}\label{subsec:hromadny_a_star}\labeltext{A*RSG}{str:a_star_arsg}

%Rozšíření A* pro více agentů, popis vylepšení.
%Parametry, pseudokód.

Tento algoritmus plánuje agenty \ref{str:rsg} strategií, čili hledá cesty pro~všechny nové agenty najednou.
Algoritmus je založen na~\nameref{sec:a_star} algoritmu.

\subsubsection{Složený hromadný A*}\label{subsubsec:arsg_slozeny_hromadny}
Předchozí \ref{str:a_star_ars} algoritmus lze poměrně jednoduše rozšířit na~variantu pro~více agentů.
Pokud potřebuji naplánovat $n$ agentů, můžu rozšířit strukturu \ref{str:a_star_ars} stavu o~vrcholy,
na~kterých se nacházejí všichni agenti.
Tedy místo jednoho vrcholu by stav obsahoval $n$-tici vrcholů.
Stav je koncový, pokud všechny vrcholy jsou v~cílových vrcholech odpovídajících agentů.

Dále potřebuji upravit hodnotu \hyperref[par:ars_vzdalenost]{vzdálenosti} jako součet vzdáleností jednotlivých agentů.
Stejným způsobem upravím hodnoty \hyperref[par:ars_uhel_zataceni]{úhlu zatáčení},
\hyperref[par:ars_pocet_zataceni]{počtu zatáčení} a \hyperref[par:ars_heuristika]{heuristiky} jako součet
odpovídajících hodnot přes všechny agenty.

Při~vytváření následujících stavů určitého stavu je nutné zjistit \hyperref[str:ars_sousedi]{sousedy} vrcholů
jednotlivých agentů stejným způsobem jako u~\ref{str:a_star_ars} (sekce \ref{subsubsec:sousedni_stavy}).
Každý prvek kartézského součinu těchto množin \hyperref[str:ars_sousedi]{sousedů} tvoří vrcholy následujícího stavu.

Oproti \ref{str:a_star_ars} je navíc nutné kontrolovat kolize mezi jednotlivými plánovanými agenty.
Pokud je nalezená kolize mezi alespoň dvěma plánovanými agenty, je daný stav zamítnut.

Bohužel prohledávací prostor má velikost $|V|^n$, kde $n$ je počet plánovaných agentů.
Například pro~čtvercovou křižovatku s~\hyperref[par:velikost_krizovatky]{velikostí} $4$,
jedním \hyperref[par:vjezdy]{vjezdem} a jedním \hyperref[par:vyjezdy]{výjezdem} (Obrázek~\ref{fig:square_type_graph})
by~měl prohledávací prostor pro~jednoho agenta velikost $24$, pro~dva agenty $576$, pro~tři $13824$.
Proto je plánování pro~velký počet agentů je téměř nemožný z~časových i~paměťových důvodů.
Naštěstí byly vyvinuty techniky na~chytřejší plánování pomocí \nameref{sec:a_star}.

