\section{Kontrola kolize}\label{sec:kolize}

%Rozbor případů, kdy může nastat mezi agenty kolize (základ v MAPF).
%Rozšíření problému na agenty s nenulovou velikostí.
%Popis pomocných datových struktur.

Při~hledání cest musí algoritmus brát v~potaz již naplánované agenty.
Pro~tyto účely jsem si vytvořil následující pomocnou datovou strukturu, do~které ukládám potřebné informace.

\paragraph{Tabulka~obsazených~pozic}\label{par:obsazene_pozice} si pamatuje pro~každý krok množinu dvojic vrcholu a agenta.
Díky této struktuře můžu jednoduše a rychle zjistit,
zda~se v~daný krok vyskytuje již naplánovaný agent na~určeném vrcholu, a popřípadě o~kterého agenta se~jedná.
Po~každém naplánování agenta je postupně přidána dvojice do~každého kroku, kdy~se agent vyskytuje na~křižovatce,
aby byla \nameref{par:obsazene_pozice} aktuální.

Kontrola kolize probíhá vždy pro~přejezd jednoho agenta $a$ mezi dvěma vrcholy.
Přejezd definuji jako výskyt agenta na některém vrcholu $u$ v~kroku $s$
a následný výskyt na některém sousedním vrcholu $v \in N(u)$ v~kroku $s + 1$.
Pokud agent zůstává na~stejném vrcholu ($u = v$), situace pořád splňuje definici přejezdu.
Z~definice \ref{str:mapf} mohu předpokládat, že se agent pohybuje pouze z~jednoho vrcholu na~jeho sousední vrchol a
agent se nachází na~některém vrcholu v~každém kroku od~kroku příjezdu do~kroku odjezdu.
Zároveň budu předpokládat, že žádný vjezd křižovatky není zároveň výjezdem.
V~těchto případech by se cesta mohla skládat pouze z~jednoho vrcholu a neobsahovala by žádný přejezd.

Kontrola zjišťuje bezpečnost přejezdu oproti určité množině agentů $B$.
Tato množina může být například množina již naplánovaných agentů, nebo množina agentů plánovaných ve~stejném kroku.
Přejezd je vyhodnocen jako nekolizní, pokud se agent $a$ během cesty nesrazí s~žádným agentem z~množiny $B$.
Avšak kontrola již nekontroluje kolize mezi agenty množiny $B$.

Agenti nejsou body pohybující se po~přímkách jako v~obecném \ref{str:mapf} problému,
proto je nutné provádět kontrolu jinak.
Budu kontrolovat každého agenta z~množiny $B$, jestli se během svého přejezdu nesrazí s~agentem $a$.

\paragraph{Safe distance}\label{par:safe_distance} (značený~$d$) určuje minimální povolenou vzdálenost mezi dvěma agenty.
\nameref{par:safe_distance} je parametr kontroly a lze nastavit před~spuštěním simulace.
Nenulová hodnota \nameref{par:safe_distance} má šanci snížit kolize,
pokud jsou zavedené nepřesnosti parametrem \nameref{par:odchylka}.

Jelikož se můžou agenti v~libovolný okamžik jakkoliv natočit, pracuji ve~výpočtech se zjednodušeným modelem agentů.
Namísto počítání složitého aktuálního natočení agenta a následné převedení na~obdélník,
je agent nahrazen pomyslným kruhem.

\paragraph{Poloměr agenta}\label{par:polomer_agenta} určuje poloměr kruhu zjednodušeného modelu
a spočítá se z~agentovo délky~$l$ a šířky~$w$ jako $\frac{\sqrt {l^2 + w^2}}{2}$.
Kontroly poté zjišťují, jestli jsou kruhy tvořené pozicí agenta a jeho poloměrem disjunktní.
Jinými slovy agenti jsou kontrolami vyhodnoceni v~kolizních trasách,
pokud se během cesty středy agentů přiblíží na~vzdálenost menší nebo rovnu součtu jejich poloměrů a \emph{safe distance}.
Toto zjednodušení nemůže způsobit kolizi, jelikož je celý agent umístěn uvnitř \hyperref[par:polomer_agenta]{poloměru agenta}.
Zároveň počítání s~kruhem značně zrychluje samotný výpočet.

\subsection{Kontrola přejezdu s jedním agentem}\label{subsec:kontrola_prejezdu_jeden_agent}

V této části popíšu, jak se určuje, zda-li nastane kolize mezi dvěma agenty při přejezdu.
Agenty si označím $a$ a $b$, a jejich pozice jako $\mathbf{a} = (a_x, a_y)$, resp. $\mathbf{b} = (b_x, b_y)$.
Vzdálenost agentů je rovna $dist = \sqrt{(a_x - b_x)^2 + (a_y - b_y)^2}$.

Vrchol, ze kterého agent $a$ cestuje si označím jako $u$ a cílový vrchol jako $v$.
Podobně si označím přejezd agenta $b$ jako cestu z $p$ do $q$.
Agenti se mezi vrcholy pohybují v čase $t \in [0, 1]$ konstantní rychlostí.
Pozice agenta $a$ je v čase $t$ rovna $a_x = (1-t)u_x + tv_x$, $a_y = (1-t)u_y + tv_y$.
Podobně je pozice agenta $b$ rovna $(1-t)\mathbf{p} + t\mathbf{q}$.
Vzdálenost agentů v čase $t$ budu značit $dist_t$.

Pokud se agenti přiblíží v nějaký čas blíže než je součet jejich \hyperref[par:polomer_agenta]{poloměrů}
zvětšený o~\nameref{par:safe_distance}, je přejezd vyhodnocen jako kolizní.
Matematicky zapsáno přejezd je kolizní pokud $(\exists t\in[0, 1])(d_a + d_b + d \geq dist_t)$,
kde $d_a$ a $d_b$ značí \hyperref[par:polomer_agent]{poloměr} agenta $a$ a agenta $b$,
a $d$ je hodnota \nameref{par:safe_distance}.

Není nutné ověřovat platnost nerovnosti pro~každé $t$, postačuje jediné ověření v~čase, kdy jsou si agenti nejblíže.
Pro~vypočítání této hodnoty spočítám derivaci $dist_t$ podle $t$ a zjistím, kdy je rovna nule.
Derivace $dist_t$ je nulová právě tehdy když je nulový výraz pod~odmocninou.
Proto stačí počítat derivaci pouze pro~$dist^2_t = (a_x - b_x)^2 + (a_y - b_y)^2$.
\[
	\frac{d}{dt} dist^2_t =
	\frac{d}{dt} \left(\left(a_x - b_x\right)^2 + \left(a_y - b_y\right)^2\right) =
	\frac{d}{dt} \left(a_x - b_x\right)^2 + \frac{d}{dt} \left(a_y - b_y\right)^2
\]
Nyní zjednoduším $\frac{d}{dt} (a_x - b_x)^2$, pro $y$ je postup stejný.
\begin{gather*}
	\frac{d}{dt} \left(a_x - b_x\right)^2 =
	\frac{d}{dt} \left(\left(1-t\right)u_x + tv_x - \left(1-t\right)p_x + tq_x\right)^2 =  \\
	\frac{d}{dt} \left(t\left(-u_x + v_x -\left(-p_x + q_x\right)\right) + u_x - p_x\right)^2 =  \\
	\left(t\left(v_x - u_x + p_x - q_x\right) + u_x - p_x\right)\left(v_x - u_x + p_x - q_x\right) =  \\
	t\left(v_x - u_x + p_x - q_x\right)^2 + \left(u_x - p_x\right)\left(v_x - u_x + p_x - q_x\right)
\end{gather*}

Pro~jednodušší počítání si označím $x_0 = v_x - u_x + p_x - q_x$ a $x_1 = u_x - p_x$.
Odtud dostávám $\frac{d}{dt} (a_x - b_x)^2 = tx^2_0 + x_0 x_1$.
Stejně si zavedu $y_0$ a $y_1$.
Celkově dostávám $\frac{d}{dt} dist^2_t = tx^2_0 + x_0 x_1 + ty^2_0 + y_0 y_1$.
Nyní určím, kdy je vzdálenost nejmenší dosazením do~rovnice.
\begin{align*}
	\frac{d}{dt} dist^2_t &= 0 \\
	tx^2_0 + x_0 x_1 + ty^2_0 + y_0 y_1 &= 0 \\
	t\left(x^2_0 + y^2_0\right) &= x_0 x_1 + y_0 y_1
\end{align*}

Nyní rozliším dva případy podle podmínky $x^2_0 = 0 \wedge y^2_0 = 0$.
Pokud podmínka platí, platí $0 = x_0 = v_x - u_x + p_x - q_x = \left(p_x - u_x\right) + \left(v_x - q_x\right)$.
Pro~$y_0$ platí to~samé.
Odkud dostávám $\mathbf{p} - \mathbf{u} = \mathbf{q} - \mathbf{v}$.
Tato~rovnost nastane pouze pokud je rozdíl počátečních vektorů vrcholů agentů shodný s~rozdílem vektorů cílových vrcholů.
Z~toho vyplývá, že agenti mají rovnoběžné trasy o~stejné vzdálenosti a jsou si stejně vzdálení v~každém okamžiku.
Proto si jsou nejblíže například pro~$t = 0$.

Pokud agenti nemají rovnoběžné trasy, $x^2_0 \neq 0 \vee y^2_0 \neq 0$.
Potom mohu jejich součtem vydělit, jelikož je nenulový.
Čas, kdy jsou si agenti nejblíže, je tedy možné spočítat pomocí $t' = \frac{x_0 x_1 + y_0 y_1}{x^2_0 + y^2_0}$.
Může se stát, že výsledný čas bude mimo čas cesty.
Proto je poslední úprava omezení času $t = \max(\min(t', 1), 0)$.

Vypočtená nejbližší vzdálenost nezáleží na~samotných agentech, ale pouze na~jejich trasách, čili $u$, $v$, $p$ a $q$.
Abych se vyhnul opakovanému počítání této hodnoty, můžu si výslednou vzdálenost předpočítat.
K~tomu budu potřebovat čtyřnásobně vnořený cyklus, který bude procházet všechny vrcholy jako $u$.
Dále se do~$v$ dosadí všichni sousedi vrcholu $u$ a $v$ samotný.
Za~$p$ se opět dosadí všechny vrcholy grafu a za~$q$ opět sousedi $p$ a opět samotný $p$.
Výsledky si mohu ukládat do~pole \ref{str:kol_vzdalenosti}\labeltext{\textrm{distances}}{str:kol_vzdalenosti}
o~velikosti $O(n^2 deg^2)$, kde $n$ je počet vrcholů grafu a $deg$ maximální stupeň grafu zvětšený o~jedna.

Při~kontrole kolize mezi dvěma agenty stačí pouze porovnat jejich \hyperref[par:polomer_agent]{poloměry}
s~patřičnou hodnotou ve~struktuře \ref{str:kol_vzdalenosti}.
Algoritmus tedy vypadá následovně.
\labeltext{\textrm{safe\_transfer}}{alg:kol_prejezd}
% @formatter:off
\begin{code}[fontsize=\footnotesize]
// tabulka vzdálenosti distances
// minimální povolená vzdálenost agentů d

// agent a s poloměrem d_a cestuje z vrcholu u do v
// agent b s poloměrem d_b cestuje z vrcholu p do q
// vrací true pokud je přejezd nekolizní, jinak false
safe_transfer(u, v, d_a, p, q, d_b)
	return distances[u][v][p][q] > d_a + d_b + d
\end{code}
% @formatter:on

\subsection{Kontrola přejezdu s množinou agentů}\label{subsec:kontrola_prejezdu_mnozina_agentu}

Zde popíšu kontrolu, zda-li se agent $a$ při~přejezdu z~$u$ do~$v$ nesrazí s~některým agentem množiny $B$.
Nejjednodušší způsob by byl projít celou množinu $B$ a zjistit, jestli není agent $a$ v~kolizi s~každým agentem v~$B$.

V~následující části budu předpokládat, že velikost agenta je oproti velikosti bloku malá a
množství agentů na~křižovatce se může blížit počtu vrcholů křižovatky.
Zároveň předpokládám, že množina $B$ bude většinou množina agentů na~křižovatce.

Abych se vyhnul počítání kolizní trasy se všemi agenty,
budu uvažovat pouze agenty na~vrcholech dostatečně blízkých aktuálnímu agentovi.
K~tomu použiji následující tvrzení:

\begin{tvrz}[Kolizní vzdálenost]
	\label{tvrz:kol_kolizni_vzdalenost}
	Nechť agent $a$ přejíždí z~$u$ do~$v$ a má poloměr $d_a$.
	Agent $b$ vyjíždí ve~stejném kroku z~vrcholu $p$ a v~následujícím kroku je taktéž na~některém vrcholu křižovatky.
	Pokud platí $(\forall q \in N(p))(\ref{str:kol_vzdalenosti}[u][v][p][q] > d_a + d_{\max} + d)$,
	potom se agenti $a$ a $b$ během přejezdu nepřiblíží na~vzdálenost menší rovnou $d$.
	Zde $\ref{str:kol_vzdalenosti}[u][v][p][q]$ značí nejmenší vzdálenost při~přejezdu vypočtenou v~předchozí části.
	$d_{\max} = \frac{\sqrt{l_{\max}^2 + w_{\max}^2}}{2}$ je maximální možný poloměr agenta,
	$l_{\max} a w_{\max}$ jsou horní meze pro~velikost agenta popsané v~kapitole \nameref{subsec:generovani_agentu}.
\end{tvrz}
\begin{dukaz}
	Délka $l_b$ a šířka $w_b$ agenta $b$ nesmějí být vyšší než nastavené horní meze.
	Tedy $l_b \leq l_{\max}$ a $w_b \leq w_{\max}$, a pro~poloměr agenta $b$ platí $d_b \leq d_{\max}$.
	Označím si $q_{\min} \in N(p)$ jako sousední vrchol $p$, který má nejmenší vzdálenost při~přejezdu.
	Formálně $(\forall q \in N(p))(\ref{str:kol_vzdalenosti}[u][v][p][q_{\min}] \geq \ref{str:kol_vzdalenosti}[u][v][p][q])$.
	Z~předpokladu vyplývá $\ref{str:kol_vzdalenosti}[u][v][p][q_{\min}] > d_a + d_{\max} + d$.
	Agent $b$ se musí v~následujícím kroku nacházet na~některém vrcholu $q'$ sousedícím s~$p$.
	Odtud dostávám $\ref{str:kol_vzdalenosti}[u][v][p][q'] \geq \ref{str:kol_vzdalenosti}[u][v][p][q_{\min}] >
	d_a + d_{\max} + d \geq d_a + d_b + d$.
	Z~definice $\ref{str:kol_vzdalenosti}$ vyplývá, že se agenti $a$ a $b$ během přejezdu nepřiblíží na~vzdálenost $d$.
\end{dukaz}


Z~tvrzení vyplývá, že nejvyšší možná kolizní vzdálenost pro~určitého agenta $a$ je $d_{safe} = d_a + d_{\max} + d$,
kde $d_a$ je \hyperref[par:polomer_agenta]{poloměr agenta} $a$, $d_{\max}$ je maximální možný poloměr agenta a
$d$ je parametr \nameref{par:safe_distance}.
Ze~struktury \ref{str:kol_vzdalenosti} se omezím na~hodnoty $\textrm{distances}_{uv} = \ref{str:kol_vzdalenosti}[u][v]$.
Následně si vyberu pouze takové vrcholy $p$, pro~které platí
$d_{safe} \geq \min_{q\in N(p)} \ref{str:kol_vzdalenosti}_{uv}[p][q]$.
Ostatní agenti $B$ se dle tvrzení nemohou s~agentem $a$ srazit.

Výběr dostatečně blízkých vrcholů není opět nutné dělat pokaždé.
Pro~tyto účely si vyrobím podobnou datovou strukturu jako \ref{str:kol_vzdalenosti}.
Ve~struktuře \ref{str:kol_serazene_vzdalenosti}\labeltext{\textrm{sort\_dist}}{str:kol_serazene_vzdalenosti}
bude pro~každou dvojici sousedů $u$ a $v$ uloženo uspořádané pole dvojic vrcholu a jeho nejmenší vzdálenosti.
Platí tedy
\begin{gather*}
(\forall u, v \in V)(\forall i \in {1, \dots, |V| - 1})(\ref{str:kol_serazene_vzdalenosti}[u][v][i - 1][1] \leq
\ref{str:kol_serazene_vzdalenosti}[u][v][i][1])
	\\
	(\forall u, v \in V)((p, d) = \ref{str:kol_serazene_vzdalenosti}[u][v][i] \rightarrow
	d = \min_{q \in N(p)}\ref{str:kol_vzdalenosti}[u][v][p][q])
\end{gather*}

Pro~jednoduché vyhledávání, zda-li je některý agent z~$B$ na~určitém vrcholu, budu vyžadovat,
aby $B$ měla stejnou strukturu jako \nameref{par:obsazene_pozice}.
Avšak poté je nutné ještě uvést krok přejezdu $s$.

Algoritmus celkové kontroly vypadá následovně:
\labeltext{\textrm{safe\_transfer\_set}}{alg:kol_prejezd_mnozina}
% @formatter:off
\begin{code}[fontsize=\footnotesize]
// tabulka seřazených vzdáleností sort_dist
// minimální povolená vzdálenost agentů d
// maximální velikost agenta d_max

// krok začátku přejezdu s
// agent a s poloměrem d_a cestuje z vrcholu u do v
// množina agentů B indexovaná kroky a poté vrcholy
// vrací true pokud je přejezd nekolizní, jinak false
safe_transfer_set(s, u, v, d_a, B)
	d_safe = d_a + d_max + d
	for p, p_dist in sort_dist[u][v]
		if p_dist > d_safe
			return true
		if B[s] is NULL
			continue
		b <- B[s][p]
		if not safe_transfer(u, v, d_a, p, b.path[s + 1], b.diameter)
			return false
	return true
\end{code}
% @formatter:on
