\chapter{Algoritmy}\label{ch:algoritmy}

%Popis společné části algoritmů (vstup, výstup).
%
%Stručný text o parametrech algoritmů.

Algoritmy tvoří plánující logiku křižovatky.
Jak bylo popsáno v~sekci \hyperref[sec:simulace]{simulace}, algoritmus hledá trasy pro~všechny agenty.
Každému úspěšně naplánovanému agentovi se přidělí nalezená trasa.

Všechny algoritmy podporují stejnou funkci \ref{alg:plan_agents},
kterou volá simulátor pro~nalezení tras agentů v~určitém kroku.
Vstup této funkce je krok simulace a množina agentů přijíždějících v~daném kroku.
Aby simulátor poznal, kterým agentům úspěšně přiřadil algoritmus trasu,
vrací funkce množinu úspěšně naplánovaných agentů.
Níže je zobrazen pseudokód výchozího chování funkce
\textrm{plan\_agents}\labeltext{\textrm{plan\_agents}}{alg:plan_agents}.

% @formatter:off
\begin{code}[xrightmargin=6em]
// vstup plánovaný krok step, množina agentů agents
// výstup množina naplánovaných agentů planned_agents
plan_agents(step, agents)
agents_mapped <- empty_set
for agent in agents
agents_mapped.add([agent, agent.entry, agent.exits])
return plan_agents(agents_mapped, step)
\end{code}
% @formatter:on

Všechny algoritmy podporují další pomocné funkce pro~zjednodušení a propojení algoritmů.
První pomocná funkce se~nazývá \ref{alg:plan_agent},
která má za~úkol naplánovat pouze jednoho agenta.
Tuto funkci využiji při~použití \ref{str:rs} strategie
(této strategii odpovídá výchozí implementace funkce \ref{alg:plan_agents}).
Funkce \textrm{plan\_agent}\labeltext{\textrm{plan\_agent}}{alg:plan_agent} dostává na~vstup krok, ve~kterém má~být agent naplánován.
Další parametry jsou agent, který má~být naplánován, vrchol,
ze~kterého agent vyjíždí a množina vrcholů, na~kterých může agent skončit.
Implementace funkce už~není jednotná a závisí na~každém algoritmu.
Funkce vrací agenta pokud byl úspěšně naplánován, jinak $NULL$.

% @formatter:off
\begin{code}[xrightmargin=14em]
// vstup plánovaný krok step, agent agent,
// vjezd entry a výjezdy exits
// výstup agent nebo NULL
plan_agent(step, agent, entry, exits)
// implementace algoritmu
if plánování úspěšné
agent.path <- planned_path
return agent
else
return NULL
\end{code}
% @formatter:on

Další pomocná funkce je rozšíření funkce \ref{alg:plan_agents}, která má na~vstupu krok plánování.
Další vstup je množina trojic, kde~první prvek trojice je agent, který má~být naplánován.
Další prvek je vrchol ze~kterého agent vyjíždí a
poslední prvek je množina cílových vrcholů, což jsou možné vrcholy, na~kterých má agent skončit.
Funkce opět vrací množinu naplánovaných agentů.
Pseudokód výchozího chování je následovný.
% @formatter:off
\begin{code}[xrightmargin=6em]
// vstup plánovaný krok step, množina trojic agentů,
// vjezdů a výjezdů agents_entries_exits
// výstup množina naplánovaných agentů planned_agents
plan_agents(step, agents_entries_exits)
planned_agents <- empty_set
for agent_entry_exits in agents_entries_exits
agent = agent_entry_exits[0]
entry = agent_entry_exits[1]
exists = agent_entry_exits[2]
planned_agent <- plan_agent(step, agent, entry, exists)
if planned_agent is not NULL
planned_agents.add(planned_agent)
return planned_agents
\end{code}
% @formatter:on

Implementované algoritmy mají různé nastavitelné parametry, které ovlivňují nalezené trasy, optimalitu či~dobu běhu.
Každý algoritmus má speciální parametry pro~něj vhodné.
Tyto~parametry proto budou popsány zvlášť a zároveň zvlášť testovány.

\section{Kontrola kolize}\label{sec:kolize}

%Rozbor případů, kdy může nastat mezi agenty kolize (základ v MAPF).
%Rozšíření problému na agenty s nenulovou velikostí.
%Popis pomocných datových struktur.

Při~hledání cest musí algoritmus brát v~potaz již naplánované agenty.
Pro~tyto účely jsem si vytvořil následující pomocnou datovou strukturu, do~které ukládám potřebné informace.

\paragraph{Tabulka~obsazených~pozic}\label{par:obsazene_pozice} si pamatuje pro~každý krok množinu dvojic vrcholu a agenta.
Díky této struktuře můžu jednoduše a rychle zjistit,
zda~se v~daný krok vyskytuje již naplánovaný agent na~určeném vrcholu, a popřípadě o~kterého agent se~jedná.
Po~každém naplánování agenta je postupně přidána dvojice do~každého kroku, kdy~se agent vyskytuje na~křižovatce,
aby byla \nameref{par:obsazene_pozice} aktuální.

Kontrola kolize probíhá třemi fázemi, kontroluje se~\nameref{subsec:bezpecnost_vrcholu},
\nameref{subsec:cesta_do_vrcholu} a \nameref{subsec:cesta_z_vrcholu}.

\paragraph{Safe distance}\label{par:safe_distance} (značený~$d$) určuje minimální povolenou vzdálenost mezi dvěma agenty.
\nameref{par:safe_distance} je společný parametr všech kontrol a lze nastavit před~spuštěním simulace.
Nenulová hodnota \nameref{par:safe_distance} má šanci snížit kolize,
pokud jsou zavedené nepřesnosti parametrem \nameref{par:odchylka}.

Jelikož se můžou agenti v~libovolný okamžik jakkoliv natočit, pracuji ve~výpočtech se zjednodušeným modelem agentů.
Namísto počítání složitého aktuálního natočení agenta a následné převedení na~obdélník,
je agent nahrazen pomyslným kruhem.

\paragraph{Poloměr agenta}\label{par:polomer_agenta} určuje poloměr kruhu zjednodušeného modelu
a spočítá se z~agentovo délky~$l$ a šířky~$w$ jako $\frac{\sqrt {l^2 + w^2}}{2}$.
Kontroly poté zjišťují, jestli jsou kruhy tvořené pozicí agenta a jeho poloměrem disjunktní.
Jinými slovy agenti jsou kontrolami vyhodnoceni v~kolizních trasách,
pokud se během cesty středy agentů přiblíží na~vzdálenost menší nebo rovnu součtu jejich poloměrů a \emph{safe distance}.
Toto zjednodušení nemůže způsobit kolizi, jelikož je celý agent umístěn uvnitř \hyperref[par:polomer_agenta]{poloměru agenta}.
Zároveň počítání s~kruhem značně zrychluje samotný výpočet.

\subsection{Bezpečnost vrcholu}\label{subsec:bezpecnost_vrcholu}

%Popis postupu kontroly, pseudokód, náčrtek.

\hyperref[subsec:bezpecnost_vrcholu]{Kontrola bezpečnosti vrcholu} zjišťuje,
zda~je bezpečný výskyt agenta v~určitém kroku na~určeném vrcholu.
Kontrola je rozšíření první \ref{str:mapf} podmínky
\uv{žádní dva agenti se nesmí nacházet na~stejném vrcholu v~jednom kroku} \eqref{eq:mapf_kolize_vrchol}.

Tato kontrola pracuje s~vrcholem~$v$, krokem~$s$ a poloměrem agenta~$r$, pro~kterého je kontrola určená.
Dále algoritmus zná maximální povolenou velikost agenta.
Z~maximální délky a šířky je předpočítán maximální poloměr agenta~$m$.

Na~obrázku (Obrázek~\ref{fig:kolize_na_vrcholu}) je ukázka situací, ve~kterých tato kontrola selže.
Vlevo dochází ke~kolizi, jelikož jsou dva agenti na~stejném vrcholu.
Vpravo nastává situace, kdy~jsou dva vrcholy a agenti na~nich příliš blízko.
Situace je kontrolou vyhodnocena jako kolizní i~když se agenti nepřekrývají.

\begin{figure}[h]
	\centering
	\includegraphics[width=\textwidth]{../img/kolize_vrchol}
	\caption{
		Ukázka kolizních situací, které jsou detekovány \hyperref[subsec:bezpecnost_vrcholu]{kontrolou bezpečnosti vrcholu}.
		Na~obrázcích jsou černě zobrazeny vrcholy.
		Obdélníky reprezentují agenty a kruhy tvoří bezpečnou zónu příslušných agentů.
	}
	\label{fig:kolize_na_vrcholu}
\end{figure}

Kontrola pro~vrchol $v$ začne procházet všechny vrcholy grafu od~nejbližšího $v$ podle eukleidovské vzdálenosti.
Pro~každý vrchol~$u$, který je blíže než~$r + d + m$, se nejprve zjistí,
jestli je na~vrcholu~$u$ v~kroku~$s$ nějaký agent.
Pokud není, pokračuje kontrola dalším vrcholem.
Jinak se spočítá poloměr agenta~$r'$ nacházejícího se v~kroku~$s$ na~vrcholu~$u$.
Pokud je eukleidovská vzdálenost vrcholů $u$ a $v$ menší nebo rovna~$r + r' + d$, kontrola selže.
V~opačném případě přejde kontrola na~další vrchol.
Pro~vrcholy vzdálené více než~$r + d + m$ uspěje kontrola triviálně,
neexistuje možnost že by agenti na~těchto vrcholech byly v~kolizi.

Níže je popsaný algoritmus na~kontrolu bezpečnosti vrcholu.
% @formatter:off
\begin{code}
// konstanty tabulka obsazených pozic t, minimální vzdálenost agentů d,
// maximální poloměr agenta m

// vstup krok s, vrchol v, poloměr agenta r
// výstup true pokud může agent být na v, jinak false
safe_vertex(s, v, r)
for u in sorted(V, x -> dist(x, v))
if dist(u, v) > r + m + d return true
else
n <- t[s][v]
r' <- diameter(n)
if dist(u, v) <= r + r' + d return false
return true
\end{code}
% @formatter:on

\subsection{Cesta do~vrcholu}\label{subsec:cesta_do_vrcholu}

%Popis postupu kontroly, pseudokód, náčrtek.

\hyperref[subsec:cesta_do_vrcholu]{Kontrola cesty do~vrcholu} zjišťuje,
zda~může plánovaný agent bezpečně přejet do~určeného vrcholu v~určitém kroku,
aniž by došlo ke~kolizi s~nějakým agentem opouštějícím daný vrchol.
Kontrola zahrnuje druhou \ref{str:mapf} podmínku
\uv{žádní dva agenti nesmí projíždět stejnou hranou v~jednom kroku} \eqref{eq:mapf_kolize_hrana}.
Avšak jelikož mají agenti nenulovou velikost, je nutné kontrolovat i~případy, kdy agenti neprojíždí stejnou hranou.
Tyto situace jsou častější čím menší úhel je mezi sousedy vrcholu.
Například pro~čtvercový typ, kde je úhel mezi sousedy $90^\circ$, nastává tato kolize pouze pro~velké agenty.
U~oktagonálního typu křižovatky dochází ke~kolizi mnohem častěji, protože úhel mezi sousedy činí $45^\circ$.
Ukázka kolizního stavu je zobrazena na~obrázku (Obrázek~\ref{fig:kolize_cesta_do}).

\begin{figure}[h]
	\centering
	\includegraphics[width=\textwidth]{../img/kolize_cesta_do}
	\caption{
		Ukázka kolizní situace, která je detekována \hyperref[subsec:cesta_do_vrcholu]{kontrolou cesty do~vrcholu}.
		Na~obrázcích jsou černě zobrazeny vrcholy.
		Obdélníky reprezentují agenty a kruhy tvoří bezpečnou zónu příslušných agentů.
		Čárkované prázdné obdélníky značí cílové stavy agentů po~jednom kroku.
		Zelený agent je již naplánovaný, černevý je aktuálně plánován.
	}
	\label{fig:kolize_cesta_do}
\end{figure}


Tato~kontrola probíhá, pokud agent~$a$ opouští vrchol~$u$ v~kroku~$s$ a přijíždí do~vrcholu~$v$ v~následujícím kroku~$s + 1$.
Algoritmus nejprve zjistí, zda existuje naplánovaný agent~$b$, který je v~kroku~$s$ na~$v$.
Pokud žádný takový agent neexistuje, kontrola uspěje a agetova cesta z~$u$ do~$v$ v~kroku $s$ je bezpečná.
Jinak se zjistí vrchol~$w$, na~kterém se nachází agent~$b$ v~následujícím kroku $s + 1$.
Pokud nastane speciální případ $u=w$, odpovídá stav zmiňované \ref{str:mapf} podmínce \eqref{eq:mapf_kolize_hrana}.

Jelikož kontrola zná pozice vrcholů $u$, $v$ a $w$, je schopna dopočítat si čas, ve~kterém se agenti nejvíce přiblíží.
Pokud se agenti srazí, musí být v~kolizní poloze i~v~čase, kdy~jsou sobě nejblíže.
Naopak pokud jsou dostatečně daleko i~když jsou si nejblíže, při~cestě nemůže dojít ke~kolizi.
Proto stačí zkontrolovat vzdálenost v~čase, kdy~jsou agenti nejblíž.

Označím polohu agenta~$a$ jako $[x_a, y_a]$ a polohu agenta~$b$ jako $[x_b, y_b]$.
Stejným způsobem si označím pozice vrcholů $u$, $v$ a $w$ jako $[x_u, y_u]$, $[x_v, y_v]$ resp. $[x_w, y_w]$.
Dále si označím čas mezi kroky~$s$ a $s + 1$ jako~$t\in[0, 1]$.
Pro~$t = 0$ je poloha agenta~$a$ shodná s~pozicí vrcholu~$u$ a poloha agenta~$b$ shodná s~pozicí vrcholu~$v$.
Analogicky v~čase $t = 1$ se nachází agent~$a$ na~vrcholu~$v$ a agent~$b$ na~vrcholu~$w$.

Jelikož se agenti pohybují po~úsečce mezi vrcholy,
pro~$t\in[0, 1]$ se agent $a$ nachází na~$x_a = tx_u + (1 - t)x_v$, $y_a = ty_u + (1 - t)y_v$.
Poloha agenta~$b$ je analogicky $x_b = tx_v + (1 - t)x_w$ a $y_b = ty_v + (1 - t)y_w$.
Vzdálenost agentů v~závislosti na~čase~$t$ je $\sqrt{(x_a - x_b)^2 + (y_a - y_b)^2}$.
Dále upravím vzorec pod~odmocninou.
\begin{align*}
	((x_a - x_b)^2 &+ (y_a - y_b)^2) = \\
	((tx_u + (1 - t)x_v - tx_v - (1 - t)x_w)^2 &+ (ty_u + (1 - t)y_v - ty_v - (1 - t)y_w)^2) = \\
	((tx_u + x_v - tx_v - tx_v - x_w + tx_w)^2 &+ (ty_u + y_v - ty_v - ty_v - y_w + ty_w)^2) = \\
	((t(x_u - x_v - x_v + x_w) + x_v - x_w)^2 &+ (t(y_u - y_v - y_v + y_w) + y_v - y_w)^2) = \\
	(t(x_u - x_v - x_v + x_w) + x_v - x_w)^2 &+ (t(y_u - y_v - y_v + y_w) + y_v - y_w)^2 = \\
	(t(x_u - 2x_v + x_w) + x_v - x_w)^2 &+ (t(y_u - 2y_v + y_w) + y_v - y_w)^2 \\
\end{align*}

Pro~zjednodušení si označím
\begin{align*}
	x_0 &= x_u - 2x_v + x_w &\qquad
	x_1 &= x_w - x_v \\
	y_0 &= y_u - 2y_v + y_w &\qquad
	y_1 &= y_w - y_v
\end{align*}

Dosazením do~předchozího vzorce dostávám
\begin{align*}
	((x_a - x_b)^2 &+ (y_a - y_b)^2) = \\
	(t(x_u - 2x_v + x_w) + x_v - x_w)^2 &+ (t(y_u - 2y_v + y_w) + y_v - y_w)^2 = \\
	(tx_0 - x_1)^2 &+ (ty_0 - y_1)^2
\end{align*}

Pro~nalezení nejmenší vzdálenosti zjistím čas~$t$, ve~kterém se agenti nacházejí nejblíže.
K~tomu spočítám derivaci vzdálenosti a zjistím, kdy je rovna nule.
Nejprve využiji faktu, že $\min\left(\sqrt{x}\right) = \min(x) \Rightarrow \frac{d}{dx} \sqrt {x} = 0 \leftrightarrow \frac{d}{dx} x=0$.
Následně
\begin{align}
	\frac{d}{dt} \sqrt{(x_a - x_b)^2 + (y_a - y_b)^2} &= 0 \nonumber \\
	\frac{d}{dt} ((x_a - x_b)^2 + (y_a - y_b)^2) &= 0 \nonumber \\
	\frac{d}{dt} ((tx_0 - x_1)^2 + (ty_0 - y_1)^2) &= 0 \nonumber \\
	\frac{d}{dt} (tx_0 - x_1)^2 + \frac{d}{dt} (ty_0 - y_1)^2 &= 0 \nonumber \\
	2(tx_0 - x_1)x_0 + 2(ty_0 - y_1)y_0 &= 0 \nonumber \\
	tx_0^2 - x_0 x_1 + ty_0 - y_0 y_1 &= 0 \nonumber \\
	t(x_0^2 + y_0^2) &= x_0 x_1 + y_0 + y_1 \label{eq:kol_d_dt}
\end{align}

Rozeberu dva případy podle podmínky
\begin{gather}
	x_0^2 + y_0^2 = 0\label{kol:nulova_podminka}
\end{gather}

Pokud je splněna podmínka~\ref{kol:nulova_podminka}, platí $x_0^2 = 0$ a $y_0^2 = 0$.
Odtud
\begin{align*}
	x_u - 2 x_v + x_w &= 0 \\
	x_u + x_w &= 2 x_v \\
	\frac{x_u + x_w}{2} &= x_v
\end{align*}
Předchozí rovnosti jsou splněné pouze když se~$x_v$ nachází přesně uprostřed $x_u$ a $x_w$.
Analogicky $y_0^2 = 0 \Leftrightarrow \frac{y_u + y_w}{2} = y_v$, tedy $y_v$ je přesně uprostřed $y_u$ a $y_w$.
Obě tyto podmínky jsou splněny jenom když se vrchol~$v$ nachází uprostřed úsečky z~$u$ do~$w$.

V~tom případě se agenti nepovoleně přiblíží pokud vzdálenost vrcholů $v$ a $w$ je menší než
součet poloměru agentů $d_a$ a $d_b$ a dovolené vzdálenosti mezi~agenty \hyperref[par:safe_distance]{safe distance}~$d$.
Dostávám tedy nerovnost $(x_w - x_v)^2 + (y_w - y_v)^2 = x_1^2 + y_1^2 > (d_a + d_b + d)^2$.


Pokud podmínka~\ref{kol:nulova_podminka} neplatí, je možné spočítat čas~$t'$,
kdy~jsou agenti nejblíže vyjádřením z~\ref{eq:kol_d_dt}.
\begin{equation}
	\label{eq:kol_t}
	t' = \frac{x_0 x_1 + y_0 y_1}{x_0^2 + y_0^2}
\end{equation}

Po~dopočítání času spočítám vzdálenost agentů $a$ a $b$ v~čase~$t'$.
Rozdíl $x$-ové souřadnice agentů je roven
\begin{gather*}
	t' x_u + (1 - t')x_v - (t' x_v + (1 - t')x_w) =
	t' x_u + x_v - t' x_v - t' x_v - x_w + t' x_w = \\
	t'(x_u - 2x_v + x_w) + x_v - x_w =
	t' x_0 - x_1
\end{gather*}

Obdobně rozdíl $y$-ové souřadnice činí $t' y_0 - y_1$.
Pro~tento případ kontrola projde jenom pokud $(t' x_0 - x_1)^2 + (t' y_0 - y_1)^2 > (r_a + r_b + d)^2$.

Pro~výsledný algoritmus si nejdříve nadefinuji
pomocnou funkci \textrm{safe\_neighbour}\labeltext{\textrm{safe\_neighbour}}{str:safe_neighbour}.
Tato~funkce pomocí postupu výše zkontroluje, zda-li dojde ke~kolizi mezi agenty $a$ a $b$,
pokud už~známe vrcholy $u$, $v$ a $w$.
% @formatter:off
\begin{code}
// konstanty bezpečná vzdálenost d

// agent a s poloměrem da cestuje z u do v,
// agent b s poloměrem db cestuje z v do w
safe_neighbour(v, u, w, da, db)
if u is w return false

x0 <- u.x - 2*v.x + w.x
x1 <- w.x - v.x
y0 <- y.u - 2*v.y + w.y
y1 <- w.y - v.y

// vzdálenost mezi středy agentů na druhou
dist <- (da + db + d) ** 2

if x0 is 0 and y0 is 0
return x1 ** 2 + y1 ** 2 > dist

else
t' <- (x0 * x1 + y0 * y1 ) / (x0 ** 2 + y0 ** 2)
x_diff <- (t' * x0 - x1) ** 2
y_diff <- (t' * y0 - y1) ** 2
return x_diff + y_diff > dist
\end{code}
\label{alg:check_neighbour}
% @formatter:on

Výsledný algoritmus kolize vypadá následovně.
% @formatter:off
\begin{code}
// konstanty tabulka obsazených pozic t

// agent a s poloměrem da cestuje z vrcholu u do v v kroku s
safe_step_to(s, u, v, da)
b <- t[s][v]
if b is Null return true

w <- b.path[s + 1]
db <- diameter(b)
return safe_neighbour(v, u, w, da, db)
\end{code}
% @formatter:on

\subsection{Cesta z~vrcholu}\label{subsec:cesta_z_vrcholu}

%Popis postupu kontroly, pseudokód.


Poslední kontrola ověřuje opačný případ předchozí kontroly.
Zjišťuje se, zda plánovaný agent~$a$ může bezpečně odjet z~vrcholu~$v$
aniž by~se srazil s~jiným již naplánovaným agentem~$b$ cestujícím do~$v$.
Ukázka kolizního stavu je zobrazena na~obrázku (Obrázek~\ref{fig:kolize_cesta_z}).

\begin{figure}[h]
	\centering
	\includegraphics[width=\textwidth]{../img/kolize_cesta_z}
	\caption{
		Ukázka kolizní situace, která je detekována \hyperref[subsec:cesta_z_vrcholu]{kontrolou cesty z~vrcholu}.
		Na~obrázcích jsou černě zobrazeny vrcholy.
		Obdélníky reprezentují agenty a kruhy tvoří bezpečnou zónu příslušných agentů.
		Čárkované prázdné obdélníky značí cílové stavy agentů po~jednom kroku.
		Zelený agent je již naplánovaný, černevý je aktuálně plánován.
	}
	\label{fig:kolize_cesta_z}
\end{figure}

Formálně agent~$a$ cestuje z~$v$ do~$w$ v~kroku~$s$ a agent~$b$ cestuje z~$u$ do~$v$ opět v~kroku~$s$.
Pokud se na~situaci podívám z~pohledu druhého agenta (prohodím agenta $a$ za $b$), dostanu předchozí případ.
Z~tohoto důvodu můžu pro~kontrolu opět použít funkci \ref{str:safe_neighbour}, akorát prohodím parametry.
Výsledný algoritmus je následovný.
% @formatter:off
\begin{code}
// konstanty tabulka obsazených pozic t

// agent a s poloměrem da cestuje z vrcholu v do w v kroku s
safe_step_from(s, v, w, da)
b <- t[s + 1][v]
if b is Null return true

u <- b.path[s]
db <- diameter(b)
return safe_neighbour(v, u, w, db, da)
\end{code}
% @formatter:on


\section{Safe lanes}\label{sec:safe_lanes}

%Převedení řešení \citet{Dresner} na graf.

%Parametry a pseudokód.

Algoritmus~\nameref{sec:safe_lanes} je založen na~křižovatkách s předem definovanými pruhy pro~auta.
Tímto způsobem řešení jsem se inspiroval u~práce \citet{Dresner}.
V~jejich práci používali jednu křižovatku s~danými pruhy.
Agentům dovolovali pouze měnit rychlost.
Já použiji jejich koncept jízdy v~pruzích, avšak moji agenti rychlost měnit nemůžou.

\citet{Dresner} plánování spadá pod \nameref{subsec:individualni_planovani}.
Plánují tedy agenty postupně jednoho po~druhém.
\nameref{sec:safe_lanes} algoritmus používá stejný přístup.
Prochází všechny přijíždějící agenty v~neurčitém pořadí a zkusí každému agentovi přiřadit nekolizní cestu.

Algoritmus~\nameref{sec:safe_lanes} se podívá na~pruh, popřípadě pruhy, podle vjezdu a výjezdu či~výjezdů daných agentem.
Pořadí procházení pruhů je dáno jeho délkou.
Délky jednotlivých pruhů si může algoritmus předem spočítat.
Pro~každý vrchol na~cestě dané pruhem provede algoritmus kontrolu popsanou v~předchozí kapitole~\ref{sec:kolize}.
Agentovi je přiřazena první nalezená nekolizní cesta.
Pokud taková cesta neexistuje, vjezd agenta je zamítnut.

Následující kód ukazuje plánování jednoho agenta.
% @formatter:off
\begin{code}
// konstanty tabulka obsazených pozic t, množina pruhů p

plan_agent(step, agent)
  r <- agent.diameter
  for exit in sorted(agent.exits, x -> dist(entry, x))
    path <- p[agent.entry, exit]
    last <- path[0]
    for i in 1, ..., path.length - 1
      s <- step + i - 1
      vertex = path[i]
      safe_transfer <- safe_transfer_set(s, last, vertex, r, t)
      if not safe_transfer
        continue
    agent.path <- path
    add_planned_agent(t, agent, s)  // přidám agenta do t
    return agent
  return NULL
\end{code}
% @formatter:on


\section{A*}\label{sec:a_star}

%Přesný popis A* algoritmu.

\nameref{sec:a_star} je známý prohledávací algoritmus.
Algoritmus potřebuje znát prohledávací prostor určený možnými stavy.
Dále je nutné uvést určitou heuristiku, která pro určitý vrchol vrátí
spodní odhad na cenu zbylé cesty z aktuálního stavu do cílového stavu.
Algoritmus si zároveň pro každý navštívený stav pamatuje cenu cesty z počátečního stavu do aktuálního.
\nameref{sec:a_star} postupně prochází navštívené stavy a pro každý stav přidá sousední stavy.
Pořadí procházených stavů je určené součtem ceny cesty do aktuálního stavu z počátečního a heuristiky v aktuálním stavu.
Tento součet je spodní odhad na minimální cenu cesty z počátku do cíle vedoucí přes navštívené stavy,
přes které byl aktuální vrchol dosažen.
\nameref{sec:a_star} zaručuje při tomto postupu optimalitu nalezené cesty.

V následujících kapitolách popíšu dvě různé implementace \nameref{sec:a_star} algoritmu pro řešení problému křižovatky.

\subsection{Individuální A* (A*RS)}\label{subsec:individualni_a_star}

%Popis úpravy A* algoritmu pro řešený problém, parametry a pseudokód.

\labeltext{A*RS}{str:individualni_a_star} patří do kategorie \ref{str:rs} algoritmů.
Plánuje totiž stejně jako \nameref{sec:safe_lanes} jednoho agenta po druhém.
Akorát algoritmus dovoluje agentům \uv{opustit} svoje pruhy.

Cena cesty se počítá podobně jako při hledání \hyperref[par:pruh]{pruhu} v křižovatce.
Cena mí více kritérií, a to \hyperref[par:ars_vzdalenost]{vzdálenost},
\hyperref[par:ars_uhel_zataceni]{úhel zatáčení} a \hyperref[par:ars_pocet_zataceni]{počet zatáčení}.

\paragraph{Vzdálenost}\label{par:ars_vzdalenost} je počet hran grafu, přes které cesta vede.
Duplicitní hrany se započítávají vícekrát.

\paragraph{Úhel zatáčení}\label{par:ars_uhel_zataceni} určuje úhel, o který se musí agent za cesty otočit.
Pro každý prostřední vrchol na cestě se dopočítá úhel mezi hranami,
přes kterou se agent na vrchol dostal a kterou odjel.
\nameref{par:ars_uhel_zataceni} je součet absolutních hodnot těchto úhlů.

\paragraph{Počet zatáčení}\label{par:ars_pocet_zataceni} udává počet
nenulových \hyperref[par:ars_uhel_zataceni]{úhlů zatáčení}.

Cesty jsou nejprve porovnávány podle \hyperref[par:ars_vzdalenost]{vzdálenosti},
poté \hyperref[par:ars_uhel_zataceni]{úhlu zatáčení} a nakonec podle \hyperref[par:ars_pocet_zataceni]{počtu zatáček}.

\paragraph{Heuristika}\label{par:ars_heuristika} v tomto případě je minimální délka cesty
z aktuálního vrcholu do nejbližšího z cílových vrcholů.
Pokud žádná taková cesta neexistuje, je hodnota \hyperref[par:ars_heuristika]{heuristiky} $\infty$.

Prohledávací prostor jsou rozšířené vrcholy křižovatky.
Pro jednodušší výpočet si u stavu mimo vrcholu pamatuji též krok, ve kterém by agent na daný vrchol přijel.
Dále si ukládám předchozí stav, cenu cesty z počátku a odhad ceny zbylé cesty dané heuristikou.
Datová struktura stavu vypadá následovně:
% @formatter:off
\begin{code}[frame=none]
stav {
	vrchol
	krok
	rodic       // předchozí stav
	vzdalenost  // vzdálenost z počátečního stavu
	uhel        // úhel zatáčení na cestě z počátečního stavu
	zatacky     // počet zatáčení na cestě z počátečního stavu
	heuristika  // hodnota heuristiky v aktuálním stavu
}
\end{code}
% @formatter:on

Následující stavy daného stavu jsou všechny validní stavy dané sousedy vrcholu aktuálního stavu.
Formálně pro vrchol $u$ jsou jeho sousedi vrcholy ${v \in V | (u,v)\in E}$, 
kde $V$ je množina vrcholů a $E$ množina hran.

\subsection{Parametry}\label{subsec:parametry}
Pro reálnější pohyby agentů po křižovatce je vhodné omezit množinu sousedů vrcholu.
Avšak určení vhodného omezení je komplikované.
Proto jsem se rozhodl umožnit omezení měnit následujícími parametry.

MAXIMUM_VERTEX_VISITS_DEF = 2;

ALLOW_AGENT_STOP_DEF = false;

MAXIMUM_PATH_DELAY_DEF = Integer.MAX_VALUE;

ALLOW_AGENT_RETURN_DEF = false;


\subsection{Hromadný A*}\label{subsec:hromadny_a_star}

Rozšíření A* pro více agentů, popis vylepšení.
Parametry, pseudokód.

\subsection{Conflict-Based Search (\ref{str:cbs})}\label{subsec:conflict_based_search}\labeltext{CBS}{str:cbs}

%Popis algoritmu, úprava pro můj problém.
%Parametry, pseudokód.

\nameref{subsec:conflict_based_search} algoritmus \citep*{Sharon} rozšiřuje jakýkoliv \ref{str:rs} algoritmus
na multiagentní plánování.
V mém případě budu rozšiřovat \ref{str:a_star_ars}.

\ref{str:cbs} začíná individuálním naplánováním všech agentů nezávisle na~sobě.
Čili agenti nesmějí mít kolize s~již cestujícími agenty,
avšak mohou mí kolizní trajektorii s~jinými aktuálně plánovanými agenty.
Poté se zkontroluje, zda-li nemají nějací agenti kolizní trajektorie.
Pokud ne, plánování úspěšně končí.
Jinak se prohledávání rozdělí na dva případy.
V obou případech je přeplánován jeden agent s podmínkou, že se musí vyhnout koliznímu místu.
Poté se opakuje opětovné hledání kolizí a rozdělování na případy.
Aby nedošlo k zacyklení, je nutné při plánování agenta vyhnout se nejen aktuální kolizi, ale také všem předchozím.
Výpočet postupně vytváří strom,
kde každý vrchol obsahuje cesty agentů (mohou být navzájem kolizní) a tabulku zakázaných pozic.
Algoritmus skončí v~prvním nalezeném vrcholu neobsahujícím kolizní trasy.

Mohlo by se stát, že plánování jednoho agenta selže.
V~tom případě je agent zcela odstraněn z~vrcholu.
Následně jsou nalezeni agenti, kteří byli v~historii přeplánováni kvůli odstraněnému agentovi.
Pro~tyto agenty jsou nalezeny nové cesty, jelikož pro~ně může existovat lepší cesta.

Algoritmus postupně prochází listy stromu výpočtu.
Pořadí průchodu je určeno počtem agentů.
Pokud je počet agentů u~více listů shodný, vybere se vrchol s nejmenší vzdáleností podobně jako u \ref{str:a_star_arsg}.
Algoritmus naplánuje pouze agenty, kteří mají cesty ve~vybraném listu, vjezd zbylých agentů je zamítnut.

\ref{str:cbs} najde optimální cestu pro všechny agenty \citep{Sharon}.
Avšak velikost stromu může být obrovská.
Proto jsem se rozhodl obětovat optimalitu s zjednodušit práci algoritmu.
Ve zjednodušeným režimu algoritmus přeplánuje takovým způsobem, aby neměl žádné kolize s ostatními plánovanými agenty.

\subsubsection{Parametry}\label{subsubsec:cbs_parametry}

\nameref{subsubsec:cbs_parametry} algoritmu jsou stejné jako u \ref{str:varsg} a mají podobný význam.
Hodnoty Maximum návštěv vrcholu (\ref{par:ars_mnv}), Povolené zastavování (\ref{par:ars_pz}),
Maximální prodleva při~cestě (\ref{par:ars_mpc}) a Povolené vracení (\ref{par:ars_pv})
algoritmus používá při~plánování jednoho agenta.
Tyto \hyperref[subsubsec:ars_parametry]{parametry} ovlivňují plánování stejně jako u \ref{str:a_star_ars}.
Hodnota parametru \ref{par:arsg_zvp} opět určuje po jak dlouhé prodlevě má algoritmus přejít na zjednodušené plánování.

\subsection{CBS-OID}\label{subsec:cbsoid}

\ref{str:cbs} lze podobně jako \ref{str:varsg} rozšířit na \ref{str:oid} variantu.
K plánovaným agentů se přidají agenti z předchozích kroků.
Jako počáteční cesty těchto agentů se použijí jejich již naplánované trasy, tudíž se znova nepočítají.
Výpočet je poté shodný, až na~případy, kdy pro~ně nebyla nalezena cesta.
Pokud k~takové situaci dojde, namísto odstranění agenta se odstraní celý list ze~stromu výpočtu.

Parametry jsou rozšířené stejně jako u \nameref{subsubsec:a_star_aoid} o~Maximální počet agentů (\ref{par:aoid_mpa})
a Počet přeplánovaných kroků (\ref{par:aoid_ppk}).
Význam těchto parametrů je shodný.



\section{SAT planner}\label{sec:sat-planner}

%Definice SAT a MAXSAT, popis řešiče.
%Rozdíl mezi optimálním ohodnocením a splňujícím ohodnocením.
%
%Popis převodu problému na SAT.
%Popis parametrů a odhad na počet proměnných a počet klauzulí.
%
%Pseudokód.

SAT je známý a prozkoumaný problém, na který existují vysoce optimalizované řešiče.
Proto není úplně zcestné pokusit se problém křižovatky převést na SAT problém.

SAT\labeltext{SAT}{str:sat} je problém určení, zda-li existuje splňující ohodnocení výrokových proměnných logické formule.
Vstupní hodnotou je tedy výroková formule a výstupem ohodnocení proměnných takové, že daná formule je splněná.
Zadaná formule většinou bývá v konjunktivní normální formě (\ref{str:sat_cnf})\labeltext{CNF}{str:sat_cnf}, což je konjunkce klauzulí.
Klauzule jsou disjunkce literálů a literál je výroková proměnná, nebo její negace.
Například formule v \ref{str:sat_cnf} pro proměnné $p_1, \dots, p_{10}$ může být
\[
	\bigwedge_{i=1}^{7}(p_i \vee p_{i+1} \vee p_{i + 3}).
\]

MAXSAT je rozšíření \ref{str:sat}.
Klauzule jsou rozdělené na dvě skupiny,
\ref{str:sat_hard}\labeltext{\emph{hard}}{str:sat_hard} a \ref{str:sat_soft}\labeltext{\emph{soft}}{str:sat_soft}.
Aby bylo ohodnocení splňující, musí být splněny všechny \ref{str:sat_hard} klauzule.
Úkolem řešiče je nalézt splňující ohodnocení, které maximalizuje počet splněných \ref{str:sat_soft} klauzulí.
Tento problém je očividně těžší, jelikož nestačí najít libovolné řešení, ale to nejlepší.

Vážený MAXSAT přidává navíc možnost přiřadit \ref{str:sat_soft} klauzulím váhy.
Řešič se nesnaží maximalizovat počet splněných klauzulí, ale součet jejich vah.

Pokud chceme naplánovat agenty pomocí váženého MAXSAT, stačí převést plánování do \ref{str:sat_cnf}.
Převedení \ref{str:mapf} problému na \ref{str:sat_cnf} bylo mnohokrát popsáno \citep{bartak}.
Z tohoto postupu budu vycházet.

\subsection{Převod do \ref{str:sat_cnf}}\label{subsec:sat_prevod_do_cnf}

Vhodný začátek převodu je nadefinování výrokových proměnných.
Poté popíšu tvorbu klauzulí.

\subsubsection{Výrokové proměnné}\label{subsubsec:sat_vyrokove_promenne}

Vytvořím si výrokové proměnné pro každého agenta, pro každý krok a pro každý vrchol grafu.
Pokud se agent $a$ vyskytuje v čase $t$ na vrcholu $v$, je výroková proměnná $p_{t,a,v}$ pravdivá, jinak je nepravdivá.
Avšak abych nedostal nekonečnou \ref{str:sat_cnf}, určím si maximální dobu cesty \ref{par:sat_mpk}.
Čas v proměnné je počet kroků od plánovaného kroku a mí hodnotu ${0, \dots, mpk}$
Počet výrokových proměnných je celkem $(mpk + 1) * |A| * |V|$, kde $A$ je množina agentů a $V$ množina vrcholů.

\subsubsection{\ref{str:sat_hard} Podmínky}\label{subsubsec:sat_hard_podminky}

V této kapitole popíšu způsob, jak vytvořit odpovídající \ref{str:sat_cnf}.
Avšak nebudu popisovat jednotlivé klauzule.
Namísto toho popíšu tvorbu klauzulí jednoduššími výrazy (např.\ pro každé $p_i$, maximálně jeden z $p_i$, \ldots).
Převod těchto výrazů do validní \ref{str:sat_cnf} je triviální.
Pokud formule obsahuje některou funkci, je možné funkci vyhodnotit předem a daný výraz přidat pokud to má smysl.

Agent přijede na vrchol \hyperref[par:vjezdy]{vjezdu} v kroku příjezdu, pokud bude úspěšně naplánován.
Zároveň musí být v jednom z časů v íli, aby byla cesta kompletní.
Z toho vyplývá první podmínka pro každého agenta, která značí, že daný agent není v počáteční čas na vjezdu,
nebo je v jenom jeden čas na právě jednom výjezdu.
Matematicky:
\[
	(\forall_{a \in A}) \left(\neg p_{0,a,a_e} + \sum_{t=1}^{mpk} \sum_{f \in a_f} p_{t, a, f} = 1\right),
\]
kde $a_e$ je vrchol vjezdu agenta $a$ a $a_f$ jeho výjezdy.

Agent nemůže nacházet na více vrcholech najednou.
Jinými slovy může být pro jednoho agenta a jeden čas maximálně jedna proměnná pravdivá:
\[
	(\forall a \in A)(\forall t \in {0, \dots, mpk})\left(\sum_{v=1}^{|V|} p_{t,a,v} \leq 1\right).
\]

Pokud je agent v určitý krok na vrcholu $v$, musí být v dalším kroku na některým vrcholu z jeho sousedů $N(v)$.
Množina sousedů může obsahovat i samotný vrchol $v$,
pokud \hyperref[par:sat_povolene_zastavovani]{povolíme zastavování}.
Toto platí až na vrcholy výjezdu, které opět pro agenta $a$ označím $a_f$, a také to neplatí pro poslední krok.
Matematickým zápisem tomu odpovídá podmínka
\[
	(\forall a \in A)(\forall t \in {0, \dots, mpk - 1})
	(\forall v \in V\\a_f)(p_{t,a,v} \rightarrow \vee_{n \in N(v)} (p_{t+1,a,n})).
\]

Počítání lze zrychlit zakázáním neplatných kombinací času a vrcholu.
Pro tyto účely si označím $d(u, v)$ jako délku nejkratší cesty mezi vrcholy $u$ a $v$.
Pokud mezi nimi cesta neexistuje, je hodnota $\infty$.
Agent nemůže být v čase $t$ na vrcholu vzdáleném více než $t$, jelikož se tam nemá jak dostat.
Stejně tak nemůže být v čase $t$ na vrcholu, který má vzdálenost k nejbližšímu cíli větší než $t$,
protože potom neexistuje způsob, jak se dostat do cíle včas.
Odtud plynou podmínka
\begin{gather*}
(\forall a \in A)(\forall t \in {0, \dots, mpk})(\forall v \in V)
	\\
	((d(a_e, v) > t \vee (\min_{f \in a_f} d(v, f)) > mpk - t) \rightarrow \neg p_{t, a, v}).
\end{gather*}

Nadále je nutné vyhnout se cestujícím agentům.
K tomu opět využiji funkce na kontrolu kolizí.
Projdu všechny vrcholy a pro každého agenta zjistím,
na kterých vrcholech se nesmí nacházet pomocí funkce \ref{alg:kol_safe_vertex}:
\[
	(\forall a \in A)(\forall t \in {0, \dots, mpk})(\forall v \in V)
	(\neg \ref{alg:kol_safe_vertex}(a_p + t, v, a_d) \rightarrow \neg p_{t, a, v}),
\]
kde $a_p$ je kro příjezdu agenta a $a_d$ je jeho \hyperref[par:polomer_agenta]{poloměr}.

Kontrola bezpečné \hyperref[subsec:cesta_do_vrcholu]{cesty do vrcholu}
a poté \hyperref[subsec:cesta_z_vrcholu]{z vrcholu} probíhá podobně.
Jediný rozdíl je, že se musí kontrolovat dvojice sousedních vrcholů.
\begin{gather*}
(\forall a \in A)(\forall t \in {0, \dots, mpk - 1})(\forall v \in V)
	(\forall n \in N(v)) \\
	(\neg(\ref{alg:kol_safe_step_to}(a_p + t, v, n, da) \wedge \ref{alg:kol_safe_step_from}(a_p + t, v, n, da))
	\rightarrow \neg p_{t, a, v}).
\end{gather*}

Poslední nutná podmínka je zamezení kolizím mezi plánovanými agenty.
Podmínky vypadají podobně předchozím klauzulím.
Je nutné projít všechny dvojice agentů a poté všechny kombinace vrcholů.
Pokud se vrcholy nacházejí moc blízko, nemůžou se agenti vyskytovat na patřičných vrcholech v jeden čas.
Zápisem:
\begin{gather*}
(\forall a, b \in {A \choose 2})(\forall v \in V)(\forall u \in \ref{alg:sat_close_vertices}(v, da, db))
	\\
	(\forall t \in {0, \dots, mpk})
	(\neg ((p_{t, a, u} \wedge p_{t, b, v}) \vee (p_{t, a, v} \wedge p_{t, b, u})))),
\end{gather*}
kde $da$ a $db$ jsou poloměry agentů $a$ resp. $b$.
Ve vzorci používám funkci \ref{alg:sat_close_vertices}, která pro daný vrchol $v$ a dvojici agentů
vrací všechny vrcholy, na kterých nesmí být některý agent, jestliže je druhý agent na $v$.
\labeltext{\textrm{close\_vertices}}{alg:sat_close_vertices}
% @formatter:off
\begin{code}[fontsize=\footnotesize]
// minimální vzdálenost agentů d

// vrchol, poloměr prvního agenta, poloměr druhého agenta
// výstup množina vrcholů nebezpečně blízká vstupnímu vrcholu
close_vertices(u, v, m, n)
	vertices <- empty
	for u in V
		if dist(u, v) <= m + n + d
		vertices.add(u)
	return vertices
\end{code}
% @formatter:on

Zároveň se agenti nesmějí srazit při cestách mezi vrcholu.
K tomu opět využiji funkci \ref{alg:kol_safe_neighbour} podobně jako při kontrole vjezdu do vrcholu.
Zkontroluji pro všechny $u$, $v$ a $w$ takové, že $v \in N(u) \wedge w \in N(v)$,
že agent $a$ může přejet mezi $u$ a $v$, a agent $b$ může přejet z $v$ do $w$.
Toto vyzkouším pro všechny dvojice agentů, včetně prohozeného pořadí agentů.
Po prohození agentů totiž podmínka odpovídá \hyperref[subsec:cesta_z_vrcholu]{kontrole cesty z vrcholu}.
Pokud není přejezd možný, nesmí se z žádných po sobě jdoucích krocích tato situace stát.
Zápisem:
\begin{gather*}
(\forall a \in A)(\forall b \neq a \in A)(\forall u \in V)
	(\forall v \in N(u))(\forall w \in N(v)) \\
	(\neg \ref{alg:kol_safe_neighbour}(v, u, w, da, db) \rightarrow
	(\forall t \in {0, \dots, mpk - 1}) \\
	(\neg (p_{t, a, u} \wedge p_{t + 1, a, v} \wedge p_{t, b, v} \wedge p_{t + 1, b, w}))
	)
\end{gather*}

\subsubsection{\ref{str:sat_soft} Podmínky}\label{subsubsec:sat_soft_podminky}

Podobně jako u všech předešlých algoritmech budu optimalizovat \ref{str:soc} metriku.
U každému agenta tedy budu chtít co nejdřívější příjezd do cíle.
Proto vytvořím jednoprvkové klauzule pro každého agenta a pro každý vrchol s cenou určenou časem.
Klauzuli v čase $t$ ($p_{t, a, v}$) přidělím váhu $mpk - t + 1$.
Tím bude mít příjezd v $t = 1$ váhu $mpk$ a v čase $t = mpk$ váhu $1$.

Abych maximalizoval počet naplánovaných agentů, vytvořím ještě pro každého agenta $a$
klauzuli $p_{0, a, a_e}$ s vahou alespoň $(mpk + 2) * (|A| - 1)$, kde $a_e$ je vrchol vjezdu agenta $a$.

\subsection{Parametry}\label{subsec:sat_parametry}

Aby byl algoritmus porovnatelný s ostatními algoritmy, přidal jsem podobné parametry použité v předešlých algoritmech.

\paragraph{Maximální počet kroků (\ref{par:sat_mpk})}\labeltext{MPK}{par:sat_mpk}
udává maximální délku plánu pro všechny agenty.

\paragraph{Maximum návštěv vrcholu (\ref{par:ars_mnv})} má stejný význam jako
parametr \ref{par:ars_mnv} u \hyperref[subsubsec:ars_parametry]{parametrů \ref{str:a_star_ars}}.
Hodnota udává maximální počet výskytů jednoho vrcholu na cestě.
Vzorcem $(\forall a \in A)(\forall v \in V)(\sum_{t=0}^{mpk} p_{t, a, v} \leq 1)$.

\paragraph{Povolené zastavování (\ref{par:ars_pz})}\label{par:sat_povolene_zastavovani} je taktéž vzatý
z \hyperref[subsubsec:ars_parametry]{parametrů \ref{str:a_star_ars}}.
Pokud je tento parametr nastaven, agent může stát na~místě.
Znamená to přidání vrcholu do množiny sousedů daného vrcholu.

\paragraph{Maximalizace} určuje, zda-li má řešič hledat libovolné splňující ohodnocení,
nebo maximalizovat váhu klauzulí.
Vypnutí optimalizace značně zrychluje výpočet, avšak může vést k mnohem horším výsledkům.
Jelikož jsem dovolil zamítnout agentovi vjezd, je možné všechny \ref{str:sat_hard} podmínky splnit
nastavením všech proměnných na $false$.

\subsection{SAT-RS}\label{subsec:sat_rs}

Nejjednodušší případ plánování je pomocí \ref{str:rs} strategie.
Algoritmus plánuje agenty sekvenčně jednoho za druhým.
Tudíž nemusí kontrolovat vzájemné kolize mezi plánovanými agenty.
Počet výrokových proměnných činí $|V| * (mpk + 1)$.

Po nalezení splňujícího ohodnocení se algoritmus podívá na proměnnou $p_{0, a, a_e}$.
Pokud je $false$, vjezd agenta je zamítnut.
Jinak se projdou všechny ostatní proměnné, a vyberou se pravdivé.
Z těch se podle $t$ sestaví cesta do prvního cíle.
Mohlo by se stát, že některé proměnné jsou nastaveny na $true$ i po čase příjezdu do cíle.
Tyto proměnné algoritmus ignoruje.

\subsection{SAT-RSG}\label{subsec:sat_rsg}
Jak název napovídá, algoritmus plánuje všechny nové agenty v daném kroku.
Počet proměnných vzroste na $|A| * |V| * (mpk + 1)$.
Jelikož počet proměnných vzroste lineárně, počet všech možných ohodnocení vzroste exponenciálně.
Po nalezení splňujícího ohodnocení se pro agenty poskládají cesty stejným postupem jako u \nameref{subsec:sat_rs}.

\subsection{SAT-RA}\label{subsec:sat_ra}

Plánování může proběhnout i pro již naplánované agenty pro nalezení lepších tras strategií \ref{str:ra}.
Avšak je nutné změnit určité podmínky.
Symbol $a_e$ u dříve naplánovaných agentů nese význam vrcholu, na kterém se v plánovaném kroku nachází agent.
Zároveň agent již vjel na křižovatku.
Proto je nutné zaručit, že bude naplánován.
To lze provést odstraněním $\neg p_{0, a, a_e}$ z první \ref{str:sat_hard} podmínky.
Dostanu tedy zjednodušenou podmínku:
\[
	(\forall_{a \in A}) \left(\sum_{t=1}^{mpk} \sum_{f \in a_f} p_{t, a, f} = 1\right).
\]

Zbytek algoritmu je stejný s \nameref{str:rsg}.

