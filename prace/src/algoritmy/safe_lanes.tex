\section{Safe lanes}\label{sec:safe_lanes}

%Převedení řešení \citet{Dresner} na graf.

%Parametry a pseudokód.

Algoritmus~\nameref{sec:safe_lanes} je založen na~křižovatkách s předem definovanými pruhy pro~auta.
Tímto způsobem řešení jsem se inspiroval u~práce \citet{Dresner}.
V~jejich práci používali jednu křižovatku s~danými pruhy.
Agentům dovolovali pouze měnit rychlost.
Já použiji jejich koncept jízdy v~pruzích, avšak mojí agenti rychlost měnit nemůžou.

\citet{Dresner} plánování spadá pod \nameref{subsec:individualni_planovani}.
Plánují tedy agenty postupně jednoho po~druhém.
\nameref{sec:safe_lanes} algoritmus používá stejný přístup.
Díky tomu může spoléhat na~výchozí implementaci obou funkcí $plan\_agents$.
Implementuje tedy jedinou funkci $plan\_agent$, která už~dostává individuální agenty namísto množiny agentů.

Algoritmus~\nameref{sec:safe_lanes} se podívá na~pruh, popřípadě pruhy, podle vjezdu a výjezdu či~výjezdů daných agentem.
Pro~každý vrchol na~cestě dané pruhem provede algoritmus kontrolu popsanou v~předchozí kapitole~\ref{sec:kolize}.
Agentovi je přiřazena první nalezená nekolizní cesta.
Pokud taková cesta neexistuje, vjezd agenta je zamítnut.

% @formatter:off
\begin{code}
// konstanty tabulka obsazených pozic t, množina pruhů p,
// minimální vzdálenost agentů d

plan_agent(step, agent_entries_exits)
  agent = agent_entries_exits[0]
  entry = agent_entries_exits[1]
  exits = agent_entries_exits[2]
  r <- diameter(agent)

  for exit in sorted(exits, x -> dist(entry, x))
    path <- p[agent.entry, exit]
    for i in 0, \ldots, path.length - 1
      vertex = path[i]
      safe_vertex_ = safe_vertex(step + i, vertex, r)
      if not safe_vertex_
        continue

      else if i > 0
        safe_to = safe_step_to(step + i, vertex, path[i - 1], r)
        if not safe_to
          continue

      else if i < path.length - 1
        safe_from = safe_step_from(step + i,vertex, path[i + 1], r)
        if not safe_from
          continue

    agent.path <- path
    return agent
  return NULL
\end{code}
% @formatter:on
