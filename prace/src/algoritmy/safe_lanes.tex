\section{Safe lanes}\label{sec:safe_lanes}

%Převedení řešení \citet{Dresner} na graf.

%Parametry a pseudokód.

Algoritmus~\nameref{sec:safe_lanes} je založen na~křižovatkách s předem definovanými pruhy pro~auta.
Tímto způsobem řešení jsem se inspiroval u~práce \citet{Dresner}.
V~jejich práci používali jednu křižovatku s~danými pruhy.
Agentům dovolovali pouze měnit rychlost.
Já použiji jejich koncept jízdy v~pruzích, avšak moji agenti rychlost měnit nemůžou.

\citet{Dresner} plánování spadá pod \nameref{subsec:individualni_planovani}.
Plánují tedy agenty postupně jednoho po~druhém.
\nameref{sec:safe_lanes} algoritmus používá stejný přístup.
Prochází všechny přijíždějící agenty v~neurčitém pořadí a zkusí každému agentovi přiřadit nekolizní cestu.

Algoritmus~\nameref{sec:safe_lanes} se podívá na~pruh, popřípadě pruhy, podle vjezdu a výjezdu či~výjezdů daných agentem.
Pořadí procházení pruhů je dáno jeho délkou.
Délky jednotlivých pruhů si může algoritmus předem spočítat.
Pro~každý vrchol na~cestě dané pruhem provede algoritmus kontrolu popsanou v~předchozí kapitole~\ref{sec:kolize}.
Agentovi je přiřazena první nalezená nekolizní cesta.
Pokud taková cesta neexistuje, vjezd agenta je zamítnut.

Následující kód ukazuje plánování jednoho agenta.
% @formatter:off
\begin{code}
// konstanty tabulka obsazených pozic t, množina pruhů p

plan_agent(step, agent)
  r <- agent.diameter
  for exit in sorted(agent.exits, x -> dist(entry, x))
    path <- p[agent.entry, exit]
    last <- path[0]
    for i in 1, ..., path.length - 1
      s <- step + i - 1
      vertex = path[i]
      safe_transfer <- safe_transfer_set(s, last, vertex, r, t)
      if not safe_transfer
        continue
    agent.path <- path
    add_planned_agent(t, agent, s)  // přidám agenta do t
    return agent
  return NULL
\end{code}
% @formatter:on
