\section{SAT planner}\label{sec:sat-planner}

%Definice SAT a MAXSAT, popis řešiče.
%Rozdíl mezi optimálním ohodnocením a splňujícím ohodnocením.
%
%Popis převodu problému na SAT.
%Popis parametrů a odhad na počet proměnných a počet klauzulí.
%
%Pseudokód.

SAT je známý a prozkoumaný problém, na který existují vysoce optimalizované řešiče.
Proto není úplně zcestné pokusit se problém křižovatky převést na SAT problém.

SAT\labeltext{SAT}{str:sat} je problém určení, zda-li existuje splňující ohodnocení výrokových proměnných logické formule.
Vstupní hodnotou je tedy výroková formule a výstupem ohodnocení proměnných takové, že daná formule je splněná.
Zadaná formule většinou bývá v konjunktivní normální formě (\ref{str:sat_cnf})\labeltext{CNF}{str:sat_cnf}, což je konjunkce klauzulí.
Klauzule jsou disjunkce literálů a literál je výroková proměnná, nebo její negace.
Například formule v \ref{str:sat_cnf} pro proměnné $p_1, \dots, p_{10}$ může být
\[
	\bigwedge_{i=1}^{7}(p_i \vee p_{i+1} \vee p_{i + 3}).
\]

MAXSAT je rozšíření \ref{str:sat}.
Klauzule jsou rozdělené na dvě skupiny,
\ref{str:sat_hard}\labeltext{\emph{hard}}{str:sat_hard} a \ref{str:sat_soft}\labeltext{\emph{soft}}{str:sat_soft}.
Aby bylo ohodnocení splňující, musí být splněny všechny \ref{str:sat_hard} klauzule.
Úkolem řešiče je nalézt splňující ohodnocení, které maximalizuje počet splněných \ref{str:sat_soft} klauzulí.
Tento problém je očividně těžší, jelikož nestačí najít libovolné řešení, ale to nejlepší.

Vážený MAXSAT přidává navíc možnost přiřadit \ref{str:sat_soft} klauzulím váhy.
Řešič se nesnaží maximalizovat počet splněných klauzulí, ale součet jejich vah.

Pokud chceme naplánovat agenty pomocí váženého MAXSAT, stačí převést plánování do \ref{str:sat_cnf}.
Převedení \ref{str:mapf} problému na \ref{str:sat_cnf} bylo mnohokrát popsáno \citep{bartak}.
Z tohoto postupu budu vycházet.

\subsection{Převod do \ref{str:sat_cnf}}\label{subsec:sat_prevod_do_cnf}

Vhodný začátek převodu je nadefinování výrokových proměnných.
Poté popíšu tvorbu klauzulí.

\subsubsection{Výrokové proměnné}\label{subsubsec:sat_vyrokove_promenne}

Vytvořím si výrokové proměnné pro každého agenta, pro každý krok a pro každý vrchol grafu.
Pokud se agent $a$ vyskytuje v čase $t$ na vrcholu $v$, je výroková proměnná $p_{t,a,v}$ pravdivá, jinak je nepravdivá.
Avšak abych nedostal nekonečnou \ref{str:sat_cnf}, určím si maximální dobu cesty \ref{par:sat_mpk}.
Čas v proměnné je počet kroků od plánovaného kroku a mí hodnotu ${0, \dots, mpk}$
Počet výrokových proměnných je celkem $(mpk + 1) * |A| * |V|$, kde $A$ je množina agentů a $V$ množina vrcholů.

\subsubsection{\ref{str:sat_hard} Podmínky}\label{subsubsec:sat_hard_podminky}

V této kapitole popíšu způsob, jak vytvořit odpovídající \ref{str:sat_cnf}.
Avšak nebudu popisovat jednotlivé klauzule.
Namísto toho popíšu tvorbu klauzulí jednoduššími výrazy (např.\ pro každé $p_i$, maximálně jeden z $p_i$, \ldots).
Převod těchto výrazů do validní \ref{str:sat_cnf} je triviální.
Pokud formule obsahuje některou funkci, je možné funkci vyhodnotit předem a daný výraz přidat pokud to má smysl.

Agent přijede na vrchol \hyperref[par:vjezdy]{vjezdu} v kroku příjezdu, pokud bude úspěšně naplánován.
Zároveň musí být v jednom z časů v íli, aby byla cesta kompletní.
Z toho vyplývá první podmínka pro každého agenta, která značí, že daný agent není v počáteční čas na vjezdu,
nebo je v jenom jeden čas na právě jednom výjezdu.
Matematicky:
\[
	(\forall_{a \in A}) \left(\neg p_{0,a,a_e} + \sum_{t=1}^{mpk} \sum_{f \in a_f} p_{t, a, f} = 1\right),
\]
kde $a_e$ je vrchol vjezdu agenta $a$ a $a_f$ jeho výjezdy.

Agent nemůže nacházet na více vrcholech najednou.
Jinými slovy může být pro jednoho agenta a jeden čas maximálně jedna proměnná pravdivá:
\[
	(\forall a \in A)(\forall t \in {0, \dots, mpk})\left(\sum_{v=1}^{|V|} p_{t,a,v} \leq 1\right).
\]

Pokud je agent v určitý krok na vrcholu $v$, musí být v dalším kroku na některým vrcholu z jeho sousedů $N(v)$.
Množina sousedů může obsahovat i samotný vrchol $v$,
pokud \hyperref[par:sat_povolene_zastavovani]{povolíme zastavování}.
Toto platí až na vrcholy výjezdu, které opět pro agenta $a$ označím $a_f$, a také to neplatí pro poslední krok.
Matematickým zápisem tomu odpovídá podmínka
\[
	(\forall a \in A)(\forall t \in {0, \dots, mpk - 1})
	(\forall v \in V\\a_f)(p_{t,a,v} \rightarrow \vee_{n \in N(v)} (p_{t+1,a,n})).
\]

Počítání lze zrychlit zakázáním neplatných kombinací času a vrcholu.
Pro tyto účely si označím $d(u, v)$ jako délku nejkratší cesty mezi vrcholy $u$ a $v$.
Pokud mezi nimi cesta neexistuje, je hodnota $\infty$.
Agent nemůže být v čase $t$ na vrcholu vzdáleném více než $t$, jelikož se tam nemá jak dostat.
Stejně tak nemůže být v čase $t$ na vrcholu, který má vzdálenost k nejbližšímu cíli větší než $t$,
protože potom neexistuje způsob, jak se dostat do cíle včas.
Odtud plynou podmínka
\begin{gather*}
(\forall a \in A)(\forall t \in {0, \dots, mpk})(\forall v \in V)
	\\
	((d(a_e, v) > t \vee (\min_{f \in a_f} d(v, f)) > mpk - t) \rightarrow \neg p_{t, a, v}).
\end{gather*}

Nadále je nutné vyhnout se cestujícím agentům.
K tomu opět využiji funkce na kontrolu kolizí.
Projdu všechny vrcholy a pro každého agenta zjistím,
na kterých vrcholech se nesmí nacházet pomocí funkce \ref{alg:kol_safe_vertex}:
\[
	(\forall a \in A)(\forall t \in {0, \dots, mpk})(\forall v \in V)
	(\neg \ref{alg:kol_safe_vertex}(a_p + t, v, a_d) \rightarrow \neg p_{t, a, v}),
\]
kde $a_p$ je kro příjezdu agenta a $a_d$ je jeho \hyperref[par:polomer_agenta]{poloměr}.

Kontrola bezpečné \hyperref[subsec:cesta_do_vrcholu]{cesty do vrcholu}
a poté \hyperref[subsec:cesta_z_vrcholu]{z vrcholu} probíhá podobně.
Jediný rozdíl je, že se musí kontrolovat dvojice sousedních vrcholů.
\begin{gather*}
(\forall a \in A)(\forall t \in {0, \dots, mpk - 1})(\forall v \in V)
	(\forall n \in N(v)) \\
	(\neg(\ref{alg:kol_safe_step_to}(a_p + t, v, n, da) \wedge \ref{alg:kol_safe_step_from}(a_p + t, v, n, da))
	\rightarrow \neg p_{t, a, v}).
\end{gather*}

Poslední nutná podmínka je zamezení kolizím mezi plánovanými agenty.
Podmínky vypadají podobně předchozím klauzulím.
Je nutné projít všechny dvojice agentů a poté všechny kombinace vrcholů.
Pokud se vrcholy nacházejí moc blízko, nemůžou se agenti vyskytovat na patřičných vrcholech v jeden čas.
Zápisem:
\begin{gather*}
(\forall a, b \in {A \choose 2})(\forall v \in V)(\forall u \in \ref{alg:sat_close_vertices}(v, da, db))
	\\
	(\forall t \in {0, \dots, mpk})
	(\neg ((p_{t, a, u} \wedge p_{t, b, v}) \vee (p_{t, a, v} \wedge p_{t, b, u})))),
\end{gather*}
kde $da$ a $db$ jsou poloměry agentů $a$ resp. $b$.
Ve vzorci používám funkci \ref{alg:sat_close_vertices}, která pro daný vrchol $v$ a dvojici agentů
vrací všechny vrcholy, na kterých nesmí být některý agent, jestliže je druhý agent na $v$.
\labeltext{\textrm{close\_vertices}}{alg:sat_close_vertices}
% @formatter:off
\begin{code}[fontsize=\footnotesize]
// minimální vzdálenost agentů d

// vrchol, poloměr prvního agenta, poloměr druhého agenta
// výstup množina vrcholů nebezpečně blízká vstupnímu vrcholu
close_vertices(u, v, m, n)
	vertices <- empty
	for u in V
		if dist(u, v) <= m + n + d
		vertices.add(u)
	return vertices
\end{code}
% @formatter:on

Zároveň se agenti nesmějí srazit při cestách mezi vrcholu.
K tomu opět využiji funkci \ref{alg:kol_safe_neighbour} podobně jako při kontrole vjezdu do vrcholu.
Zkontroluji pro všechny $u$, $v$ a $w$ takové, že $v \in N(u) \wedge w \in N(v)$,
že agent $a$ může přejet mezi $u$ a $v$, a agent $b$ může přejet z $v$ do $w$.
Toto vyzkouším pro všechny dvojice agentů, včetně prohozeného pořadí agentů.
Po prohození agentů totiž podmínka odpovídá \hyperref[subsec:cesta_z_vrcholu]{kontrole cesty z vrcholu}.
Pokud není přejezd možný, nesmí se z žádných po sobě jdoucích krocích tato situace stát.
Zápisem:
\begin{gather*}
(\forall a \in A)(\forall b \neq a \in A)(\forall u \in V)
	(\forall v \in N(u))(\forall w \in N(v)) \\
	(\neg \ref{alg:kol_safe_neighbour}(v, u, w, da, db) \rightarrow
	(\forall t \in {0, \dots, mpk - 1}) \\
	(\neg (p_{t, a, u} \wedge p_{t + 1, a, v} \wedge p_{t, b, v} \wedge p_{t + 1, b, w}))
	)
\end{gather*}

\subsubsection{\ref{str:sat_soft} Podmínky}\label{subsubsec:sat_soft_podminky}

Podobně jako u všech předešlých algoritmech budu optimalizovat \ref{str:soc} metriku.
U každému agenta tedy budu chtít co nejdřívější příjezd do cíle.
Proto vytvořím jednoprvkové klauzule pro každého agenta a pro každý vrchol s cenou určenou časem.
Klauzuli v čase $t$ ($p_{t, a, v}$) přidělím váhu $mpk - t + 1$.
Tím bude mít příjezd v $t = 1$ váhu $mpk$ a v čase $t = mpk$ váhu $1$.

Abych maximalizoval počet naplánovaných agentů, vytvořím ještě pro každého agenta $a$
klauzuli $p_{0, a, a_e}$ s vahou alespoň $(mpk + 2) * (|A| - 1)$, kde $a_e$ je vrchol vjezdu agenta $a$.

\subsection{Parametry}\label{subsec:sat_parametry}

Aby byl algoritmus porovnatelný s ostatními algoritmy, přidal jsem podobné parametry použité v předešlých algoritmech.

\paragraph{Maximální počet kroků (\ref{par:sat_mpk})}\labeltext{MPK}{par:sat_mpk}
udává maximální délku plánu pro všechny agenty.

\paragraph{Maximum návštěv vrcholu (\ref{par:ars_mnv})} má stejný význam jako
parametr \ref{par:ars_mnv} u \hyperref[subsubsec:ars_parametry]{parametrů \ref{str:a_star_ars}}.
Hodnota udává maximální počet výskytů jednoho vrcholu na cestě.
Vzorcem $(\forall a \in A)(\forall v \in V)(\sum_{t=0}^{mpk} p_{t, a, v} \leq 1)$.

\paragraph{Povolené zastavování (\ref{par:ars_pz})}\label{par:sat_povolene_zastavovani} je taktéž vzatý
z \hyperref[subsubsec:ars_parametry]{parametrů \ref{str:a_star_ars}}.
Pokud je tento parametr nastaven, agent může stát na~místě.
Znamená to přidání vrcholu do množiny sousedů daného vrcholu.

\paragraph{Maximalizace} určuje, zda-li má řešič hledat libovolné splňující ohodnocení,
nebo maximalizovat váhu klauzulí.
Vypnutí optimalizace značně zrychluje výpočet, avšak může vést k mnohem horším výsledkům.
Jelikož jsem dovolil zamítnout agentovi vjezd, je možné všechny \ref{str:sat_hard} podmínky splnit
nastavením všech proměnných na $false$.

\subsection{SAT-RS}\label{subsec:sat_rs}

Nejjednodušší případ plánování je pomocí \ref{str:rs} strategie.
Algoritmus plánuje agenty sekvenčně jednoho za druhým.
Tudíž nemusí kontrolovat vzájemné kolize mezi plánovanými agenty.
Počet výrokových proměnných činí $|V| * (mpk + 1)$.

Po nalezení splňujícího ohodnocení se algoritmus podívá na proměnnou $p_{0, a, a_e}$.
Pokud je $false$, vjezd agenta je zamítnut.
Jinak se projdou všechny ostatní proměnné, a vyberou se pravdivé.
Z těch se podle $t$ sestaví cesta do prvního cíle.
Mohlo by se stát, že některé proměnné jsou nastaveny na $true$ i po čase příjezdu do cíle.
Tyto proměnné algoritmus ignoruje.

\subsection{SAT-RSG}\label{subsec:sat_rsg}
Jak název napovídá, algoritmus plánuje všechny nové agenty v daném kroku.
Počet proměnných vzroste na $|A| * |V| * (mpk + 1)$.
Jelikož počet proměnných vzroste lineárně, počet všech možných ohodnocení vzroste exponenciálně.
Po nalezení splňujícího ohodnocení se pro agenty poskládají cesty stejným postupem jako u \nameref{subsec:sat_rs}.

\subsection{SAT-RA}\label{subsec:sat_ra}

Plánování může proběhnout i pro již naplánované agenty pro nalezení lepších tras strategií \ref{str:ra}.
Avšak je nutné změnit určité podmínky.
Symbol $a_e$ u dříve naplánovaných agentů nese význam vrcholu, na kterém se v plánovaném kroku nachází agent.
Zároveň agent již vjel na křižovatku.
Proto je nutné zaručit, že bude naplánován.
To lze provést odstraněním $\neg p_{0, a, a_e}$ z první \ref{str:sat_hard} podmínky.
Dostanu tedy zjednodušenou podmínku:
\[
	(\forall_{a \in A}) \left(\sum_{t=1}^{mpk} \sum_{f \in a_f} p_{t, a, f} = 1\right).
\]

Zbytek algoritmu je stejný s \nameref{str:rsg}.
