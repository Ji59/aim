\subsection{Vylepšený A*RSG}\label{subsec:vylepseny_hromadny_a_star}

\citep{Standley_2010} se zabýval zlepšením multiagentního \nameref{sec:a_star} algoritmu.
Nakonec ve své práci \citet*{Standley_2010} sepsal techniky, které ve většině případů výrazně zrychlily prohledávání.
Já se zaměřím pouze na \nameref{subsubsec:varsg_independence_detection},
jelikož dle mého názoru přináší největší zrychlení.

\subsubsection{Independence Detection}\label{subsubsec:varsg_independence_detection}

Tato metoda rozdělí plánované agenty do skupin.
Poté nalezne cesty pro jednotlivé skupiny zvlášť bez ohledu na ostatní skupiny.
Když jsou všechny skupiny naplánovány, hledají se kolize mezi jednotlivými skupinami.
Pokud nejsou žádné kolize nalezeny, všichni agenti mají už naplánovanou validní cestu.
Jinak existuje kolize mezi agenty dvou skupin.
Algoritmus tedy zkusí přeplánovat první skupinu v kolizi tak,
aby už žádný agent z této skupiny neměl kolizní trasu s žádným agentem první i druhé skupiny.
Pokud se nepodařilo takové trasy najít, algoritmus zkusí najít nové trasy pro agenty druhé skupiny.
Jestliže se nepodařilo úspěšně přeplánovat ani druhou skupinu, spojí se obě skupiny do jedné.
Potom se najdou trasy pro tuto novou skupinu.
Při všech přeplánování se vždy kontrolují kolize s agenty naplánovanými v předešlých krocích.

Při přeplánování by mohlo dojít k zacyklení.
Například mám $3$ skupiny $g_1, g_2, g_3$ a $g_1$ je v kolizi s $g_2$.
Povede se mi přeplánovat skupinu $g_1$,avšak nyní je $g_1$ v kolizi s $g_3$.
Po opětovném přeplánování $g_1$ může oět dojít ke kolizi s $g_2$.
Proto je nutné udržovat si historii skupin, se kterými byla aktuálně přeplánovaná skupina v kolizi.
Při hledání nových tras se agenti vyhýbají nejen druhé skupině, ale i všem předešlým skupinám.

\paragraph{Illegal moves table}\label{par:varsg_illegal_moves_table} je datová struktura,
kterou algoritmus používá při přeplánování k určení kolize s agenty jiné skupiny.
Struktura této tabulky je shodná se strukturou \hyperref[par:obsazene_pozice]{tabulkou obsazených pozic}.
Před každým přeplánováním skupiny si algoritmus uloží pro~každý krok množinu dvojic vrcholu a agenta.
Pro každého agenta ze skupiny z historie nebo kolizní skupiny je do tabulky přidána pro každý krok, kdy agent cestuje,
dvojice zmíněného agenta a vrcholu z jeho aktuální trasy, kde se vyskytuje ve zmíněný krok.
Kontrola sousedních vrcholů pro jednoho agenta probíhá podobně jako při kontrole u \ref{str:a_star_ars}.
Avšak kromě kontroly kolize s naplánovanými agenty pomocí \hyperref[par:obsazene_pozice]{tabulkou obsazených pozic}
se podobnými kontrolami zaručuje nekolizní přejezd s agenty jiných skupin pomocí
\hyperref[par:varsg_illegal_moves_table]{Illegal moves table}.

K rychlejšímu naplánování chceme minimalizovat počet přeplánování.
Proto by bylo vhodné již při plánování skupiny preferovat cesty,
které mají co nejméně kolizních tras s ostatními skupinami.
Přesněji je důležitý počet agentů jiných skupin, které kolidují s některým agentem aktuální skupiny.
Avšak pro zaručení optimality je nutné dodržovat stejné uspořádání jako při \nameref{str:a_star_arsg}.
Porovnání na počet kolizních agentů tedy budu používat jenom tehdy, mají-li stavy totožnou
\hyperref[par:ars_vzdalenost]{vzdálenost}, \hyperref[par:ars_uhel_zataceni]{úhlu zatáčení},
\hyperref[par:ars_pocet_zataceni]{počtu zatáčení} i \hyperref[par:ars_heuristika]{heuristiku}.
Pokud bude shodný i počet agentů s kolizní trajektorií, budu upřednostňovat stavy s menším počtem celkových kolizí.
Při tomto součtu už započítávám stejného agenta vícekrát.

\paragraph{Conflict avoidance table}\label{par:varsg_conflict_avoidance_table} je tabulka, pomocí níž
algoritmus zjišťuje množství těchto kolizí.
Strukturou je podobná \nameref{par:varsg_illegal_moves_table}.
Agenti jiných skupin mohou ale být v kolizi, a proto může být naplánováno více agentů na stejném vrcholu ve stejný krok.
Proto je výhodnější pamatovat si místo dvojice vrchol-agent dvojici vrchol-množina agentů.
Zjišťování kolizních agentů lze opět provést podobnými způsoby popsanými v sekci \nameref{sec:kolize}.
Rozdíl je jenom v používané tabulce a namísto kontrolování s jedním agentem
se musí zkontrolovat kolize se všemi agenty v dané množině.

Algoritmus potřebuje na začátku výpočtu rozdělit agenty do skupin.
Vhodný způsob rozdělení je přiřazení každému agentovi svojí skupinu,
protože složitost plánování roste exponenciálně s množstvím agentů ve skupině.
Všichni agenti jsou naplánováni zvlášť stejným stejně jako u \ref{str:a_star_ars}.
Pokud nějaký agent nemůže být naplánován, je jeho skupina vyřazena.

Může se stát, že algoritmus nebude moci najít trasy pro všechny agenty po spojení dvou skupin.
V tom případě je nutné vyřadit určité agenty ze skupiny.
Pro zachování co nejlepších výsledků jsem se rozhodl, že budu zkoušet všechny podmnožiny agentů dané skupiny,
dokud algoritmus nenalezne trasy pro danou podmnožinu.

\subsubsection{Parametry}\label{subsubsec:arsg_parametry}
Algoritmus má opět několik nastavitelných parametrů, kterými je možné ovlivnit chování a složitost výpočtu.
Jelikož je algoritmus rozšířením \nameref{str:a_star_ars}, ponechávám algoritmu všechny tyto parametry.
Omezení parametrem platí pro všechny plánované agenty ve skupině.

Počet podmnožin agentů u spojování skupin je exponenciální.
Při testování jsem zjistil, že algoritmu občas zabere hodně času najít vhodnou podmnožinu.
Proto jsem se rozhodl dovolit algoritmu použít\ zjednodušený výpočet.
Zjednodušený výpočet znamená, že algoritmus místo spojování skupin ponechá pouze skupinu větší velikosti.

\paragraph{Zjednodušený výpočet po (\ref{par:arsg_zvp})}\labeltext{ZVP}{par:arsg_zvp} je jediný nový parametr.
Tento parametr udává, po jak dlouhé prodlevě má algoritmus začít používat zjednodušené počítání.
