\chapter*{Závěr}

%Stručné shrnutí výsledků, problémy řešení, nápady na vylepšení.

Zjistil jsem, že vzhled křižovatky a její velikost ovlivňuje chování algoritmů.
Proto neexistuje jeden nejlepší algoritmus a ani jedno nejlepší nastavení algoritmů.

Pokud není křižovatka rozdělená na mnoho bloků, \ref{str:cbs} se choval velmi dobře.
\ref{str:a_star_ars} je dost spolehlivý a dává taktéž dobré výsledky.

Naopak varianty přeplánovávající agenty měly vysoké časy plánování, díky čemu často nedoběhly.
A pokud doběhly, patřily k těm horším algoritmům.

\ref{str:sat} algoritmy byly velmi pomalé.
Myslím si, že to je způsobeno mým rozhodnutím používat MAX-SAT\@.
Zároveň mohly být výsledky ovlivněny výběrem \ref{str:sat} řešiče.

\ref{str:varsg} a \ref{subsubsec:a_star_aoid} měly problémy s pamětí díky vysokému počtu možných kombinací
pozic agentů, pokud se jich plánuje více najednou.
Zjednodušený režim zaručoval alespoň nějaké řešení v rozumném čase, avšak podle mého způsoboval horší výsledky,
než měl \ref{str:a_star_ars}.

\ref{str:cbs} vyřešil problém s pamětí u \ref{str:varsg} algoritmu.
Avšak pro vyšší množství agentů se tvořily velké stromy, které značně zvyšovaly čas plánování.
Toto bylo obzvláště vidět u \nameref{subsec:cbsoid} algoritmu.
Podle mého názoru zde značně zapůsobil zjednodušený režim, jelikož pokud některý agent nešel naplánovat,
bez tohoto režimu musel algoritmus vyzkoušet všechny možné kombinace vrcholů a hran.
S použitím zjednodušení se větve výpočtu uzavíraly, a musely se nejvýše projít pouze aktuálně existující vrcholy.
Toto je podle mého názoru hlavní důvod, proč si tento algoritmus vedl tak dobře.

Řekl bych, že na menších křižovatkách se vyplatí agenty více omezovat, jelikož na těchto křižovatkách není tolik
místa pro manévrování agentů.
Zdá se mi, že se zde více vyplatí počkat s agentem před křižovatkou a poté projet kratší cestou,
než naplánovat dlouho cestu po křižovatce v aktuálním kroku.
Naopak pro větší křižovatky se obecně vyplatilo omezit agenty méně.
Myslím si, že to je kvůli většímu místu.
Zároveň menší omezení agentů dovoluje naplánování agenta, který by byl jinak zamítnut.
Díky tomu se pro tohoto agenta nemusí zdlouhavě procházet všechny možné cesty, což navyšuje časovou náročnost.

Umožnění agentům diagonální jízdy při přechodu ze čtvercové křižovatky na oktagonální na malé křižovatce značně uškodilo.
Avšak při velké velikosti křižovatky to naopak pomohlo.
Myslím si, že to má podobný důvod jako u parametrů.
Na menší křižovatce agenti v nových vrcholech spíše překáží ostatním, což prodlužuje cesty všech.
Naopak na velké křižovatce umožňují vyšší počet způsobů, jak se navzájem vyhnout.

\addcontentsline{toc}{chapter}{Závěr}
