\section{Simulace}\label{sec:simulace}

Simulace se spouští pomocí mnou vytvořeného programu.
V~každém kroku se pokusí simulace naplánovat agenty použitím vybraného algoritmu.
Pro~naplánování jsou vybráni agenti z~každého vjezdu, kteří přijeli nejdříve.
Nemůže se stát, že by agent předjel jiného agenta čekajícího před~ním.

Simulace nabízí různé statistiky pomocí nichž je možné jednotlivé algoritmy porovnat.
Tyto statistiky jsou popsány níže.

\begin{itemize}
	\item \emph{Zdržení} agentů. \emph{Zdržení} jednoho agenta se~spočte jako součet
	rozdílu délky cesty od~optimální cesty a doba čekání před vjezdem do~křižovatky.
	Jinými slovy značí \emph{zdržení} počet kroků o~které agent vyjel z~křižovatky později oproti situaci,
	kdyby přijel na~prázdnou křižovatku (křižovatku bez~jiných agentů) a ihned by~projel nejkratší možnou cestou.
	\emph{Zdržení} všech agentů je součet \emph{zdržení} přes~všechny agenty, co se~pokusili křižovatkou projet.
	\item Počet \emph{zamítnutých} agentů udává množství agentů, kteří čekali na~křižovatce příliš dlouho a
	vzdali~se čekání ve~frontě.
	I~když křižovatkou neprojeli, pořád se kroky jejich čekání přičítají do~celkového \emph{zdržení}.
	\item Počet \emph{kolizí} udává množství sražených agentů.
	\item \emph{Doba běhu} algoritmu udává kolik nanosekund běžel algoritmus v~součtu přes všechny kroky.
\end{itemize}
