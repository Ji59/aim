\section{Hromadné výsledky}\label{sec:hromadne_vysledky}

%Porovnání algoritmů mezi sebou s nejlepšími parametry.
%
%Porovnání čtvercové a oktagonální křižovatky.

Nyní budu porovnávat nejlepší nastavení algoritmů mezi sebou.
Porovnání by mělo být férové, jelikož všechny algoritmy běžely s podobným nastavením parametrů.
Nejlepší nastavení většiny parametrů je zobrazeno v prvních sloupcích tabulky hned za názvem algoritmu.

Zároveň je v tabulkách obsažen algoritmus \nameref{sec:safe_lanes}.
Ten byl doposud vynechán, jelikož nemá žádné nastavitelné parametry.

\subsection{Porovnání algoritmů na malé křižovatce}\label{subsec:porovnani_algoritmu_na_male_krizovatce}

V této kapitole porovnám jednotlivé algoritmy na malých typech křižovatek.

Jediný algoritmus, který nedoběhl ani na jednom typu malé křižovatky je \nameref{subsec:cbsoid}.
Z toho usuzuji, že \ref{str:cbs} je nejvíce citlivý na přidávání agentů.


Nejprve začnu čtvercovou křižovatkou, jejíž výsledky jsou v tabulce \ref{tab:all_exp_mala_ctvercova}.
V tabulce není zapsána hodnota parametru maximálního počtu přeplánovaných agentů (\ref{par:aoid_mpa}) u~\ref{str:oid} variant.
Na této křižovatce měli \nameref{subsubsec:a_star_aoid} a \nameref{subsec:cbsoid}
největší úspěšnost s \ref{par:aoid_mpa} nastaveným na $16$ a \nameref{subsec:sat_ra} na $12$.

Nejlepší algoritmus zde vyšel \ref{str:cbs}, v těsném závěsu za ním je \ref{str:a_star_ars}.
Tyto algoritmy mají zároveň nejmenší zpoždění agentů a v průměru nejméně agentů na křižovatce.
Zároveň měli nejnižší dobu plánování po \nameref{sec:safe_lanes}.

Po těchto algoritmech měly nejmenší počet zamítnutí \ref{str:sat} algoritmy a průměrná zpoždění byla srovnatelná.
Avšak po \nameref{subsec:cbsoid} měly nejvyšší čas plánování.

\nameref{sec:safe_lanes} sice zvládlo plánovat agenty mnohem rychleji než ostatní algoritmy,
avšak zároveň má mnohem více zamítnutých agentů, než všechny algoritmy, které doběhly.

\nameref{subsubsec:a_star_aoid} měl nejvyšší průměrný počet agentů na křižovatce,
avšak zároveň měl jednoznačně nejvyšší průměrné zpoždění agentů.

Verze algoritmů plánující menší počet agentů preferovala spíše volnější pohyb pro agenty.
Naopak všechny \ref{str:oid} varianty, \nameref{subsec:sat_ra} a \ref{str:a_star_arsg} dosáhly nejlepších výsledků s nejvíce omezenými parametry.

\begin{table}[h]
%	\centering
	\begin{adjustwidth}{-0.5cm}{}
		\begin{tabular}{c c c c | r r D{.}{,}{2.2} D{.}{,}{2.2} D{.}{,}{7.2}}
			\toprule \\
			\pulrad{\B{Alg}} & \pulrad{\B{Omez}} & \pulrad{\B{\ref{str:ars_mnv}}} &
			\pulrad{\B{\ref{str:ars_mpc}}} & \pulrad{\B{Krok}}  & \pulrad{\B{Zam}} &
			\mc{\pulrad{\B{pAg}}} & \mc{\pulrad{\B{pZp}}} & \mc{\pulrad{\B{Čas}}} \\
			\midrule
			\nameref{sec:safe_lanes}        & -  & 1 & inf & 32782 & 4056   & 11.19                                & 6.89                                & \multicolumn{1}{B{.}{,}{7.2}}{16.75} \\
			\hline
			\ref{str:a_star_ars}            & s  & 2 & inf & 32775 & 76     & 13.68                                & 3.92                                & 47.77                                \\
			\ref{str:varsg}           & -  & 1 & 8   & 32779 & 568    & 14.20                                & 5.72                                & 262.71                               \\
			\nameref{subsubsec:a_star_aoid} & -  & 1 & 8   & 32786 & 1373   & \multicolumn{1}{B{.}{,}{2.2}}{18.04} & 15.10 & 5582.03                                                            \\  % 16
			\hline
			\ref{str:cbs}                   & sr & 2 & inf & 32776 & \B{70} & 13.41                                & \multicolumn{1}{B{.}{,}{2.2}}{3.66} & 194.12                               \\
			\nameref{subsec:cbsoid}         & -  & 1 & 16  & 6838  & 51759  & 12.61                                & 8.45                                & 1054679.77                           \\  % 16
			\hline
			\nameref{subsec:sat_rsg}        & s  & 2 & 24  & 32778 & 255    & 13.46                                & 3.90                                & 131515.34                            \\
			\nameref{subsec:sat_ra}         & -  & 1 & 10  & 32776 & 363    & 15.07                                & 5.44                                & 9893.10                              \\  % 12
			\bottomrule
%		\multicolumn{6}{l}{\footnotesize \textit{Pozn:}
%		\textrm{Zam} - počet zamítnutí, \textrm{pAgen} - průměrný počet agentů v jeden krok na křižovatce, \\
%		\textrm{sAgen} - směrodatná odchylka počtu agentů na křižovatce, \\
%		\textrm{Zpož} - součet spoždění přes všechny agenty, \textrm{pZpož} - průměrné zpoždění agentů
%		}  TODO
		\end{tabular}
		\caption{Porovnání algoritmů na malé čtvercové křižovatce.}\label{tab:all_exp_mala_ctvercova}
	\end{adjustwidth}
\end{table}

Na oktagonální křižovatce je chování algoritmů podobné (tabulka \ref{tab:all_exp_mala_oktagonalni}).
Avšak zde ani jeden z algoritmů přeplánovající naplánované agenty nestihl doběhnout.
Zde měli \nameref{subsubsec:a_star_aoid} a \nameref{subsec:cbsoid}
největší úspěšnost opět s \ref{par:aoid_mpa} nastaveným na $16$ a \nameref{subsec:sat_ra} na $12$.

Pořadí algoritmů, které doběhly, s ohledem na počet zamítnutých agentů se nezměnilo,
stále si nejlépe vedlo \ref{str:cbs}.

\nameref{subsubsec:a_star_aoid} opět mělo v průměru nejvíce agentů na křižovatce, ale opět nejvyšší zpoždění agentů.

\nameref{sec:safe_lanes} sice zvládlo má opět mnohem více zamítnutých agentů, než všechny algoritmy, které doběhly.
Oproti čtvercové křižovatce se plánovací časy snížily,
avšak při tak nízkých časech to může být způsobeno zatížením systému jinými procesy.

Běhové časy se u ostatních algoritmů výrazně zvýšily.
To může být způsobeno výrazným nárůstem počtu zamítnutých agentů.
Podle mého názoru je tento nárůst způsoben zmenšením celkové plochy křižovatky.
Diagonální vrcholy size umožňují více možných pozic pro agenty,
avšak tyto vrcholy jsou blízko svým sousedům, což způsobuje větší vzájemné překážení jednotlivých agentů.

Obecně bych řekl, že zde algoritmy více profitovaly z méně omezenéných parametrů,
až na \ref{str:a_star_arsg} a \nameref{subsubsec:a_star_aoid}.

\begin{table}[h]
%	\centering
	\begin{adjustwidth}{-0.5cm}{}
		\begin{tabular}{c c c c | r r D{.}{,}{2.2} D{.}{,}{2.2} D{.}{,}{7.2}}
			\toprule \\
			\pulrad{\B{Alg}} & \pulrad{\B{Omez}} & \pulrad{\B{\ref{str:ars_mnv}}} &
			\pulrad{\B{\ref{str:ars_mpc}}} & \pulrad{\B{Krok}}  & \pulrad{\B{Zam}} &
			\mc{\pulrad{\B{pAg}}} & \mc{\pulrad{\B{pZp}}} & \mc{\pulrad{\B{Čas}}} \\
			\midrule
			\nameref{sec:safe_lanes}        & -  & 1 & inf & 32785 & 13488    & 9.34                                 & 10.36                               & \multicolumn{1}{B{.}{,}{7.2}}{11.30} \\
			\hline
			\ref{str:a_star_ars}            & sr & 2 & inf & 32779 & 1338     & 14.19                                & 6.83                                & 89.03                                \\
			\ref{str:varsg}           & -  & 1 & 8   & 32785 & 6276     & 13.14                                & 9.46                                & 237.60                               \\
			\nameref{subsubsec:a_star_aoid} & -  & 1 & 8   & 16976 & 32774    & \multicolumn{1}{B{.}{,}{2.2}}{18.49} & 17.57 & 424199.33                                                          \\  % 16
			\hline
			\ref{str:cbs}                   & sr & 2 & inf & 32780 & \B{1079} & 13.91                                & \multicolumn{1}{B{.}{,}{2.2}}{6.46} & 1761.27                              \\
			\nameref{subsec:cbsoid}         & s  & 2 & inf & 2889  & 59641    & 11.99                                & 10.23                               & 2497072.06                           \\  % 16
			\hline
			\nameref{subsec:sat_rsg}        & s  & 2 & 24  & 32782 & 2868     & 13.81                                & 7.77                                & 133605.81                            \\
			\nameref{subsec:sat_ra}         & s  & 1 & 10  & 20821 & 25521    & 15.16                                & 8.57                                & 345728.00                            \\  %  8
			\bottomrule
%		\multicolumn{6}{l}{\footnotesize \textit{Pozn:}
%		\textrm{Zam} - počet zamítnutí, \textrm{pAgen} - průměrný počet agentů v jeden krok na křižovatce, \\
%		\textrm{sAgen} - směrodatná odchylka počtu agentů na křižovatce, \\
%		\textrm{Zpož} - součet spoždění přes všechny agenty, \textrm{pZpož} - průměrné zpoždění agentů
%		}  TODO
		\end{tabular}
		\caption{Porovnání algoritmů na malé oktagonální křižovatce.}\label{tab:all_exp_mala_oktagonalni}
	\end{adjustwidth}
\end{table}

Jelikož tato křižovatka obsahuje téměř dvakrát více vrcholů, než oktagonální, $48$ oproti $28$, čekal jsem,
že běhové časy algoritmů budou značně horší než na oktagonální křižovatce.
Avšak první pohled na tabulku \ref{tab:all_exp_mala_hexagonalni}, kde jsou zapsaná výsledky, napovídá opaku.

Jediný algoritmus, který nedoběhl, je opět \nameref{subsec:cbsoid}.
Avšak pokud srovnám časy plánování mezi předchozí křižovatkou a touto, jediné algoritmy, které si polepšily,
jsou \nameref{subsubsec:a_star_aoid} a \ref{str:sat} algoritmy.
Podle mého názoru je tomuto jevu značně pomohlo snížení počtu přeplánovaných agentů.
Ten totiž měl u \nameref{subsubsec:a_star_aoid} hodnotu $12$,
u \nameref{subsec:cbsoid} $24$ a u \nameref{subsec:sat_ra} $8$.

Zároveň u \nameref{subsec:sat_rsg} stoupla průměrná zaplňenost křižovatky z přibližně $49,32\%$ na $54,56\%$.
Jak jsem popsal u v kapitole s výsledky pro \ref{str:sat} (Kapitola \ref{subsubsec:sat_zavislost_casu_a_agentu}),
\ref{str:sat} algoritmy mají menší plánovací čas s více zaplněnou křižovatkou.
Avšak nemyslím si, že je toto hlavní důvod zrychlení, jelikož i když je zaplněnost křižovatky vyšší,
celkový počet vrcholů je značně vyšší, čili průměrný počet volných vrcholů se zvýšil.
Osobně si myslím, že je rozdíl v časech způsobený nižším nastavením parametru \ref{par:ars_mpc}
u~\nameref{subsec:sat_rsg}, a nižším počtem přeplánovaných agentů u~\nameref{subsec:sat_ra}.

Překvapivě na této křižovatce má nejméně zamítnutých agentů a nejmenší zpoždění zmiňovaný \nameref{subsec:sat_ra}.
Mimo to je pořadí algoritmů z pohledu na počet zamítnutých agentů totožné s pořadím na čtvercové křižovatce.
Druhý nejlepší je \ref{cbs}, následuje \ref{str:a_star_ars}, \nameref{subsec:sat_rsg}, \ref{str:a_star_arsg}
a \nameref{subsubsec:a_star_aoid}.
Nejhorším úspěšně doběhnutým algoritmem je opět \nameref{sec:safe_lanes}.

\begin{table}[h]
%	\centering
	\begin{adjustwidth}{-0.5cm}{}
		\begin{tabular}{c c c c | r r D{.}{,}{2.2} D{.}{,}{2.2} D{.}{,}{7.2}}
			\toprule \\
			\pulrad{\B{Alg}} & \pulrad{\B{Omez}} & \pulrad{\B{\ref{str:ars_mnv}}} &
			\pulrad{\B{\ref{str:ars_mpc}}} & \pulrad{\B{Krok}}  & \pulrad{\B{Zam}} &
			\mc{\pulrad{\B{pAg}}} & \mc{\pulrad{\B{pZp}}} & \mc{\pulrad{\B{Čas}}} \\
			\midrule
			\nameref{sec:safe_lanes}        & -  & 1 & inf & 32796 & 32273    & 14.15                                & 18.18                               & \multicolumn{1}{B{.}{,}{7.2}}{87.44} \\
			\hline
			\ref{str:a_star_ars}            & s  & 2 & inf & 32790 & 3021     & 26.54                                & 11.08                               & 734.08                               \\
			\ref{str:varsg}           & -  & 1 & 16  & 32792 & 5872     & 25.94                                & 13.15                               & 1195.23                              \\
			\nameref{subsubsec:a_star_aoid} & s  & 2 & 16  & 32800 & 9472     & \multicolumn{1}{B{.}{,}{2.2}}{32.67} & 21.81                               & 10276.77                             \\  %  12
			\hline
			\ref{str:cbs}                   & sr & 2 & inf & 32790 & 2698     & 25.74                                & 10.80                               & 3128.19                              \\
			\nameref{subsec:cbsoid}         & s  & 2 & inf & 1704  & 93403    & 22.16                                & 21.18                               & 4255409.05                           \\  % 24
			\hline
			\nameref{subsec:sat_rsg}        & -  & 1 & 16  & 32787 & 4403     & 26.19                                & 12.08                               & 78214.86                             \\
			\nameref{subsec:sat_ra}         & -  & 1 & 10  & 32788 & \B{1905} & 26.06                                & \multicolumn{1}{B{.}{,}{2.2}}{8.85}  & 34920.59                            \\  %  8
			\bottomrule
%		\multicolumn{6}{l}{\footnotesize \textit{Pozn:}
%		\textrm{Zam} - počet zamítnutí, \textrm{pAgen} - průměrný počet agentů v jeden krok na křižovatce, \\
%		\textrm{sAgen} - směrodatná odchylka počtu agentů na křižovatce, \\
%		\textrm{Zpož} - součet spoždění přes všechny agenty, \textrm{pZpož} - průměrné zpoždění agentů
%		}  TODO
		\end{tabular}
		\caption{Porovnání algoritmů na malé hexagonální křižovatce.}\label{tab:all_exp_mala_hexagonalni}
	\end{adjustwidth}
\end{table}

Z uvedených výsledků usuzuji, že na malých křižovatkách se mnohem více vyplatí jednodušší, více omezené algoritmy.
Vypadá to, že pokud algoritmus přeplánovává agenty, vyplatí se měnit pouze malé množství agentů.
\nameref{sec:safe_lanes} plánuje agenty hodně rychle, avšak za cenu vysokého počtu zamítnutých agentů.

\subsection{Porovnání algoritmů na velké křižovatce bez výjezdů}
\label{subsec:porovnani_algoritmu_na_velke_krizovatce_bez_vyjezdu}

V této sekci jsou porovnány pouze algoritmy, které zvládly alespoň částečně něco spočítat.
\ref{str:sat} algoritmy nezvládly výpočet ani na jediném typu křižovatky.
\nameref{subsubsec:a_star_aoid} úspěšně běžel jenom na čtvercové křižovatce, na zbylých padal kvůli paměťové náročnosti.
\nameref{subsec:cbsoid} byl kvůli času neúspěšný na hexagonálním typu.

Výsledky pro čtvercovou křižovatku (\ref{tab:all_exp_velka_ctvercova_bez_vyjezdu}) obsahují určité překvapivé hodnoty.
Zde algoritmy \nameref{subsubsec:a_star_aoid} i \nameref{subsec:cbsoid} plánovaly nejvýše $32$ agentů každý krok.

\nameref{subsubsec:a_star_aoid} opět dosáhl největšího zaplnění křižovatky.
Avšak tentokrát měl i nejnižší počet zamítnutých agentů.

Proto mě překvapuje, že \ref{str:a_star_arsg} měl druhý nejvyšší počet zamítnutí hned po \nameref{sec:safe_lanes}.
Nepočítám tedy \nameref{subsec:cbsoid}, který nespočítal skoro žádné kroky.

Jediný důvod, proč toto nastalo, je podle mě vysoké množství spojování skupin u \ref{str:a_star_arsg},
které vede k dlouhé době plánování.
To způsobí, že algoritmus přejde do zjednodušeného počítání, ve kterém zamítá agenty místo snahy o přeplánování.
Ačkoliv \nameref{subsubsec:a_star_aoid} má mnohem vyšší dobu běhu, je možné, že se algoritmus \uv{zasekne}
na pár krocích, ale ve zbytku kroků je rychlejší a úspěšnější.

\ref{str:cbs} dosáhl menšího počtu zamítnutí než \ref{str:a_star_ars}, avšak za cenu vyššího průměrného zpoždění.

Řekl bych, že na této křižovatce algoritmy preferovaly menší omezení parametrů.
Algoritmy, u~kterých nezvítězily nejméně omezené varianty, byly většinou s vysokým časem běhu.

\begin{table}[h]
%	\centering
	\begin{adjustwidth}{-1cm}{}
	\begin{tabular}{c c c c | r r D{.}{,}{3.2} D{.}{,}{2.2} D{.}{,}{7.2}}
		\toprule \\
		\pulrad{\B{Alg}} & \pulrad{\B{Omez}} & \pulrad{\B{\ref{par:ars_mnv}}} &
		\pulrad{\B{\ref{par:ars_mpc}}} & \pulrad{\B{Krok}}  & \pulrad{\B{Zam}} &
		\mc{\pulrad{\B{pAg}}} & \mc{\pulrad{\B{pZp}}} & \mc{\pulrad{\B{Čas}}} \\
		\midrule
		\nameref{sec:safe_lanes}        & -  & 1 & inf & 32849 & 75983     & 86.91                                 & 44.18                                & \multicolumn{1}{B{.}{,}{7.2}}{1449.67} \\
		\hline
		\ref{str:a_star_ars}            & sr & 2 & inf & 32847 & 16461     & 138.36                                & \multicolumn{1}{B{.}{,}{2.2}}{26.71} & 11011.56   \\
		\ref{str:a_star_arsg}           & sr & 2 & inf & 32847 & 22451     & 145.58                                & 43.62                                & 21190.45                               \\
		\nameref{subsubsec:a_star_aoid} & -  & 1 & 32  & 32840 & \B{11285} & \multicolumn{1}{B{.}{,}{3.2}}{159.15} & 35.46 & 204258.46  \\  % 32
		\hline
		\ref{str:cbs}                   & s  & 2 & inf & 32848 & 13193     & 137.94                                & 31.45                                & 69699.31                               \\
		\nameref{subsec:cbsoid}         & -  & 1 & 32  & 965   & 255430    & 121.15                                & 45.84                                & 7631472.87                             \\  % 32
		\bottomrule
%		\multicolumn{6}{l}{\footnotesize \textit{Pozn:}
%		\textrm{Zam} - počet zamítnutí, \textrm{pAgen} - průměrný počet agentů v jeden krok na křižovatce, \\
%		\textrm{sAgen} - směrodatná odchylka počtu agentů na křižovatce, \\
%		\textrm{Zpož} - součet spoždění přes všechny agenty, \textrm{pZpož} - průměrné zpoždění agentů
%		}  TODO
	\end{tabular}
	\caption{Porovnání algoritmů na velké čtvercové křižovatce bez výjezdů.}\label{tab:all_exp_velka_ctvercova_bez_vyjezdu}
	\end{adjustwidth}
\end{table}


Pro oktagonální křižovatku jsou výsledky v tabulce \ref{tab:all_exp_velka_oktagonalni_bez_vyjezdu}.
Zde jediné algoritmy, které úspěšně dopočítaly všechny kroky, byly
\nameref{sec:safe_lanes}, \ref{str:a_star_ars} a \ref{str:a_star_arsg}, proto se zaměřím převážně na ně.

\ref{str:a_star_ars} zde dosáhl nejnižšího počtu zamítnutých agentů a zaplněnosti křižovatky.

Jelikož \ref{str:a_star_arsg} měla průměrně méně agentů na křižovatce než \ref{str:a_star_ars}, myslím si,
že \ref{str:a_star_arsg} často přešel do zjednodušeného režimu, čož asi způsobilo takové velké množství zamítnutí.
Díky tomu ale nejspíše dosáhl nejnižšího průměrného zpoždění agentů, jelikož křižovatka byla v průměru volnější.

Překvapením jsou plánovací časy algoritmů.
\ref{str:a_star_arsg} dosáhla výrazně rychlejšího plánování, než \ref{str:a_star_ars}.
\ref{str:a_star_ars} měl méně omezené cesty agentů, což mohlo způsobit tento velký rozdíl.
Množství možných cest pro agenta je díky tomu mnohem vyšší, což vede na mnohem více výpočtů,
než algoritmus zamítne vjezd agentovi.

\begin{table}[h]
%	\centering
	\begin{adjustwidth}{-1cm}{}
	\begin{tabular}{c c c c | r r D{.}{,}{3.2} D{.}{,}{2.2} D{.}{,}{8.2}}
		\toprule \\
		\pulrad{\B{Alg}} & \pulrad{\B{Omez}} & \pulrad{\B{\ref{str:ars_mnv}}} &
		\pulrad{\B{\ref{str:ars_mpc}}} & \pulrad{\B{Krok}}  & \pulrad{\B{Zam}} &
		\mc{\pulrad{\B{pAg}}} & \mc{\pulrad{\B{pZp}}} & \mc{\pulrad{\B{Čas}}} \\
		\midrule
		\nameref{sec:safe_lanes} & -  & 1 & inf & 32846 & 107830   & 71.98                                 & 58.92                                & \multicolumn{1}{B{.}{,}{8.2}}{1914.34} \\
		\hline
		\ref{str:a_star_ars}     & s  & 2 & inf & 32837 & \B{6046} & \multicolumn{1}{B{.}{,}{3.2}}{157.83} & 25.65 & 117705.81   \\
		\ref{str:varsg}    & -  & 1 & 32  & 32840 & 12990    & 142.63                                & \multicolumn{1}{B{.}{,}{2.2}}{19.84} & 62133.51    \\
		\hline
		\ref{str:cbs}            & sr & 2 & inf & 10221 & 184171   & 146.21                                & 26.42                                & 705203.89                              \\
		\nameref{subsec:cbsoid}  & sr & 2 & inf & 638   & 257862   & 118.79                                & 47.18                                & 11697421.69                            \\  % 32
		\bottomrule
%		\multicolumn{6}{l}{\footnotesize \textit{Pozn:}
%		\textrm{Zam} - počet zamítnutí, \textrm{pAgen} - průměrný počet agentů v jeden krok na křižovatce, \\
%		\textrm{sAgen} - směrodatná odchylka počtu agentů na křižovatce, \\
%		\textrm{Zpož} - součet spoždění přes všechny agenty, \textrm{pZpož} - průměrné zpoždění agentů
%		}  TODO
	\end{tabular}
	\caption{Porovnání algoritmů na velké oktagonální křižovatce bez výjezdů.}\label{tab:all_exp_velka_oktagonalni_bez_vyjezdu}
	\end{adjustwidth}
\end{table}


Hexagonální křižovatka obsahuje mnohem více vrcholů, než oktagonální.
Proto není divu,
že tabulka s výsledky pro tuto křižovatku obsahuje ještě méně záznamů (\ref{tab:all_exp_velka_hexagonalni_bez_vyjezdu}).

Tabulka neobsahuje moc zajímavých hodnot, jelikož jediné algoritmy, které úspěšně dopočítaly všechny kroky až do konce,
jsou \nameref{sec:safe_lanes} a \ref{str:a_star_ars}.
\ref{str:a_star_ars} má vyšší výpočetní dobu, avšak mnohem nižší počet zamítnutí.

Překvapilo mě jenom, že i když \ref{str:a_star_ars} mělo méně omezené agenty než na oktagonální křižovatce,
časy plánování zůstaly přibližně stejné.
Na oktagonální křižovatce tento algoritmus zamítl přibližně $2,3\%$ agentů, avšak na hexagonální celých $13,56\%$.
Avšak zaplnění křižovatky byl na oktagonální křižovatce $31.07\%$, zatímco na hexagonální $37.4\%$.
Zároveň na hexagonální křižovatce bylo mnohem vyšší průměrné zpoždění agentů.
Proto si myslím, že na hexagonální křižovatce si agenti mnohem více překážejí.
Myslím si, že to způsobuje snazší a rychlejší zjištění, že neexistuje pro daného agenta žádná cesta.
Podle mého je toto hlavní důvod, proč není výrazné zvýšení plánovacího času.

\begin{table}[h]
%	\centering
	\begin{adjustwidth}{-1cm}{}
	\begin{tabular}{c c c c | r r D{.}{,}{3.2} D{.}{,}{2.2} D{.}{,}{7.2}}
		\toprule \\
		\pulrad{\B{Alg}} & \pulrad{\B{Omez}} & \pulrad{\B{\ref{str:ars_mnv}}} &
		\pulrad{\B{\ref{str:ars_mpc}}} & \pulrad{\B{Krok}}  & \pulrad{\B{Zam}} &
		\mc{\pulrad{\B{pAg}}} & \mc{\pulrad{\B{pZp}}} & \mc{\pulrad{\B{Čas}}} \\
		\midrule
		\nameref{sec:safe_lanes} & -  & 1 & inf & 32894 & 212628    & 109.63                                & 92.37                                & \multicolumn{1}{B{.}{,}{7.2}}{4221.17} \\
		\hline
		\ref{str:a_star_ars}     & sr & 2 & inf & 32893 & \B{35567} & 287.20                                & \multicolumn{1}{B{.}{,}{2.2}}{48.34} & 128608.45  \\
		\ref{str:varsg}    & s  & 2 & inf & 20359 & 198896    & \multicolumn{1}{B{.}{,}{3.2}}{299.14} & 88.14 & 353255.32  \\
		\hline
		\ref{str:cbs}            & s  & 2 & inf & 4011  & 347852    & 292.50                                & 59.07                                & 1810300.99                             \\
		\bottomrule
%		\multicolumn{6}{l}{\footnotesize \textit{Pozn:}
%		\textrm{Zam} - počet zamítnutí, \textrm{pAgen} - průměrný počet agentů v jeden krok na křižovatce, \\
%		\textrm{sAgen} - směrodatná odchylka počtu agentů na křižovatce, \\
%		\textrm{Zpož} - součet spoždění přes všechny agenty, \textrm{pZpož} - průměrné zpoždění agentů
%		}  TODO
	\end{tabular}
	\caption{Porovnání algoritmů na velké hexagonální křižovatce bez výjezdů.}\label{tab:all_exp_velka_hexagonalni_bez_vyjezdu}
	\end{adjustwidth}
\end{table}


\subsection{Porovnání algoritmů na velké křižovatce s výjezdy}
\label{subsec:porovnani_algoritmu_na_velke_krizovatce_s_vyjezdy}



\begin{table}[h]
%	\centering
	\begin{adjustwidth}{-1cm}{}
		\begin{tabular}{c c c c | r r D{.}{,}{3.2} D{.}{,}{2.2} D{.}{,}{7.2}}
			\toprule \\
			\pulrad{\B{Alg}} & \pulrad{\B{Omez}} & \pulrad{\B{\ref{par:ars_mnv}}} &
			\pulrad{\B{\ref{par:ars_mpc}}} & \pulrad{\B{Krok}}  & \pulrad{\B{Zam}} &
			\mc{\pulrad{\B{pAg}}} & \mc{\pulrad{\B{pZp}}} & \mc{\pulrad{\B{Čas}}} \\
			\midrule
			\nameref{sec:safe_lanes} & -  & 1 & inf & 32850 & 111987    & 77.02                                 & 60.70                                & \multicolumn{1}{B{.}{,}{7.2}}{1244.95} \\
			\hline
			\ref{str:a_star_ars}     & sr & 2 & inf & 32846 & 26667     & 155.96                                & \multicolumn{1}{B{.}{,}{2.2}}{35.99} & 14428.00   \\
			\ref{str:a_star_arsg}    & sr & 2 & inf & 32855 & 40413     & \multicolumn{1}{B{.}{,}{3.2}}{159.59} & 52.69 & 49668.90   \\
			\hline
			\ref{str:cbs}            & sr & 2 & inf & 32846 & \B{23949} & 155.63                                & 36.41                                & 160418.65                              \\
			\nameref{subsec:cbsoid}  & -  & 1 & 64  & 957   & 255758    & 136.49                                & 52.40                                & 7736125.71                             \\  % 24
			\bottomrule
%		\multicolumn{6}{l}{\footnotesize \textit{Pozn:}
%		\textrm{Zam} - počet zamítnutí, \textrm{pAgen} - průměrný počet agentů v jeden krok na křižovatce, \\
%		\textrm{sAgen} - směrodatná odchylka počtu agentů na křižovatce, \\
%		\textrm{Zpož} - součet spoždění přes všechny agenty, \textrm{pZpož} - průměrné zpoždění agentů
%		}  TODO
		\end{tabular}
		\caption{Porovnání algoritmů na velké čtvercové křižovatce s výjezdy.}\label{tab:all_exp_velka_ctvercova_s_vyjezdy}
	\end{adjustwidth}
\end{table}

\begin{table}[h]
%	\centering
	\begin{adjustwidth}{-1cm}{}
		\begin{tabular}{c c c c | r r D{.}{,}{3.2} D{.}{,}{2.2} D{.}{,}{7.2}}
			\toprule \\
			\pulrad{\B{Alg}} & \pulrad{\B{Omez}} & \pulrad{\B{\ref{par:ars_mnv}}} &
			\pulrad{\B{\ref{par:ars_mpc}}} & \pulrad{\B{Krok}}  & \pulrad{\B{Zam}} &
			\mc{\pulrad{\B{pAg}}} & \mc{\pulrad{\B{pZp}}} & \mc{\pulrad{\B{Čas}}} \\
			\midrule
			\nameref{sec:safe_lanes} & -  & 1 & inf & 32849 & 138867    & 61.62                                 & 60.30                                & \multicolumn{1}{B{.}{,}{7.2}}{937.51} \\
			\hline
			\ref{str:a_star_ars}     & sr & 2 & inf & 32844 & \B{24714} & \multicolumn{1}{B{.}{,}{3.2}}{164.83} & \multicolumn{1}{B{.}{,}{2.2}}{34.25} & 72598.43   \\
			\ref{str:a_star_arsg}    & sr & 2 & inf & 32855 & 36807     & 163.73                                & 52.16                                & 145805.57                             \\
			\hline
			\ref{str:cbs}            & s  & 2 & inf & 6774  & 212658    & 159.57                                & 43.74                                & 1065891.74                            \\

			\bottomrule
%		\multicolumn{6}{l}{\footnotesize \textit{Pozn:}
%		\textrm{Zam} - počet zamítnutí, \textrm{pAgen} - průměrný počet agentů v jeden krok na křižovatce, \\
%		\textrm{sAgen} - směrodatná odchylka počtu agentů na křižovatce, \\
%		\textrm{Zpož} - součet spoždění přes všechny agenty, \textrm{pZpož} - průměrné zpoždění agentů
%		}  TODO
		\end{tabular}
		\caption{Porovnání algoritmů na velké oktagonální křižovatce s výjezdy.}\label{tab:all_exp_velka_oktagonalni_s_vyjezdy}
	\end{adjustwidth}
\end{table}

\begin{table}[h]
%	\centering
	\begin{adjustwidth}{-1cm}{}
		\begin{tabular}{c c c c | r r D{.}{,}{3.2} D{.}{,}{2.2} D{.}{,}{7.2}}
			\toprule \\
			\pulrad{\B{Alg}} & \pulrad{\B{Omez}} & \pulrad{\B{\ref{str:ars_mnv}}} &
			\pulrad{\B{\ref{str:ars_mpc}}} & \pulrad{\B{Krok}}  & \pulrad{\B{Zam}} &
			\mc{\pulrad{\B{pAg}}} & \mc{\pulrad{\B{pZp}}} & \mc{\pulrad{\B{Čas}}} \\
			\midrule
			\nameref{sec:safe_lanes} & -  & 1 & inf & 32893 & 234142    & 99.14                                 & 92.83                                & \multicolumn{1}{B{.}{,}{7.2}}{2399.98} \\
			\hline
			\ref{str:a_star_ars}     & sr & 2 & inf & 32894 & \B{35385} & 316.87                                & \multicolumn{1}{B{.}{,}{2.2}}{53.93} & 211489.82  \\
			\ref{str:varsg}    & s  & 2 & inf & 15656 & 240581    & \multicolumn{1}{B{.}{,}{3.2}}{331.02} & 93.02 & 459814.95  \\
			\hline
			\ref{str:cbs}            & s  & 2 & inf & 4580  & 341093    & 312.39                                & 67.51                                & 1583005.68                             \\
			\bottomrule
%		\multicolumn{6}{l}{\footnotesize \textit{Pozn:}
%		\textrm{Zam} - počet zamítnutí, \textrm{pAgen} - průměrný počet agentů v jeden krok na křižovatce, \\
%		\textrm{sAgen} - směrodatná odchylka počtu agentů na křižovatce, \\
%		\textrm{Zpož} - součet spoždění přes všechny agenty, \textrm{pZpož} - průměrné zpoždění agentů
%		}  TODO
		\end{tabular}
		\caption{Porovnání algoritmů na velké hexagonální křižovatce s výjezdy.}\label{tab:all_exp_velka_hexagonalni_s_vyjezdy}
	\end{adjustwidth}
\end{table}

