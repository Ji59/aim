\section{Testovací data}\label{sec:testovaci_data}

%Popis experimentálních dat - velikost křižovatky, počet agentů a kroků, safe distance.

\subsubsection{Délka simulace}

Simulace bude přidávat agenty po $32768$ kroků.
Pokud bude plánování trvat příliš dlouhou dobu, bude simulace předčasně ukončena.

\subsubsection{Křižovatka}

Algoritmy budu testovat na každém typu křižovatky o třech velikosti.
\paragraph{Malá}\label{par:data_mala} křižovatka bude mít \hyperref[par:granularita]{granularitu} rovnou $4$.
Tato křižovatka bude mít pouze jeden \hyperref[par:vjezdy]{vjezd} a jeden \hyperref[par:vyjezdy]{výjezd}.
\paragraph{Střední}\label{par:data_stredni} křižovatka bude mít \hyperref[par:granularita]{granularitu} rovnou $8$.
\hyperref[par:vjezdy]{Počet vjezdů} a \hyperref[par:vyjezdy]{výjezdů} bude činit $3$.
\paragraph{Velká}\label{par:data_velka} křižovatka bude mít \hyperref[par:granularita]{granularitu} rovnou $16$.
\hyperref[par:vjezdy]{Počet vjezdů} a \hyperref[par:vyjezdy]{výjezdů} je zvýšen na $5$.

\subsubsection{Agenti}

Agenti budou nejdříve náhodně vygenerováni zvlášť pro každou velikost a typ křižovatky.
Poté se ti samí agenti použijí k porovnání na jednotlivých křižovatkách.
V každý kroku přibude mezi $0 - en$ agentů, kde $en$ je celkový počet vjezdů do křižovatky ze všech stran.
Přesný počet je uniformě náhodně vygenerován každý krok.

\ref{sec:agent}Délka agentů bude $0.56$ \hyperref[par:velikost_bloku]{velikosti bloku} křižovatky
a šířka $0.35$ \hyperref[par:velikost_bloku]{velikosti bloku}.
Tyto hodnoty jsem zvolil, jelikož umožňují nekolizní pozice agentů na sousedních vrcholech.
Zároveň ale agenti nejsou příliš malí na to, aby jejich velikost nehrála roli.

Dále budu porovnávat případy, kdy agenti mají daný přesný výjezd, nebo pouze směr výjezdu.
Tyto případy má cenu porovnávat pouze u \hyperref[par:data_stredni]{střední} a
\hyperref[par:data_velka]{velké} křižovatky, jelikož obsahují více výjezdů.
