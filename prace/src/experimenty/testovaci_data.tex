\section{Testovací data}\label{sec:testovaci_data}

%Popis experimentálních dat - velikost křižovatky, počet agentů a kroků, safe distance.

\subsection{Délka simulace}\label{subsec:delka_simulace}

Simulace bude přidávat agenty po $32768$ kroků.
Pokud bude plánování trvat příliš dlouhou dobu, bude simulace předčasně ukončena.
Maximální dobu běhu simulace jsem omezil na 2 hodiny.

\subsubsection{Křižovatka}

Algoritmy budu testovat na každém typu křižovatky o dvou velikostech.
\paragraph{Malá}\label{par:data_mala} křižovatka bude mít \hyperref[par:velikost_krizovatky]{velikost} rovnou~$4$.
Tato křižovatka bude mít pouze jeden \hyperref[par:vjezdy]{vjezd} a jeden \hyperref[par:vyjezdy]{výjezd}.
%\paragraph{Střední}\label{par:data_stredni} křižovatka bude mít \hyperref[par:velikost_krizovatky]{velikost}~$8$.
%\hyperref[par:vjezdy]{Počet vjezdů} a \hyperref[par:vyjezdy]{výjezdů} bude činit $3$.
\paragraph{Velká}\label{par:data_velka} křižovatka bude mít \hyperref[par:velikost_krizovatky]{velikost}~$16$.
Počet \hyperref[par:vjezdy]{vjezdů} a \hyperref[par:vyjezdy]{výjezdů} je zvýšen na $4$.

\subsubsection{Agenti}

Agenti budou nejdříve náhodně vygenerovaní zvlášť pro každou velikost.
Generování proběhne dvakrát pro každou velikost,
jednou pro hexagonální typ a jednou společně pro čtvercový a oktagonální typ.
Poté se ti samí agenti použijí k porovnání jednotlivých algoritmů, abych snížil vliv náhody na výsledky.
V každém kroku přibude mezi $0 - en$ agentů, kde $en$ je celkový počet vjezdů do křižovatky ze všech stran.
Přesný počet je náhodně vygenerován každý krok.
Popis generování agentů je blíže popsán v sekci \ref{subsec:generovani_agentu}.

Délka agentů bude $0.56$ \hyperref[par:velikost_bloku]{velikosti bloku} křižovatky
a šířka $0.35$ \hyperref[par:velikost_bloku]{velikosti bloku}.
Tyto hodnoty jsem zvolil, jelikož umožňují nekolizní pozice agentů na sousedních vrcholech u všech typů křižovatek.
Zároveň ale agenti nejsou příliš malí na to, aby jejich velikost nehrála žádnou roli.

Dále budu porovnávat případy, kdy agenti mají daný přesný výjezd, nebo pouze směr výjezdu.
Tyto případy má cenu porovnávat pouze u
%\hyperref[par:data_stredni]{střední} a
\hyperref[par:data_velka]{velké} křižovatky, jelikož obsahuje více výjezdů.
