\subsection{\ref{str:sat} porovnání parametrů}\label{subsec:sat_porovnani_parametru}

U~\ref{str:sat} algoritmů jsem zkoušel stejné nastavení parametrů až na maximální počet prodlevu cesty (\ref{par:ars_mpc}),
jelikož musí být vždy nastaven na některou hodnotu.
Zjistil jsem, že složitost hledání roste velmi rychle se zvyšující \ref{par:ars_mpc},
proto jsem nastavil tento parametr maximálně na $24$.
I přes to měl algoritmus problémy.

S žádným nastavením mi algoritmy nedoběhly na velké křižovatce, pokud nepočítám neoptimalizovanou variantu.
Neoptimalizované běhy měli mnohem nižší čas plánování, avšak algoritmus mnoho agentů zamítal jen proto, že \ref{str:sat}
řešič vrátil model, který vjezd odmítl, i když by pro něj validní cesta existovala.
Proto jsem se rozhodl tento parametr nenastavovat, všechny zobrazené běhy jsou vyřešené \textrm{MAX-SAT}em.

\subsubsection{\nameref{subsec:sat_rsg} na \hyperref[par:data_mala]{malé} křižovatce}
\label{subsubsec:exp_satsg_mala_krizovatka}

\begin{table}[b!]
	\centering
%	\begin{adjustwidth}{-1.5cm}{}
	\begin{tabular}{c c c c | r r D{.}{,}{2.2} D{.}{,}{2.2} D{.}{,}{7.2}}
		\toprule \\
		\pulrad{\B{Typ}} & \pulrad{\B{Omez}} & \pulrad{\B{\ref{par:ars_mnv}}} &
		\pulrad{\B{\ref{par:ars_mpc}}} & \pulrad{\B{Krok}}  & \pulrad{\B{Zam}} &
		\mc{\pulrad{\B{pAg}}} & \mc{\pulrad{\B{pZp}}} & \mc{\pulrad{\B{Čas}}} \\
		\midrule
		S & - & 1 & 16 & 32782 & 1095     & \multicolumn{1}{B{.}{,}{2.2}}{14.29} & 5.99                                & \multicolumn{1}{B{.}{,}{7.2}}{4109.71}  \\
		S & s & 2 & 24 & 32778 & \B{255}  & 13.46                                & \multicolumn{1}{B{.}{,}{2.2}}{3.90} & 131515.34                               \\
		\hline
		O & - & 1 & 16 & 32784 & 4657     & 13.22                                & 8.65                                & \multicolumn{1}{B{.}{,}{7.2}}{15154.02} \\
		O & s & 2 & 24 & 32782 & \B{2868} & \multicolumn{1}{B{.}{,}{2.2}}{13.81} & \multicolumn{1}{B{.}{,}{2.2}}{7.77}  & 133605.81  \\
		\hline
		H & - & 1 & 16 & 32787 & \B{4403} & \multicolumn{1}{B{.}{,}{2.2}}{26.19} & 12.08                               & \multicolumn{1}{B{.}{,}{7.2}}{78214.86} \\
		H & s & 2 & 22 & 0     & 94573    & 1.02                                 & \multicolumn{1}{B{.}{,}{2.2}}{7.60} & 7200869.77                              \\
		\bottomrule
%		\multicolumn{6}{l}{\footnotesize \textit{Pozn:}
%		\textrm{Zam} - počet zamítnutí, \textrm{pAgen} - průměrný počet agentů v jeden krok na křižovatce, \\
%		\textrm{sAgen} - směrodatná odchylka počtu agentů na křižovatce, \\
%		\textrm{Zpož} - součet spoždění přes všechny agenty, \textrm{pZpož} - průměrné zpoždění agentů
%		}  TODO
	\end{tabular}
	\caption{Porovnání vlivu parametrů u \nameref{subsec:sat_rsg} na různých typech malé křižovatky.}\label{tab:sat_exp_mala}
%	\end{adjustwidth}
\end{table}


\ref{subsec:sat_rsg} se choval poměrně smysluplně, snižující omezení výpočtu vedlo k lepším výsledkům
za cenu vyšší doby plánování.
Tyto výsledky jsou v tabulce \ref{subsubsec:exp_satsg_mala_krizovatka}.

Jedinou výjimkou je hexagonální křižovatka, kde volnější varianta nespočítala skoro nic.

Je vidět vysoký nárůst složitosti při přechodu na graf s více vrcholy.
Opět si ale čtvercová křižovatka vedla mnohem lépe než hexagonální.


\subsubsection{\nameref{subsec:sat_ra} na \hyperref[par:data_mala]{malé} křižovatce}
\label{subsubsec:exp_sata_mala_krizovatka}

\ref{subsec:cbsoid} algoritmus ukazuje extrémní nárůst složitosti se zvyšujícím počtem plánovaných agentů.
Proto jsem značně snížil i \ref{par:ars_mpc}.

\begin{table}[b!]
	\centering
%	\begin{adjustwidth}{-1cm}{}
	\begin{tabular}{c c c c c | r r D{.}{,}{2.2} D{.}{,}{1.2} D{.}{,}{8.2}}
		\toprule \\
		\pulrad{\B{Typ}} & \pulrad{\B{Omez}} & \pulrad{\B{\ref{par:ars_mnv}}} &
		\pulrad{\B{\ref{par:ars_mpc}}} & \pulrad{\B{\ref{par:aoid_mpa}}} & \pulrad{\B{Krok}} &
		\pulrad{\B{Zam}} & \mc{\pulrad{\B{pAg}}} & \mc{\pulrad{\B{pZp}}} & \mc{\pulrad{\B{Čas}}} \\
		\midrule
		S & - & 1 & 10 & 12 & 32776 & \B{363}   & \multicolumn{1}{B{.}{,}{2.2}}{15.07} & 5.44                                & \multicolumn{1}{B{.}{,}{8.2}}{9893.10}    \\
		S & - & 2 & 14 & 12 & 16634 & 32273     & 7.56                                 & \multicolumn{1}{B{.}{,}{1.2}}{4.58} & 432905.36                                 \\
		\hline
		O & - & 1 & 10 & 8  & 9289  & 47575     & \multicolumn{1}{B{.}{,}{2.2}}{15.16} & 8.74                                & 297656.39                                 \\
		O & s & 1 & 10 & 8  & 20821 & \B{25521} & \multicolumn{1}{B{.}{,}{2.2}}{15.16} & 8.57                                & 345728.00                                 \\
		O & s & 2 & 9  & 9  & 2740  & 59957     & 1.98                                 & 5.15                                & \multicolumn{1}{B{.}{,}{8.2}}{2632600.55} \\
		O & s & 2 & 9  & 10 & 99    & 65125     & 0.07                                 & \multicolumn{1}{B{.}{,}{1.2}}{3.07} & 30123061.30                               \\
		\hline
		H & - & 1 & 10 & 8  & 32788 & \B{1905}  & \multicolumn{1}{B{.}{,}{2.2}}{26.06} & 8.85                                & \multicolumn{1}{B{.}{,}{8.2}}{34920.59}   \\
		H & s & 2 & 12 & 8  & 11448 & 64109     & 9.30                                 & \multicolumn{1}{B{.}{,}{2.2}}{1.22} & 1240239.54                                \\
		\bottomrule
%		\multicolumn{6}{l}{\footnotesize \textit{Pozn:}
%		\textrm{Zam} - počet zamítnutí, \textrm{pAgen} - průměrný počet agentů v jeden krok na křižovatce, \\
%		\textrm{sAgen} - směrodatná odchylka počtu agentů na křižovatce, \\
%		\textrm{Zpož} - součet spoždění přes všechny agenty, \textrm{pZpož} - průměrné zpoždění agentů
%		}  TODO
	\end{tabular}
	\caption{Porovnání vlivu parametrů u \nameref{subsec:sat_ra} na různých typech malé křižovatky.}\label{tab:sata_exp_mala}
%	\end{adjustwidth}
\end{table}
Výsledky jsou zapsané v tabulce \ref{tab:sata_exp_mala}.
Nejpřekvapivější jsou výsledky pro oktagonální křižovatku.
Nejprve mi přišlo zvláštní, že algoritmus, kterému vyšel nejnižší průměrný čas plánování nemá nejméně vypočtených kroků.
Podle mého je to způsobeno tím, že se \ref{str:sat} řešič na některém kroku zasekne a nepodaří se mu rozhodnout
určitý krok v čase, než uplynou dvě hodiny od počátku simulace.
