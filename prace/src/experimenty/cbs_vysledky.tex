\subsection{\ref{str:cbs} porovnání parametrů}\label{subsec:cbs_porovnani_parametru}

V této kapitole porovnám vliv parametrů u algoritmů \ref{str:cbs} a \nameref{subsec:cbsoid}.

Tyto algoritmy rozšiřují \ref{str:a_star_ars},
čili budu používat všechny parametry z \ref{str:a_star_ars} se stejnými hodnotami.
Parametry od algoritmu \ref{str:a_star_ars} jsou \hyperref[par:ars_mnv]{maximum návštěv vrcholu},
\hyperref[par:ars_pz]{povolené zastavování}, \hyperref[par:ars_mpc]{maximální prodleva při cestě} a
\hyperref[par:ars_pv]{povolené vracení}.
Zároveň obsahuje \ref{str:cbs} parametr \ref{par:arsg_zvp} určující,
po jak dlouhé době má výpočet přejít na zjednodušený režim.
Tento parametr bude vždy nastaven na jednu sekundu.

\subsubsection{\ref{str:cbs} na \hyperref[par:data_mala]{malé} křižovatce}
\label{subsubsec:exp_cbssg_mala_krizovatka}

\ref{str:cbs} algoritmus se choval poměrně předvídatelně, jak je možné vidět v tabulce \ref{tab:cbssg_exp_mala}.
Zároveň všechny experimenty úspěšně stihly doběhnout.
\begin{table}[b!]
	\centering
%	\begin{adjustwidth}{-1.5cm}{}
	\begin{tabular}{c c c c | r r D{.}{,}{2.2} D{.}{,}{2.2} D{.}{,}{5.2}}
		\toprule \\
		\pulrad{\B{Typ}} & \pulrad{\B{Omez}} & \pulrad{\B{\ref{par:ars_mnv}}} &
		\pulrad{\B{\ref{par:ars_mpc}}} & \pulrad{\B{Krok}}  & \pulrad{\B{Zam}} &
		\mc{\pulrad{\B{pAg}}} & \mc{\pulrad{\B{pZp}}} & \mc{\pulrad{\B{Čas}}} \\
		\midrule
		S & -  & 1 & inf & 32779 & 915      & 14.14                                & 6.00                                 & \multicolumn{1}{B{.}{,}{5.2}}{132.36}  \\
		S & s  & 2 & inf & 32777 & 82       & \multicolumn{1}{B{.}{,}{2.2}}{13.41} & \multicolumn{1}{B{.}{,}{2.2}}{3.62}  & 169.34   \\
		S & sr & 2 & inf & 32776 & \B{70}   & \multicolumn{1}{B{.}{,}{2.2}}{13.41} & 3.66                                 & 194.12                                 \\
		\hline
		O & -  & 1 & 16  & 32779 & 2264     & 13.40                                & 7.57                                 & 6902.67                                \\
		O & s  & 2 & inf & 32781 & 1490     & \multicolumn{1}{B{.}{,}{2.2}}{13.96} & 7.39                                 & 2595.20                                \\
		O & sr & 2 & inf & 32780 & \B{1079} & 13.91                                & \multicolumn{1}{B{.}{,}{2.2}}{6.46}  & \multicolumn{1}{B{.}{,}{5.2}}{1761.27} \\
		\hline
		H & -  & 1 & 24  & 32792 & 5846     & 25.48                                & 13.71                                & 24000.77                               \\
		H & s  & 2 & inf & 32793 & 2878     & \multicolumn{1}{B{.}{,}{2.2}}{26.17} & 11.15                                & 3762.95                                \\
		H & sr & 2 & inf & 32790 & \B{2698} & 25.74                                & \multicolumn{1}{B{.}{,}{2.2}}{10.80} & \multicolumn{1}{B{.}{,}{5.2}}{3128.19} \\
		\bottomrule
%		\multicolumn{6}{l}{\footnotesize \textit{Pozn:}
%		\textrm{Zam} - počet zamítnutí, \textrm{pAgen} - průměrný počet agentů v jeden krok na křižovatce, \\
%		\textrm{sAgen} - směrodatná odchylka počtu agentů na křižovatce, \\
%		\textrm{Zpož} - součet spoždění přes všechny agenty, \textrm{pZpož} - průměrné zpoždění agentů
%		}  TODO
	\end{tabular}
	\caption{Porovnání vlivu parametrů u \ref{str:cbs} na různých typech malé křižovatky.}\label{tab:cbssg_exp_mala}
%	\end{adjustwidth}
\end{table}


Na čtvercové křižovatce se se snižujícím omezením pohybu agentů snižoval počet zamítnutých agentů,
avšak za cenu rostoucí doby výpočtu.
Varianta s povoleným zastavování ale bez vracení měla mírně vyšší počet zamítnutí, o~12 agentů ($~17,14\%$).
Průměrné zpoždění měl dokonce o~kousek nižší, přesněji o~$~1,09\%$.
Čas výpočtu byl přibližně o $12,77\%$ nižší.

Na oktagonální i hexagonální křižovatce vyšla jednoznačně nejlepší nejméně omezená křižovatka.
Oproti předchozímu případu se se snižujícím omezením výrazně snižuje nejen počet zamítnutých agentů,
ale i průměrné zpoždění a čas plánování.
Dle mého názoru to je způsobeno vysokým počtem možností, kde se cesty agentů můžou křížit.
Pokud tedy některý agent nelze naplánovat, algoritmus musí vyzkoušet všechny možnosti
a pokaždé tvořit dva podpřípady, kopírovat do každého tabulky kolizí, \dots,
a nakonec zatřídit nové vrcholy do prioritní fronty.
Jelikož poslední varianta umožňuje nejvíce tras pro agenta, z počtu zamítnutých agentů usuzuji,
že v každém kroku dokáže úspěšně naplánovat více agentů, a proto je také nejrychlejší.

\subsubsection{\ref{str:cbs} na \hyperref[par:data_velka]{velké} křižovatce bez výjezdů}
\label{subsubsec:exp_cbssg_velka_krizovatka_bez_vyjezdu}

Na velké křižovatce úspěšně doběhl algoritmus pouze na čtvercovém typu.
Pro čtvercovou křižovatku měl nejmenší počet zamítnutí prostřední běh.
Avšak nejmenší zpoždění měla třetí, nejméně omezená varianta.
Opět zde platí, že se snižujícím omezením pohybu agentů se snižuje průměrné zpoždění
a roste zaplnění křižovatky, ale za cenu rostoucího času výpočtu.
Rozdíl v počtu zamítnutých agentů mezi variantami s povoleným zastavováním byl malý, přibližně $6,93\%$.
Dle mého názoru není žádný důvod, proč běh bez povoleného vracení měl počet agentů nižší,
myslím si, že to je způsobeno náhodným generováním agentů, které více \uv{sedlo} této variantě.

Na oktagonálním typu je vidět podobný trend jako na malé křižovatce.
Se snižujícím omezením agentů se snižuje plánovací čas.
Z toho důvodu se snižuje i počet zamítnutých agentů.
Zároveň je ale nižší i průměrný počet zpoždění agentů.
Překvapivé pro mě je, že přechod ze čtvercové křižovatky na oktagonální zvýší průměrný čas plánování alespoň desetkrát.
Zvýšení počtu vrcholů křižovatky při tomto přechodu je přibližně $77\%$.



\begin{table}[b!]
	\centering
%	\begin{adjustwidth}{-1.5cm}{}
	\begin{tabular}{c c c c | r r D{.}{,}{3.2} D{.}{,}{2.2} D{.}{,}{7.2}}
		\toprule \\
		\pulrad{\B{Typ}} & \pulrad{\B{Omez}} & \pulrad{\B{\ref{par:ars_mnv}}} &
		\pulrad{\B{\ref{par:ars_mpc}}} & \pulrad{\B{Krok}}  & \pulrad{\B{Zam}} &
		\mc{\pulrad{\B{pAg}}} & \mc{\pulrad{\B{pZp}}} & \mc{\pulrad{\B{Čas}}} \\
		\midrule
		S & -  & 1 & 32  & 32845 & 39340      & 132.53                                & 37.71                                & \multicolumn{1}{B{.}{,}{7.2}}{29068.97}   \\
		S & s  & 2 & inf & 32848 & \B{13193}  & 137.94                                & 31.45                                & 69699.31                                  \\
		S & sr & 2 & inf & 32847 & 14176      & \multicolumn{1}{B{.}{,}{3.2}}{138.72} & \multicolumn{1}{B{.}{,}{2.2}}{26.17} & 73197.52   \\
		\hline
		O & -  & 1 & 32  & 5064  & 224520     & \multicolumn{1}{B{.}{,}{3.2}}{148.76} & 34.61                                & 1428883.80                                \\
		O & s  & 2 & inf & 7751  & 202167     & 115.16                                & 27.90                                & 930942.60                                 \\
		O & sr & 2 & inf & 10221 & \B{184171} & 146.21                                & \multicolumn{1}{B{.}{,}{2.2}}{26.42} & \multicolumn{1}{B{.}{,}{7.2}}{705203.89}  \\
		\hline
		H & -  & 1 & 64  & 2710  & 363556     & 280.49                                & 56.42                                & 2690812.00                                \\
		H & s  & 2 & inf & 4011  & \B{347852} & \multicolumn{1}{B{.}{,}{3.2}}{292.50} & 59.07                                & \multicolumn{1}{B{.}{,}{7.2}}{1810300.99} \\
		H & sr & 2 & inf & 3607  & 352717     & 282.48                                & \multicolumn{1}{B{.}{,}{2.2}}{48.71} & 2015672.53                                \\
		\bottomrule
%		\multicolumn{6}{l}{\footnotesize \textit{Pozn:}
%		\textrm{Zam} - počet zamítnutí, \textrm{pAgen} - průměrný počet agentů v jeden krok na křižovatce, \\
%		\textrm{sAgen} - směrodatná odchylka počtu agentů na křižovatce, \\
%		\textrm{Zpož} - součet spoždění přes všechny agenty, \textrm{pZpož} - průměrné zpoždění agentů
%		}  TODO
	\end{tabular}
	\caption{Porovnání vlivu parametrů u \ref{str:cbs} na různých typech velké křižovatky bez výjezdů.}\label{tab:cbs_exp_velka_bez_vyjezdu}
%	\end{adjustwidth}
\end{table}

\subsubsection{\ref{str:cbs} na \hyperref[par:data_velka]{velké} křižovatce s výjezdem}
\label{subsubsec:exp_cbssg_velka_krizovatka_s_vyjezdem}

\subsubsection{\nameref{subsec:cbsoid} na \hyperref[par:data_mala]{malé} křižovatce}
\label{subsubsec:exp_cbsoid_mala_krizovatka}
\begin{table}[h]
	\centering
%	\begin{adjustwidth}{-1cm}{}
	\begin{tabular}{c c c c c | r r D{.}{,}{2.2} D{.}{,}{2.2} D{.}{,}{7.2}}
		\toprule \\
		\pulrad{\B{Typ}} & \pulrad{\B{Omez}} & \pulrad{\B{\ref{par:ars_mnv}}} &
		\pulrad{\B{\ref{par:ars_mpc}}} & \pulrad{\B{\ref{par:aoid_mpa}}} & \pulrad{\B{Krok}} &
		\pulrad{\B{Zam}} & \mc{\pulrad{\B{pAg}}} & \mc{\pulrad{\B{pZp}}} & \mc{\pulrad{\B{Čas}}} \\
		\midrule
		S & -  & 1 & 16  & 16  & 6838 & \B{51759} & \multicolumn{1}{B{.}{,}{2.2}}{12.61} & \multicolumn{1}{B{.}{,}{2.2}}{8.45}  & \multicolumn{1}{B{.}{,}{7.2}}{1054679.77} \\
		S & s  & 2 & inf & 24  & 3755 & 57904     & 7.06                                 & 10.50                                & 1920988.21                                \\ % TODO
		S & sr & 2 & inf & 24  & 1530 & 62295     & 2.88                                 & 10.77                                & 2044461.87                                \\ % TODO 24?
		\hline
		O & -  & 1 & 16  & inf & 2466 & 60424     & \multicolumn{1}{B{.}{,}{2.2}}{12.11} & \multicolumn{1}{B{.}{,}{2.2}}{8.69}  & 2699251.20                                \\
		O & s  & 2 & inf & 16  & 2889 & \B{59641} & 11.99                                & 10.23                                & \multicolumn{1}{B{.}{,}{7.2}}{2497072.06} \\  % TODO 16?
		O & sr & 2 & inf & 16  & 855  & 63627     & 4.16                                 & 9.45                                 & 2675054.74                                \\  % TODO 16?
		\hline
		H & -  & 1 & 24  & 24  & 1227 & 94802     & 21.09                                & 19.49                                & 5906285.18                                \\
		H & s  & 2 & inf & 24  & 1704 & \B{93403} & \multicolumn{1}{B{.}{,}{2.2}}{22.16} & 21.18                                & \multicolumn{1}{B{.}{,}{7.2}}{4255409.05} \\  % TODO 24?
		H & sr & 2 & inf & 24  & 1229 & 94792     & 21.25                                & \multicolumn{1}{B{.}{,}{2.2}}{18.52} & 5893074.13                                \\  % TODO 24?
		\bottomrule
%		\multicolumn{6}{l}{\footnotesize \textit{Pozn:}
%		\textrm{Zam} - počet zamítnutí, \textrm{pAgen} - průměrný počet agentů v jeden krok na křižovatce, \\
%		\textrm{sAgen} - směrodatná odchylka počtu agentů na křižovatce, \\
%		\textrm{Zpož} - součet spoždění přes všechny agenty, \textrm{pZpož} - průměrné zpoždění agentů
%		}  TODO
	\end{tabular}
	\caption{Porovnání vlivu parametrů u \nameref{subsec:cbsoid} na různých typech malé křižovatky.}\label{tab:cbsoid_exp_mala}
%	\end{adjustwidth}
\end{table}

\subsubsection{\nameref{subsec:cbsoid} na \hyperref[par:data_velka]{velké} křižovatce bez výjezdů}
\label{subsubsec:exp_cbsoid_velka_krizovatka_bez_vyjezdu}
\begin{table}[b!]
%	\centering
	\begin{adjustwidth}{-1.5cm}{}
		\begin{tabular}{c c c c c | r r D{.}{,}{3.2} D{.}{,}{2.2} D{.}{,}{8.2}}
			\toprule \\
			\pulrad{\B{Typ}} & \pulrad{\B{Omez}} & \pulrad{\B{\ref{str:ars_mnv}}} &
			\pulrad{\B{\ref{str:ars_mpc}}} & \pulrad{\B{\ref{str:aoid_mpa}}} & \pulrad{\B{Krok}} &
			\pulrad{\B{Zam}} & \mc{\pulrad{\B{pAg}}} & \mc{\pulrad{\B{pZp}}} & \mc{\pulrad{\B{Čas}}} \\
			\midrule
			S & -  & 1 & 32  & 32 & 965 & \B{255430} & \multicolumn{1}{B{.}{,}{3.2}}{121.15} & 45.84                                & \multicolumn{1}{B{.}{,}{8.2}}{7631472.87}  \\
			S & s  & 2 & inf & 64 & 396 & 259643     & 45.16                                 & \multicolumn{1}{B{.}{,}{2.2}}{35.35} & 19118433.27                                \\
			\hline
			O & -  & 2 & inf & 32 & 527 & 258775     & 111.77                                & 51.48                                & 14274680.74                                \\
			O & sr & 2 & inf & 32 & 638 & \B{257862} & \multicolumn{1}{B{.}{,}{3.2}}{118.79} & \multicolumn{1}{B{.}{,}{2.2}}{47.18} & \multicolumn{1}{B{.}{,}{8.2}}{11697421.69} \\
			\bottomrule
%		\multicolumn{6}{l}{\footnotesize \textit{Pozn:}
%		\textrm{Zam} - počet zamítnutí, \textrm{pAgen} - průměrný počet agentů v jeden krok na křižovatce, \\
%		\textrm{sAgen} - směrodatná odchylka počtu agentů na křižovatce, \\
%		\textrm{Zpož} - součet spoždění přes všechny agenty, \textrm{pZpož} - průměrné zpoždění agentů
%		}  TODO
		\end{tabular}
		\caption{Porovnání vlivu parametrů u \nameref{subsec:cbsoid} na různých typech velké křižovatky bez výjezdů.}\label{tab:cbsoid_exp_velka_bez_vyjezdu}
	\end{adjustwidth}
\end{table}

\begin{table}[b!]
%	\centering
	\begin{adjustwidth}{-1.5cm}{}
		\begin{tabular}{c c c c c | r r D{.}{,}{3.2} D{.}{,}{2.2} D{.}{,}{8.2}}
			\toprule \\
			\pulrad{\B{Typ}} & \pulrad{\B{Omez}} & \pulrad{\B{\ref{str:ars_mnv}}} &
			\pulrad{\B{\ref{str:ars_mpc}}} & \pulrad{\B{\ref{str:aoid_mpa}}} & \pulrad{\B{Krok}} &
			\pulrad{\B{Zam}} & \mc{\pulrad{\B{pAg}}} & \mc{\pulrad{\B{pZp}}} & \mc{\pulrad{\B{Čas}}} \\
			\midrule
			S & - & 1 & 64  & 24 & 957 & \B{255758} & \multicolumn{1}{B{.}{,}{3.2}}{136.49} & 52.40                                & \multicolumn{1}{B{.}{,}{8.2}}{7736125.71} \\
			S & s & 2 & inf & 32 & 553 & 258602     & 72.57                                 & \multicolumn{1}{B{.}{,}{2.2}}{51.77} & 13573333.56                               \\
			\bottomrule
%		\multicolumn{6}{l}{\footnotesize \textit{Pozn:}
%		\textrm{Zam} - počet zamítnutí, \textrm{pAgen} - průměrný počet agentů v jeden krok na křižovatce, \\
%		\textrm{sAgen} - směrodatná odchylka počtu agentů na křižovatce, \\
%		\textrm{Zpož} - součet spoždění přes všechny agenty, \textrm{pZpož} - průměrné zpoždění agentů
%		}  TODO
		\end{tabular}
		\caption{Porovnání vlivu parametrů u \nameref{subsec:cbsoid} na různých typech velké křižovatky s výjezdy.}\label{tab:cbsoid_exp_velka_s_vyjezdy}
	\end{adjustwidth}
\end{table}


\subsubsection{\nameref{subsec:cbsoid} na \hyperref[par:data_velka]{velké} křižovatce s výjezdem}
\label{subsubsec:exp_cbsoid_velka_krizovatka_s_vyjezdem}

