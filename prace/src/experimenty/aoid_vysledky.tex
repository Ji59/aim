\subsubsection{\nameref{subsubsec:a_star_aoid} na \hyperref[par:data_mala]{malé} křižovatce}
\label{subsubsec:exp_aoid_mala_krizovatka}

Tento algoritmus rozšiřuje \ref{str:varsg}, proto jsem se snažil použít stejné nastavení parametrů.
Bohužel to nebylo vždy možné, jelikož přidaní agenti značně zvýšili časovou náročnost.

Algoritmus má několik parametrů navíc oproti \ref{str:varsg}.
Těmi jsou \ref{str:aoid_mpa} udávající maximální počet plánovaných agentů
a \ref{str:aoid_ppk}, který značí, kolik kroků může být agent na cestě, aby byl zvážen pro přeplánování.
\ref{str:aoid_ppk} bude vždy nastaveno na $8$ kroků,
\ref{str:aoid_mpa} je proměnlivé, vybraná hodnota je vidět v tabulce.

Algoritmus opět přejde na zjednodušený výpočet po jedné sekundě.

V tabulce (Tabulka \ref{tab:aoid_exp_mala}) jsou vidět výsledky na všech typech křižovatky
velikosti 4, jedním vjezdem a výjezdem.

Na čtvercové křižovatce všechny varianty doběhly,
avšak rozdíl v době plánování je mezi nimi mnohem větší než v předchozích případech.
Delší časy mohly způsobit častější přechod na zjednodušený režim, což může vést k horším výsledkům.
Zároveň je vidět zvyšování zamítnutí a průměrného zpoždění se zvyšujícím se zaplněním křižovatky.
Opět by to nasvědčovalo možnosti, kdy vyšší omezení naplánuje obecně lepší trasu o krok později.

Stejné nastavení parametrů nebylo úspěšné na oktagonální křižovatce, ani jedno nastavení nedoběhlo do konce.
Nárůst složitosti může být způsoben vyšším počtem vrcholů grafu křižovatky
nebo faktem, že přidané vrcholy jsou mnohem blíže sebe
a agenti si tak navzájem omezují pohyb více než na čtvercovém typu.

Na hexagonálním typu křižovatky jsem zkusil stejné nastavení parametrů pro nejvíce omezenou variantu.
Algoritmus spočítal mnohem méně kroků než na oktagonálním typu, což by nasvědčovalo hypotéze,
že za přidanou složitost může vyšší počet vrcholů křižovatky.
U zbylých dvou experimentů jsem na této křižovatce snížil maximální počet plánovaných agentů.
Je vidět, že snížení tohoto počtu z $16$ na $14$ zaručí, že algoritmus bez problémů doběhne do konce.
Zároveň počet zamítnutých agentů a průměrná prodleva se příliš nemění pro počet plánovaných agentů $12$ a $14$.
Mohlo by to být způsobeno jiným nastavením nejvyšší prodlevy cesty,
avšak dle mého názoru povolení přeplánování více agentům nemá na výsledky tak velký vliv.

\begin{table}[h]
%	\centering
	\begin{adjustwidth}{-1cm}{}
		\begin{tabular}{c c c c c | r r D{.}{,}{2.2} D{.}{,}{2.2} D{.}{,}{7.2}}
			\toprule \\
			\pulrad{\B{Typ}} & \pulrad{\B{Omez}} & \pulrad{\B{\ref{str:ars_mnv}}} &
			\pulrad{\B{\ref{str:ars_mpc}}} & \pulrad{\B{\ref{str:aoid_mpa}}} & \pulrad{\B{Krok}} &
			\pulrad{\B{Zam}} & \mc{\pulrad{\B{pAg}}} & \mc{\pulrad{\B{pZp}}} & \mc{\pulrad{\B{Čas}}} \\
			\midrule
%		1 & 0 & \B{701} & \multicolumn{1}{B{.}{,}{2.2}}{11.85} & \multicolumn{1}{B{.}{,}{1.2}}{2.06}
%		& \B{267\,141} & \multicolumn{1}{B{.}{,}{1.2}}{4.13} \\
			S & -  & 1 & 8   & 16 & 32786 & \B{1373}  & 18.04                                & \multicolumn{1}{B{.}{,}{2.2}}{15.10} & \multicolumn{1}{B{.}{,}{7.2}}{5582.03}   \\
			S & s  & 2 & inf & 16 & 32792 & 3382      & 19.95                                & 17.71                                & 34867.77                                 \\
			S & sr & 2 & inf & 16 & 32790 & 4586      & \multicolumn{1}{B{.}{,}{2.2}}{20.52} & 19.02                                & 120604.89                                \\
			\hline
			O & -  & 1 & 8   & 16 & 16976 & \B{32774} & \multicolumn{1}{B{.}{,}{2.2}}{18.49} & \multicolumn{1}{B{.}{,}{2.2}}{17.57} & \multicolumn{1}{B{.}{,}{7.2}}{424199.33} \\
			O & s  & 2 & inf & 16 & 1443  & 62691     & 1.68                                 & 18.28                                & 5038828.24                               \\
			\hline
			H & -  & 1 & 12  & 16 & 3955  & 87635     & 32.77                                & 25.07                                & 1956276.57                               \\
			H & s  & 2 & 16  & 12 & 32800 & \B{9472}  & 32.67                                & \multicolumn{1}{B{.}{,}{2.2}}{21.81} & \multicolumn{1}{B{.}{,}{7.2}}{10276.77}  \\
			H & s  & 2 & inf & 14 & 32800 & 9482      & \multicolumn{1}{B{.}{,}{2.2}}{34.11} & 23.50                                & 42300.29                                 \\
			\bottomrule
%		\multicolumn{6}{l}{\footnotesize \textit{Pozn:}
%		\textrm{Zam} - počet zamítnutí, \textrm{pAgen} - průměrný počet agentů v jeden krok na křižovatce, \\
%		\textrm{sAgen} - směrodatná odchylka počtu agentů na křižovatce, \\
%		\textrm{Zpož} - součet spoždění přes všechny agenty, \textrm{pZpož} - průměrné zpoždění agentů
%		}  TODO
		\end{tabular}
		\caption{Porovnání vlivu parametrů u \nameref{subsubsec:a_star_aoid} na různých typech malé křižovatky.}\label{tab:aoid_exp_mala}
	\end{adjustwidth}
\end{table}

\subsubsection{\nameref{subsubsec:a_star_aoid} na \hyperref[par:data_velka]{velké} křižovatce bez výjezdů}
\label{subsubsec:exp_aoid_velka_krizovatka_bez_vyjezdu}

Tabulky \ref{tab:aoid_exp_velka_bez_vyjezdu} a \ref{tab:aoid_exp_velka_s_vyjezdy} obsahují pouze omezené výsledky,
jelikož zbylé varianty zaplnily paměť a selhaly.

Z toho usuzuji, že počet přeplánovaných agentů musí být velice malý pro velké křižovatky.
To činí tento algoritmus značně nepoužitelný.

\begin{table}[h]
%	\centering
	\begin{adjustwidth}{-1cm}{}
		\begin{tabular}{c c c c c | r r D{.}{,}{3.2} D{.}{,}{2.2} D{.}{,}{6.2}}
			\toprule \\
			\pulrad{\B{Typ}} & \pulrad{\B{Omez}} & \pulrad{\B{\ref{par:ars_mnv}}} &
			\pulrad{\B{\ref{par:ars_mpc}}} & \pulrad{\B{\ref{par:aoid_mpa}}} & \pulrad{\B{Krok}} &
			\pulrad{\B{Zam}} & \mc{\pulrad{\B{pAg}}} & \mc{\pulrad{\B{pZp}}} & \mc{\pulrad{\B{Čas}}} \\
			\midrule
			S & -  & 1 & 32  & 32 & 32840 & \B{11285} & \multicolumn{1}{B{.}{,}{3.2}}{159.15} & 35.46                                & \multicolumn{1}{B{.}{,}{6.2}}{204258.46} \\
			S & s  & 2 & inf & 48 & 10320 & 182402    & 51.39                                 & 21.06                                & 557650.01                                \\
			S & sr & 2 & inf & 64 & 674   & 257255    & 3.25                                  & \multicolumn{1}{B{.}{,}{2.2}}{14.67} & 701142.19                                \\
%			\hline
%			\hline
			\bottomrule
%		\multicolumn{6}{l}{\footnotesize \textit{Pozn:}
%		\textrm{Zam} - počet zamítnutí, \textrm{pAgen} - průměrný počet agentů v jeden krok na křižovatce, \\
%		\textrm{sAgen} - směrodatná odchylka počtu agentů na křižovatce, \\
%		\textrm{Zpož} - součet spoždění přes všechny agenty, \textrm{pZpož} - průměrné zpoždění agentů
%		}  TODO
		\end{tabular}
		\caption{Porovnání vlivu parametrů u \nameref{subsubsec:a_star_aoid} na různých typech velké křižovatky.}\label{tab:aoid_exp_velka_bez_vyjezdu}
	\end{adjustwidth}
\end{table}

\begin{table}[h]
%	\centering
	\begin{adjustwidth}{-1cm}{}
		\begin{tabular}{c c c c c | r r D{.}{,}{3.2} D{.}{,}{2.2} D{.}{,}{7.2}}
			\toprule \\
			\pulrad{\B{Typ}} & \pulrad{\B{Omez}} & \pulrad{\B{\ref{par:ars_mnv}}} &
			\pulrad{\B{\ref{par:ars_mpc}}} & \pulrad{\B{\ref{par:aoid_mpa}}} & \pulrad{\B{Krok}} &
			\pulrad{\B{Zam}} & \mc{\pulrad{\B{pAg}}} & \mc{\pulrad{\B{pZp}}} & \mc{\pulrad{\B{Čas}}} \\
			\midrule
			S & - & 1 & 32 & 22 & 32853 & \B{33043} & \multicolumn{1}{B{.}{,}{3.2}}{183.06} & 55.96                                & \multicolumn{1}{B{.}{,}{7.2}}{92660.76} \\
			O & - & 1 & 32 & 22 & 963   & 255040    & 5.23                                  & \multicolumn{1}{B{.}{,}{2.2}}{34.00} & 2625309.35                              \\
%			\hline
%			\hline
			\bottomrule
%		\multicolumn{6}{l}{\footnotesize \textit{Pozn:}
%		\textrm{Zam} - počet zamítnutí, \textrm{pAgen} - průměrný počet agentů v jeden krok na křižovatce, \\
%		\textrm{sAgen} - směrodatná odchylka počtu agentů na křižovatce, \\
%		\textrm{Zpož} - součet spoždění přes všechny agenty, \textrm{pZpož} - průměrné zpoždění agentů
%		}  TODO
		\end{tabular}
		\caption{Porovnání vlivu parametrů u \nameref{subsubsec:a_star_aoid} na různých typech velké křižovatky s výjezdy.}\label{tab:aoid_exp_velka_s_vyjezdy}
	\end{adjustwidth}
\end{table}

