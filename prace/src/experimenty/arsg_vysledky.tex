\subsubsection{\ref{str:a_star_arsg} na \hyperref[par:data_mala]{malé} křižovatce}
\label{subsubsec:exp_arsg_mala_krizovatka}

Jelikož se jedná o rozšíření \ref{str:a_star_ars}, použil jsem stejná nastavení parametrů.
Tedy pro čtvercový a oktagonální typ jsem nastavil \ref{par:ars_mpc} na hodnotu $8$, a pro hexagonální na $16$.
Při bězích s \ref{par:ars_mnv} nastaveno na $2$ bude \hyperref[par:ars_mpc]{prodleva cesty} opět neomezená.

Algoritmus má jediný parametr navíc, a to \nameref{par:arsg_zvp}, který udává,
po jak dlouhé prodlevě má výpočet přejít na zjednodušený režim.
U všech experimentů je tento parametr nastaven na jednu sekundu.

\begin{table}[h]
	\centering
%	\begin{adjustwidth}{-1.5cm}{}
	\begin{tabular}{c c c c | r r D{.}{,}{2.2} D{.}{,}{2.2} D{.}{,}{4.2}}
		\toprule \\
		\pulrad{\B{Typ}} & \pulrad{\B{Omez}} & \pulrad{\B{\ref{str:ars_mnv}}} &
		\pulrad{\B{\ref{str:ars_mpc}}} & \pulrad{\B{Krok}}  & \pulrad{\B{Zam}} &
		\mc{\pulrad{\B{pAg}}} & \mc{\pulrad{\B{pZp}}} & \mc{\pulrad{\B{Čas}}} \\
		\midrule
%		1 & 0 & \B{701} & \multicolumn{1}{B{.}{,}{2.2}}{11.85} & \multicolumn{1}{B{.}{,}{1.2}}{2.06}
%		& \B{267\,141} & \multicolumn{1}{B{.}{,}{1.2}}{4.13} \\
		S & -  & 1 & 8   & 32779 & \B{568}  & 14.20                                & \multicolumn{1}{B{.}{,}{2.2}}{5.72}  & 262.71                                 \\
		S & s  & 2 & inf & 32787 & 4130     & \multicolumn{1}{B{.}{,}{2.2}}{14.84} & 10.68                                & \multicolumn{1}{B{.}{,}{4.2}}{218.39}  \\
		S & sr & 2 & inf & 32780 & 2397     & 14.79                                & 9.08                                 & 258.56                                 \\
		\hline
		O & -  & 1 & 8   & 32785 & \B{6276} & 13.14                                & \multicolumn{1}{B{.}{,}{2.2}}{9.46}  & \multicolumn{1}{B{.}{,}{4.2}}{237.60}  \\
		O & s  & 2 & inf & 32789 & 12220    & 13.34                                & 13.42                                & 278.54                                 \\
		O & sr & 2 & inf & 32787 & 10357    & \multicolumn{1}{B{.}{,}{2.2}}{13.48} & 12.74                                & 284.36                                 \\
		\hline
		H & -  & 1 & 16  & 32792 & \B{5872} & 25.94                                & \multicolumn{1}{B{.}{,}{2.2}}{13.15} & \multicolumn{1}{B{.}{,}{4.2}}{1195.23} \\
		H & s  & 2 & inf & 32801 & 17624    & \multicolumn{1}{B{.}{,}{2.2}}{26.35} & 20.95                                & 1208.47                                \\
		H & sr & 2 & inf & 32798 & 15252    & 26.26                                & 19.84                                & 1484.21                                \\
		\bottomrule
%		\multicolumn{6}{l}{\footnotesize \textit{Pozn:}
%		\textrm{Zam} - počet zamítnutí, \textrm{pAgen} - průměrný počet agentů v jeden krok na křižovatce, \\
%		\textrm{sAgen} - směrodatná odchylka počtu agentů na křižovatce, \\
%		\textrm{Zpož} - součet spoždění přes všechny agenty, \textrm{pZpož} - průměrné zpoždění agentů
%		}  TODO
	\end{tabular}
	\caption{Porovnání vlivu parametrů u \ref{str:varsg} na různých typech malé křižovatky.}\label{tab:arsg_exp_mala}
%	\end{adjustwidth}
\end{table}

V tabulce (Tabulka \ref{tab:arsg_exp_mala}) jsou vidět výsledky na všech typech křižovatky
velikostí $4$ a jedním vjezdem a výjezdem.

Překvapivě si v této simulaci na všech křižovatkách vedla nejlépe nejvíce omezená varianta.
Měla výrazně menší počet zamítnutých agentů a průměrné zpoždění.
Avšak měla nejmenší počet agentů na křižovatce.
To napovídá možnosti, kdy méně omezené varianty dokáží naplánovat složitější a delší cesty,
zatímco omezenější běhy by tyto agenty ve stejném kroku zamítly a naplánovaly jim optimálnější trasy následující krok.

Běhy s povoleným zastavováním si vždy vedly hůře než běhy s povoleným vracením.
Avšak měly rychlejší plánování srovnatelné s nejvíce omezenou variantou.

\subsubsection{\ref{str:a_star_arsg} na \hyperref[par:data_velka]{velké} křižovatce bez výjezdů}
\label{subsubsec:exp_arsg_velka_krizovatka_bez_vyjezdu}

\ref{par:ars_mpc} je opět nastaveno pro čtvercový a oktagonální typ na hodnotu $32$,
a pro hexagonální na $64$, pokud je nastavený \ref{par:ars_mnv} na $1$.
Por běhy s \ref{par:ars_mnv} $2$ jsou délky cest opět neomezené.

\begin{table}[b!]
	\centering
%	\begin{adjustwidth}{-1.5cm}{}
	\begin{tabular}{c c c c | r r D{.}{,}{3.2} D{.}{,}{2.2} D{.}{,}{6.2}}
		\toprule \\
		\pulrad{\B{Typ}} & \pulrad{\B{Omez}} & \pulrad{\B{\ref{par:ars_mnv}}} &
		\pulrad{\B{\ref{par:ars_mpc}}} & \pulrad{\B{Krok}}  & \pulrad{\B{Zam}} &
		\mc{\pulrad{\B{pAg}}} & \mc{\pulrad{\B{pZp}}} & \mc{\pulrad{\B{Čas}}} \\
		\midrule
		S & -  & 1 & 32  & 32845 & 22791  & 143.20                                & \multicolumn{1}{B{.}{,}{2.2}}{30.23} & \multicolumn{1}{B{.}{,}{6.2}}{12972.80}  \\
		S & s  & 2 & inf & 32851 & 35971  & 139.62                                & 47.06                                & 16372.90                                 \\
		S & sr & 2 & inf & 32847 & \B{22451}  & \multicolumn{1}{B{.}{,}{3.2}}{145.58} & 43.62                                & 21190.45                                 \\
		\hline
		O & -  & 1 & 32  & 32840 & \B{12990}  & 142.63                                & \multicolumn{1}{B{.}{,}{2.2}}{19.84} & \multicolumn{1}{B{.}{,}{6.2}}{62133.51}  \\
		O & s  & 2 & inf & 32849 & 33990  & 143.50                                & 44.20                                & 80076.10                                 \\
		O & sr & 2 & inf & 32850 & 21928  & \multicolumn{1}{B{.}{,}{3.2}}{148.34} & 41.81                                & 68197.91                                 \\
		\hline
		H & -  & 1 & 64  & 11604 & 263886 & 298.78                                & \multicolumn{1}{B{.}{,}{2.2}}{49.01} & 621399.93                                \\
		H & s  & 2 & inf & 20359 & \B{198896} & \multicolumn{1}{B{.}{,}{3.2}}{299.14} & 88.14                                & \multicolumn{1}{B{.}{,}{6.2}}{353255.32} \\
		H & sr & 2 & inf & 18381 & 212523 & 298.47                                & 81.40                                & 391228.41                                \\
		\bottomrule
%		\multicolumn{6}{l}{\footnotesize \textit{Pozn:}
%		\textrm{Zam} - počet zamítnutí, \textrm{pAgen} - průměrný počet agentů v jeden krok na křižovatce, \\
%		\textrm{sAgen} - směrodatná odchylka počtu agentů na křižovatce, \\
%		\textrm{Zpož} - součet spoždění přes všechny agenty, \textrm{pZpož} - průměrné zpoždění agentů
%		}  TODO
	\end{tabular}
	\caption{Porovnání vlivu parametrů u \ref{str:a_star_arsg} na různých typech velké křižovatky.}\label{tab:arsg_exp_velka}
%	\end{adjustwidth}
\end{table}


V tabulce (Tabulka \ref{tab:arsg_exp_velka_bez_vyjezdu}) jsou vidět výsledky na všech typech křižovatky
velikostí $16$ se $4$ vjezdy a $4$ výjezdy.

V této tabulce mě většina hodnot překvapila, výsledky mezi jednotlivými typy křižovatek jsou navzájem nesrovnatelné.
Na čtvercovém typu křižovatky měl nejmenší počet zamítnutých agentů běh s nejméně omezenými agenty.
Zároveň měl nejvíce agentů na křižovatce.
Nejméně omezená varianta měla mírně víc zamítnutých agentů a mírně menší počet agentů na křižovatce.
Avšak dosáhla výrazně nižšího průměrného zpoždění.
Zároveň zvládla plánovat agenty nejrychleji.
Prostředního běh měl nejhorší výsledky až na dobu plánování.

U oktagonálního typu jednoznačně vyhrála první varianta,
dosáhla nejmenšího počtu zamítnutí a průměrného zpoždění.
Ačkoliv se plánovací časy oproti čtvercové křižovatce zpětinásobily, pořád byly prot tento typ nejmenší.
Nejméně omezený běh měl výrazně horší počet zamítnutých agentů, průměrné zpoždění a plánovací čas.
Avšak má vyšší počet agentů na křižovatce.
To by nasvědčovalo mé hypotéze, že se snížením omezení pohybu dokáže algoritmus na plánovat dříve nenaplánované agenty,
avšak za cenu delších cest a blokování budoucích agentů.
Dalším překvapením je výsledek prostředního běhu.
Ačkoliv si oproti čtvercovému typu pomohl, celkově vyšel nejhůře, včetně plánovací doby.
Je možné, že algoritmus často přechází na zjednodušený režim, ve kterém plánuje méně agentů, než by jinak zvládl.
Dále je možné, že opět dochází ke vzájemnému překážení agentů, jako u předchozí varianty.
Je totiž vidět, že i tento běh má vyšší zaplněnost křižovatky než běh první.

Na hexagonální křižovatce žádná varianta nedoběhla do konce, takže data tím mohou být zkreslená.
Výsledky jsou zde opačné oproti předchozímu případu.
Prostřední varianta zvládla vypočítat nejvíce kroků a nejspíše díky tomu dosáhnout nejmenšího počtu zamítnutí.
Zaplněnost křižovatky byla přibližně stejná, avšak průměrné zpoždění měl nejméně omezený běh.
To může být způsobeno výrazně menším počtem kroků, a tedy i menším počtem naplánovaných agentů.
Plány agentů ze začátku simulace jsou mnohem kratší,
jelikož na křižovatce nejsou žádní agenti, kteří by blokovali tyto trasy.
Intuitivně měl tento běh nejvyšší čas plánování, více než o třetinu vyšší než zbylé dva testy.

Nejsem si jistý, proč ta dříve nejrychlejší varianta byla tak pomalá.
Dle mého názoru existuje optimální omezení agentů, které se liší pro každý typ a velikost křižovatky.
Pokud agenta omezím více, rychleji zjistím, že pro něj neexistuje cesta.
Avšak zároveň můžou nastat situace, ve kterých agentovi chyběl jediný krok do cíle
a agent by byl úspěšně naplánován.


\subsubsection{\ref{str:a_star_arsg} na \hyperref[par:data_velka]{velké} křižovatce s výjezdem}
\label{subsubsec:exp_arsg_velka_krizovatka_s_vyjezdem}