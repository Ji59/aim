\chapter{Experimenty}\label{ch:experimenty}

%Podrobnější popis sledovaných dat.

Po křižovatce požadujeme, aby zaručovala bezpečnou cestu autům.
Nyní budu předpokládat, že agenti dávají přesnou informaci křižovatce a plně dodržují plán.
Za tohoto předpokladu nám algoritmy zaručuje nekolizní trasy.

Následujícím faktorem je zpoždění aut.
Budu sledovat \hyperref[par:zamitnuti]{počet zamítnutých agentů} a \hyperref[par:zdrzeni]{zdržení} naplánovaných agentů.
V tabulkách budu porovnávat celkový \hyperref[par:zamitnuti]{počet zamítnutých agentů} (Zam).
Dále budu porovnávat průměrné \hyperref[par:zdrzeni]{zdržení} na agenta (pZp) vypočítané ze všech agentů,
kteří byli úspěšně naplánovaní.

Další zajímavý faktor je obsazenost křižovatky,
který udává v kolika krocích ze všech byl některý agent na určitém bloku křižovatky.
Ve výsledcích budu počítat průměrný počet agentů na křižovatce (pAg) přes všechny kroky.

Důležitá je také doba běhu algoritmu.
Budu sledovat pro každý krok, kolik uběhlo času od spuštění plánování daného kroku.
Odtud vypočtu průměrný čas na jeden krok v~mikrosekundách (Čas).

Nejprve provedu testy pro porovnání parametrů jednotlivých agentů.
Poté srovnám nejlepší nastavení algoritmů proti sobě.
%V poslední sekci zavedu k agentům určité nepřesnosti, které můžou způsobit kolize.
%Poté budu pozorovat vliv těchto nepřesností na počet kolizí u jednotlivých algoritmů.


\section{Testovací data}\label{sec:testovaci_data}

%Popis experimentálních dat - velikost křižovatky, počet agentů a kroků, safe distance.

\subsection{Délka simulace}\label{subsec:delka_simulace}

Simulace bude přidávat agenty po $32768$ kroků.
Pokud bude plánování trvat příliš dlouhou dobu, bude simulace předčasně ukončena.
Maximální dobu běhu simulace jsem omezil na 2 hodiny.

\subsubsection{Křižovatka}

Algoritmy budu testovat na každém typu křižovatky o dvou velikostech.
\paragraph{Malá}\label{par:data_mala} křižovatka bude mít \hyperref[par:velikost_krizovatky]{velikost} rovnou~$4$.
Tato křižovatka bude mít pouze jeden \hyperref[par:vjezdy]{vjezd} a jeden \hyperref[par:vyjezdy]{výjezd}.
%\paragraph{Střední}\label{par:data_stredni} křižovatka bude mít \hyperref[par:velikost_krizovatky]{velikost}~$8$.
%\hyperref[par:vjezdy]{Počet vjezdů} a \hyperref[par:vyjezdy]{výjezdů} bude činit $3$.
\paragraph{Velká}\label{par:data_velka} křižovatka bude mít \hyperref[par:velikost_krizovatky]{velikost}~$16$.
Počet \hyperref[par:vjezdy]{vjezdů} a \hyperref[par:vyjezdy]{výjezdů} je zvýšen na $4$.

\subsubsection{Agenti}

Agenti budou nejdříve náhodně vygenerovaní zvlášť pro každou velikost.
Generování proběhne dvakrát pro každou velikost,
jednou pro hexagonální typ a jednou společně pro čtvercový a oktagonální typ.
Poté se ti samí agenti použijí k porovnání jednotlivých algoritmů, abych snížil vliv náhody na výsledky.
V každém kroku přibude mezi $0 - en$ agentů, kde $en$ je celkový počet vjezdů do křižovatky ze všech stran.
Přesný počet je náhodně vygenerován každý krok.
Popis generování agentů je blíže popsán v sekci \ref{subsec:generovani_agentu}.

Délka agentů bude $0.56$ \hyperref[par:velikost_bloku]{velikosti bloku} křižovatky
a šířka $0.35$ \hyperref[par:velikost_bloku]{velikosti bloku}.
Tyto hodnoty jsem zvolil, jelikož umožňují nekolizní pozice agentů na sousedních vrcholech u všech typů křižovatek.
Zároveň ale agenti nejsou příliš malí na to, aby jejich velikost nehrála žádnou roli.

Dále budu porovnávat případy, kdy agenti mají daný přesný výjezd, nebo pouze směr výjezdu.
Tyto případy má cenu porovnávat pouze u
%\hyperref[par:data_stredni]{střední} a
\hyperref[par:data_velka]{velké} křižovatky, jelikož obsahuje více výjezdů.



\section{Parametry algoritmů}\label{sec:parametry_algoritmu}

%Podrobnější výsledky pro každý algoritmus zvlášť, sledování vlivu parametrů algoritmů na výsledky.
%
%Vyhodnocení nejlepších parametrů.

V této kapitole budu porovnávat vliv parametrů algoritmů na běh algoritmů.
V obecnosti budu testovat algoritmy s hodně omezenými parametry a s málo omezujícím nastavením parametrů.

Algoritmus \nameref{sec:safe_lanes} nemá žádné nastavitelné parametry, proto jej v této části přeskočím.

%\begin{table}[h]
	\centering
%	\begin{adjustwidth}{-1.5cm}{}
	\begin{tabular}{c c c | r r D{.}{,}{3.2} D{.}{,}{2.2} D{.}{,}{4.2}}
		\toprule \\
		\pulrad{\B{Typ}} & \pulrad{\B{Vel}} & \pulrad{\B{Výj}} &
		\pulrad{\B{Krok}} & \pulrad{\B{Zam}} & \mc{\pulrad{\B{pAg}}} &
		\mc{\pulrad{\B{pZp}}} & \mc{\pulrad{\B{Čas}}} \\
		\midrule
		S & m & - & 32782 & \B{4056}   & \multicolumn{1}{B{.}{,}{3.2}}{11.19}  & \multicolumn{1}{B{.}{,}{2.2}}{6.89}  & 16.75   \\
		O & m & - & 32785 & 13488      & 9.34                                  & 10.36                                & \multicolumn{1}{B{.}{,}{4.2}}{11.30}   \\
		H & m & - & 32796 & 32273      & 14.15                                 & 18.18                                & 87.44                                  \\
		\hline
		S & v & e & 32850 & 111987     & 77.02                                 & 60.70                                & \multicolumn{1}{B{.}{,}{4.2}}{1244.95} \\
		S & v & n & 32849 & \B{75983}  & \multicolumn{1}{B{.}{,}{3.2}}{86.91}  & \multicolumn{1}{B{.}{,}{2.2}}{44.18} & 1449.67 \\
		\hline
		O & v & e & 32849 & 138867     & 61.62                                 & 60.30                                & \multicolumn{1}{B{.}{,}{4.2}}{937.51}  \\
		O & v & n & 32846 & \B{107830} & \multicolumn{1}{B{.}{,}{3.2}}{71.98}  & \multicolumn{1}{B{.}{,}{2.2}}{58.92} & 1914.34 \\
		\hline
		H & v & e & 32893 & 234142     & 99.14                                 & 92.83                                & \multicolumn{1}{B{.}{,}{4.2}}{2399.98} \\
		H & v & n & 32894 & \B{212628} & \multicolumn{1}{B{.}{,}{3.2}}{109.63} & \multicolumn{1}{B{.}{,}{2.2}}{92.37} & 4221.17 \\
		\bottomrule
		\multicolumn{8}{l}{\footnotesize
		\textrm{Typ} - Typ křižovatky (\textrm{S}~\nameref{subsec:ctvercovy_typ}, \textrm{O}~\nameref{subsec:oktagonalni_typ}, \textrm{H}~\nameref{subsec:hexagonalni_typ})
		} \\
		\multicolumn{8}{l}{\footnotesize
		\textrm{Vel} - velikost křižovatky (\textrm{m}~\nameref{par:data_mala}, \textrm{v}~\nameref{par:data_velka})
		} \\
		\multicolumn{8}{l}{\footnotesize
		\textrm{Výj} - Výjezdy u~velké křižovatky (\textrm{e}~jediný daný výjezd, \textrm{n}~libovolný výjezd)
		} \\
		\multicolumn{8}{l}{\footnotesize
		\textrm{Krok} - počet kroků simulace, \textrm{Zam} - počet zamítnutí
		} \\
		\multicolumn{8}{l}{\footnotesize
		\textrm{pAg} - průměrný počet agentů v jeden krok na křižovatce
		} \\
		\multicolumn{8}{l}{\footnotesize
		\textrm{pZp} - průměrné zpoždění agentů
		} \\
		\multicolumn{8}{l}{\footnotesize
		\textrm{Čas} - průměrný počet mikrosekund na plánování jednoho kroku
		}
	\end{tabular}
	\caption{Porovnání vlivu parametrů u \nameref{sec:safe_lanes} na různých typech křižovatkek.}\label{tab:safe_lanes_exp}
%	\end{adjustwidth}
\end{table}

\subsection{\nameref{sec:a_star} porovnání parametrů}\label{subsec:a_star_porovnani_parametru}

V této kapitole porovnám vliv parametrů u algoritmů \ref{str:a_star_ars}, \ref{str:varsg} a \nameref{subsubsec:a_star_aoid}.

Všechny společné parametry těchto algoritmů (\hyperref[par:ars_mnv]{maximum návštěv vrcholu},
\hyperref[par:ars_pz]{povolené zastavování},\hyperref[par:ars_mpc]{maximální prodleva při cestě}
i~\hyperref[par:ars_pv]{povolené vracení}) omezují prohledávací prostor při plánování.
Proto bych čekal s větším omezením kratší dobu plánování, avšak za cenu horších výsledků.

Pro všechny typy křižovatky vyzkouším omezit \hyperref[par:ars_mnv]{maximum návštěv vrcholu (\ref{str:ars_mnv})}
na $1$ či $2$.
Pokud bude hodnota \ref{str:ars_mnv} nastavena na $1$, omezím
\hyperref[par:ars_mpc]{maximální prodlevu cesty (\ref{str:ars_mpc})} podle typu a velikosti křižovatky.
Pro čtvercovou a oktagonální na hodnotu $8$, a pro hexagonální většinou na $16$.
Zároveň nedovolím agentům \hyperref[par:ars_pz]{zastavování} (\ref{str:ars_pz})
ani \hyperref[par:ars_pv]{vracení} (\ref{str:ars_pv}).

Při nastavení \ref{str:ars_mnv} na $2$, \hyperref[par:ars_mpc]{maximální prodlevu cesty}
neomezím většinou vůbec.
Vyzkouším možnosti, kdy dovolím agentům pouze \hyperref[par:ars_pz]{zastavování},
nebo \hyperref[par:ars_pz]{zastavování} a zároveň \hyperref[par:ars_pv]{vracení} (\ref{str:ars_pv}).
V tabulkách s výsledky bude toto nastavení zobrazeno ve sloupci \textrm{Omez}.
Pokud bude povolené zastavování, bude sloupec obsahovat hodnotu~$s$.
Pokud dovolím agentům vracení, bude ve~sloupci napsáno~$r$.

\subsubsection{\ref{str:a_star_ars} na \hyperref[par:data_mala]{malé} křižovatce}
\label{subsubsec:exp_ars_mala_krizovatka}

Pokud bude nastaven \ref{str:ars_mnv} na $1$, omezím \ref{str:ars_mpc}
pro čtvercový a oktagonální typ na hodnotu $8$, a pro hexagonální na $16$.
Pro běhy s \ref{str:ars_mnv} $2$ bude \hyperref[par:ars_mpc]{prodleva cesty} neomezená, značená hodnotou $inf$.
Jelikož má křižovatka $16$ vrcholů kromě vjezdů a výjezdů, není rozdíl mezi neomezenými cestami a
cestami omezenými na $34$ kroků pro výpočet s \ref{str:ars_mnv} nastavené na $2$.

V tabulce (Tabulka \ref{tab:ars_exp_mala}) jsou vidět výsledky na všech typech křižovatky
velikosti 4, jedním vjezdem a výjezdem.


Z výsledků je znatelné, že více omezený prohledávací prostor vede ke značně horším výsledkům,
avšak průměrný čas plánování jednoho kroku je nejnižší.

Dovolení či zakázání vracení má různý vliv na výsledky u různých typů křižovatek.
U čtvercového typu byly rozdíly mezi variantami minimální.
U oktagonálního si vedla lépe varianta s povoleným vracením, avšak u hexagonálního typu si vedla hůře.

Překvapilo mě, že oktagonální typ měl mnohem více zamítnutí než čtvercový typ.
Dle mého názoru je tento jev způsoben menším manipulativním prostorem pro auta.
Oktagonální typ oproti čtvercovému nemá čtyři rohové vrcholy.
Zároveň pokud stojí auto na čtvercovém vrcholu reprezentujícím diagonální přejezd mezi dvěma oktagonálními vrcholy,
blokuje jiné přejezdy na těchto sousedních vrcholech. % TODO obrázek
Toto je dle mého názoru i důvod, proč si varianta s vracením vedla nejlépe na této křižovatce.
Umožňuje agentovi větší pohyblivost, tudíž i více možností vyhnout se jiným agentům.

Povolené vracení vedlo k nejnižšímu zpoždění ve všech třech typech křižovatky,
ačkoliv počet zamítnutí nebyl vždy nejmenší.
Mohlo by to být způsobeno tím, že zpoždění je určeno pouze nezamítnutými agenty.
Jelikož tato varianta mimo oktagonální typ křižovatky naplánovala méně agentů,
dochází zde k menšímu ovlivňování cesty jinými agenty.
Zároveň ale tato varianta rozšiřuje množinu možných cest a tyto nové cesty jsou asi kratší,
než nalezené cesty bez vracení.

Dalším zajímavým rozdílem mezi čtvercovou a oktagonální křižovatkou je rozdíl v průměrném počtu agentů na křižovatce.
Při nejvíce omezených cestách přechod z čtvercové na oktagonální křižovatku tento průměr snížil.
Ve zbylých dvou případech průměr vzrostl, ačkoliv celkový počet cestujících agentů klesl.
Z~toho lze usoudit, že na čtvercové křižovatce byly naplánované trasy značně kratší.

\begin{table}[h]
	\centering
%	\begin{adjustwidth}{-1.5cm}{}
	\begin{tabular}{c c c c | r r D{.}{,}{2.2} D{.}{,}{2.2} D{.}{,}{3.2}}
		\toprule \\
		\pulrad{\B{Typ}} & \pulrad{\B{Omez}} & \pulrad{\B{\ref{par:ars_mnv}}} &
		\pulrad{\B{\ref{par:ars_mpc}}} & \pulrad{\B{Krok}}  & \pulrad{\B{Zam}} &
		\mc{\pulrad{\B{pAg}}} & \mc{\pulrad{\B{pZp}}} & \mc{\pulrad{\B{Čas}}} \\
		\midrule
		S & -  & 1 & 8   & 32779 & 785      & \multicolumn{1}{B{.}{,}{2.2}}{14.21} & 5.95                                 & \multicolumn{1}{B{.}{,}{3.2}}{43.19}  \\
		S & s  & 2 & inf & 32775 & \B{76}   & 13.68                                & 3.92                                 & 47.77                                 \\
		S & sr & 2 & inf & 32775 & 78       & 13.62                                & \multicolumn{1}{B{.}{,}{2.2}}{3.73}  & 44.13                                 \\
		\hline
		O & -  & 1 & 8   & 32781 & 2332     & 13.60                                & 7.72                                 & \multicolumn{1}{B{.}{,}{3.2}}{69.54}  \\
		O & s  & 2 & inf & 32780 & 1616     & \multicolumn{1}{B{.}{,}{2.2}}{14.27} & 7.66                                 & 86.42                                 \\
		O & sr & 2 & inf & 32779 & \B{1338} & 14.19                                & \multicolumn{1}{B{.}{,}{2.2}}{6.83}  & 89.03                                 \\
		\hline
		H & -  & 1 & 16  & 32795 & 6081     & 25.82                                & 13.62                                & \multicolumn{1}{B{.}{,}{3.2}}{542.81} \\
		H & s  & 2 & inf & 32790 & \B{3021} & \multicolumn{1}{B{.}{,}{2.2}}{26.54} & 11.08                                & 734.08                                \\
		H & sr & 2 & inf & 32792 & 3312     & 25.99                                & \multicolumn{1}{B{.}{,}{2.2}}{10.95} & 798.29                                \\
		\bottomrule
%		\multicolumn{6}{l}{\footnotesize \textit{Pozn:}
%		\textrm{Zam} - počet zamítnutí, \textrm{pAgen} - průměrný počet agentů v jeden krok na křižovatce, \\
%		\textrm{sAgen} - směrodatná odchylka počtu agentů na křižovatce, \\
%		\textrm{Zpož} - součet spoždění přes všechny agenty, \textrm{pZpož} - průměrné zpoždění agentů
%		}  TODO
	\end{tabular}
	\caption{Porovnání vlivu parametrů u \ref{str:a_star_ars} na různých typech malé křižovatky.}\label{tab:ars_exp_mala}
%	\end{adjustwidth}
\end{table}

\subsubsection{\ref{str:a_star_ars} na \hyperref[par:data_velka]{velké} křižovatce bez výjezdů}
\label{subsubsec:exp_ars_velka_krizovatka_bez_vyjezdu}

Pro tyto testy jsem přirozeně zvýšil omezení \ref{str:ars_mpc} pro čtvercový a oktagonální typ na hodnotu $32$,
a pro hexagonální na $64$, pokud je nastavený \ref{str:ars_mnv} na $1$.
Pro běhy s \ref{str:ars_mnv} $2$ jsou délky cest opět neomezené.

V tabulce (Tabulka \ref{tab:ars_exp_velka_bez_vyjezdu}) jsou vidět výsledky na všech typech křižovatky
velikosti $16$ se $4$ vjezdy a $4$ výjezdy.

Z výsledků je dobře vidět souvislost mezi nejmenším počtem zamítnutých agentů, nejvyšším počtem agentů na křižovatce a
nejmenším průměrným zpožděním agenta.

Avšak zbytek výsledků je dosti překvapivý.
Na čtvercovém a hexagonálním typu křižovatky byl nejlepší nejméně omezený algoritmus,
zatímco na oktagonální křižovatce vyšla značně lépe varianta s povoleným zastavováním a $2$ návštěvami vrcholu.
Přidání diagonálních přejezdů tentokrát výrazně pomohlo u všech tří běhů.

Nejpřekvapivější pro mě byly časy plánování.
Nejrychlejší u čtvercového typu byl nejvíce omezený algoritmus,
avšak u zbylých dvou typů běžel nejrychleji nejméně omezený běh.
Běh s povoleným zastavováním byl u všech tří typů nejpomalejší a pro hexagonální křižovatky ani nestihl doběhnout.
Dle mého názoru je to způsobené vysokým počtem plánování agentů, která nejsou úspěšná.
Ale aby tento fakt algoritmus zjistil musí vyzkoušet vysoký počet možností.
Pokud ale povolím vracení agenta, dle mého názoru mnoha těmto agentům umožním jet relativně krátkou trasou.

\input{experimenty/ars_big_table_no_exits}

\subsubsection{\ref{str:a_star_ars} na \hyperref[par:data_velka]{velké} křižovatce s výjezdem}
\label{subsubsec:exp_ars_velka_krizovatka_s_vyjezdem}

Zde jsem použil totožné nastavení parametrů jako u běhů bez specifikovaných výjezdů.

V tabulce \ref{tab:ars_exp_velka_s_vyjezdy} jsou zobrazeny výsledky.

Algoritmus se zde chová většinově podle očekávání.
Nejméně omezené varianty dávají nejlepší výsledky na všech typech křižovatky.
Pokud všechny varianty doběhly, největší omezení vede k největšímu počtu zamítnutých agentů a
největšímu průměrnému zpoždění.

Jediné překvapení je ve sloupci s dobou plánování, avšak pořadí je stejné jako při nespecifikovaných výjezdech.
Na všech typech křižovatky plánovala prostřední varianta v průměru nejpomaleji.
Na čtvercové byla nejrychlejší první, nejvíce omezená varianta.
Na zbylých typech křižovatky nejméně omezené běhy.
Důvody jsou podle mě stejné jako u situace bez výjezdů.


\begin{table}[h]
	\centering
%	\begin{adjustwidth}{-1.5cm}{}
	\begin{tabular}{c c c c | r r D{.}{,}{3.2} D{.}{,}{2.2} D{.}{,}{6.2}}
		\toprule \\
		\pulrad{\B{Typ}} & \pulrad{\B{Omez}} & \pulrad{\B{\ref{str:ars_mnv}}} &
		\pulrad{\B{\ref{str:ars_mpc}}} & \pulrad{\B{Krok}}  & \pulrad{\B{Zam}} &
		\mc{\pulrad{\B{pAg}}} & \mc{\pulrad{\B{pZp}}} & \mc{\pulrad{\B{Čas}}} \\
		\midrule
		S & -  & 1 & 32  & 32855 & 56384     & 143.71                                & 49.13                                & \multicolumn{1}{B{.}{,}{6.2}}{11972.49}  \\
		S & s  & 2 & inf & 32852 & 28566     & 152.22                                & 47.19                                & 20772.51                                 \\
		S & sr & 2 & inf & 32846 & \B{26667} & \multicolumn{1}{B{.}{,}{3.2}}{155.96} & \multicolumn{1}{B{.}{,}{2.2}}{35.99} & 14428.00  \\
		\hline
		O & -  & 1 & 32  & 32849 & 37589     & 157.28                                & 43.90                                & 86298.33                                 \\
		O & s  & 2 & inf & 32848 & 25990     & 162.69                                & 38.05                                & 157218.08                                \\
		O & sr & 2 & inf & 32844 & \B{24714} & \multicolumn{1}{B{.}{,}{3.2}}{164.83} & \multicolumn{1}{B{.}{,}{2.2}}{34.25} & \multicolumn{1}{B{.}{,}{6.2}}{72598.43}  \\
		\hline
		H & -  & 1 & 64  & 28256 & 105284    & 300.94                                & 61.20                                & 253753.98                                \\
		H & s  & 2 & inf & 13850 & 241515    & 135.19                                & 55.48                                & 519920.12                                \\
		H & sr & 2 & inf & 32894 & \B{35385} & \multicolumn{1}{B{.}{,}{3.2}}{316.87} & \multicolumn{1}{B{.}{,}{2.2}}{53.93} & \multicolumn{1}{B{.}{,}{6.2}}{211489.82} \\
		\bottomrule
%		\multicolumn{6}{l}{\footnotesize \textit{Pozn:}
%		\textrm{Zam} - počet zamítnutí, \textrm{pAgen} - průměrný počet agentů v jeden krok na křižovatce, \\
%		\textrm{sAgen} - směrodatná odchylka počtu agentů na křižovatce, \\
%		\textrm{Zpož} - součet spoždění přes všechny agenty, \textrm{pZpož} - průměrné zpoždění agentů
%		}  TODO
	\end{tabular}
	\caption{Porovnání vlivu parametrů u \ref{str:a_star_ars} na různých typech velké křižovatky se specifikovanými výjezdy.}
	\label{tab:ars_exp_velka_s_vyjezdy}
%	\end{adjustwidth}
\end{table}


\subsubsection{\ref{str:a_star_arsg} na \hyperref[par:data_mala]{malé} křižovatce}
\label{subsubsec:exp_arsg_mala_krizovatka}

Jelikož se jedná o rozšíření \ref{str:a_star_ars}, použil jsem stejná nastavení parametrů.
Tedy pro čtvercový a oktagonální typ jsem nastavil \ref{par:ars_mpc} na hodnotu $8$, a pro hexagonální na $16$.
Při bězích s \ref{par:ars_mnv} nastaveno na $2$ bude \hyperref[par:ars_mpc]{prodleva cesty} opět neomezená.

Algoritmus má jediný parametr navíc, a to \nameref{par:arsg_zvp}, který udává,
po jak dlouhé prodlevě má výpočet přejít na zjednodušený režim.
U všech experimentů je tento parametr nastaven na jednu sekundu.

V tabulce (Tabulka \ref{tab:arsg_exp_mala}) jsou vidět výsledky na všech typech křižovatky
velikostí $4$ a jedním vjezdem a výjezdem.

Překvapivě si v této simulaci na všech křižovatkách vedla nejlépe nejvíce omezená varianta.
Měla výrazně menší počet zamítnutých agentů a průměrné zpoždění.
Avšak měla nejmenší počet agentů na křižovatce.
To napovídá možnosti, kdy méně omezené varianty dokáží naplánovat složitější a delší cesty,
zatímco omezenější běhy by tyto agenty ve stejném kroku zamítly a naplánovaly jim optimálnější trasy následující krok.

Běhy s povoleným zastavováním si vždy vedly hůře než běhy s povoleným vracením.
Avšak měly rychlejší plánování srovnatelné s nejvíce omezenou variantou.

\begin{table}[b!]
	\centering
%	\begin{adjustwidth}{-1.5cm}{}
	\begin{tabular}{c c c c | r r D{.}{,}{2.2} D{.}{,}{2.2} D{.}{,}{4.2}}
		\toprule \\
		\pulrad{\B{Typ}} & \pulrad{\B{Omez}} & \pulrad{\B{\ref{par:ars_mnv}}} &
		\pulrad{\B{\ref{par:ars_mpc}}} & \pulrad{\B{Krok}}  & \pulrad{\B{Zam}} &
		\mc{\pulrad{\B{pAg}}} & \mc{\pulrad{\B{pZp}}} & \mc{\pulrad{\B{Čas}}} \\
		\midrule
%		1 & 0 & \B{701} & \multicolumn{1}{B{.}{,}{2.2}}{11.85} & \multicolumn{1}{B{.}{,}{1.2}}{2.06}
%		& \B{267\,141} & \multicolumn{1}{B{.}{,}{1.2}}{4.13} \\
		S & -  & 1 & 8   & 32779 & \B{568}  & 14.20                                & \multicolumn{1}{B{.}{,}{2.2}}{5.72}  & 262.71                                 \\
		S & s  & 2 & inf & 32787 & 4130     & \multicolumn{1}{B{.}{,}{2.2}}{14.84} & 10.68                                & \multicolumn{1}{B{.}{,}{4.2}}{218.39}  \\
		S & sr & 2 & inf & 32780 & 2397     & 14.79                                & 9.08                                 & 258.56                                 \\
		\hline
		O & -  & 1 & 8   & 32785 & \B{6276} & 13.14                                & \multicolumn{1}{B{.}{,}{2.2}}{9.46}  & \multicolumn{1}{B{.}{,}{4.2}}{237.60}  \\
		O & s  & 2 & inf & 32789 & 12220    & 13.34                                & 13.42                                & 278.54                                 \\
		O & sr & 2 & inf & 32787 & 10357    & \multicolumn{1}{B{.}{,}{2.2}}{13.48} & 12.74                                & 284.36                                 \\
		\hline
		H & -  & 1 & 16  & 32792 & \B{5872} & 25.94                                & \multicolumn{1}{B{.}{,}{2.2}}{13.15} & \multicolumn{1}{B{.}{,}{4.2}}{1195.23} \\
		H & s  & 2 & inf & 32801 & 17624    & \multicolumn{1}{B{.}{,}{2.2}}{26.35} & 20.95                                & 1208.47                                \\
		H & sr & 2 & inf & 32798 & 15252    & 26.26                                & 19.84                                & 1484.21                                \\
		\bottomrule
%		\multicolumn{6}{l}{\footnotesize \textit{Pozn:}
%		\textrm{Zam} - počet zamítnutí, \textrm{pAgen} - průměrný počet agentů v jeden krok na křižovatce, \\
%		\textrm{sAgen} - směrodatná odchylka počtu agentů na křižovatce, \\
%		\textrm{Zpož} - součet spoždění přes všechny agenty, \textrm{pZpož} - průměrné zpoždění agentů
%		}  TODO
	\end{tabular}
	\caption{Porovnání vlivu parametrů u \ref{str:a_star_arsg} na různých typech malé křižovatky.}\label{tab:arsg_exp_mala}
%	\end{adjustwidth}
\end{table}

\subsubsection{\ref{str:a_star_arsg} na \hyperref[par:data_velka]{velké} křižovatce bez výjezdů}
\label{subsubsec:exp_arsg_velka_krizovatka_bez_vyjezdu}

\ref{par:ars_mpc} je opět nastaveno pro čtvercový a oktagonální typ na hodnotu $32$,
a pro hexagonální na $64$, pokud je nastavený \ref{par:ars_mnv} na $1$.
Por běhy s \ref{par:ars_mnv} $2$ jsou délky cest opět neomezené.

V tabulce (Tabulka \ref{tab:arsg_exp_velka_bez_vyjezdu}) jsou vidět výsledky na všech typech křižovatky
velikostí $16$ se $4$ vjezdy a $4$ výjezdy.

V této tabulce mě většina hodnot překvapila, výsledky mezi jednotlivými typy křižovatek jsou navzájem nesrovnatelné.
Na čtvercovém typu křižovatky měl nejmenší počet zamítnutých agentů běh s nejméně omezenými agenty.
Zároveň měl nejvíce agentů na křižovatce.
Nejméně omezená varianta měla mírně víc zamítnutých agentů a mírně menší počet agentů na křižovatce.
Avšak dosáhla výrazně nižšího průměrného zpoždění.
Zároveň zvládla plánovat agenty nejrychleji.
Prostředního běh měl nejhorší výsledky až na dobu plánování.

U oktagonálního typu jednoznačně vyhrála první varianta,
dosáhla nejmenšího počtu zamítnutí a průměrného zpoždění.
Ačkoliv se plánovací časy oproti čtvercové křižovatce zpětinásobily, pořád byly prot tento typ nejmenší.
Nejméně omezený běh měl výrazně horší počet zamítnutých agentů, průměrné zpoždění a plánovací čas.
Avšak má vyšší počet agentů na křižovatce.
To by nasvědčovalo mé hypotéze, že se snížením omezení pohybu dokáže algoritmus na plánovat dříve nenaplánované agenty,
avšak za cenu delších cest a blokování budoucích agentů.
Dalším překvapením je výsledek prostředního běhu.
Ačkoliv si oproti čtvercovému typu pomohl, celkově vyšel nejhůře, včetně plánovací doby.
Je možné, že algoritmus často přechází na zjednodušený režim, ve kterém plánuje méně agentů, než by jinak zvládl.
Dále je možné, že opět dochází ke vzájemnému překážení agentů, jako u předchozí varianty.
Je totiž vidět, že i tento běh má vyšší zaplněnost křižovatky než běh první.

Na hexagonální křižovatce žádná varianta nedoběhla do konce, takže data tím mohou být zkreslená.
Výsledky jsou zde opačné oproti předchozímu případu.
Prostřední varianta zvládla vypočítat nejvíce kroků a nejspíše díky tomu dosáhnout nejmenšího počtu zamítnutí.
Zaplněnost křižovatky byla přibližně stejná, avšak průměrné zpoždění měl nejméně omezený běh.
To může být způsobeno výrazně menším počtem kroků, a tedy i menším počtem naplánovaných agentů.
Plány agentů ze začátku simulace jsou mnohem kratší,
jelikož na křižovatce nejsou žádní agenti, kteří by blokovali tyto trasy.
Intuitivně měl tento běh nejvyšší čas plánování, více než o třetinu vyšší než zbylé dva testy.

Nejsem si jistý, proč ta dříve nejrychlejší varianta byla tak pomalá.
Dle mého názoru existuje optimální omezení agentů, které se liší pro každý typ a velikost křižovatky.
Pokud agenta omezím více, rychleji zjistím, že pro něj neexistuje cesta.
Avšak zároveň můžou nastat situace, ve kterých agentovi chyběl jediný krok do cíle
a agent by byl úspěšně naplánován.

\begin{table}[h]
	\centering
%	\begin{adjustwidth}{-1.5cm}{}
	\begin{tabular}{c c c c | r r D{.}{,}{3.2} D{.}{,}{2.2} D{.}{,}{6.2}}
		\toprule \\
		\pulrad{\B{Typ}} & \pulrad{\B{Omez}} & \pulrad{\B{\ref{par:ars_mnv}}} &
		\pulrad{\B{\ref{par:ars_mpc}}} & \pulrad{\B{Krok}}  & \pulrad{\B{Zam}} &
		\mc{\pulrad{\B{pAg}}} & \mc{\pulrad{\B{pZp}}} & \mc{\pulrad{\B{Čas}}} \\
		\midrule
		S & -  & 1 & 32  & 32845 & 22791      & 143.20                                & \multicolumn{1}{B{.}{,}{2.2}}{30.23} & \multicolumn{1}{B{.}{,}{6.2}}{12972.80}  \\
		S & s  & 2 & inf & 32851 & 35971      & 139.62                                & 47.06                                & 16372.90                                 \\
		S & sr & 2 & inf & 32847 & \B{22451}  & \multicolumn{1}{B{.}{,}{3.2}}{145.58} & 43.62                                & 21190.45                                 \\
		\hline
		O & -  & 1 & 32  & 32840 & \B{12990}  & 142.63                                & \multicolumn{1}{B{.}{,}{2.2}}{19.84} & \multicolumn{1}{B{.}{,}{6.2}}{62133.51}  \\
		O & s  & 2 & inf & 32849 & 33990      & 143.50                                & 44.20                                & 80076.10                                 \\
		O & sr & 2 & inf & 32850 & 21928      & \multicolumn{1}{B{.}{,}{3.2}}{148.34} & 41.81                                & 68197.91                                 \\
		\hline
		H & -  & 1 & 64  & 11604 & 263886     & 298.78                                & \multicolumn{1}{B{.}{,}{2.2}}{49.01} & 621399.93                                \\
		H & s  & 2 & inf & 20359 & \B{198896} & \multicolumn{1}{B{.}{,}{3.2}}{299.14} & 88.14                                & \multicolumn{1}{B{.}{,}{6.2}}{353255.32} \\
		H & sr & 2 & inf & 18381 & 212523     & 298.47                                & 81.40                                & 391228.41                                \\
		\bottomrule
%		\multicolumn{6}{l}{\footnotesize \textit{Pozn:}
%		\textrm{Zam} - počet zamítnutí, \textrm{pAgen} - průměrný počet agentů v jeden krok na křižovatce, \\
%		\textrm{sAgen} - směrodatná odchylka počtu agentů na křižovatce, \\
%		\textrm{Zpož} - součet spoždění přes všechny agenty, \textrm{pZpož} - průměrné zpoždění agentů
%		}  TODO
	\end{tabular}
	\caption{Porovnání vlivu parametrů u \ref{str:a_star_arsg} na různých typech velké křižovatky bez výjezdů.}\label{tab:arsg_exp_velka_bez_vyjezdu}
%	\end{adjustwidth}
\end{table}


\subsubsection{\ref{str:a_star_arsg} na \hyperref[par:data_velka]{velké} křižovatce s výjezdem}
\label{subsubsec:exp_arsg_velka_krizovatka_s_vyjezdem}

Testoval jsem opět totožné nastavení parametrů, výsledky jsou v tabulce \ref{tab:arsg_exp_velka_s_vyjezdy}.

Na čtvercové křižovatce se algoritmus choval podle očekávání, nejvíce omezený běh měl nejvíce zamítnutých agentů,
ale nejkratší plánovací čas.
Tento běh měl nejnižší počet zamítnutých agentů,
avšak to je podle mého způsobeno menším počtem celkově naplánovaných agentů,
což vede k menšímu počtu jedoucích agentů, kterým se musí plánovaní agenti vyhnout.
Nejméně omezená varianta měla opačné hodnoty, nejméně zamítnutých agentů za cenu vyšší časové náročnosti.

Na oktagonální i hexagonální křižovatce si vedla nejhůře nejvíce omezená varianta,
jelikož měla nejvyšší časový běh, což vedlo k nejméně naplánovaným agentům.
Naopak prostřední varianta má nejnižší čas plánování, což jí učinilo favoritem na hexagonálním typu.
Myslím si, že tato varianta dokáže úspěšně naplánovat vyšší počet agentů i v situacích,
kdy by je první varianta odmítla.
V takových případů musí první varianta zkusit pro agenta všechny možné trasy,
zatímco druhé variantě může stačit mnohem menší podprostor.
Vypadá to, že zde opravdu tato situace nastává.
Avšak rozšíření prohledávaného prostoru už negativně ovlivnilo časové nároky.

Nejméně omezená varianta si udržela nejmenší počet zamítnutí na oktagonálním typu křižovatky,
avšak nejspíše díky dlouhému času běhu měla vyšší počet zamítnutých agentů
než druhá varianta na hexagonálním typu křižovatky.


\begin{table}[h]
	\centering
%	\begin{adjustwidth}{-1.5cm}{}
	\begin{tabular}{c c c c | r r D{.}{,}{3.2} D{.}{,}{2.2} D{.}{,}{7.2}}
		\toprule \\
		\pulrad{\B{Typ}} & \pulrad{\B{Omez}} & \pulrad{\B{\ref{par:ars_mnv}}} &
		\pulrad{\B{\ref{par:ars_mpc}}} & \pulrad{\B{Krok}}  & \pulrad{\B{Zam}} &
		\mc{\pulrad{\B{pAg}}} & \mc{\pulrad{\B{pZp}}} & \mc{\pulrad{\B{Čas}}} \\
		\midrule
		S & -  & 1 & 32  & 32857 & 52093      & \multicolumn{1}{B{.}{,}{3.2}}{160.87} & \multicolumn{1}{B{.}{,}{2.2}}{51.32} & \multicolumn{1}{B{.}{,}{7.2}}{28779.47}  \\
		S & s  & 2 & inf & 32853 & 48406      & 155.67                                & 54.77                                & 41633.29                                 \\
		S & sr & 2 & inf & 32855 & \B{40413}  & 159.59                                & 52.69                                & 49668.90                                 \\
		\hline
		O & -  & 1 & 32  & 29599 & 48658      & \multicolumn{1}{B{.}{,}{3.2}}{168.16} & \multicolumn{1}{B{.}{,}{2.2}}{36.72} & 243226.57                                \\
		O & s  & 2 & inf & 32853 & 44435      & 159.65                                & 54.12                                & \multicolumn{1}{B{.}{,}{7.2}}{124598.82} \\
		O & sr & 2 & inf & 32855 & \B{36807}  & 163.73                                & 52.16                                & 145805.57                                \\
		\hline
		H & -  & 1 & 64  & 5837  & 328488     & \multicolumn{1}{B{.}{,}{3.2}}{334.24} & \multicolumn{1}{B{.}{,}{2.2}}{50.71} & 1240806.06                               \\
		H & s  & 2 & inf & 15656 & \B{240581} & 331.02                                & 93.02                                & \multicolumn{1}{B{.}{,}{7.2}}{459814.95} \\
		H & sr & 2 & inf & 13655 & 256584     & 330.02                                & 86.25                                & 527606.52                                \\
		\bottomrule
%		\multicolumn{6}{l}{\footnotesize \textit{Pozn:}
%		\textrm{Zam} - počet zamítnutí, \textrm{pAgen} - průměrný počet agentů v jeden krok na křižovatce, \\
%		\textrm{sAgen} - směrodatná odchylka počtu agentů na křižovatce, \\
%		\textrm{Zpož} - součet spoždění přes všechny agenty, \textrm{pZpož} - průměrné zpoždění agentů
%		}  TODO
	\end{tabular}
	\caption{Porovnání vlivu parametrů u \ref{str:a_star_arsg} na různých typech velké křižovatky s výjezdy.}\label{tab:arsg_exp_velka_s_vyjezdy}
%	\end{adjustwidth}
\end{table}



\subsubsection{\nameref{subsubsec:a_star_aoid} na \hyperref[par:data_mala]{malé} křižovatce}
\label{subsubsec:exp_aoid_mala_krizovatka}

Tento algoritmus rozšiřuje \ref{str:varsg}, proto jsem se snažil použít stejné nastavení parametrů.
Bohužel to nebylo vždy možné, jelikož přidaní agenti značně zvýšili časovou náročnost.

Algoritmus má pár parametrů navíc oproti \ref{str:varsg}.
Těmi jsou \ref{str:aoid_mpa} udávající maximální počet plánovaných agentů
a \ref{str:aoid_ppk}, který značí, kolik kroků může být agent na cestě, aby byl zvážen pro přeplánování.
\ref{str:aoid_ppk} bude vždy nastaveno na $8$ kroků,
\ref{str:aoid_mpa} je proměnlivé, vybraná hodnota je vidět v tabulce.

Algoritmus opět přejde na zjednodušený výpočet po jedné sekundě.

V tabulce (Tabulka \ref{tab:aoid_exp_mala}) jsou vidět výsledky na všech typech křižovatky
velikosti 4, jedním vjezdem a výjezdem.

Na čtvercové křižovatce všechny varianty doběhly,
avšak rozdíl v době plánování je mezi nimi mnohem větší než v předchozích případech.
Delší časy mohly způsobit častější přechod na zjednodušený režim, což může vést k horším výsledkům.
Zároveň je vidět zvyšování zamítnutí a průměrného zpoždění se zvyšujícím se zaplněním křižovatky.
Opět by to nasvědčovalo možnosti, kdy vyšší omezení naplánuje obecně lepší trasu o krok později.

Stejné nastavení parametrů nebylo úspěšné na oktagonální křižovatce, ani jedno nastavení nedoběhlo do konce.
Nárůst složitosti může být způsoben vyšším počtem vrcholů grafu křižovatky
nebo faktem, že přidané vrcholy jsou mnohem blíže sebe
a agenti si tak navzájem omezují pohyb více než na čtvercovém typu.

Na hexagonálním typu křižovatky jsem zkusil stejné nastavení parametrů pro nejvíce omezenou variantu.
Algoritmus spočítal mnohem méně kroků než na oktagonálním typu, což by nasvědčovalo hypotéze,
že za přidanou složitost může vyšší počet vrcholů křižovatky.
U zbylých dvou experimentů jsem na této křižovatce snížil maximální počet plánovaných agentů.
Je vidět, že snížení tohoto počtu z $16$ na $14$ zaručí, že algoritmus bez problémů doběhne do konce.
Zároveň počet zamítnutých agentů a průměrná prodleva se příliš nemění pro počet plánovaných agentů $12$ a $14$.
Mohlo by to být způsobeno jiným nastavením nejvyšší prodlevy cesty,
avšak dle mého názoru povolení přeplánování více agentům nemá na výsledky tak velký vliv.

\begin{table}[b!]
%	\centering
	\begin{adjustwidth}{-1cm}{}
		\begin{tabular}{c c c c c | r r D{.}{,}{2.2} D{.}{,}{2.2} D{.}{,}{7.2}}
			\toprule \\
			\pulrad{\B{Typ}} & \pulrad{\B{Omez}} & \pulrad{\B{\ref{par:ars_mnv}}} &
			\pulrad{\B{\ref{par:ars_mpc}}} & \pulrad{\B{\ref{par:aoid_mpa}}} & \pulrad{\B{Krok}} &
			\pulrad{\B{Zam}} & \mc{\pulrad{\B{pAg}}} & \mc{\pulrad{\B{pZp}}} & \mc{\pulrad{\B{Čas}}} \\
			\midrule
%		1 & 0 & \B{701} & \multicolumn{1}{B{.}{,}{2.2}}{11.85} & \multicolumn{1}{B{.}{,}{1.2}}{2.06}
%		& \B{267\,141} & \multicolumn{1}{B{.}{,}{1.2}}{4.13} \\
			S & -  & 1 & 8   & 16 & 32786 & \B{1373}  & 18.04                                & \multicolumn{1}{B{.}{,}{2.2}}{15.10} & \multicolumn{1}{B{.}{,}{7.2}}{5582.03}   \\
			S & s  & 2 & inf & 16 & 32792 & 3382      & 19.95                                & 17.71                                & 34867.77                                 \\
			S & sr & 2 & inf & 16 & 32790 & 4586      & \multicolumn{1}{B{.}{,}{2.2}}{20.52} & 19.02                                & 120604.89                                \\
			\hline
			O & -  & 1 & 8   & 16 & 16976 & \B{32774} & \multicolumn{1}{B{.}{,}{2.2}}{18.49} & \multicolumn{1}{B{.}{,}{2.2}}{17.57} & \multicolumn{1}{B{.}{,}{7.2}}{424199.33} \\
			O & s  & 2 & inf & 16 & 1443  & 62691     & 1.68                                 & 18.28                                & 5038828.24                               \\
			\hline
			H & -  & 1 & 12  & 16 & 3955  & 87635     & 32.77                                & 25.07                                & 1956276.57                               \\
			H & s  & 2 & 16  & 12 & 32800 & \B{9472}  & 32.67                                & \multicolumn{1}{B{.}{,}{2.2}}{21.81} & \multicolumn{1}{B{.}{,}{7.2}}{10276.77}  \\
			H & s  & 2 & inf & 14 & 32800 & 9482      & \multicolumn{1}{B{.}{,}{2.2}}{34.11} & 23.50                                & 42300.29                                 \\
			\bottomrule
%		\multicolumn{6}{l}{\footnotesize \textit{Pozn:}
%		\textrm{Zam} - počet zamítnutí, \textrm{pAgen} - průměrný počet agentů v jeden krok na křižovatce, \\
%		\textrm{sAgen} - směrodatná odchylka počtu agentů na křižovatce, \\
%		\textrm{Zpož} - součet spoždění přes všechny agenty, \textrm{pZpož} - průměrné zpoždění agentů
%		}  TODO
		\end{tabular}
		\caption{Porovnání vlivu parametrů u \nameref{subsubsec:a_star_aoid} na různých typech malé křižovatky.}\label{tab:aoid_exp_mala}
	\end{adjustwidth}
\end{table}

\subsubsection{\nameref{subsubsec:a_star_aoid} na \hyperref[par:data_velka]{velké} křižovatce bez výjezdů}
\label{subsubsec:exp_aoid_velka_krizovatka_bez_vyjezdu}

Tabulky \ref{tab:aoid_exp_velka_bez_vyjezdu} a \ref{tab:aoid_exp_velka_s_vyjezdy} obsahují pouze omezené výsledky,
jelikož zbylé varianty zaplnily paměť a selhaly.

Z toho usuzuji, že počet přeplánovaných agentů musí být velice malý pro velké křižovatky.
To činí tento algoritmus značně nepoužitelný.

\begin{table}[h]
%	\centering
	\begin{adjustwidth}{-1cm}{}
		\begin{tabular}{c c c c c | r r D{.}{,}{3.2} D{.}{,}{2.2} D{.}{,}{6.2}}
			\toprule \\
			\pulrad{\B{Typ}} & \pulrad{\B{Omez}} & \pulrad{\B{\ref{str:ars_mnv}}} &
			\pulrad{\B{\ref{str:ars_mpc}}} & \pulrad{\B{\ref{str:aoid_mpa}}} & \pulrad{\B{Krok}} &
			\pulrad{\B{Zam}} & \mc{\pulrad{\B{pAg}}} & \mc{\pulrad{\B{pZp}}} & \mc{\pulrad{\B{Čas}}} \\
			\midrule
			S & -  & 1 & 32  & 32 & 32840 & \B{11285} & \multicolumn{1}{B{.}{,}{3.2}}{159.15} & 35.46                                & \multicolumn{1}{B{.}{,}{6.2}}{204258.46} \\
			S & s  & 2 & inf & 48 & 10320 & 182402    & 51.39                                 & 21.06                                & 557650.01                                \\
			S & sr & 2 & inf & 64 & 674   & 257255    & 3.25                                  & \multicolumn{1}{B{.}{,}{2.2}}{14.67} & 701142.19                                \\
%			\hline
%			\hline
			\bottomrule
%		\multicolumn{6}{l}{\footnotesize \textit{Pozn:}
%		\textrm{Zam} - počet zamítnutí, \textrm{pAgen} - průměrný počet agentů v jeden krok na křižovatce, \\
%		\textrm{sAgen} - směrodatná odchylka počtu agentů na křižovatce, \\
%		\textrm{Zpož} - součet spoždění přes všechny agenty, \textrm{pZpož} - průměrné zpoždění agentů
%		}  TODO
		\end{tabular}
		\caption{Porovnání vlivu parametrů u \nameref{subsubsec:a_star_aoid} na různých typech velké křižovatky bez výjezdů.}\label{tab:aoid_exp_velka_bez_vyjezdu}
	\end{adjustwidth}
\end{table}

\begin{table}[h]
%	\centering
	\begin{adjustwidth}{-1cm}{}
		\begin{tabular}{c c c c c | r r D{.}{,}{3.2} D{.}{,}{2.2} D{.}{,}{7.2}}
			\toprule \\
			\pulrad{\B{Typ}} & \pulrad{\B{Omez}} & \pulrad{\B{\ref{str:ars_mnv}}} &
			\pulrad{\B{\ref{str:ars_mpc}}} & \pulrad{\B{\ref{str:aoid_mpa}}} & \pulrad{\B{Krok}} &
			\pulrad{\B{Zam}} & \mc{\pulrad{\B{pAg}}} & \mc{\pulrad{\B{pZp}}} & \mc{\pulrad{\B{Čas}}} \\
			\midrule
			S & - & 1 & 32 & 22 & 32853 & \B{33043} & \multicolumn{1}{B{.}{,}{3.2}}{183.06} & 55.96                                & \multicolumn{1}{B{.}{,}{7.2}}{92660.76} \\
			O & - & 1 & 32 & 22 & 963   & 255040    & 5.23                                  & \multicolumn{1}{B{.}{,}{2.2}}{34.00} & 2625309.35                              \\
%			\hline
%			\hline
			\bottomrule
%		\multicolumn{6}{l}{\footnotesize \textit{Pozn:}
%		\textrm{Zam} - počet zamítnutí, \textrm{pAgen} - průměrný počet agentů v jeden krok na křižovatce, \\
%		\textrm{sAgen} - směrodatná odchylka počtu agentů na křižovatce, \\
%		\textrm{Zpož} - součet spoždění přes všechny agenty, \textrm{pZpož} - průměrné zpoždění agentů
%		}  TODO
		\end{tabular}
		\caption{Porovnání vlivu parametrů u \nameref{subsubsec:a_star_aoid} na různých typech velké křižovatky s výjezdy.}\label{tab:aoid_exp_velka_s_vyjezdy}
	\end{adjustwidth}
\end{table}






\subsection{\ref{str:cbs} porovnání parametrů}\label{subsec:cbs_porovnani_parametru}

V této kapitole porovnám vliv parametrů u algoritmů \ref{str:cbs} a \nameref{subsec:cbsoid}.

Tyto algoritmy rozšiřují \ref{str:a_star_ars},
čili budu používat všechny parametry z \ref{str:a_star_ars} se stejnými hodnotami.
Parametry od algoritmu \ref{str:a_star_ars} jsou \hyperref[par:ars_mnv]{maximum návštěv vrcholu},
\hyperref[par:ars_pz]{povolené zastavování}, \hyperref[par:ars_mpc]{maximální prodleva při cestě} a
\hyperref[par:ars_pv]{povolené vracení}.
Zároveň obsahuje \ref{str:cbs} parametr \ref{par:arsg_zvp} určující,
po jak dlouhé době má výpočet přejít na zjednodušený režim.
Tento parametr bude vždy nastaven na jednu sekundu.

\subsubsection{\ref{str:cbs} na \hyperref[par:data_mala]{malé} křižovatce}
\label{subsubsec:exp_cbssg_mala_krizovatka}

\ref{str:cbs} algoritmus se choval poměrně předvídatelně, jak je možné vidět v tabulce \ref{tab:cbssg_exp_mala}.
Zároveň všechny experimenty úspěšně stihly doběhnout.
\begin{table}[h]
	\centering
%	\begin{adjustwidth}{-1.5cm}{}
	\begin{tabular}{c c c c | r r D{.}{,}{2.2} D{.}{,}{2.2} D{.}{,}{5.2}}
		\toprule \\
		\pulrad{\B{Typ}} & \pulrad{\B{Omez}} & \pulrad{\B{\ref{str:ars_mnv}}} &
		\pulrad{\B{\ref{str:ars_mpc}}} & \pulrad{\B{Krok}}  & \pulrad{\B{Zam}} &
		\mc{\pulrad{\B{pAg}}} & \mc{\pulrad{\B{pZp}}} & \mc{\pulrad{\B{Čas}}} \\
		\midrule
		S & -  & 1 & inf & 32779 & 915      & 14.14                                & 6.00                                 & \multicolumn{1}{B{.}{,}{5.2}}{132.36}  \\
		S & s  & 2 & inf & 32777 & 82       & \multicolumn{1}{B{.}{,}{2.2}}{13.41} & \multicolumn{1}{B{.}{,}{2.2}}{3.62}  & 169.34   \\
		S & sr & 2 & inf & 32776 & \B{70}   & \multicolumn{1}{B{.}{,}{2.2}}{13.41} & 3.66                                 & 194.12                                 \\
		\hline
		O & -  & 1 & 16  & 32779 & 2264     & 13.40                                & 7.57                                 & 6902.67                                \\
		O & s  & 2 & inf & 32781 & 1490     & \multicolumn{1}{B{.}{,}{2.2}}{13.96} & 7.39                                 & 2595.20                                \\
		O & sr & 2 & inf & 32780 & \B{1079} & 13.91                                & \multicolumn{1}{B{.}{,}{2.2}}{6.46}  & \multicolumn{1}{B{.}{,}{5.2}}{1761.27} \\
		\hline
		H & -  & 1 & 24  & 32792 & 5846     & 25.48                                & 13.71                                & 24000.77                               \\
		H & s  & 2 & inf & 32793 & 2878     & \multicolumn{1}{B{.}{,}{2.2}}{26.17} & 11.15                                & 3762.95                                \\
		H & sr & 2 & inf & 32790 & \B{2698} & 25.74                                & \multicolumn{1}{B{.}{,}{2.2}}{10.80} & \multicolumn{1}{B{.}{,}{5.2}}{3128.19} \\
		\bottomrule
%		\multicolumn{6}{l}{\footnotesize \textit{Pozn:}
%		\textrm{Zam} - počet zamítnutí, \textrm{pAgen} - průměrný počet agentů v jeden krok na křižovatce, \\
%		\textrm{sAgen} - směrodatná odchylka počtu agentů na křižovatce, \\
%		\textrm{Zpož} - součet spoždění přes všechny agenty, \textrm{pZpož} - průměrné zpoždění agentů
%		}  TODO
	\end{tabular}
	\caption{Porovnání vlivu parametrů u \ref{str:cbs} na různých typech malé křižovatky.}\label{tab:cbssg_exp_mala}
%	\end{adjustwidth}
\end{table}


Na čtvercové křižovatce se se snižujícím omezením pohybu agentů snižoval počet zamítnutých agentů,
avšak za cenu rostoucí doby výpočtu.
Varianta s povoleným zastavování ale bez vracení měla mírně vyšší počet zamítnutí, o~12 agentů ($~17,14\%$).
Průměrné zpoždění měl dokonce o~kousek nižší, přesněji o~$~1,09\%$.
Čas výpočtu byl přibližně o $12,77\%$ nižší.

Na oktagonální i hexagonální křižovatce vyšla jednoznačně nejlepší nejméně omezená křižovatka.
Oproti předchozímu případu se se snižujícím omezením výrazně snižuje nejen počet zamítnutých agentů,
ale i průměrné zpoždění a čas plánování.
Dle mého názoru to je způsobeno vysokým počtem možností, kde se cesty agentů můžou křížit.
Pokud tedy některý agent nelze naplánovat, algoritmus musí vyzkoušet všechny možnosti
a pokaždé tvořit dva podpřípady, kopírovat do každého tabulky kolizí, \dots,
a nakonec zatřídit nové vrcholy do prioritní fronty.
Jelikož poslední varianta umožňuje nejvíce tras pro agenta, z počtu zamítnutých agentů usuzuji,
že v každém kroku dokáže úspěšně naplánovat více agentů, a proto je také nejrychlejší.

\subsubsection{\ref{str:cbs} na \hyperref[par:data_velka]{velké} křižovatce bez výjezdů}
\label{subsubsec:exp_cbssg_velka_krizovatka_bez_vyjezdu}

Na velké křižovatce úspěšně doběhl algoritmus pouze na čtvercovém typu.
Pro čtvercovou křižovatku měl nejmenší počet zamítnutí prostřední běh.
Avšak nejmenší zpoždění měla třetí, nejméně omezená varianta.
Opět zde platí, že se snižujícím omezením pohybu agentů se snižuje průměrné zpoždění
a roste zaplnění křižovatky, ale za cenu rostoucího času výpočtu.
Rozdíl v počtu zamítnutých agentů mezi variantami s povoleným zastavováním byl malý, přibližně $6,93\%$.
Dle mého názoru není žádný důvod, proč běh bez povoleného vracení měl počet agentů nižší,
myslím si, že to je způsobeno náhodným generováním agentů, které více \uv{sedlo} této variantě.

Na oktagonálním typu je vidět podobný trend jako na malé křižovatce.
Se snižujícím omezením agentů se snižuje plánovací čas.
Z toho důvodu se snižuje i počet zamítnutých agentů.
Zároveň je ale nižší i průměrný počet zpoždění agentů.
Překvapivé pro mě je, že přechod ze čtvercové křižovatky na oktagonální zvýší průměrný čas plánování alespoň desetkrát.
Pro porovnání, zvýšení počtu vrcholů křižovatky při tomto přechodu je přibližně $77\%$.

Na hexagonálním typu křižovatky byly časy výpočtu opět mnohem vyšší než na oktagonálním typu,
což vedlo k velkému snížení naplánovaných kroků.
Tentokrát si nejvíce pohoršil nejméně omezený běh, zvládl spočítat přibližně $35.29\%$ kroků.
Nejlépe si vedla prostřední varianta.
Důvod, proč si nejméně omezená varianta tolik pohoršila, je dle mého názoru vysokým nárůstem možných cest.
Tento nárůst na této křižovatce podle mého předčil výhodu z vyššího počtu naplánovaných agentů,
která asi způsobovala nejlepší výsledky na předchozích typech křižovatky.

\input{experimenty/cbssg_big_table_bez_vyjezdu}

\subsubsection{\ref{str:cbs} na \hyperref[par:data_velka]{velké} křižovatce s výjezdem}
\label{subsubsec:exp_cbssg_velka_krizovatka_s_vyjezdem}

\subsubsection{\nameref{subsec:cbsoid}}
\label{subsubsec:exp_cbsoid}

Zkusil jsem plánování i pomocí \nameref{subsec:cbsoid} algoritmu, avšak nepodařilo se mi najít kombinaci parametrů,
která by běžela v rozumném čase.
\begin{table}[h]
%	\centering
	\begin{adjustwidth}{-1cm}{}
		\begin{tabular}{c c c c c | r r D{.}{,}{2.2} D{.}{,}{2.2} D{.}{,}{7.2}}
			\toprule \\
			\pulrad{\B{Typ}} & \pulrad{\B{Omez}} & \pulrad{\B{\ref{str:ars_mnv}}} &
			\pulrad{\B{\ref{str:ars_mpc}}} & \pulrad{\B{\ref{str:aoid_mpa}}} & \pulrad{\B{Krok}} &
			\pulrad{\B{Zam}} & \mc{\pulrad{\B{pAg}}} & \mc{\pulrad{\B{pZp}}} & \mc{\pulrad{\B{Čas}}} \\
			\midrule
			S & -  & 1 & 16  & 16  & 6838 & \B{51759} & \multicolumn{1}{B{.}{,}{2.2}}{12.61} & \multicolumn{1}{B{.}{,}{2.2}}{8.45}  & \multicolumn{1}{B{.}{,}{7.2}}{1054679.77} \\
			S & s  & 2 & inf & 24  & 3755 & 57904     & 7.06                                 & 10.50                                & 1920988.21                                \\ % TODO
			S & sr & 2 & inf & 24  & 1530 & 62295     & 2.88                                 & 10.77                                & 2044461.87                                \\ % TODO 24?
			\hline
			O & -  & 1 & 16  & inf & 2466 & 60424     & \multicolumn{1}{B{.}{,}{2.2}}{12.11} & \multicolumn{1}{B{.}{,}{2.2}}{8.69}  & 2699251.20                                \\
			O & s  & 2 & inf & 16  & 2889 & \B{59641} & 11.99                                & 10.23                                & \multicolumn{1}{B{.}{,}{7.2}}{2497072.06} \\  % TODO 16?
			O & sr & 2 & inf & 16  & 855  & 63627     & 4.16                                 & 9.45                                 & 2675054.74                                \\  % TODO 16?
			\hline
			H & -  & 1 & 24  & 24  & 1227 & 94802     & 21.09                                & 19.49                                & 5906285.18                                \\
			H & s  & 2 & inf & 24  & 1704 & \B{93403} & \multicolumn{1}{B{.}{,}{2.2}}{22.16} & 21.18                                & \multicolumn{1}{B{.}{,}{7.2}}{4255409.05} \\  % TODO 24?
			H & sr & 2 & inf & 24  & 1229 & 94792     & 21.25                                & \multicolumn{1}{B{.}{,}{2.2}}{18.52} & 5893074.13                                \\  % TODO 24?
			\bottomrule
%		\multicolumn{6}{l}{\footnotesize \textit{Pozn:}
%		\textrm{Zam} - počet zamítnutí, \textrm{pAgen} - průměrný počet agentů v jeden krok na křižovatce, \\
%		\textrm{sAgen} - směrodatná odchylka počtu agentů na křižovatce, \\
%		\textrm{Zpož} - součet spoždění přes všechny agenty, \textrm{pZpož} - průměrné zpoždění agentů
%		}  TODO
		\end{tabular}
		\caption{Porovnání vlivu parametrů u \nameref{subsec:cbsoid} na různých typech malé křižovatky.}\label{tab:cbsoid_exp_mala}
	\end{adjustwidth}
\end{table}
\begin{table}[b!]
%	\centering
	\begin{adjustwidth}{-1.5cm}{}
		\begin{tabular}{c c c c c | r r D{.}{,}{3.2} D{.}{,}{2.2} D{.}{,}{8.2}}
			\toprule \\
			\pulrad{\B{Typ}} & \pulrad{\B{Omez}} & \pulrad{\B{\ref{par:ars_mnv}}} &
			\pulrad{\B{\ref{par:ars_mpc}}} & \pulrad{\B{\ref{par:aoid_mpa}}} & \pulrad{\B{Krok}} &
			\pulrad{\B{Zam}} & \mc{\pulrad{\B{pAg}}} & \mc{\pulrad{\B{pZp}}} & \mc{\pulrad{\B{Čas}}} \\
			\midrule
			S & -  & 1 & 32  & 32 & 965 & \B{255430} & \multicolumn{1}{B{.}{,}{3.2}}{121.15} & 45.84                                & \multicolumn{1}{B{.}{,}{8.2}}{7631472.87}  \\
			S & s  & 2 & inf & 64 & 396 & 259643     & 45.16                                 & \multicolumn{1}{B{.}{,}{2.2}}{35.35} & 19118433.27                                \\
			\hline
			O & -  & 2 & inf & 32 & 527 & 258775     & 111.77                                & 51.48                                & 14274680.74                                \\
			O & sr & 2 & inf & 32 & 638 & \B{257862} & \multicolumn{1}{B{.}{,}{3.2}}{118.79} & \multicolumn{1}{B{.}{,}{2.2}}{47.18} & \multicolumn{1}{B{.}{,}{8.2}}{11697421.69} \\
			\bottomrule
%		\multicolumn{6}{l}{\footnotesize \textit{Pozn:}
%		\textrm{Zam} - počet zamítnutí, \textrm{pAgen} - průměrný počet agentů v jeden krok na křižovatce, \\
%		\textrm{sAgen} - směrodatná odchylka počtu agentů na křižovatce, \\
%		\textrm{Zpož} - součet spoždění přes všechny agenty, \textrm{pZpož} - průměrné zpoždění agentů
%		}  TODO
		\end{tabular}
		\caption{Porovnání vlivu parametrů u \nameref{subsec:cbsoid} na různých typech velké křižovatky bez výjezdů.}\label{tab:cbsoid_exp_velka_bez_vyjezdu}
	\end{adjustwidth}
\end{table}

\begin{table}[b!]
%	\centering
	\begin{adjustwidth}{-1.5cm}{}
		\begin{tabular}{c c c c c | r r D{.}{,}{3.2} D{.}{,}{2.2} D{.}{,}{8.2}}
			\toprule \\
			\pulrad{\B{Typ}} & \pulrad{\B{Omez}} & \pulrad{\B{\ref{par:ars_mnv}}} &
			\pulrad{\B{\ref{par:ars_mpc}}} & \pulrad{\B{\ref{par:aoid_mpa}}} & \pulrad{\B{Krok}} &
			\pulrad{\B{Zam}} & \mc{\pulrad{\B{pAg}}} & \mc{\pulrad{\B{pZp}}} & \mc{\pulrad{\B{Čas}}} \\
			\midrule
			S & - & 1 & 64  & 24 & 957 & \B{255758} & \multicolumn{1}{B{.}{,}{3.2}}{136.49} & 52.40                                & \multicolumn{1}{B{.}{,}{8.2}}{7736125.71} \\
			S & s & 2 & inf & 32 & 553 & 258602     & 72.57                                 & \multicolumn{1}{B{.}{,}{2.2}}{51.77} & 13573333.56                               \\
			\bottomrule
%		\multicolumn{6}{l}{\footnotesize \textit{Pozn:}
%		\textrm{Zam} - počet zamítnutí, \textrm{pAgen} - průměrný počet agentů v jeden krok na křižovatce, \\
%		\textrm{sAgen} - směrodatná odchylka počtu agentů na křižovatce, \\
%		\textrm{Zpož} - součet spoždění přes všechny agenty, \textrm{pZpož} - průměrné zpoždění agentů
%		}  TODO
		\end{tabular}
		\caption{Porovnání vlivu parametrů u \nameref{subsec:cbsoid} na různých typech velké křižovatky s výjezdy.}\label{tab:cbsoid_exp_velka_s_vyjezdy}
	\end{adjustwidth}
\end{table}

Na malé, ani velké křižovatce algoritmus ani jednou nedoběhl.
Výsledky jsou vidět v tabulkách \ref{tab:cbsoid_exp_mala} a \ref{tab:cbsoid_exp_velka_bez_vyjezdu}.

\subsection{\ref{str:sat} porovnání parametrů}\label{subsec:sat_porovnani_parametru}

\subsubsection{\nameref{subsec:sat_rsg} na \hyperref[par:data_mala]{malé} křižovatce}
\label{subsubsec:exp_satsg_mala_krizovatka}
\begin{table}[h]
	\centering
%	\begin{adjustwidth}{-1.5cm}{}
	\begin{tabular}{c c c c | r r D{.}{,}{2.2} D{.}{,}{2.2} D{.}{,}{7.2}}
		\toprule \\
		\pulrad{\B{Typ}} & \pulrad{\B{Omez}} & \pulrad{\B{\ref{par:ars_mnv}}} &
		\pulrad{\B{\ref{par:ars_mpc}}} & \pulrad{\B{Krok}}  & \pulrad{\B{Zam}} &
		\mc{\pulrad{\B{pAg}}} & \mc{\pulrad{\B{pZp}}} & \mc{\pulrad{\B{Čas}}} \\
		\midrule
		S & - & 1 & 16 & 32782 & 1095     & \multicolumn{1}{B{.}{,}{2.2}}{14.29} & 5.99                                & \multicolumn{1}{B{.}{,}{7.2}}{4109.71}  \\
		S & s & 2 & 24 & 32778 & \B{255}  & 13.46                                & \multicolumn{1}{B{.}{,}{2.2}}{3.90} & 131515.34                               \\
		\hline
		O & - & 1 & 16 & 32784 & 4657     & 13.22                                & 8.65                                & \multicolumn{1}{B{.}{,}{7.2}}{15154.02} \\
		O & s & 2 & 24 & 32782 & \B{2868} & \multicolumn{1}{B{.}{,}{2.2}}{13.81} & \multicolumn{1}{B{.}{,}{2.2}}{7.77}  & 133605.81  \\
		\hline
		H & - & 1 & 16 & 32787 & \B{4403} & \multicolumn{1}{B{.}{,}{2.2}}{26.19} & 12.08                               & \multicolumn{1}{B{.}{,}{7.2}}{78214.86} \\
		H & s & 2 & 22 & 0     & 94573    & 1.02                                 & \multicolumn{1}{B{.}{,}{2.2}}{7.60} & 7200869.77                              \\
		\bottomrule
%		\multicolumn{6}{l}{\footnotesize \textit{Pozn:}
%		\textrm{Zam} - počet zamítnutí, \textrm{pAgen} - průměrný počet agentů v jeden krok na křižovatce, \\
%		\textrm{sAgen} - směrodatná odchylka počtu agentů na křižovatce, \\
%		\textrm{Zpož} - součet spoždění přes všechny agenty, \textrm{pZpož} - průměrné zpoždění agentů
%		}  TODO
	\end{tabular}
	\caption{Porovnání vlivu parametrů u \nameref{subsec:sat_rsg} na různých typech malé křižovatky.}\label{tab:sat_exp_mala}
%	\end{adjustwidth}
\end{table}


\subsubsection{\nameref{subsec:sat_ra} na \hyperref[par:data_mala]{malé} křižovatce}
\label{subsubsec:exp_sata_mala_krizovatka}
\begin{table}[h]
	\centering
%	\begin{adjustwidth}{-1cm}{}
	\begin{tabular}{c c c c c | r r D{.}{,}{2.2} D{.}{,}{1.2} D{.}{,}{8.2}}
		\toprule \\
		\pulrad{\B{Typ}} & \pulrad{\B{Omez}} & \pulrad{\B{\ref{par:ars_mnv}}} &
		\pulrad{\B{\ref{par:ars_mpc}}} & \pulrad{\B{\ref{par:aoid_mpa}}} & \pulrad{\B{Krok}} &
		\pulrad{\B{Zam}} & \mc{\pulrad{\B{pAg}}} & \mc{\pulrad{\B{pZp}}} & \mc{\pulrad{\B{Čas}}} \\
		\midrule
		S & - & 1 & 10 & 12 & 32776 & \B{363}   & \multicolumn{1}{B{.}{,}{2.2}}{15.07} & 5.44                                & \multicolumn{1}{B{.}{,}{8.2}}{9893.10}    \\
		S & - & 2 & 14 & 12 & 16634 & 32273     & 7.56                                 & \multicolumn{1}{B{.}{,}{1.2}}{4.58} & 432905.36                                 \\
		\hline
		O & - & 1 & 10 & 8  & 9289  & 47575     & \multicolumn{1}{B{.}{,}{2.2}}{15.16} & 8.74                                & 297656.39                                 \\
		O & s & 1 & 10 & 8  & 20821 & \B{25521} & \multicolumn{1}{B{.}{,}{2.2}}{15.16} & 8.57                                & 345728.00                                 \\
		O & s & 2 & 9  & 9  & 2740  & 59957     & 1.98                                 & 5.15                                & \multicolumn{1}{B{.}{,}{8.2}}{2632600.55} \\
		O & s & 2 & 9  & 10 & 99    & 65125     & 0.07                                 & \multicolumn{1}{B{.}{,}{1.2}}{3.07} & 30123061.30                               \\
		\hline
		H & - & 1 & 10 & 8  & 32788 & \B{1905}  & \multicolumn{1}{B{.}{,}{2.2}}{26.06} & 8.85                                & \multicolumn{1}{B{.}{,}{8.2}}{34920.59}   \\
		H & s & 2 & 12 & 8  & 11448 & 64109     & 9.30                                 & \multicolumn{1}{B{.}{,}{2.2}}{1.22} & 1240239.54                                \\
		\bottomrule
%		\multicolumn{6}{l}{\footnotesize \textit{Pozn:}
%		\textrm{Zam} - počet zamítnutí, \textrm{pAgen} - průměrný počet agentů v jeden krok na křižovatce, \\
%		\textrm{sAgen} - směrodatná odchylka počtu agentů na křižovatce, \\
%		\textrm{Zpož} - součet spoždění přes všechny agenty, \textrm{pZpož} - průměrné zpoždění agentů
%		}  TODO
	\end{tabular}
	\caption{Porovnání vlivu parametrů u \nameref{subsec:sat_ra} na různých typech malé křižovatky.}\label{tab:sata_exp_mala}
%	\end{adjustwidth}
\end{table}


\section{Hromadné výsledky}\label{sec:hromadne_vysledky}

%Porovnání algoritmů mezi sebou s nejlepšími parametry.
%
%Porovnání čtvercové a oktagonální křižovatky.

Nyní budu porovnávat nejlepší nastavení algoritmů mezi sebou.
Porovnání by mělo být férové, jelikož všechny algoritmy běžely s podobným nastavením parametrů.
Nejlepší nastavení většiny parametrů je zobrazeno v prvních sloupcích tabulky hned za názvem algoritmu.

Zároveň je v tabulkách obsažen algoritmus \nameref{sec:safe_lanes}.
Ten byl doposud vynechán, jelikož nemá žádné nastavitelné parametry.

\subsection{Porovnání algoritmů na malé křižovatce}\label{subsec:porovnani_algoritmu_na_male_krizovatce}

Jediný algoritmus, který nedoběhl ani na jednom typu malé křižovatky je \nameref{subsec:cbsoid}.
Z toho usuzuji, že \ref{str:cbs} je nejvíce citlivý na přidávání agentů.


Nejprve začnu čtvercovou křižovatkou, jejíž výsledky jsou v tabulce \ref{tab:all_exp_mala_ctvercova}.
V tabulce není zapsána hodnota parametru maximálního počtu přeplánovaných agentů (\ref{par:aoid_mpa}) u~\ref{str:oid} variant.
Na této křižovatce měli \nameref{subsubsec:a_star_aoid} a \nameref{subsec:cbsoid}
největší úspěšnost s \ref{par:aoid_mpa} nastaveným na $16$ a \nameref{subsec:sat_ra} na $12$.

Nejlepší algoritmus zde vyšel \ref{str:cbs}, v těsném závěsu za ním je \ref{str:a_star_ars}.
Tyto algoritmy mají zároveň nejmenší zpoždění agentů a v průměru nejméně agentů na křižovatce.
Zároveň měli nejnižší dobu plánování po \nameref{sec:safe_lanes}.

Po těchto algoritmech měly nejmenší počet zamítnutí \ref{str:sat} algoritmy a průměrná zpoždění byla srovnatelná.
Avšak po \nameref{subsec:cbsoid} měly nejvyšší čas plánování.

\nameref{sec:safe_lanes} sice zvládlo plánovat agenty mnohem rychleji než ostatní algoritmy,
avšak zároveň má mnohem více zamítnutých agentů, než všechny algoritmy, které doběhly.

\nameref{subsubsec:a_star_aoid} měl nejvyšší průměrný počet agentů na křižovatce,
avšak zároveň měl jednoznačně nejvyšší průměrné zpoždění agentů.

Verze algoritmů plánující menší počet agentů preferovala spíše volnější pohyb pro agenty.
Naopak všechny \ref{str:oid} varianty, \nameref{subsec:sat_ra} a \ref{str:a_star_arsg} dosáhly nejlepších výsledků s nejvíce omezenými parametry.

\begin{table}[h]
%	\centering
	\begin{adjustwidth}{-0.5cm}{}
		\begin{tabular}{c c c c | r r D{.}{,}{2.2} D{.}{,}{2.2} D{.}{,}{7.2}}
			\toprule \\
			\pulrad{\B{Alg}} & \pulrad{\B{Omez}} & \pulrad{\B{\ref{str:ars_mnv}}} &
			\pulrad{\B{\ref{str:ars_mpc}}} & \pulrad{\B{Krok}}  & \pulrad{\B{Zam}} &
			\mc{\pulrad{\B{pAg}}} & \mc{\pulrad{\B{pZp}}} & \mc{\pulrad{\B{Čas}}} \\
			\midrule
			\nameref{sec:safe_lanes}        & -  & 1 & inf & 32782 & 4056   & 11.19                                & 6.89                                & \multicolumn{1}{B{.}{,}{7.2}}{16.75} \\
			\hline
			\ref{str:a_star_ars}            & s  & 2 & inf & 32775 & 76     & 13.68                                & 3.92                                & 47.77                                \\
			\ref{str:varsg}           & -  & 1 & 8   & 32779 & 568    & 14.20                                & 5.72                                & 262.71                               \\
			\nameref{subsubsec:a_star_aoid} & -  & 1 & 8   & 32786 & 1373   & \multicolumn{1}{B{.}{,}{2.2}}{18.04} & 15.10 & 5582.03                                                            \\  % 16
			\hline
			\ref{str:cbs}                   & sr & 2 & inf & 32776 & \B{70} & 13.41                                & \multicolumn{1}{B{.}{,}{2.2}}{3.66} & 194.12                               \\
			\nameref{subsec:cbsoid}         & -  & 1 & 16  & 6838  & 51759  & 12.61                                & 8.45                                & 1054679.77                           \\  % 16
			\hline
			\nameref{subsec:sat_rsg}        & s  & 2 & 24  & 32778 & 255    & 13.46                                & 3.90                                & 131515.34                            \\
			\nameref{subsec:sat_ra}         & -  & 1 & 10  & 32776 & 363    & 15.07                                & 5.44                                & 9893.10                              \\  % 12
			\bottomrule
%		\multicolumn{6}{l}{\footnotesize \textit{Pozn:}
%		\textrm{Zam} - počet zamítnutí, \textrm{pAgen} - průměrný počet agentů v jeden krok na křižovatce, \\
%		\textrm{sAgen} - směrodatná odchylka počtu agentů na křižovatce, \\
%		\textrm{Zpož} - součet spoždění přes všechny agenty, \textrm{pZpož} - průměrné zpoždění agentů
%		}  TODO
		\end{tabular}
		\caption{Porovnání algoritmů na malé čtvercové křižovatce.}\label{tab:all_exp_mala_ctvercova}
	\end{adjustwidth}
\end{table}

Na oktagonální křižovatce je chování algoritmů podobné (tabulka \ref{tab:all_exp_mala_oktagonalni}).
Avšak zde ani jeden z algoritmů přeplánovající naplánované agenty nestihl doběhnout.
Zde měli \nameref{subsubsec:a_star_aoid} a \nameref{subsec:cbsoid}
největší úspěšnost opět s \ref{par:aoid_mpa} nastaveným na $16$ a \nameref{subsec:sat_ra} na $12$.

Pořadí algoritmů, které doběhly, s ohledem na počet zamítnutých agentů se nezměnilo,
stále si nejlépe vedlo \ref{str:cbs}.

\nameref{subsubsec:a_star_aoid} opět mělo v průměru nejvíce agentů na křižovatce, ale opět nejvyšší zpoždění agentů.

\nameref{sec:safe_lanes} sice zvládlo má opět mnohem více zamítnutých agentů, než všechny algoritmy, které doběhly.
Oproti čtvercové křižovatce se plánovací časy snížily,
avšak při tak nízkých časech to může být způsobeno zatížením systému jinými procesy.

Běhové časy se u ostatních algoritmů výrazně zvýšily.
To může být způsobeno výrazným nárůstem počtu zamítnutých agentů.
Podle mého názoru je tento nárůst způsoben zmenšením celkové plochy křižovatky.
Diagonální vrcholy size umožňují více možných pozic pro agenty,
avšak tyto vrcholy jsou blízko svým sousedům, což způsobuje větší vzájemné překážení jednotlivých agentů.

Algoritmy zde více profitovaly z méně omezenéných parametrů, až na \ref{str:a_star_arsg} a \nameref{subsubsec:a_star_aoid}.

\begin{table}[h]
%	\centering
	\begin{adjustwidth}{-0.5cm}{}
		\begin{tabular}{c c c c | r r D{.}{,}{2.2} D{.}{,}{2.2} D{.}{,}{7.2}}
			\toprule \\
			\pulrad{\B{Alg}} & \pulrad{\B{Omez}} & \pulrad{\B{\ref{par:ars_mnv}}} &
			\pulrad{\B{\ref{par:ars_mpc}}} & \pulrad{\B{Krok}}  & \pulrad{\B{Zam}} &
			\mc{\pulrad{\B{pAg}}} & \mc{\pulrad{\B{pZp}}} & \mc{\pulrad{\B{Čas}}} \\
			\midrule
			\nameref{sec:safe_lanes}        & -  & 1 & inf & 32785 & 13488    & 9.34                                 & 10.36                               & \multicolumn{1}{B{.}{,}{7.2}}{11.30} \\
			\hline
			\ref{str:a_star_ars}            & sr & 2 & inf & 32779 & 1338     & 14.19                                & 6.83                                & 89.03                                \\
			\ref{str:a_star_arsg}           & -  & 1 & 8   & 32785 & 6276     & 13.14                                & 9.46                                & 237.60                               \\
			\nameref{subsubsec:a_star_aoid} & -  & 1 & 8   & 16976 & 32774    & \multicolumn{1}{B{.}{,}{2.2}}{18.49} & 17.57 & 424199.33                                                          \\  % 16
			\hline
			\ref{str:cbs}                   & sr & 2 & inf & 32780 & \B{1079} & 13.91                                & \multicolumn{1}{B{.}{,}{2.2}}{6.46} & 1761.27                              \\
			\nameref{subsec:cbsoid}         & s  & 2 & inf & 2889  & 59641    & 11.99                                & 10.23                               & 2497072.06                           \\  % 16
			\hline
			\nameref{subsec:sat_rsg}        & s  & 2 & 24  & 32782 & 2868     & 13.81                                & 7.77                                & 133605.81                            \\
			\nameref{subsec:sat_ra}         & s  & 1 & 10  & 20821 & 25521    & 15.16                                & 8.57                                & 345728.00                            \\  %  8
			\bottomrule
%		\multicolumn{6}{l}{\footnotesize \textit{Pozn:}
%		\textrm{Zam} - počet zamítnutí, \textrm{pAgen} - průměrný počet agentů v jeden krok na křižovatce, \\
%		\textrm{sAgen} - směrodatná odchylka počtu agentů na křižovatce, \\
%		\textrm{Zpož} - součet spoždění přes všechny agenty, \textrm{pZpož} - průměrné zpoždění agentů
%		}  TODO
		\end{tabular}
		\caption{Porovnání algoritmů na malé oktagonální křižovatce.}\label{tab:all_exp_mala_oktagonalni}
	\end{adjustwidth}
\end{table}


\begin{table}[h]
%	\centering
	\begin{adjustwidth}{-0.5cm}{}
		\begin{tabular}{c c c c | r r D{.}{,}{2.2} D{.}{,}{2.2} D{.}{,}{7.2}}
			\toprule \\
			\pulrad{\B{Alg}} & \pulrad{\B{Omez}} & \pulrad{\B{\ref{str:ars_mnv}}} &
			\pulrad{\B{\ref{str:ars_mpc}}} & \pulrad{\B{Krok}}  & \pulrad{\B{Zam}} &
			\mc{\pulrad{\B{pAg}}} & \mc{\pulrad{\B{pZp}}} & \mc{\pulrad{\B{Čas}}} \\
			\midrule
			\nameref{sec:safe_lanes}        & -  & 1 & inf & 32796 & 32273    & 14.15                                & 18.18                               & \multicolumn{1}{B{.}{,}{7.2}}{87.44} \\
			\hline
			\ref{str:a_star_ars}            & s  & 2 & inf & 32790 & 3021     & 26.54                                & 11.08                               & 734.08                               \\
			\ref{str:varsg}           & -  & 1 & 16  & 32792 & 5872     & 25.94                                & 13.15                               & 1195.23                              \\
			\nameref{subsubsec:a_star_aoid} & s  & 2 & 16  & 32800 & 9472     & \multicolumn{1}{B{.}{,}{2.2}}{32.67} & 21.81                               & 10276.77                             \\  %  12
			\hline
			\ref{str:cbs}                   & sr & 2 & inf & 32790 & 2698     & 25.74                                & 10.80                               & 3128.19                              \\
			\nameref{subsec:cbsoid}         & s  & 2 & inf & 1704  & 93403    & 22.16                                & 21.18                               & 4255409.05                           \\  % 24
			\hline
			\nameref{subsec:sat_rsg}        & -  & 1 & 16  & 32787 & 4403     & 26.19                                & 12.08                               & 78214.86                             \\
			\nameref{subsec:sat_ra}         & -  & 1 & 10  & 32788 & \B{1905} & 26.06                                & \multicolumn{1}{B{.}{,}{2.2}}{8.85}  & 34920.59                            \\  %  8
			\bottomrule
%		\multicolumn{6}{l}{\footnotesize \textit{Pozn:}
%		\textrm{Zam} - počet zamítnutí, \textrm{pAgen} - průměrný počet agentů v jeden krok na křižovatce, \\
%		\textrm{sAgen} - směrodatná odchylka počtu agentů na křižovatce, \\
%		\textrm{Zpož} - součet spoždění přes všechny agenty, \textrm{pZpož} - průměrné zpoždění agentů
%		}  TODO
		\end{tabular}
		\caption{Porovnání algoritmů na malé hexagonální křižovatce.}\label{tab:all_exp_mala_hexagonalni}
	\end{adjustwidth}
\end{table}

Z uvedených výsledků usuzuji, že na této křižovatce se mnohem více vyplatí jednodušší, více omezené algoritmy.

\begin{table}[h]
%	\centering
	\begin{adjustwidth}{-1cm}{}
	\begin{tabular}{c c c c | r r D{.}{,}{3.2} D{.}{,}{2.2} D{.}{,}{7.2}}
		\toprule \\
		\pulrad{\B{Alg}} & \pulrad{\B{Omez}} & \pulrad{\B{\ref{par:ars_mnv}}} &
		\pulrad{\B{\ref{par:ars_mpc}}} & \pulrad{\B{Krok}}  & \pulrad{\B{Zam}} &
		\mc{\pulrad{\B{pAg}}} & \mc{\pulrad{\B{pZp}}} & \mc{\pulrad{\B{Čas}}} \\
		\midrule
		\nameref{sec:safe_lanes}        & -  & 1 & inf & 32849 & 75983     & 86.91                                 & 44.18                                & \multicolumn{1}{B{.}{,}{7.2}}{1449.67} \\
		\hline
		\ref{str:a_star_ars}            & sr & 2 & inf & 32847 & 16461     & 138.36                                & \multicolumn{1}{B{.}{,}{2.2}}{26.71} & 11011.56   \\
		\ref{str:a_star_arsg}           & sr & 2 & inf & 32847 & 22451     & 145.58                                & 43.62                                & 21190.45                               \\
		\nameref{subsubsec:a_star_aoid} & -  & 1 & 32  & 32840 & \B{11285} & \multicolumn{1}{B{.}{,}{3.2}}{159.15} & 35.46 & 204258.46  \\  % 32
		\hline
		\ref{str:cbs}                   & s  & 2 & inf & 32848 & 13193     & 137.94                                & 31.45                                & 69699.31                               \\
		\nameref{subsec:cbsoid}         & -  & 1 & 32  & 965   & 255430    & 121.15                                & 45.84                                & 7631472.87                             \\  % 32
		\bottomrule
%		\multicolumn{6}{l}{\footnotesize \textit{Pozn:}
%		\textrm{Zam} - počet zamítnutí, \textrm{pAgen} - průměrný počet agentů v jeden krok na křižovatce, \\
%		\textrm{sAgen} - směrodatná odchylka počtu agentů na křižovatce, \\
%		\textrm{Zpož} - součet spoždění přes všechny agenty, \textrm{pZpož} - průměrné zpoždění agentů
%		}  TODO
	\end{tabular}
	\caption{Porovnání algoritmů na velké čtvercové křižovatce bez výjezdů.}\label{tab:all_exp_velka_ctvercova_bez_vyjezdu}
	\end{adjustwidth}
\end{table}

\begin{table}[h]
%	\centering
	\begin{adjustwidth}{-1cm}{}
	\begin{tabular}{c c c c | r r D{.}{,}{3.2} D{.}{,}{2.2} D{.}{,}{8.2}}
		\toprule \\
		\pulrad{\B{Alg}} & \pulrad{\B{Omez}} & \pulrad{\B{\ref{str:ars_mnv}}} &
		\pulrad{\B{\ref{str:ars_mpc}}} & \pulrad{\B{Krok}}  & \pulrad{\B{Zam}} &
		\mc{\pulrad{\B{pAg}}} & \mc{\pulrad{\B{pZp}}} & \mc{\pulrad{\B{Čas}}} \\
		\midrule
		\nameref{sec:safe_lanes} & -  & 1 & inf & 32846 & 107830   & 71.98                                 & 58.92                                & \multicolumn{1}{B{.}{,}{8.2}}{1914.34} \\
		\hline
		\ref{str:a_star_ars}     & s  & 2 & inf & 32837 & \B{6046} & \multicolumn{1}{B{.}{,}{3.2}}{157.83} & 25.65 & 117705.81   \\
		\ref{str:varsg}    & -  & 1 & 32  & 32840 & 12990    & 142.63                                & \multicolumn{1}{B{.}{,}{2.2}}{19.84} & 62133.51    \\
		\hline
		\ref{str:cbs}            & sr & 2 & inf & 10221 & 184171   & 146.21                                & 26.42                                & 705203.89                              \\
		\nameref{subsec:cbsoid}  & sr & 2 & inf & 638   & 257862   & 118.79                                & 47.18                                & 11697421.69                            \\  % 32
		\bottomrule
%		\multicolumn{6}{l}{\footnotesize \textit{Pozn:}
%		\textrm{Zam} - počet zamítnutí, \textrm{pAgen} - průměrný počet agentů v jeden krok na křižovatce, \\
%		\textrm{sAgen} - směrodatná odchylka počtu agentů na křižovatce, \\
%		\textrm{Zpož} - součet spoždění přes všechny agenty, \textrm{pZpož} - průměrné zpoždění agentů
%		}  TODO
	\end{tabular}
	\caption{Porovnání algoritmů na velké oktagonální křižovatce bez výjezdů.}\label{tab:all_exp_velka_oktagonalni_bez_vyjezdu}
	\end{adjustwidth}
\end{table}

\begin{table}[h]
%	\centering
	\begin{adjustwidth}{-1cm}{}
	\begin{tabular}{c c c c | r r D{.}{,}{3.2} D{.}{,}{2.2} D{.}{,}{7.2}}
		\toprule \\
		\pulrad{\B{Alg}} & \pulrad{\B{Omez}} & \pulrad{\B{\ref{par:ars_mnv}}} &
		\pulrad{\B{\ref{par:ars_mpc}}} & \pulrad{\B{Krok}}  & \pulrad{\B{Zam}} &
		\mc{\pulrad{\B{pAg}}} & \mc{\pulrad{\B{pZp}}} & \mc{\pulrad{\B{Čas}}} \\
		\midrule
		\nameref{sec:safe_lanes} & -  & 1 & inf & 32894 & 212628    & 109.63                                & 92.37                                & \multicolumn{1}{B{.}{,}{7.2}}{4221.17} \\
		\hline
		\ref{str:a_star_ars}     & sr & 2 & inf & 32893 & \B{35567} & 287.20                                & \multicolumn{1}{B{.}{,}{2.2}}{48.34} & 128608.45  \\
		\ref{str:a_star_arsg}    & s  & 2 & inf & 20359 & 198896    & \multicolumn{1}{B{.}{,}{3.2}}{299.14} & 88.14 & 353255.32  \\
		\hline
		\ref{str:cbs}            & s  & 2 & inf & 4011  & 347852    & 292.50                                & 59.07                                & 1810300.99                             \\
		\bottomrule
%		\multicolumn{6}{l}{\footnotesize \textit{Pozn:}
%		\textrm{Zam} - počet zamítnutí, \textrm{pAgen} - průměrný počet agentů v jeden krok na křižovatce, \\
%		\textrm{sAgen} - směrodatná odchylka počtu agentů na křižovatce, \\
%		\textrm{Zpož} - součet spoždění přes všechny agenty, \textrm{pZpož} - průměrné zpoždění agentů
%		}  TODO
	\end{tabular}
	\caption{Porovnání algoritmů na velké hexagonální křižovatce bez výjezdů.}\label{tab:all_exp_velka_hexagonalni_bez_vyjezdu}
	\end{adjustwidth}
\end{table}


\begin{table}[h]
%	\centering
	\begin{adjustwidth}{-1cm}{}
		\begin{tabular}{c c c c | r r D{.}{,}{3.2} D{.}{,}{2.2} D{.}{,}{7.2}}
			\toprule \\
			\pulrad{\B{Alg}} & \pulrad{\B{Omez}} & \pulrad{\B{\ref{str:ars_mnv}}} &
			\pulrad{\B{\ref{str:ars_mpc}}} & \pulrad{\B{Krok}}  & \pulrad{\B{Zam}} &
			\mc{\pulrad{\B{pAg}}} & \mc{\pulrad{\B{pZp}}} & \mc{\pulrad{\B{Čas}}} \\
			\midrule
			\nameref{sec:safe_lanes} & -  & 1 & inf & 32850 & 111987    & 77.02                                 & 60.70                                & \multicolumn{1}{B{.}{,}{7.2}}{1244.95} \\
			\hline
			\ref{str:a_star_ars}     & sr & 2 & inf & 32846 & 26667     & 155.96                                & \multicolumn{1}{B{.}{,}{2.2}}{35.99} & 14428.00   \\
			\ref{str:varsg}    & sr & 2 & inf & 32855 & 40413     & \multicolumn{1}{B{.}{,}{3.2}}{159.59} & 52.69 & 49668.90   \\
			\hline
			\ref{str:cbs}            & sr & 2 & inf & 32846 & \B{23949} & 155.63                                & 36.41                                & 160418.65                              \\
			\nameref{subsec:cbsoid}  & -  & 1 & 64  & 957   & 255758    & 136.49                                & 52.40                                & 7736125.71                             \\  % 24
			\bottomrule
%		\multicolumn{6}{l}{\footnotesize \textit{Pozn:}
%		\textrm{Zam} - počet zamítnutí, \textrm{pAgen} - průměrný počet agentů v jeden krok na křižovatce, \\
%		\textrm{sAgen} - směrodatná odchylka počtu agentů na křižovatce, \\
%		\textrm{Zpož} - součet spoždění přes všechny agenty, \textrm{pZpož} - průměrné zpoždění agentů
%		}  TODO
		\end{tabular}
		\caption{Porovnání algoritmů na velké čtvercové křižovatce s výjezdy.}\label{tab:all_exp_velka_ctvercova_s_vyjezdy}
	\end{adjustwidth}
\end{table}

\begin{table}[h]
%	\centering
	\begin{adjustwidth}{-1cm}{}
		\begin{tabular}{c c c c | r r D{.}{,}{3.2} D{.}{,}{2.2} D{.}{,}{7.2}}
			\toprule \\
			\pulrad{\B{Alg}} & \pulrad{\B{Omez}} & \pulrad{\B{\ref{par:ars_mnv}}} &
			\pulrad{\B{\ref{par:ars_mpc}}} & \pulrad{\B{Krok}}  & \pulrad{\B{Zam}} &
			\mc{\pulrad{\B{pAg}}} & \mc{\pulrad{\B{pZp}}} & \mc{\pulrad{\B{Čas}}} \\
			\midrule
			\nameref{sec:safe_lanes} & -  & 1 & inf & 32849 & 138867    & 61.62                                 & 60.30                                & \multicolumn{1}{B{.}{,}{7.2}}{937.51} \\
			\hline
			\ref{str:a_star_ars}     & sr & 2 & inf & 32844 & \B{24714} & \multicolumn{1}{B{.}{,}{3.2}}{164.83} & \multicolumn{1}{B{.}{,}{2.2}}{34.25} & 72598.43   \\
			\ref{str:a_star_arsg}    & sr & 2 & inf & 32855 & 36807     & 163.73                                & 52.16                                & 145805.57                             \\
			\hline
			\ref{str:cbs}            & s  & 2 & inf & 6774  & 212658    & 159.57                                & 43.74                                & 1065891.74                            \\

			\bottomrule
%		\multicolumn{6}{l}{\footnotesize \textit{Pozn:}
%		\textrm{Zam} - počet zamítnutí, \textrm{pAgen} - průměrný počet agentů v jeden krok na křižovatce, \\
%		\textrm{sAgen} - směrodatná odchylka počtu agentů na křižovatce, \\
%		\textrm{Zpož} - součet spoždění přes všechny agenty, \textrm{pZpož} - průměrné zpoždění agentů
%		}  TODO
		\end{tabular}
		\caption{Porovnání algoritmů na velké oktagonální křižovatce s výjezdy.}\label{tab:all_exp_velka_oktagonalni_s_vyjezdy}
	\end{adjustwidth}
\end{table}

\begin{table}[h]
%	\centering
	\begin{adjustwidth}{-1cm}{}
		\begin{tabular}{c c c c | r r D{.}{,}{3.2} D{.}{,}{2.2} D{.}{,}{7.2}}
			\toprule \\
			\pulrad{\B{Alg}} & \pulrad{\B{Omez}} & \pulrad{\B{\ref{par:ars_mnv}}} &
			\pulrad{\B{\ref{par:ars_mpc}}} & \pulrad{\B{Krok}}  & \pulrad{\B{Zam}} &
			\mc{\pulrad{\B{pAg}}} & \mc{\pulrad{\B{pZp}}} & \mc{\pulrad{\B{Čas}}} \\
			\midrule
			\nameref{sec:safe_lanes} & -  & 1 & inf & 32893 & 234142    & 99.14                                 & 92.83                                & \multicolumn{1}{B{.}{,}{7.2}}{2399.98} \\
			\hline
			\ref{str:a_star_ars}     & sr & 2 & inf & 32894 & \B{35385} & 316.87                                & \multicolumn{1}{B{.}{,}{2.2}}{53.93} & 211489.82  \\
			\ref{str:a_star_arsg}    & s  & 2 & inf & 15656 & 240581    & \multicolumn{1}{B{.}{,}{3.2}}{331.02} & 93.02 & 459814.95  \\
			\hline
			\ref{str:cbs}            & s  & 2 & inf & 4580  & 341093    & 312.39                                & 67.51                                & 1583005.68                             \\
			\bottomrule
%		\multicolumn{6}{l}{\footnotesize \textit{Pozn:}
%		\textrm{Zam} - počet zamítnutí, \textrm{pAgen} - průměrný počet agentů v jeden krok na křižovatce, \\
%		\textrm{sAgen} - směrodatná odchylka počtu agentů na křižovatce, \\
%		\textrm{Zpož} - součet spoždění přes všechny agenty, \textrm{pZpož} - průměrné zpoždění agentů
%		}  TODO
		\end{tabular}
		\caption{Porovnání algoritmů na velké hexagonální křižovatce s výjezdy.}\label{tab:all_exp_velka_hexagonalni_s_vyjezdy}
	\end{adjustwidth}
\end{table}



%\section{Neoptimální agenti}\label{sec:neoptimalni_agenti}

%Vzájemné porovnání algoritmů při datech, kdy křižovatka má nepřesná data o agentech.
