\chapter{Experimenty}\label{ch:experimenty}

%Podrobnější popis sledovaných dat.

Po křižovatce požadujeme, aby zaručovala bezpečnou cestu autům.
Pokud agenti hlásí správné hodnoty křižovatce, algoritmy zaručují nekolizní cesty.
Předpoklady na přesnou informaci budu nyní předpokládat.
V následujících kapitolách počet kolizí neřeším, jelikož je nulový ve všech měřeních.

Následujícím faktorem je zpoždění aut.
Budu sledovat počet zamítnutých agentů a \hyperref[par:zdrzeni]{zdržení} naplánovaných agentů.
\hyperref[par:zdrzeni]{Zdržení} budu porovnávat pomocí průměru a směrodatné odchylky vypočítaných ze všech agentů.

Další zajímavý faktor je obsazenost křižovatky,
který udává v kolika krocích ze všech byl některý agent na určitém bloku křižovatky.
Budu také sledovat průměrnou délku naplánovaných cest.

Důležitá je také doba běhu algoritmu.
Budu sledovat pro každý krok, kolik uběhlo času od spuštění plánování.

Nejprve provedu testy pro porovnání parametrů jednotlivých agentů.
Poté srovnám nejlepší nastavení algoritmů proti sobě.
V poslední sekci zavedu k agentům určité nepřesnosti, které můžou způsobit kolize.
Poté budu pozorovat vliv těchto nepřesností na počet kolizí u jednotlivých algoritmů.


\section{Testovací data}\label{sec:testovaci_data}

%Popis experimentálních dat - velikost křižovatky, počet agentů a kroků, safe distance.

\subsection{Délka simulace}\label{subsec:delka_simulace}

Simulace bude přidávat agenty po $32768$ kroků.
Pokud bude plánování trvat příliš dlouhou dobu, bude simulace předčasně ukončena.
Maximální dobu běhu simulace jsem omezil na 2 hodiny.

\subsubsection{Křižovatka}

Algoritmy budu testovat na každém typu křižovatky o dvou velikostech.
\paragraph{Malá}\label{par:data_mala} křižovatka bude mít \hyperref[par:velikost_krizovatky]{velikost} rovnou~$4$.
Tato křižovatka bude mít pouze jeden \hyperref[par:vjezdy]{vjezd} a jeden \hyperref[par:vyjezdy]{výjezd}.
%\paragraph{Střední}\label{par:data_stredni} křižovatka bude mít \hyperref[par:velikost_krizovatky]{velikost}~$8$.
%\hyperref[par:vjezdy]{Počet vjezdů} a \hyperref[par:vyjezdy]{výjezdů} bude činit $3$.
\paragraph{Velká}\label{par:data_velka} křižovatka bude mít \hyperref[par:velikost_krizovatky]{velikost}~$16$.
Počet \hyperref[par:vjezdy]{vjezdů} a \hyperref[par:vyjezdy]{výjezdů} je zvýšen na $4$.

\subsubsection{Agenti}

Agenti budou nejdříve náhodně vygenerováni zvlášť pro každou velikost.
Generování proběhne dvakrát pro každou velikost, jednou pro hexagonální typ a jednou společně pro čtvercový a oktagonální typ.
Poté se ti samí agenti použijí k porovnání jednotlivých algoritmů, abych snížil vliv náhody na výsledky.
V každý kroku přibude mezi $0 - en$ agentů, kde $en$ je celkový počet vjezdů do křižovatky ze všech stran.
Přesný počet je uniformě náhodně vygenerován každý krok.

\ref{sec:agent}Délka agentů bude $0.56$ \hyperref[par:velikost_bloku]{velikosti bloku} křižovatky
a šířka $0.35$ \hyperref[par:velikost_bloku]{velikosti bloku}.
Tyto hodnoty jsem zvolil, jelikož umožňují nekolizní pozice agentů na sousedních vrcholech.
Zároveň ale agenti nejsou příliš malí na to, aby jejich velikost nehrála žádnou roli.

Dále budu porovnávat případy, kdy agenti mají daný přesný výjezd, nebo pouze směr výjezdu.
Tyto případy má cenu porovnávat pouze u
%\hyperref[par:data_stredni]{střední} a
\hyperref[par:data_velka]{velké} křižovatky, jelikož obsahuje více výjezdů.



\section{Parametry algoritmů}\label{sec:parametry_algoritmu}

%Podrobnější výsledky pro každý algoritmus zvlášť, sledování vlivu parametrů algoritmů na výsledky.
%
%Vyhodnocení nejlepších parametrů.

V této kapitole budu porovnávat vliv parametrů algoritmů na běh algoritmů.

\subsubsection{\ref{str:a_star_ars} na \hyperref[par:data_mala]{malé} křižovatce}
\label{subsubsec:exp_ars_mala_krizovatka}

Pokud bude nastaven \ref{str:ars_mnv} na $1$, omezím \ref{str:ars_mpc}
pro čtvercový a oktagonální typ na hodnotu $8$, a pro hexagonální na $16$.
Por běhy s \ref{str:ars_mnv} $2$ bude \hyperref[par:ars_mpc]{prodleva cesty} neomezená, značená hodnotou $inf$.
Jelikož má křižovatka $16$ vrcholů kromě vjezdů a výjezdů, není rozdíl mezi neomezenými cestami a
cestami omezenými na $34$ kroků pro výpočet s \ref{str:ars_mnv} nastavené na $2$.

V tabulce (Tabulka \ref{tab:ars_exp_mala}) jsou vidět výsledky na všech typech křižovatky
velikosti 4, jedním vjezdem a výjezdem.


Z výsledků je znatelné, že více omezený prohledávací prostor vede ke značně horším výsledkům,
avšak průměrný čas plánování jednoho kroku je nejnižší.

Dovolení či zakázání vracení má různý vliv na výsledky u různých typů křižovatek.
U čtvercového typu byly rozdíly mezi variantami minimální.
U oktagonálního si vedla lépe varianta s povoleným vracením, avšak u hexagonálního typu si vedla hůře.

Překvapilo mě, že oktagonální typ měl mnohem více zamítnutí než čtvercový typ.
Dle mého názoru je tento jev způsoben menším manipulativním prostorem pro auta.
Oktagonální typ oproti čtvercovému nemá čtyři rohové vrcholy.
Zároveň pokud stojí auto na čtvercovém vrcholu reprezentujícím diagonální přejezd mezi dvěma oktagonálními vrcholy,
blokuje jiné přejezdy na těchto sousedních vrcholech. % TODO obrázek
Toto je dle mého názoru i důvod, proč si varianta s vracením vedla nejlépe na této křižovatce.
Umožňuje agentovi větší pohyblivost, tudíž i více možností vyhnout se jiným agentům.

Povolené vracení vedlo k nejnižšímu zpoždění ve všech třech typech křižovatky,
ačkoliv počet zamítnutí nebyl vždy nejmenší.
Mohlo by to být způsobeno tím, že zpoždění je určeno pouze nezamítnutými agenty.
Jelikož tato varianta mimo oktagonální typ křižovatky naplánovala méně agentů,
dochází zde k menšímu ovlivňování cesty jinými agenty.
Zároveň ale tato varianta rozšiřuje množinu možných cest a tyto nové cesty jsou asi kratší,
než nalezené cesty bez vracení.

Dalším zajímavým rozdílem mezi čtvercovou a oktagonální křižovatkou je rozdíl v průměrném počtu agentů na křižovatce.
Při nejvíce omezených cestách přechod z čtvercové na oktagonální křižovatku tento průměr snížil.
Ve zbylých dvou případech průměr vzrostl, ačkoliv celkový počet cestujících agentů klesl.
Z~toho lze usoudit, že na čtvercové křižovatce byly naplánované trasy značně kratší.

\begin{table}[h]
	\centering
%	\begin{adjustwidth}{-1.5cm}{}
	\begin{tabular}{c c c c | r r D{.}{,}{2.2} D{.}{,}{2.2} D{.}{,}{3.2}}
		\toprule \\
		\pulrad{\B{Typ}} & \pulrad{\B{Omez}} & \pulrad{\B{\ref{par:ars_mnv}}} &
		\pulrad{\B{\ref{par:ars_mpc}}} & \pulrad{\B{Krok}}  & \pulrad{\B{Zam}} &
		\mc{\pulrad{\B{pAg}}} & \mc{\pulrad{\B{pZp}}} & \mc{\pulrad{\B{Čas}}} \\
		\midrule
		S & -  & 1 & 8   & 32779 & 785      & \multicolumn{1}{B{.}{,}{2.2}}{14.21} & 5.95                                 & \multicolumn{1}{B{.}{,}{3.2}}{43.19}  \\
		S & s  & 2 & inf & 32775 & \B{76}   & 13.68                                & 3.92                                 & 47.77                                 \\
		S & sr & 2 & inf & 32775 & 78       & 13.62                                & \multicolumn{1}{B{.}{,}{2.2}}{3.73}  & 44.13                                 \\
		\hline
		O & -  & 1 & 8   & 32781 & 2332     & 13.60                                & 7.72                                 & \multicolumn{1}{B{.}{,}{3.2}}{69.54}  \\
		O & s  & 2 & inf & 32780 & 1616     & \multicolumn{1}{B{.}{,}{2.2}}{14.27} & 7.66                                 & 86.42                                 \\
		O & sr & 2 & inf & 32779 & \B{1338} & 14.19                                & \multicolumn{1}{B{.}{,}{2.2}}{6.83}  & 89.03                                 \\
		\hline
		H & -  & 1 & 16  & 32795 & 6081     & 25.82                                & 13.62                                & \multicolumn{1}{B{.}{,}{3.2}}{542.81} \\
		H & s  & 2 & inf & 32790 & \B{3021} & \multicolumn{1}{B{.}{,}{2.2}}{26.54} & 11.08                                & 734.08                                \\
		H & sr & 2 & inf & 32792 & 3312     & 25.99                                & \multicolumn{1}{B{.}{,}{2.2}}{10.95} & 798.29                                \\
		\bottomrule
%		\multicolumn{6}{l}{\footnotesize \textit{Pozn:}
%		\textrm{Zam} - počet zamítnutí, \textrm{pAgen} - průměrný počet agentů v jeden krok na křižovatce, \\
%		\textrm{sAgen} - směrodatná odchylka počtu agentů na křižovatce, \\
%		\textrm{Zpož} - součet spoždění přes všechny agenty, \textrm{pZpož} - průměrné zpoždění agentů
%		}  TODO
	\end{tabular}
	\caption{Porovnání vlivu parametrů u \ref{str:a_star_ars} na různých typech malé křižovatky.}\label{tab:ars_exp_mala}
%	\end{adjustwidth}
\end{table}

\subsubsection{\ref{str:a_star_ars} na \hyperref[par:data_velka]{velké} křižovatce bez výjezdů}
\label{subsubsec:exp_ars_velka_krizovatka_bez_vyjezdu}

Pro tyto testy jsem přirozeně zvýšil omezení \ref{str:ars_mpc} pro čtvercový a oktagonální typ na hodnotu $32$,
a pro hexagonální na $64$, pokud je nastavený \ref{str:ars_mnv} na $1$.
Pro běhy s \ref{str:ars_mnv} $2$ jsou délky cest opět neomezené.

V tabulce (Tabulka \ref{tab:ars_exp_velka_bez_vyjezdu}) jsou vidět výsledky na všech typech křižovatky
velikosti $16$ se $4$ vjezdy a $4$ výjezdy.

Z výsledků je dobře vidět souvislost mezi nejmenším počtem zamítnutých agentů, nejvyšším počtem agentů na křižovatce a
nejmenším průměrným zpožděním agenta.

Avšak zbytek výsledků je dosti překvapivý.
Na čtvercovém a hexagonálním typu křižovatky byl nejlepší nejméně omezený algoritmus,
zatímco na oktagonální křižovatce vyšla značně lépe varianta s povoleným zastavováním a $2$ návštěvami vrcholu.
Přidání diagonálních přejezdů tentokrát výrazně pomohlo u všech tří běhů.

Nejpřekvapivější pro mě byly časy plánování.
Nejrychlejší u čtvercového typu byl nejvíce omezený algoritmus,
avšak u zbylých dvou typů běžel nejrychleji nejméně omezený běh.
Běh s povoleným zastavováním byl u všech tří typů nejpomalejší a pro hexagonální křižovatky ani nestihl doběhnout.
Dle mého názoru je to způsobené vysokým počtem plánování agentů, která nejsou úspěšná.
Ale aby tento fakt algoritmus zjistil musí vyzkoušet vysoký počet možností.
Pokud ale povolím vracení agenta, dle mého názoru mnoha těmto agentům umožním jet relativně krátkou trasou.

\begin{table}[h]
	\centering
%	\begin{adjustwidth}{-1.5cm}{}
	\begin{tabular}{c c c c | r r D{.}{,}{3.2} D{.}{,}{2.2} D{.}{,}{6.2}}
		\toprule \\
		\pulrad{\B{Typ}} & \pulrad{\B{Omez}} & \pulrad{\B{\ref{str:ars_mnv}}} &
		\pulrad{\B{\ref{str:ars_mpc}}} & \pulrad{\B{Krok}}  & \pulrad{\B{Zam}} &
		\mc{\pulrad{\B{pAg}}} & \mc{\pulrad{\B{pZp}}} & \mc{\pulrad{\B{Čas}}} \\
		\midrule
		S & -  & 1 & 32  & 32845 & 39931     & 132.60                                & 38.67                                & \multicolumn{1}{B{.}{,}{6.2}}{9035.89}   \\
		S & s  & 2 & inf & 32846 & 16614     & 137.41                                & 35.17                                & 14117.29                                 \\
		S & sr & 2 & inf & 32847 & \B{16461} & \multicolumn{1}{B{.}{,}{3.2}}{138.36} & \multicolumn{1}{B{.}{,}{2.2}}{26.71} & 11011.56                                 \\
		\hline
		O & -  & 1 & 32  & 32846 & 21100     & 153.26                                & 34.08                                & 74162.84                                 \\
		O & s  & 2 & inf & 32837 & \B{6046}  & \multicolumn{1}{B{.}{,}{3.2}}{157.83} & \multicolumn{1}{B{.}{,}{2.2}}{25.65} & 117705.81                                \\
		O & sr & 2 & inf & 32839 & 11305     & 149.00                                & 26.27                                & \multicolumn{1}{B{.}{,}{6.2}}{44289.81}  \\
		\hline
		H & -  & 1 & 64  & 32893 & 59119     & 282.72                                & 54.76                                & 168720.43                                \\
		H & s  & 2 & inf & 21441 & 159904    & 194.99                                & 55.64                                & 334908.27                                \\
		H & sr & 2 & inf & 32893 & \B{35567} & \multicolumn{1}{B{.}{,}{3.2}}{287.20} & \multicolumn{1}{B{.}{,}{2.2}}{48.34} & \multicolumn{1}{B{.}{,}{6.2}}{128608.45} \\
		\bottomrule
%		\multicolumn{6}{l}{\footnotesize \textit{Pozn:}
%		\textrm{Zam} - počet zamítnutí, \textrm{pAgen} - průměrný počet agentů v jeden krok na křižovatce, \\
%		\textrm{sAgen} - směrodatná odchylka počtu agentů na křižovatce, \\
%		\textrm{Zpož} - součet spoždění přes všechny agenty, \textrm{pZpož} - průměrné zpoždění agentů
%		}  TODO
	\end{tabular}
	\caption{Porovnání vlivu parametrů u \ref{str:a_star_ars} na různých typech velké křižovatky bez specifikovaných výjezdů.}
	\label{tab:ars_exp_velka_bez_vyjezdu}
%	\end{adjustwidth}
\end{table}


\subsubsection{\ref{str:a_star_ars} na \hyperref[par:data_velka]{velké} křižovatce s výjezdem}
\label{subsubsec:exp_ars_velka_krizovatka_s_vyjezdem}

Zde jsem použil totožné nastavení parametrů jako u běhů bez specifikovaných výjezdů.

V tabulce \ref{tab:ars_exp_velka_s_vyjezdy} jsou zobrazeny výsledky.

Algoritmus se zde chová většinově podle očekávání.
Nejméně omezené varianty dávají nejlepší výsledky na všech typech křižovatky.
Pokud všechny varianty doběhly, největší omezení vede k největšímu počtu zamítnutých agentů a
největšímu průměrnému zpoždění.

Jediné překvapení je ve sloupci s dobou plánování, avšak pořadí je stejné jako při nespecifikovaných výjezdech.
Na všech typech křižovatky plánovala prostřední varianta v průměru nejpomaleji.
Na čtvercové byla nejrychlejší první, nejvíce omezená varianta.
Na zbylých typech křižovatky nejméně omezené běhy.
Důvody jsou podle mě stejné jako u situace bez výjezdů.


\begin{table}[h]
	\centering
%	\begin{adjustwidth}{-1.5cm}{}
	\begin{tabular}{c c c c | r r D{.}{,}{3.2} D{.}{,}{2.2} D{.}{,}{6.2}}
		\toprule \\
		\pulrad{\B{Typ}} & \pulrad{\B{Omez}} & \pulrad{\B{\ref{str:ars_mnv}}} &
		\pulrad{\B{\ref{str:ars_mpc}}} & \pulrad{\B{Krok}}  & \pulrad{\B{Zam}} &
		\mc{\pulrad{\B{pAg}}} & \mc{\pulrad{\B{pZp}}} & \mc{\pulrad{\B{Čas}}} \\
		\midrule
		S & -  & 1 & 32  & 32855 & 56384     & 143.71                                & 49.13                                & \multicolumn{1}{B{.}{,}{6.2}}{11972.49}  \\
		S & s  & 2 & inf & 32852 & 28566     & 152.22                                & 47.19                                & 20772.51                                 \\
		S & sr & 2 & inf & 32846 & \B{26667} & \multicolumn{1}{B{.}{,}{3.2}}{155.96} & \multicolumn{1}{B{.}{,}{2.2}}{35.99} & 14428.00  \\
		\hline
		O & -  & 1 & 32  & 32849 & 37589     & 157.28                                & 43.90                                & 86298.33                                 \\
		O & s  & 2 & inf & 32848 & 25990     & 162.69                                & 38.05                                & 157218.08                                \\
		O & sr & 2 & inf & 32844 & \B{24714} & \multicolumn{1}{B{.}{,}{3.2}}{164.83} & \multicolumn{1}{B{.}{,}{2.2}}{34.25} & \multicolumn{1}{B{.}{,}{6.2}}{72598.43}  \\
		\hline
		H & -  & 1 & 64  & 28256 & 105284    & 300.94                                & 61.20                                & 253753.98                                \\
		H & s  & 2 & inf & 13850 & 241515    & 135.19                                & 55.48                                & 519920.12                                \\
		H & sr & 2 & inf & 32894 & \B{35385} & \multicolumn{1}{B{.}{,}{3.2}}{316.87} & \multicolumn{1}{B{.}{,}{2.2}}{53.93} & \multicolumn{1}{B{.}{,}{6.2}}{211489.82} \\
		\bottomrule
%		\multicolumn{6}{l}{\footnotesize \textit{Pozn:}
%		\textrm{Zam} - počet zamítnutí, \textrm{pAgen} - průměrný počet agentů v jeden krok na křižovatce, \\
%		\textrm{sAgen} - směrodatná odchylka počtu agentů na křižovatce, \\
%		\textrm{Zpož} - součet spoždění přes všechny agenty, \textrm{pZpož} - průměrné zpoždění agentů
%		}  TODO
	\end{tabular}
	\caption{Porovnání vlivu parametrů u \ref{str:a_star_ars} na různých typech velké křižovatky se specifikovanými výjezdy.}
	\label{tab:ars_exp_velka_s_vyjezdy}
%	\end{adjustwidth}
\end{table}



\section{Hromadné výsledky}\label{sec:hromadne_vysledky}

Porovnání algoritmů mezi sebou s nejlepšími parametry.

Porovnání čtvercové a oktagonální křižovatky.


\section{Neoptimální agenti}\label{sec:neoptimalni_agenti}

Vzájemné porovnání algoritmů při datech, kdy křižovatka má nepřesná data o agentech.
