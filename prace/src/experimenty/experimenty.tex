\chapter{Experimenty}\label{ch:experimenty}

%Podrobnější popis sledovaných dat.

Po křižovatce požadujeme, aby zaručovala bezpečnou cestu autům.
Nyní budu předpokládat, že agenti dávají přesnou informaci křižovatce a plně dodržují plán.
Za tohoto předpokladu nám algoritmus zaručuje nekolizní trasy.

Následujícím faktorem je zpoždění aut.
Budu sledovat \hyperref[par:zamitnuti]{počet zamítnutých agentů} a \hyperref[par:zdrzeni]{zdržení} naplánovaných agentů.
V tabulkách budu porovnávat celkový \hyperref[par:zamitnuti]{počet zamítnutých agentů} (Zam).
U \hyperref[par:zdrzeni]{zdržení} budu počítat celkový součet (Zpož)
a průměrné \hyperref[par:zdrzeni]{zdržení} na agenta (pZpož) vypočítané ze všech agentů.

Další zajímavý faktor je obsazenost křižovatky,
který udává v kolika krocích ze všech byl některý agent na určitém bloku křižovatky.
Ve výsledkách budu počítat průměrný počet agentů na křižovatce (pAgen) přes všechny kroky.
Budu také sledovat průměrnou délku naplánovaných cest.

Důležitá je také doba běhu algoritmu.
Budu sledovat pro každý krok, kolik uběhlo času od spuštění plánování daného kroku.

Nejprve provedu testy pro porovnání parametrů jednotlivých agentů.
Poté srovnám nejlepší nastavení algoritmů proti sobě.
%V poslední sekci zavedu k agentům určité nepřesnosti, které můžou způsobit kolize.
%Poté budu pozorovat vliv těchto nepřesností na počet kolizí u jednotlivých algoritmů.


\section{Testovací data}\label{sec:testovaci_data}

%Popis experimentálních dat - velikost křižovatky, počet agentů a kroků, safe distance.

\subsection{Délka simulace}\label{subsec:delka_simulace}

Simulace bude přidávat agenty po $32768$ kroků.
Pokud bude plánování trvat příliš dlouhou dobu, bude simulace předčasně ukončena.
Maximální dobu běhu simulace jsem omezil na 2 hodiny.

\subsubsection{Křižovatka}

Algoritmy budu testovat na každém typu křižovatky o dvou velikostech.
\paragraph{Malá}\label{par:data_mala} křižovatka bude mít \hyperref[par:velikost_krizovatky]{velikost} rovnou~$4$.
Tato křižovatka bude mít pouze jeden \hyperref[par:vjezdy]{vjezd} a jeden \hyperref[par:vyjezdy]{výjezd}.
%\paragraph{Střední}\label{par:data_stredni} křižovatka bude mít \hyperref[par:velikost_krizovatky]{velikost}~$8$.
%\hyperref[par:vjezdy]{Počet vjezdů} a \hyperref[par:vyjezdy]{výjezdů} bude činit $3$.
\paragraph{Velká}\label{par:data_velka} křižovatka bude mít \hyperref[par:velikost_krizovatky]{velikost}~$16$.
Počet \hyperref[par:vjezdy]{vjezdů} a \hyperref[par:vyjezdy]{výjezdů} je zvýšen na $4$.

\subsubsection{Agenti}

Agenti budou nejdříve náhodně vygenerováni zvlášť pro každou velikost.
Generování proběhne dvakrát pro každou velikost, jednou pro hexagonální typ a jednou společně pro čtvercový a oktagonální typ.
Poté se ti samí agenti použijí k porovnání jednotlivých algoritmů, abych snížil vliv náhody na výsledky.
V každý kroku přibude mezi $0 - en$ agentů, kde $en$ je celkový počet vjezdů do křižovatky ze všech stran.
Přesný počet je uniformě náhodně vygenerován každý krok.

\ref{sec:agent}Délka agentů bude $0.56$ \hyperref[par:velikost_bloku]{velikosti bloku} křižovatky
a šířka $0.35$ \hyperref[par:velikost_bloku]{velikosti bloku}.
Tyto hodnoty jsem zvolil, jelikož umožňují nekolizní pozice agentů na sousedních vrcholech.
Zároveň ale agenti nejsou příliš malí na to, aby jejich velikost nehrála žádnou roli.

Dále budu porovnávat případy, kdy agenti mají daný přesný výjezd, nebo pouze směr výjezdu.
Tyto případy má cenu porovnávat pouze u
%\hyperref[par:data_stredni]{střední} a
\hyperref[par:data_velka]{velké} křižovatky, jelikož obsahuje více výjezdů.



\section{Parametry algoritmů}\label{sec:parametry_algoritmu}

%Podrobnější výsledky pro každý algoritmus zvlášť, sledování vlivu parametrů algoritmů na výsledky.
%
%Vyhodnocení nejlepších parametrů.

V této kapitole budu porovnávat vliv parametrů algoritmů na běh algoritmů.
V obecnosti budu testovat algoritmy s hodně omezenými parametry a s málo omezujícím nastavením parametrů.

\subsection{\ref{sec:a_star} porovnání parametrů}\label{subsec:a_star_porovnani_parametru}

V této kapitole porovnám vliv parametrů u algoritmů \ref{str:a_star_ars}, \ref{str:a_star_arsg} a \ref{subsubsec:a_star_aoid}.

Všechny parametry algoritmu \hyperref[par:ars_mnv]{maximum návštěv vrcholu}, \hyperref[par:ars_pz]{povolené zastavování},
\hyperref[par:ars_mpc]{maximální prodleva při cestě} i \hyperref[par:ars_pv]{povolené vracení} omezují prohledávací
prostor algoritmu.
Proto bych čekal s větším omezením kratší dobu plánování, avšak za cenu horších výsledků.

Pro všechny typy křižovatky vyzkouším omezit \hyperref[par:ars_mnv]{maximum návštěv vrcholu (\ref{par:ars_mnv})}
na $1$ či $2$.
Pokud bude hodnota \ref{par:ars_mnv} nastavena na $1$, omezím
\hyperref[par:ars_mpc]{maximální prodlevu cesty (\ref{par:ars_mpc})} podle typu a velikosti křižovatky.
Pro čtvercovou a oktagonální na hodnotu $8$, a pro hexagonální většinou na $16$.
Zároveň nedovolím agentům \hyperref[par:ars_pz]{zastavování} (\ref{par:ars_pz})
ani \hyperref[par:ars_pv]{vracení} (\ref{par:ars_pv}).

Při nastavení \ref{par:ars_mnv} na $2$, \hyperref[par:ars_mpc]{maximální prodlevu cesty}
neomezím většinou vůbec \nameref{subsubsec:a_star_aoid}.
Vyzkouším možnosti, kdy dovolím agentům pouze \hyperref[par:ars_pz]{zastavování},
nebo \hyperref[par:ars_pz]{zastavování} a zároveň \hyperref[par:ars_pv]{vracení} (\ref{par:ars_pv}).

\subsubsection{\ref{str:a_star_ars} na \hyperref[par:data_mala]{malé} křižovatce}
\label{subsubsec:exp_ars_mala_krizovatka}

Pokud bude nastaven \ref{par:ars_mnv} na $1$, omezím \ref{par:ars_mpc}
pro čtvercový a oktagonální typ na hodnotu $8$, a pro hexagonální převážně na $16$.
Por běhy s \ref{par:ars_mnv} $2$ bude \hyperref[par:ars_mpc]{prodleva cesty} neomezená, značená hodnotou $neom$.
Jelikož má křižovatka $16$ vrcholů kromě vjezdů a výjezdů, není rozdíl mezi neomezenými cestami a
cestami omezenými na $34$ kroků pro výpočet s \ref{par:ars_mnv} nastavené na $2$.

V tabulce (Tabulka \ref{tab:ars_exp_mala}) jsou vidět výsledky na všech typech křižovatky
velikostí $4$ a jedním vjezdem a výjezdem.

Nejhorší výsledek dalo nastavení, kdy agent směl navštívit každý vrchol nejvýše jednou a délka cesty byla neomezená.

\begin{table}[b!]
	\begin{adjustwidth}{-1cm}{}
		\begin{tabular}{c c c c | r r D{.}{,}{2.2} r D{.}{,}{2.2} D{.}{,}{3.2}}
			\toprule \\
			\pulrad{\textbf{Typ}} & \pulrad{\textbf{Omez}} & \pulrad{\textbf{\ref{par:ars_mnv}}} &
			\pulrad{\textbf{\ref{par:ars_mpc}}} & \pulrad{\textbf{Krok}}  & \pulrad{\textbf{Zam}} &
			\mc{\pulrad{\textbf{pAg}}} & \pulrad{\textbf{Zpož}} &
			\mc{\pulrad{\textbf{pZp}}} & \mc{\pulrad{\textbf{Čas}}} \\
			\midrule
			S & n  & 1 & 8   & 32793 & 785           & \multicolumn{1}{B{.}{,}{2.2}}{14.21} & 384062           & 5.95  & \multicolumn{1}{B{.}{,}{2.2}}{43.19}  \\
			S & s  & 2 & inf & 32793 & \textbf{76}   & 13.68                                & 255588           & 3.92  & 47.77                                 \\
			S & sr & 2 & inf & 32793 & 78            & 13.62                                & \textbf{243381}  & 3.73  & 44.13                                 \\
			\hline
			O & n  & 1 & 8   & 32782 & 2332          & 13.60                                & 486431           & 7.72  & \multicolumn{1}{B{.}{,}{2.2}}{69.54}  \\
			O & s  & 2 & inf & 32782 & 1616          & \multicolumn{1}{B{.}{,}{2.2}}{14.27} & 488003           & 7.66  & 86.42                                 \\
			O & sr & 2 & inf & 32782 & \textbf{1338} & 14.19                                & \textbf{437360}  & 6.83  & 89.03                                 \\
			\hline
			H & n  & 1 & 16  & 32802 & 6081          & 25.82                                & 1256874          & 13.62 & \multicolumn{1}{B{.}{,}{3.2}}{542.81} \\
			H & s  & 2 & inf & 32802 & \textbf{3021} & \multicolumn{1}{B{.}{,}{2.2}}{26.54} & 1056474          & 11.08 & 734.08 \\
			H & sr & 2 & inf & 32802 & 3312          & 25.99                                & \textbf{1041074} & 10.95 & 798.29                                \\
			\bottomrule
%		\multicolumn{6}{l}{\footnotesize \textit{Pozn:}
%		\textrm{Zam} - počet zamítnutí, \textrm{pAgen} - průměrný počet agentů v jeden krok na křižovatce, \\
%		\textrm{sAgen} - směrodatná odchylka počtu agentů na křižovatce, \\
%		\textrm{Zpož} - součet spoždění přes všechny agenty, \textrm{pZpož} - průměrné zpoždění agentů
%		}  TODO
		\end{tabular}
		\caption{Porovnání vlivu parametrů u \ref{str:a_star_ars} na různých typech křižovatky.}\label{tab:ars_exp_mala}
	\end{adjustwidth}
\end{table}

%
%\begin{table}[b!]
%	\centering
%	\begin{tabular}{c c c c | r r D{.}{,}{2.2}D{.}{,}{1.2} r D{.}{,}{2.2} D{.}{,}{4.1}}
%		\toprule \\
%		\pulrad{\textbf{Typ}} & \pulrad{\textbf{Omez}} & \pulrad{\textbf{\ref{par:ars_mnv}}} &
%		\pulrad{\textbf{\ref{par:ars_mpc}}} & \pulrad{\textbf{Kroky}} & \pulrad{\textbf{Zam}} & \mc{\pulrad{\textbf{pAgen}}} &
%		\mc{\pulrad{\textbf{sAgen}}} & \pulrad{\textbf{Zpož}} & \mc{\pulrad{\textbf{pZpož}}} \\
%		\midrule
%		1 & 0 & \textbf{701} & \multicolumn{1}{B{.}{,}{2.2}}{11.85} & \multicolumn{1}{B{.}{,}{1.2}}{2.06}
%		& \textbf{267\,141} & \multicolumn{1}{B{.}{,}{1.2}}{4.13} \\
%		ar_n_1_8: 32793, 568, 14.20, 1.85, 370408, 5.72 & 262.71136 \\
%		ar_sr_2_inf: 32793, 2397, 14.79, 1.57, 571703, 9.08 & 258.56440 \\
%		ar_s_2_inf: 32793, 4130, 14.84, 1.52, 653819, 10.68 & 218.39257 \\
%		\hline
%		ar_n_1_8: 32786, 6276, 13.14, 1.59, 558883, 9.46 & 237.59660 \\
%		ar_rs_2_inf: 32789, 10357, 13.48, 1.58, 700508, 12.74 & 284.35575 \\
%		ar_s_2_inf: 32789, 12220, 13.34, 1.55, 713116, 13.42 & 278.53918 \\
%		\hline
%		asg_n_1_16: 32801, 5872, 25.94, 2.21, 1216694, 13.15 & 1195.22746 \\
%		asg_sr_2_inf: 32801, 15252, 26.26, 1.97, 1649275, 19.84 & 1484.20947 \\
%		asg_s_2_inf: 32802, 17624, 26.35, 1.95, 1691950, 20.95 & 1208.47404 \\
%		\bottomrule
%%		\multicolumn{6}{l}{\footnotesize \textit{Pozn:}
%%		\textrm{Zam} - počet zamítnutí, \textrm{pAgen} - průměrný počet agentů v jeden krok na křižovatce, \\
%%		\textrm{sAgen} - směrodatná odchylka počtu agentů na křižovatce, \\
%%		\textrm{Zpož} - součet spoždění přes všechny agenty, \textrm{pZpož} - průměrné zpoždění agentů
%%		}  TODO
%	\end{tabular}
%	\caption{Porovnání vlivu \ref{par:ars_mnv} a \ref{par:ars_mpc} u \ref{str:a_star_arsg} na \hyperref[par:data_mala]{malém} čtv. typu.}\label{tab:arsg_exp_male_ctvercova}
%\end{table}
%
%\begin{table}[b!]
%	\centering
%	\begin{tabular}{c c c c | r r D{.}{,}{2.2}D{.}{,}{1.2} r D{.}{,}{2.2} D{.}{,}{7}}
%		\toprule \\
%		\pulrad{\textbf{Typ}} & \pulrad{\textbf{Omez}} & \pulrad{\textbf{\ref{par:ars_mnv}}} &
%		\pulrad{\textbf{\ref{par:ars_mpc}}} & \pulrad{\textbf{Kroky}} & \pulrad{\textbf{Zam}} & \mc{\pulrad{\textbf{pAgen}}} &
%		\mc{\pulrad{\textbf{sAgen}}} & \pulrad{\textbf{Zpož}} & \mc{\pulrad{\textbf{pZpož}}} \\
%		\midrule
%		1 & 0 & \textbf{701} & \multicolumn{1}{B{.}{,}{2.2}}{11.85} & \multicolumn{1}{B{.}{,}{1.2}}{2.06}
%		& \textbf{267\,141} & \multicolumn{1}{B{.}{,}{1.2}}{4.13} \\
%		aoid_n_1_8_16: 32787, 1373, 18.04, 2.31, 966321, 15.10 & 5582.03456 \\
%		aoid_sr_2_inf: 32791, 4586, 20.52, 2.53, 1155703, 19.02 & 120604.89302 \\
%		aoid_s_2_inf: 32793, 3382, 19.95, 2.53, 1097273, 17.71 & 34867.76845 \\
%		\hline
%		aoid_n_1_8_16: 16977, 32774, 18.49, 2.68, 572267, 17.57 & 424199.33218 \\
%		aoid_s_2_inf: 16977, 62691, 1.68, 5.60, 48636, 18.28 & 5038828.24188 \\
%		\hline
%		aoid_n_1_12_16: 3956, 87635, 32.77, 3.62, 269081, 25.07 & 1956276.56642 \\
%		aoid_s_2_16_12: 32801, 9472, 32.67, 2.62, 1938903, 21.81 & 10276.77279 \\
%		aoid_s_2_inf_14: 32801, 9482, 34.11, 2.90, 2088963, 23.50 & 42300.28815 \\
%		\bottomrule
%%		\multicolumn{6}{l}{\footnotesize \textit{Pozn:}
%%		\textrm{Zam} - počet zamítnutí, \textrm{pAgen} - průměrný počet agentů v jeden krok na křižovatce, \\
%%		\textrm{sAgen} - směrodatná odchylka počtu agentů na křižovatce, \\
%%		\textrm{Zpož} - součet spoždění přes všechny agenty, \textrm{pZpož} - průměrné zpoždění agentů
%%		}  TODO
%	\end{tabular}
%	\caption{Porovnání vlivu \ref{par:ars_mnv} a \ref{par:ars_mpc} u \ref{subsubsec:a_star_aoid} na \hyperref[par:data_mala]{malém} čtv. typu.}\label{tab:aoid_exp_male_ctvercova}
%\end{table}



\section{Hromadné výsledky}\label{sec:hromadne_vysledky}

Porovnání algoritmů mezi sebou s nejlepšími parametry.

Porovnání čtvercové a oktagonální křižovatky.


%\section{Neoptimální agenti}\label{sec:neoptimalni_agenti}

%Vzájemné porovnání algoritmů při datech, kdy křižovatka má nepřesná data o agentech.
