\subsubsection{\ref{str:a_star_ars} na \hyperref[par:data_mala]{malé} křižovatce}
\label{subsubsec:exp_ars_mala_krizovatka}

Pokud bude nastaven \ref{par:ars_mnv} na $1$, omezím \ref{par:ars_mpc}
pro čtvercový a oktagonální typ na hodnotu $8$, a pro hexagonální na $16$.
Por běhy s \ref{par:ars_mnv} $2$ bude \hyperref[par:ars_mpc]{prodleva cesty} neomezená, značená hodnotou $inf$.
Jelikož má křižovatka $16$ vrcholů kromě vjezdů a výjezdů, není rozdíl mezi neomezenými cestami a
cestami omezenými na $34$ kroků pro výpočet s \ref{par:ars_mnv} nastavené na $2$.

V tabulce (Tabulka \ref{tab:ars_exp_mala}) jsou vidět výsledky na všech typech křižovatky
velikostí $4$ a jedním vjezdem a výjezdem.


Z výsledků je znatelné, že více omezený prohledávací prostor vede ke značně horším výsledkům,
avšak průměrný čas plánování jednoho kroku je nejnižší.

Dovolení či zakázání vracení má různý vliv na výsledky u různých typů křižovatek.
U čtvercového typu byly rozdíly mezi variantami minimální.
U oktagonálního si vedla lépe varianta s povoleným vracením, avšak u hexagonálního typu si vedla hůře.

Překvapilo mě, že oktagonální typ měl mnohem více zamítnutí než čtvercový typ.
Dle mého názoru je tento jev způsoben menším manipulativním prostorem pro auta.
Oktagonální typ oproti čtvercovému nemá čtyři rohové vrcholy.
Zároveň pokud stojí auto na čtvercovém vrcholu reprezentující diagonální mezi dvěma oktagonálními vrcholy,
blokuje jiné přejezdy na těchto sousedních vrcholech. % TODO obrázek
Toto je dle mého názoru i důvod, proč si varianta s vracením vedla nejlépe na této křižovatce.
Umožňuje agentovi větší pohyblivost, tudíž i více možností vyhnout se jiným agentům.

Povolené vracení vedlo k nejnižšímu zpoždění ve všech třech typech křižovatky, ačkoliv počet zamítnutí nebyl nejmenší.
Mohlo by to být způsobeno tím, že do zpoždění se u zamítnutých agentů počítá pouze čekací doba před křižovatkou,
zatímco pro nezamítnuté agenty čekací doba a zároveň prodleva cesty oproti optimální.
Pokud tedy mnoho agentů čeká dlouho vev frontě a poté jedou značně delší cestou, mají v součtu vyšší zpoždění,
než kdyby byli všichni agenti zamítnutí.

Dalším zajímavým rozdílem mezi čtvercovou a oktagonální křižovatkou je rozdíl v průměrném počtu agentů na křižovatce.
Při nejvíce omezených cestách se přechod z čtvercové na oktagonální křižovatku se tento průměr snížil.
Ve zbylých dvou případech průměr vzrostl, ačkoliv celkový počet cestujících agentů klesl.
Z toho lze usoudit, že na čtvercové křižovatce byly naplánované trasy značně kratší.

\begin{table}[h]
	\centering
%	\begin{adjustwidth}{-1.5cm}{}
	\begin{tabular}{c c c c | r r D{.}{,}{2.2} D{.}{,}{2.2} D{.}{,}{3.2}}
		\toprule \\
		\pulrad{\B{Typ}} & \pulrad{\B{Omez}} & \pulrad{\B{\ref{par:ars_mnv}}} &
		\pulrad{\B{\ref{par:ars_mpc}}} & \pulrad{\B{Krok}}  & \pulrad{\B{Zam}} &
		\mc{\pulrad{\B{pAg}}} & \mc{\pulrad{\B{pZp}}} & \mc{\pulrad{\B{Čas}}} \\
		\midrule
		S & -  & 1 & 8   & 32779 & 785      & \multicolumn{1}{B{.}{,}{2.2}}{14.21} & 5.95                                 & \multicolumn{1}{B{.}{,}{3.2}}{43.19}  \\
		S & s  & 2 & inf & 32775 & \B{76}   & 13.68                                & 3.92                                 & 47.77                                 \\
		S & sr & 2 & inf & 32775 & 78       & 13.62                                & \multicolumn{1}{B{.}{,}{2.2}}{3.73}  & 44.13                                 \\
		\hline
		O & -  & 1 & 8   & 32781 & 2332     & 13.60                                & 7.72                                 & \multicolumn{1}{B{.}{,}{3.2}}{69.54}  \\
		O & s  & 2 & inf & 32780 & 1616     & \multicolumn{1}{B{.}{,}{2.2}}{14.27} & 7.66                                 & 86.42                                 \\
		O & sr & 2 & inf & 32779 & \B{1338} & 14.19                                & \multicolumn{1}{B{.}{,}{2.2}}{6.83}  & 89.03                                 \\
		\hline
		H & -  & 1 & 16  & 32795 & 6081     & 25.82                                & 13.62                                & \multicolumn{1}{B{.}{,}{3.2}}{542.81} \\
		H & s  & 2 & inf & 32790 & \B{3021} & \multicolumn{1}{B{.}{,}{2.2}}{26.54} & 11.08                                & 734.08                                \\
		H & sr & 2 & inf & 32792 & 3312     & 25.99                                & \multicolumn{1}{B{.}{,}{2.2}}{10.95} & 798.29                                \\
		\bottomrule
%		\multicolumn{6}{l}{\footnotesize \textit{Pozn:}
%		\textrm{Zam} - počet zamítnutí, \textrm{pAgen} - průměrný počet agentů v jeden krok na křižovatce, \\
%		\textrm{sAgen} - směrodatná odchylka počtu agentů na křižovatce, \\
%		\textrm{Zpož} - součet spoždění přes všechny agenty, \textrm{pZpož} - průměrné zpoždění agentů
%		}  TODO
	\end{tabular}
	\caption{Porovnání vlivu parametrů u \ref{str:a_star_ars} na různých typech malé křižovatky.}\label{tab:ars_exp_mala}
%	\end{adjustwidth}
\end{table}

\subsubsection{\ref{str:a_star_ars} na \hyperref[par:data_velka]{velké} křižovatce bez výjezdů}
\label{subsubsec:exp_ars_velka_krizovatka_bez_vyjezdu}

Pro tyto testy jsem přirozeně zvýšil omezení \ref{par:ars_mpc} pro čtvercový a oktagonální typ na hodnotu $32$,
a pro hexagonální na $64$, pokud je nastavený \ref{par:ars_mnv} na $1$.
Por běhy s \ref{par:ars_mnv} $2$ jsou délky cest opět neomezené.

V tabulce (Tabulka \ref{tab:ars_exp_velka_bez_vyjezdu}) jsou vidět výsledky na všech typech křižovatky
velikostí $16$ se $4$ vjezdy a $4$ výjezdy.

Z výsledků je dobře vidět souvislost mezi nejmenším počtem zamítnutých agentů, nejvyšším počtem agentů na křižovatce a
nejmenším průměrným zpožděním agenta.

Avšak zbytek výsledků je dosti překvapivý.
Na čtvercovém a hexagonálním typu křižovatky byl nejlepší nejméně omezený algoritmus,
zatímco na oktagonální křižovatce vyšla značně lépe varianta s povoleným zastavováním a $2$ návštěvami vrcholu.
Přidání diagonálních přejezdů tentokrát výrazně pomohlo u všech tří běhů.

Nejpřekvapivější pro mě byly časy plánování.
Nejrychlejší u čtvercového typu byl nejvíce omezený algoritmus,
avšak u zbylých dvou typů běžel nejrychleji nejméně omezený běh.
Běh s povoleným zastavováním byl u všech tří typů nejpomalejší a pro hexagonální křižovatky ani nestihl doběhnout.
Dle mého názoru je to způsobené vysokým počtem plánování agentů, které nejsou úspěšné.
Ale aby tento fakt algoritmus zjistil musí vyzkoušet vysoká počet možností.
Pokud ale povolím vracení agenta, dle mého názoru mnoha těmto agentům umožním jet relativně krátkou trasou.

\begin{table}[h]
	\centering
%	\begin{adjustwidth}{-1.5cm}{}
	\begin{tabular}{c c c c | r r D{.}{,}{3.2} D{.}{,}{2.2} D{.}{,}{6.2}}
		\toprule \\
		\pulrad{\B{Typ}} & \pulrad{\B{Omez}} & \pulrad{\B{\ref{str:ars_mnv}}} &
		\pulrad{\B{\ref{str:ars_mpc}}} & \pulrad{\B{Krok}}  & \pulrad{\B{Zam}} &
		\mc{\pulrad{\B{pAg}}} & \mc{\pulrad{\B{pZp}}} & \mc{\pulrad{\B{Čas}}} \\
		\midrule
		S & -  & 1 & 32  & 32845 & 39931     & 132.60                                & 38.67                                & \multicolumn{1}{B{.}{,}{6.2}}{9035.89}   \\
		S & s  & 2 & inf & 32846 & 16614     & 137.41                                & 35.17                                & 14117.29                                 \\
		S & sr & 2 & inf & 32847 & \B{16461} & \multicolumn{1}{B{.}{,}{3.2}}{138.36} & \multicolumn{1}{B{.}{,}{2.2}}{26.71} & 11011.56                                 \\
		\hline
		O & -  & 1 & 32  & 32846 & 21100     & 153.26                                & 34.08                                & 74162.84                                 \\
		O & s  & 2 & inf & 32837 & \B{6046}  & \multicolumn{1}{B{.}{,}{3.2}}{157.83} & \multicolumn{1}{B{.}{,}{2.2}}{25.65} & 117705.81                                \\
		O & sr & 2 & inf & 32839 & 11305     & 149.00                                & 26.27                                & \multicolumn{1}{B{.}{,}{6.2}}{44289.81}  \\
		\hline
		H & -  & 1 & 64  & 32893 & 59119     & 282.72                                & 54.76                                & 168720.43                                \\
		H & s  & 2 & inf & 21441 & 159904    & 194.99                                & 55.64                                & 334908.27                                \\
		H & sr & 2 & inf & 32893 & \B{35567} & \multicolumn{1}{B{.}{,}{3.2}}{287.20} & \multicolumn{1}{B{.}{,}{2.2}}{48.34} & \multicolumn{1}{B{.}{,}{6.2}}{128608.45} \\
		\bottomrule
%		\multicolumn{6}{l}{\footnotesize \textit{Pozn:}
%		\textrm{Zam} - počet zamítnutí, \textrm{pAgen} - průměrný počet agentů v jeden krok na křižovatce, \\
%		\textrm{sAgen} - směrodatná odchylka počtu agentů na křižovatce, \\
%		\textrm{Zpož} - součet spoždění přes všechny agenty, \textrm{pZpož} - průměrné zpoždění agentů
%		}  TODO
	\end{tabular}
	\caption{Porovnání vlivu parametrů u \ref{str:a_star_ars} na různých typech velké křižovatky bez specifikovaných výjezdů.}
	\label{tab:ars_exp_velka_bez_vyjezdu}
%	\end{adjustwidth}
\end{table}


\subsubsection{\ref{str:a_star_ars} na \hyperref[par:data_velka]{velké} křižovatce s výjezdem}
\label{subsubsec:exp_ars_velka_krizovatka_s_vyjezdem}

Zde jsem použil totožné nastavení parametrů jako u běhů bez specifikovaní výjezdů.

V tabulce \ref{tab:ars_exp_velka_s_vyjezdy} jsou zobrazeny výsledky.

Algoritmus se zde chová většinově podle očekávání.
Nejméně omezené varianty dávají nejlepší výsledky na všech typech křižovatky.
Pokud všechny varianty doběhly, největší omezení vede k největšímu počtu zamítnutých agentů a
největšímu průměrnému zpoždění.

Jediné překvapení je ve sloupci s dobou plánování, avšak pořadí je stejné jako při nespecifikovaných výjezdech.
Na všech typech křižovatky plánovala prostřední varianta v průměru nejpomaleji.
Na čtvercové byla nejrychlejší první, nejvíce omezená varianta.
Na zbylých typech křižovatky nejméně omezené běhy.
Důvody jsou podle mě stejné jako u situace bez výjezdů.


\begin{table}[h]
	\centering
%	\begin{adjustwidth}{-1.5cm}{}
	\begin{tabular}{c c c c | r r D{.}{,}{3.2} D{.}{,}{2.2} D{.}{,}{6.2}}
		\toprule \\
		\pulrad{\B{Typ}} & \pulrad{\B{Omez}} & \pulrad{\B{\ref{str:ars_mnv}}} &
		\pulrad{\B{\ref{str:ars_mpc}}} & \pulrad{\B{Krok}}  & \pulrad{\B{Zam}} &
		\mc{\pulrad{\B{pAg}}} & \mc{\pulrad{\B{pZp}}} & \mc{\pulrad{\B{Čas}}} \\
		\midrule
		S & -  & 1 & 32  & 32855 & 56384     & 143.71                                & 49.13                                & \multicolumn{1}{B{.}{,}{6.2}}{11972.49}  \\
		S & s  & 2 & inf & 32852 & 28566     & 152.22                                & 47.19                                & 20772.51                                 \\
		S & sr & 2 & inf & 32846 & \B{26667} & \multicolumn{1}{B{.}{,}{3.2}}{155.96} & \multicolumn{1}{B{.}{,}{2.2}}{35.99} & 14428.00  \\
		\hline
		O & -  & 1 & 32  & 32849 & 37589     & 157.28                                & 43.90                                & 86298.33                                 \\
		O & s  & 2 & inf & 32848 & 25990     & 162.69                                & 38.05                                & 157218.08                                \\
		O & sr & 2 & inf & 32844 & \B{24714} & \multicolumn{1}{B{.}{,}{3.2}}{164.83} & \multicolumn{1}{B{.}{,}{2.2}}{34.25} & \multicolumn{1}{B{.}{,}{6.2}}{72598.43}  \\
		\hline
		H & -  & 1 & 64  & 28256 & 105284    & 300.94                                & 61.20                                & 253753.98                                \\
		H & s  & 2 & inf & 13850 & 241515    & 135.19                                & 55.48                                & 519920.12                                \\
		H & sr & 2 & inf & 32894 & \B{35385} & \multicolumn{1}{B{.}{,}{3.2}}{316.87} & \multicolumn{1}{B{.}{,}{2.2}}{53.93} & \multicolumn{1}{B{.}{,}{6.2}}{211489.82} \\
		\bottomrule
%		\multicolumn{6}{l}{\footnotesize \textit{Pozn:}
%		\textrm{Zam} - počet zamítnutí, \textrm{pAgen} - průměrný počet agentů v jeden krok na křižovatce, \\
%		\textrm{sAgen} - směrodatná odchylka počtu agentů na křižovatce, \\
%		\textrm{Zpož} - součet spoždění přes všechny agenty, \textrm{pZpož} - průměrné zpoždění agentů
%		}  TODO
	\end{tabular}
	\caption{Porovnání vlivu parametrů u \ref{str:a_star_ars} na různých typech velké křižovatky se specifikovanými výjezdy.}
	\label{tab:ars_exp_velka_s_vyjezdy}
%	\end{adjustwidth}
\end{table}
