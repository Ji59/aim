\subsubsection{\ref{str:a_star_ars} na \hyperref[par:data_mala]{malé} křižovatce}
\label{subsubsec:exp_ars_mala_krizovatka}

Pokud bude nastaven \ref{par:ars_mnv} na $1$, omezím \ref{par:ars_mpc}
pro čtvercový a oktagonální typ na hodnotu $8$, a pro hexagonální převážně na $16$.
Por běhy s \ref{par:ars_mnv} $2$ bude \hyperref[par:ars_mpc]{prodleva cesty} neomezená, značená hodnotou $inf$.
Jelikož má křižovatka $16$ vrcholů kromě vjezdů a výjezdů, není rozdíl mezi neomezenými cestami a
cestami omezenými na $34$ kroků pro výpočet s \ref{par:ars_mnv} nastavené na $2$.

V tabulce (Tabulka \ref{tab:ars_exp_mala}) jsou vidět výsledky na všech typech křižovatky
velikostí $4$ a jedním vjezdem a výjezdem.

\begin{table}[h]
	\centering
%	\begin{adjustwidth}{-1.5cm}{}
	\begin{tabular}{c c c c | r r D{.}{,}{2.2} D{.}{,}{2.2} D{.}{,}{3.2}}
		\toprule \\
		\pulrad{\B{Typ}} & \pulrad{\B{Omez}} & \pulrad{\B{\ref{par:ars_mnv}}} &
		\pulrad{\B{\ref{par:ars_mpc}}} & \pulrad{\B{Krok}}  & \pulrad{\B{Zam}} &
		\mc{\pulrad{\B{pAg}}} & \mc{\pulrad{\B{pZp}}} & \mc{\pulrad{\B{Čas}}} \\
		\midrule
		S & -  & 1 & 8   & 32779 & 785      & \multicolumn{1}{B{.}{,}{2.2}}{14.21} & 5.95                                 & \multicolumn{1}{B{.}{,}{3.2}}{43.19}  \\
		S & s  & 2 & inf & 32775 & \B{76}   & 13.68                                & 3.92                                 & 47.77                                 \\
		S & sr & 2 & inf & 32775 & 78       & 13.62                                & \multicolumn{1}{B{.}{,}{2.2}}{3.73}  & 44.13                                 \\
		\hline
		O & -  & 1 & 8   & 32781 & 2332     & 13.60                                & 7.72                                 & \multicolumn{1}{B{.}{,}{3.2}}{69.54}  \\
		O & s  & 2 & inf & 32780 & 1616     & \multicolumn{1}{B{.}{,}{2.2}}{14.27} & 7.66                                 & 86.42                                 \\
		O & sr & 2 & inf & 32779 & \B{1338} & 14.19                                & \multicolumn{1}{B{.}{,}{2.2}}{6.83}  & 89.03                                 \\
		\hline
		H & -  & 1 & 16  & 32795 & 6081     & 25.82                                & 13.62                                & \multicolumn{1}{B{.}{,}{3.2}}{542.81} \\
		H & s  & 2 & inf & 32790 & \B{3021} & \multicolumn{1}{B{.}{,}{2.2}}{26.54} & 11.08                                & 734.08                                \\
		H & sr & 2 & inf & 32792 & 3312     & 25.99                                & \multicolumn{1}{B{.}{,}{2.2}}{10.95} & 798.29                                \\
		\bottomrule
%		\multicolumn{6}{l}{\footnotesize \textit{Pozn:}
%		\textrm{Zam} - počet zamítnutí, \textrm{pAgen} - průměrný počet agentů v jeden krok na křižovatce, \\
%		\textrm{sAgen} - směrodatná odchylka počtu agentů na křižovatce, \\
%		\textrm{Zpož} - součet spoždění přes všechny agenty, \textrm{pZpož} - průměrné zpoždění agentů
%		}  TODO
	\end{tabular}
	\caption{Porovnání vlivu parametrů u \ref{str:a_star_ars} na různých typech malé křižovatky.}\label{tab:ars_exp_mala}
%	\end{adjustwidth}
\end{table}

Z výsledků je znatelné, že více omezený prohledávací prostor vede ke značně horším výsledkům,
avšak průměrný čas plánování jednoho kroku je nejnižší.

Dovolení či zakázání vracení má různý vliv na výsledky u různých typů křižovatek.
U čtvercového typu byly rozdíly mezi variantami minimální.
U oktagonálního si vedla lépe varianta s povoleným vracením, avšak u hexagonálního typu si vedla hůře.

Překvapilo mě, že oktagonální typ měl mnohem více zamítnutí než čtvercový typ.
Dle mého názoru je tento jev způsoben menším manipulativním prostorem pro auta.
Oktagonální typ oproti čtvercovému nemá čtyři rohové vrcholy.
Zároveň pokud stojí auto na čtvercovém vrcholu reprezentující diagonální mezi dvěma oktagonálními vrcholy,
blokuje jiné přejezdy na těchto sousedních vrcholech. % TODO obrázek
Toto je dle mého názoru i důvod, proč si varianta s vracením vedla nejlépe na této křižovatce.
Umožňuje agentovi větší pohyblivost, tudíž i více možností vyhnout se jiným agentům.

Povolené vracení vedlo k nejnižšímu zpoždění ve všech třech typech křižovatky, ačkoliv počet zamítnutí nebyl nejmenší.
Mohlo by to být způsobeno tím, že do zpoždění se u zamítnutých agentů počítá pouze čekací doba před křižovatkou,
zatímco pro nezamítnuté agenty čekací doba a zároveň prodleva cesty oproti optimální.
Pokud tedy mnoho agentů čeká dlouho vev frontě a poté jedou značně delší cestou, mají v součtu vyšší zpoždění,
než kdyby byli všichni agenti zamítnutí.

Dalším zajímavým rozdílem mezi čtvercovou a oktagonální křižovatkou je rozdíl v průměrném počtu agentů na křižovatce.
Při nejvíce omezených cestách se přechod z čtvercové na oktagonální křižovatku se tento průměr snížil.
Ve zbylých dvou případech průměr vzrostl, ačkoliv celkový počet cestujících agentů klesl.
Z toho lze usoudit, že na čtvercové křižovatce byly naplánované trasy značně kratší.

\subsubsection{\ref{str:a_star_ars} na \hyperref[par:data_velka]{velké} křižovatce bez výjezdů}
\label{subsubsec:exp_ars_velka_krizovatka_bez_vyjezdu}

\begin{table}[b!]
	\centering
%	\begin{adjustwidth}{-1.5cm}{}
	\begin{tabular}{c c c c | r r D{.}{,}{3.2} D{.}{,}{2.2} D{.}{,}{6.2}}
		\toprule \\
		\pulrad{\B{Typ}} & \pulrad{\B{Omez}} & \pulrad{\B{\ref{par:ars_mnv}}} &
		\pulrad{\B{\ref{par:ars_mpc}}} & \pulrad{\B{Krok}}  & \pulrad{\B{Zam}} &
		\mc{\pulrad{\B{pAg}}} & \mc{\pulrad{\B{pZp}}} & \mc{\pulrad{\B{Čas}}} \\
		\midrule
		S & -  & 1 & 32  & 32845 & 39931     & 132.60                                & 38.67                                & \multicolumn{1}{B{.}{,}{6.2}}{9035.89}   \\
		S & s  & 2 & inf & 32846 & 16614     & 137.41                                & 35.17                                & 14117.29                                 \\
		S & sr & 2 & inf & 32847 & \B{16461} & \multicolumn{1}{B{.}{,}{3.2}}{138.36} & \multicolumn{1}{B{.}{,}{2.2}}{26.71} & 11011.56                                 \\
		\hline
		O & -  & 1 & 32  & 32846 & 21100     & 153.26                                & 34.08                                & 74162.84                                 \\
		O & s  & 2 & inf & 32837 & \B{6046}  & \multicolumn{1}{B{.}{,}{3.2}}{157.83} & \multicolumn{1}{B{.}{,}{2.2}}{25.65} & 117705.81                                \\
		O & sr & 2 & inf & 32839 & 11305     & 149.00                                & 26.27                                & \multicolumn{1}{B{.}{,}{6.2}}{44289.81}  \\
		\hline
		H & -  & 1 & 64  & 32893 & 59119     & 282.72                                & 54.76                                & 168720.43                                \\
		H & s  & 2 & inf & 21441 & 159904    & 194.99                                & 55.64                                & 334908.27                                \\
		H & sr & 2 & inf & 32893 & \B{35567} & \multicolumn{1}{B{.}{,}{3.2}}{287.20} & \multicolumn{1}{B{.}{,}{2.2}}{48.34} & \multicolumn{1}{B{.}{,}{6.2}}{128608.45} \\
		\bottomrule
%		\multicolumn{6}{l}{\footnotesize \textit{Pozn:}
%		\textrm{Zam} - počet zamítnutí, \textrm{pAgen} - průměrný počet agentů v jeden krok na křižovatce, \\
%		\textrm{sAgen} - směrodatná odchylka počtu agentů na křižovatce, \\
%		\textrm{Zpož} - součet spoždění přes všechny agenty, \textrm{pZpož} - průměrné zpoždění agentů
%		}  TODO
	\end{tabular}
	\caption{Porovnání vlivu parametrů u \ref{str:a_star_ars} na různých typech velké křižovatky bez specifikovaných výjezdů.}
	\label{tab:ars_exp_velka_bez_vyjezdu}
%	\end{adjustwidth}
\end{table}

\begin{table}[b!]
%	\centering
	\begin{adjustwidth}{-1cm}{}
		\begin{tabular}{c c c c c | r r D{.}{,}{2.2} D{.}{,}{2.2} D{.}{,}{7.2}}
			\toprule \\
			\pulrad{\B{Typ}} & \pulrad{\B{Omez}} & \pulrad{\B{\ref{par:ars_mnv}}} &
			\pulrad{\B{\ref{par:ars_mpc}}} & \pulrad{\B{\ref{par:aoid_mpa}}} & \pulrad{\B{Krok}} &
			\pulrad{\B{Zam}} & \mc{\pulrad{\B{pAg}}} & \mc{\pulrad{\B{pZp}}} & \mc{\pulrad{\B{Čas}}} \\
			\midrule
%		1 & 0 & \B{701} & \multicolumn{1}{B{.}{,}{2.2}}{11.85} & \multicolumn{1}{B{.}{,}{1.2}}{2.06}
%		& \B{267\,141} & \multicolumn{1}{B{.}{,}{1.2}}{4.13} \\
			S & -  & 1 & 8   & 16 & 32786 & \B{1373}  & 18.04                                & \multicolumn{1}{B{.}{,}{2.2}}{15.10} & \multicolumn{1}{B{.}{,}{7.2}}{5582.03}   \\
			S & s  & 2 & inf & 16 & 32792 & 3382      & 19.95                                & 17.71                                & 34867.77                                 \\
			S & sr & 2 & inf & 16 & 32790 & 4586      & \multicolumn{1}{B{.}{,}{2.2}}{20.52} & 19.02                                & 120604.89                                \\
			\hline
			O & -  & 1 & 8   & 16 & 16976 & \B{32774} & \multicolumn{1}{B{.}{,}{2.2}}{18.49} & \multicolumn{1}{B{.}{,}{2.2}}{17.57} & \multicolumn{1}{B{.}{,}{7.2}}{424199.33} \\
			O & s  & 2 & inf & 16 & 1443  & 62691     & 1.68                                 & 18.28                                & 5038828.24                               \\
			\hline
			H & -  & 1 & 12  & 16 & 3955  & 87635     & 32.77                                & 25.07                                & 1956276.57                               \\
			H & s  & 2 & 16  & 12 & 32800 & \B{9472}  & 32.67                                & \multicolumn{1}{B{.}{,}{2.2}}{21.81} & \multicolumn{1}{B{.}{,}{7.2}}{10276.77}  \\
			H & s  & 2 & inf & 14 & 32800 & 9482      & \multicolumn{1}{B{.}{,}{2.2}}{34.11} & 23.50                                & 42300.29                                 \\
			\bottomrule
%		\multicolumn{6}{l}{\footnotesize \textit{Pozn:}
%		\textrm{Zam} - počet zamítnutí, \textrm{pAgen} - průměrný počet agentů v jeden krok na křižovatce, \\
%		\textrm{sAgen} - směrodatná odchylka počtu agentů na křižovatce, \\
%		\textrm{Zpož} - součet spoždění přes všechny agenty, \textrm{pZpož} - průměrné zpoždění agentů
%		}  TODO
		\end{tabular}
		\caption{Porovnání vlivu parametrů u \nameref{subsubsec:a_star_aoid} na různých typech malé křižovatky.}\label{tab:aoid_exp_mala}
	\end{adjustwidth}
\end{table}
\input{experimenty/aoid_big_table}
\begin{table}[b!]
	\centering
%	\begin{adjustwidth}{-1.5cm}{}
	\begin{tabular}{c c c c | r r D{.}{,}{2.2} D{.}{,}{2.2} D{.}{,}{4.2}}
		\toprule \\
		\pulrad{\B{Typ}} & \pulrad{\B{Omez}} & \pulrad{\B{\ref{par:ars_mnv}}} &
		\pulrad{\B{\ref{par:ars_mpc}}} & \pulrad{\B{Krok}}  & \pulrad{\B{Zam}} &
		\mc{\pulrad{\B{pAg}}} & \mc{\pulrad{\B{pZp}}} & \mc{\pulrad{\B{Čas}}} \\
		\midrule
%		1 & 0 & \B{701} & \multicolumn{1}{B{.}{,}{2.2}}{11.85} & \multicolumn{1}{B{.}{,}{1.2}}{2.06}
%		& \B{267\,141} & \multicolumn{1}{B{.}{,}{1.2}}{4.13} \\
		S & -  & 1 & 8   & 32779 & \B{568}  & 14.20                                & \multicolumn{1}{B{.}{,}{2.2}}{5.72}  & 262.71                                 \\
		S & s  & 2 & inf & 32787 & 4130     & \multicolumn{1}{B{.}{,}{2.2}}{14.84} & 10.68                                & \multicolumn{1}{B{.}{,}{4.2}}{218.39}  \\
		S & sr & 2 & inf & 32780 & 2397     & 14.79                                & 9.08                                 & 258.56                                 \\
		\hline
		O & -  & 1 & 8   & 32785 & \B{6276} & 13.14                                & \multicolumn{1}{B{.}{,}{2.2}}{9.46}  & \multicolumn{1}{B{.}{,}{4.2}}{237.60}  \\
		O & s  & 2 & inf & 32789 & 12220    & 13.34                                & 13.42                                & 278.54                                 \\
		O & sr & 2 & inf & 32787 & 10357    & \multicolumn{1}{B{.}{,}{2.2}}{13.48} & 12.74                                & 284.36                                 \\
		\hline
		H & -  & 1 & 16  & 32792 & \B{5872} & 25.94                                & \multicolumn{1}{B{.}{,}{2.2}}{13.15} & \multicolumn{1}{B{.}{,}{4.2}}{1195.23} \\
		H & s  & 2 & inf & 32801 & 17624    & \multicolumn{1}{B{.}{,}{2.2}}{26.35} & 20.95                                & 1208.47                                \\
		H & sr & 2 & inf & 32798 & 15252    & 26.26                                & 19.84                                & 1484.21                                \\
		\bottomrule
%		\multicolumn{6}{l}{\footnotesize \textit{Pozn:}
%		\textrm{Zam} - počet zamítnutí, \textrm{pAgen} - průměrný počet agentů v jeden krok na křižovatce, \\
%		\textrm{sAgen} - směrodatná odchylka počtu agentů na křižovatce, \\
%		\textrm{Zpož} - součet spoždění přes všechny agenty, \textrm{pZpož} - průměrné zpoždění agentů
%		}  TODO
	\end{tabular}
	\caption{Porovnání vlivu parametrů u \ref{str:a_star_arsg} na různých typech malé křižovatky.}\label{tab:arsg_exp_mala}
%	\end{adjustwidth}
\end{table}
\input{experimenty/arsg_big_table}
\begin{table}[h]
	\centering
%	\begin{adjustwidth}{-1.5cm}{}
	\begin{tabular}{c c c c | r r D{.}{,}{2.2} D{.}{,}{2.2} D{.}{,}{5.2}}
		\toprule \\
		\pulrad{\B{Typ}} & \pulrad{\B{Omez}} & \pulrad{\B{\ref{str:ars_mnv}}} &
		\pulrad{\B{\ref{str:ars_mpc}}} & \pulrad{\B{Krok}}  & \pulrad{\B{Zam}} &
		\mc{\pulrad{\B{pAg}}} & \mc{\pulrad{\B{pZp}}} & \mc{\pulrad{\B{Čas}}} \\
		\midrule
		S & -  & 1 & inf & 32779 & 915      & 14.14                                & 6.00                                 & \multicolumn{1}{B{.}{,}{5.2}}{132.36}  \\
		S & s  & 2 & inf & 32777 & 82       & \multicolumn{1}{B{.}{,}{2.2}}{13.41} & \multicolumn{1}{B{.}{,}{2.2}}{3.62}  & 169.34   \\
		S & sr & 2 & inf & 32776 & \B{70}   & \multicolumn{1}{B{.}{,}{2.2}}{13.41} & 3.66                                 & 194.12                                 \\
		\hline
		O & -  & 1 & 16  & 32779 & 2264     & 13.40                                & 7.57                                 & 6902.67                                \\
		O & s  & 2 & inf & 32781 & 1490     & \multicolumn{1}{B{.}{,}{2.2}}{13.96} & 7.39                                 & 2595.20                                \\
		O & sr & 2 & inf & 32780 & \B{1079} & 13.91                                & \multicolumn{1}{B{.}{,}{2.2}}{6.46}  & \multicolumn{1}{B{.}{,}{5.2}}{1761.27} \\
		\hline
		H & -  & 1 & 24  & 32792 & 5846     & 25.48                                & 13.71                                & 24000.77                               \\
		H & s  & 2 & inf & 32793 & 2878     & \multicolumn{1}{B{.}{,}{2.2}}{26.17} & 11.15                                & 3762.95                                \\
		H & sr & 2 & inf & 32790 & \B{2698} & 25.74                                & \multicolumn{1}{B{.}{,}{2.2}}{10.80} & \multicolumn{1}{B{.}{,}{5.2}}{3128.19} \\
		\bottomrule
%		\multicolumn{6}{l}{\footnotesize \textit{Pozn:}
%		\textrm{Zam} - počet zamítnutí, \textrm{pAgen} - průměrný počet agentů v jeden krok na křižovatce, \\
%		\textrm{sAgen} - směrodatná odchylka počtu agentů na křižovatce, \\
%		\textrm{Zpož} - součet spoždění přes všechny agenty, \textrm{pZpož} - průměrné zpoždění agentů
%		}  TODO
	\end{tabular}
	\caption{Porovnání vlivu parametrů u \ref{str:cbs} na různých typech malé křižovatky.}\label{tab:cbssg_exp_mala}
%	\end{adjustwidth}
\end{table}

\begin{table}[h]
%	\centering
	\begin{adjustwidth}{-1cm}{}
		\begin{tabular}{c c c c c | r r D{.}{,}{2.2} D{.}{,}{2.2} D{.}{,}{7.2}}
			\toprule \\
			\pulrad{\B{Typ}} & \pulrad{\B{Omez}} & \pulrad{\B{\ref{str:ars_mnv}}} &
			\pulrad{\B{\ref{str:ars_mpc}}} & \pulrad{\B{\ref{str:aoid_mpa}}} & \pulrad{\B{Krok}} &
			\pulrad{\B{Zam}} & \mc{\pulrad{\B{pAg}}} & \mc{\pulrad{\B{pZp}}} & \mc{\pulrad{\B{Čas}}} \\
			\midrule
			S & -  & 1 & 16  & 16  & 6838 & \B{51759} & \multicolumn{1}{B{.}{,}{2.2}}{12.61} & \multicolumn{1}{B{.}{,}{2.2}}{8.45}  & \multicolumn{1}{B{.}{,}{7.2}}{1054679.77} \\
			S & s  & 2 & inf & 24  & 3755 & 57904     & 7.06                                 & 10.50                                & 1920988.21                                \\ % TODO
			S & sr & 2 & inf & 24  & 1530 & 62295     & 2.88                                 & 10.77                                & 2044461.87                                \\ % TODO 24?
			\hline
			O & -  & 1 & 16  & inf & 2466 & 60424     & \multicolumn{1}{B{.}{,}{2.2}}{12.11} & \multicolumn{1}{B{.}{,}{2.2}}{8.69}  & 2699251.20                                \\
			O & s  & 2 & inf & 16  & 2889 & \B{59641} & 11.99                                & 10.23                                & \multicolumn{1}{B{.}{,}{7.2}}{2497072.06} \\  % TODO 16?
			O & sr & 2 & inf & 16  & 855  & 63627     & 4.16                                 & 9.45                                 & 2675054.74                                \\  % TODO 16?
			\hline
			H & -  & 1 & 24  & 24  & 1227 & 94802     & 21.09                                & 19.49                                & 5906285.18                                \\
			H & s  & 2 & inf & 24  & 1704 & \B{93403} & \multicolumn{1}{B{.}{,}{2.2}}{22.16} & 21.18                                & \multicolumn{1}{B{.}{,}{7.2}}{4255409.05} \\  % TODO 24?
			H & sr & 2 & inf & 24  & 1229 & 94792     & 21.25                                & \multicolumn{1}{B{.}{,}{2.2}}{18.52} & 5893074.13                                \\  % TODO 24?
			\bottomrule
%		\multicolumn{6}{l}{\footnotesize \textit{Pozn:}
%		\textrm{Zam} - počet zamítnutí, \textrm{pAgen} - průměrný počet agentů v jeden krok na křižovatce, \\
%		\textrm{sAgen} - směrodatná odchylka počtu agentů na křižovatce, \\
%		\textrm{Zpož} - součet spoždění přes všechny agenty, \textrm{pZpož} - průměrné zpoždění agentů
%		}  TODO
		\end{tabular}
		\caption{Porovnání vlivu parametrů u \nameref{subsec:cbsoid} na různých typech malé křižovatky.}\label{tab:cbsoid_exp_mala}
	\end{adjustwidth}
\end{table}
\begin{table}[h]
	\centering
%	\begin{adjustwidth}{-1.5cm}{}
	\begin{tabular}{c c c c | r r D{.}{,}{2.2} D{.}{,}{2.2} D{.}{,}{7.2}}
		\toprule \\
		\pulrad{\B{Typ}} & \pulrad{\B{Omez}} & \pulrad{\B{\ref{par:ars_mnv}}} &
		\pulrad{\B{\ref{par:ars_mpc}}} & \pulrad{\B{Krok}}  & \pulrad{\B{Zam}} &
		\mc{\pulrad{\B{pAg}}} & \mc{\pulrad{\B{pZp}}} & \mc{\pulrad{\B{Čas}}} \\
		\midrule
		S & - & 1 & 16 & 32782 & 1095     & \multicolumn{1}{B{.}{,}{2.2}}{14.29} & 5.99                                & \multicolumn{1}{B{.}{,}{7.2}}{4109.71}  \\
		S & s & 2 & 24 & 32778 & \B{255}  & 13.46                                & \multicolumn{1}{B{.}{,}{2.2}}{3.90} & 131515.34                               \\
		\hline
		O & - & 1 & 16 & 32784 & 4657     & 13.22                                & 8.65                                & \multicolumn{1}{B{.}{,}{7.2}}{15154.02} \\
		O & s & 2 & 24 & 32782 & \B{2868} & \multicolumn{1}{B{.}{,}{2.2}}{13.81} & \multicolumn{1}{B{.}{,}{2.2}}{7.77}  & 133605.81  \\
		\hline
		H & - & 1 & 16 & 32787 & \B{4403} & \multicolumn{1}{B{.}{,}{2.2}}{26.19} & 12.08                               & \multicolumn{1}{B{.}{,}{7.2}}{78214.86} \\
		H & s & 2 & 22 & 0     & 94573    & 1.02                                 & \multicolumn{1}{B{.}{,}{2.2}}{7.60} & 7200869.77                              \\
		\bottomrule
%		\multicolumn{6}{l}{\footnotesize \textit{Pozn:}
%		\textrm{Zam} - počet zamítnutí, \textrm{pAgen} - průměrný počet agentů v jeden krok na křižovatce, \\
%		\textrm{sAgen} - směrodatná odchylka počtu agentů na křižovatce, \\
%		\textrm{Zpož} - součet spoždění přes všechny agenty, \textrm{pZpož} - průměrné zpoždění agentů
%		}  TODO
	\end{tabular}
	\caption{Porovnání vlivu parametrů u \nameref{subsec:sat_rsg} na různých typech malé křižovatky.}\label{tab:sat_exp_mala}
%	\end{adjustwidth}
\end{table}

\begin{table}[h]
	\centering
%	\begin{adjustwidth}{-1cm}{}
	\begin{tabular}{c c c c c | r r D{.}{,}{2.2} D{.}{,}{1.2} D{.}{,}{8.2}}
		\toprule \\
		\pulrad{\B{Typ}} & \pulrad{\B{Omez}} & \pulrad{\B{\ref{par:ars_mnv}}} &
		\pulrad{\B{\ref{par:ars_mpc}}} & \pulrad{\B{\ref{par:aoid_mpa}}} & \pulrad{\B{Krok}} &
		\pulrad{\B{Zam}} & \mc{\pulrad{\B{pAg}}} & \mc{\pulrad{\B{pZp}}} & \mc{\pulrad{\B{Čas}}} \\
		\midrule
		S & - & 1 & 10 & 12 & 32776 & \B{363}   & \multicolumn{1}{B{.}{,}{2.2}}{15.07} & 5.44                                & \multicolumn{1}{B{.}{,}{8.2}}{9893.10}    \\
		S & - & 2 & 14 & 12 & 16634 & 32273     & 7.56                                 & \multicolumn{1}{B{.}{,}{1.2}}{4.58} & 432905.36                                 \\
		\hline
		O & - & 1 & 10 & 8  & 9289  & 47575     & \multicolumn{1}{B{.}{,}{2.2}}{15.16} & 8.74                                & 297656.39                                 \\
		O & s & 1 & 10 & 8  & 20821 & \B{25521} & \multicolumn{1}{B{.}{,}{2.2}}{15.16} & 8.57                                & 345728.00                                 \\
		O & s & 2 & 9  & 9  & 2740  & 59957     & 1.98                                 & 5.15                                & \multicolumn{1}{B{.}{,}{8.2}}{2632600.55} \\
		O & s & 2 & 9  & 10 & 99    & 65125     & 0.07                                 & \multicolumn{1}{B{.}{,}{1.2}}{3.07} & 30123061.30                               \\
		\hline
		H & - & 1 & 10 & 8  & 32788 & \B{1905}  & \multicolumn{1}{B{.}{,}{2.2}}{26.06} & 8.85                                & \multicolumn{1}{B{.}{,}{8.2}}{34920.59}   \\
		H & s & 2 & 12 & 8  & 11448 & 64109     & 9.30                                 & \multicolumn{1}{B{.}{,}{2.2}}{1.22} & 1240239.54                                \\
		\bottomrule
%		\multicolumn{6}{l}{\footnotesize \textit{Pozn:}
%		\textrm{Zam} - počet zamítnutí, \textrm{pAgen} - průměrný počet agentů v jeden krok na křižovatce, \\
%		\textrm{sAgen} - směrodatná odchylka počtu agentů na křižovatce, \\
%		\textrm{Zpož} - součet spoždění přes všechny agenty, \textrm{pZpož} - průměrné zpoždění agentů
%		}  TODO
	\end{tabular}
	\caption{Porovnání vlivu parametrů u \nameref{subsec:sat_ra} na různých typech malé křižovatky.}\label{tab:sata_exp_mala}
%	\end{adjustwidth}
\end{table}


