\subsection{\ref{str:a_star_ars} porovnání parametrů}\label{subsec:ars_porovnani_parametru}

Všechny parametry algoritmu \hyperref[par:ars_mnv]{maximum návštěv vrcholu}, \hyperref[par:ars_pz]{povolené zastavování},
\hyperref[par:ars_mpc]{maximální prodleva při cestě} i \hyperref[par:ars_pv]{povolené vracení} omezují prohledávací
prostor algoritmu.
Proto bych čekal s větším omezením kratší dobu plánování, avšak za cenu horších výsledků.
Výsledky by mohly být závislé na kombinaci hodnot parametrů.
Proto budu testovat různé kombinace parametrů namísto měnění pouze jednoho z nich.
Zároveň budu měnit hodnoty v závislosti na křižovatce.

\subsubsection{\nameref{par:data_mala} \hyperref[subsec:ctvercovy_typ]{čtvercová} křižovatka}
\label{subsubsec:exp_ars_mala_ctvercova_krizovatka}

Nejprve budu testovat algoritmus s nastavením \hyperref[par:ars_mnv]{maximuma návštěv vrcholu (\ref{par:ars_mnv})} na $1, 2, 4$.
Nepředpokládám ale, že na této malé křižovatce bude mít hodnota parametru velký vliv na výsledky.
Dále budu používat hodnoty $0, 4, 16$ a $neom.$ pro \hyperref[par:ars_mpc]{maximální prodlevu cesty (\ref{par:ars_mpc})}.
Zde hodnota $neom$ znamená neomezenou délku cesty.
Jelikož má křižovatka $16$ vrcholů kromě vjezdů a výjezdů, není rozdíl mezi neomezenými cestami a
cestami omezenými na $16$ kroků pro výpočet s \ref{par:ars_mnv} nastavené na jedna.
Zároveň pokud je hodnota \ref{par:ars_mpc} rovna nule, nemůže se stát,
že by algoritmus naplánoval cestu obsahující více stejných vrcholů.
Proto nemá význam testovat parametry s \ref{par:ars_mnv} větší než jedna s nulovým nastavením \ref{par:ars_mpc}.

V tabulce (Tabulka \ref{tab:ars_exp_male_ctvercova}) jsou vidět výsledky pro tyto dva parametry.
Z výsledků překvapivě vyplývá, že největší omezení agentů vede k nejmenšímu zdržení a nejméně zamítnutým agentům.
Při tomto omezení mohou agenti jezdit pouze cestami srovnatelnými s jízdou v pruhu.
Další, už ne tolik překvapivý výsledek je, že v tomto nastavení je nejméně agentů na křižovatce.

O něco hůře si vedly testy s $\ref{par:ars_mpc} = 4$.
Měly o něco více zamítnutých agentů, avšak výrazně stoupl jak počet zamítnutých agentů, tak i zpoždění agentů.
Testy s nastavením \ref{par:ars_mpc} na $16$ mají totožné výsledky s testy bez omezení délky cest.
Stejně tak zvýšení maximálních návštěv vrcholu ze $2$ na $4$ nemělo žádný vliv.
Z toho usuzuji, že agenti navštívili každý vrchol nejvýše dvakrát, a nejdelší cesta byla kratší než $16$.

Nejhorší výsledek dalo nastavení, kdy agent směl navštívit každý vrchol nejvýše jednou a délka cesty byla neomezená.

\begin{table}[b!]
	\centering
	\begin{tabular}{c c r D{.}{,}{2.2}D{.}{,}{1.2} r D{.}{,}{1.2}}
		\toprule \\
		\pulrad{\textbf{\ref{par:ars_mnv}}} & \pulrad{\textbf{\ref{par:ars_mpc}}} & \pulrad{\textbf{Zam}} &
		\mc{\pulrad{\textbf{pAgen}}} & \mc{\pulrad{\textbf{sAgen}}} & \pulrad{\textbf{Zpož}} & \mc{\pulrad{\textbf{pZpož}}} \\
		\midrule
		1 & 0 & \textbf{701} & \multicolumn{1}{B{.}{,}{2.2}}{11.85} & \multicolumn{1}{B{.}{,}{1.2}}{2.06}
		& \textbf{267\,141} & \multicolumn{1}{B{.}{,}{1.2}}{4.13} \\
		1 & 4    & 837    & 14.09 & 2.12 & 376\,332 & 5.82 \\
		1 & neom & 1\,147 & 14.22 & 2.10 & 401\,513 & 6.24 \\
		2 & 4    & 764    & 14.18 & 2.14 & 380\,404 & 5.88 \\
		2 & 16   & 1\,064 & 14.32 & 2.13 & 402\,771 & 6.26 \\
		2 & neom & 1\,064 & 14.32 & 2.13 & 402\,771 & 6.26 \\
		4 & 4    & 764    & 14.18 & 2.14 & 380\,404 & 5.88 \\
		4 & 16   & 1\,054 & 14.32 & 2.13 & 402\,743 & 6.25 \\
		4 & neom & 1\,054 & 14.32 & 2.13 & 402\,743 & 6.25 \\
		\bottomrule
%		\multicolumn{6}{l}{\footnotesize \textit{Pozn:}
%		\textrm{Zam} - počet zamítnutí, \textrm{pAgen} - průměrný počet agentů v jeden krok na křižovatce, \\
%		\textrm{sAgen} - směrodatná odchylka počtu agentů na křižovatce, \\
%		\textrm{Zpož} - součet spoždění přes všechny agenty, \textrm{pZpož} - průměrné zpoždění agentů
%		}  TODO
	\end{tabular}
	\caption{Porovnání vlivu \ref{par:ars_mnv} a \ref{par:ars_mpc} u \ref{str:a_star_ars} na \hyperref[par:data_mala]{malém} čtv. typu.}\label{tab:ars_exp_male_ctvercova}
\end{table}

Dále budu testova zbylé parametry.
Zastavování a vracení algoritmů má vliv pouze pokud je \ref{par:ars_mnv} větší než jedna
a zároveň je hodnota \ref{par:ars_mpc} nenulová.
Proto budu testovat pouze tyto možnosti.

Tabulka \ref{tab:ars_exp_male_ctvercova_z} obsahuje výsledky pro parametry s povoleným zastavováním.
Všechny testy daly mnohem lepší výsledky než testy bez zastavování
jak v počtu zamítnutých agentů, tak v průměrném zpoždění agentů.
Zde opět má nejmenší počet zamítnutých agentů a celkové zdržení nejvíce omezený algoritmus.
Počet agentů na křižovatce v jeden krok a průměrné zpoždění agenta je u všech testů srovnatelný.
Opět zde nejsou rozdíli mezi neomezenými cestami a běhy s \ref{par:ars_mpc} $16$.
Nejvíce zamítnutých agentů měl tentokrát běh s nejvýše $4$ návštěvami vrcholů a prodlevou cesty nejvýše $4$.

\begin{table}[b!]
	\centering
	\begin{tabular}{c c r D{.}{,}{2.2}D{.}{,}{1.2} r D{.}{,}{1.2}}
		\toprule \\
		\pulrad{\textbf{\ref{par:ars_mnv}}} & \pulrad{\textbf{\ref{par:ars_mpc}}} & \pulrad{\textbf{Zam}} &
		\mc{\pulrad{\textbf{pAgen}}} & \mc{\pulrad{\textbf{sAgen}}} & \pulrad{\textbf{Zpož}} & \mc{\pulrad{\textbf{pZpož}}} \\
		\midrule
		2 & 4    & \textbf{166} & 13.63 & 2.44 & \textbf{261\,459} & 4.00 \\
		2 & 16   & 184          & 13.68 & 2.46 & 266\,204          & 4.08 \\
		2 & neom & 184          & 13.68 & 2.46 & 266\,204          & 4.08 \\
		4 & 4    & 201          & 13.62 & 2.44 & 264\,071          & 4.05 \\
		4 & 16   & 189          & 13.66 & 2.44 & 269\,379          & 4.13 \\
		4 & neom & 189          & 13.66 & 2.44 & 269\,379          & 4.13 \\
		\bottomrule
%		\multicolumn{6}{l}{\footnotesize \textit{Pozn:}
%		\textrm{Zam} - počet zamítnutí, \textrm{pAgen} - průměrný počet agentů v jeden krok na křižovatce, \\
%		\textrm{sAgen} - směrodatná odchylka počtu agentů na křižovatce, \\
%		\textrm{Zpož} - součet spoždění přes všechny agenty, \textrm{pZpož} - průměrné zpoždění agentů
%		}  TODO
	\end{tabular}
	\caption{Porovnání testů při povoleném vracení u \ref{str:a_star_ars} na \hyperref[par:data_mala]{malém} čtv. typu.}\label{tab:ars_exp_male_ctvercova_z}
\end{table}

Další tabulka \ref{tab:ars_exp_male_ctvercova_v} ukazuje porovnání pro běhy,
pokud dovolím vracet se agentům na předchozí vrchol.
Všechny naměřené hodnoty jsou horší než když povolíme zastavování,
avšak opět jsou všechny hodnoty lepší než při testech bez zastavování a vracení.
Oproti předchozímu případu zde dávala varianta s \ref{par:ars_mnv} $=$ \ref{par:ars_mpc} $= 4$ nejlepší výsledky.
O něco hůře na tom byla varianta s dvěma návštěvami vrcholu a prodlevou $4$ na cestě.
Neomezená prodleva aprodleva $16$ opět dává totožné výsledky.

\begin{table}[b!]
	\centering
	\begin{tabular}{c c r D{.}{,}{2.2}D{.}{,}{1.2} r D{.}{,}{1.2}}
		\toprule \\
		\pulrad{\textbf{\ref{par:ars_mnv}}} & \pulrad{\textbf{\ref{par:ars_mpc}}} & \pulrad{\textbf{Zam}} &
		\mc{\pulrad{\textbf{pAgen}}} & \mc{\pulrad{\textbf{sAgen}}} & \pulrad{\textbf{Zpož}} & \mc{\pulrad{\textbf{pZpož}}} \\
		\midrule
		2 & 4    & 541          & 14.29 & 2.28 & 346\,471          & 5.34                                \\
		2 & 16   & 658          & 14.46 & 2.26 & 371\,912          & 5.74                                \\
		2 & neom & 658          & 14.46 & 2.26 & 371\,912          & 5.74                                \\
		4 & 4    & \textbf{457} & 14.29 & 2.29 & \textbf{344\,654} & \multicolumn{1}{B{.}{,}{1.2}}{5.30} \\
		4 & 16   & 648          & 14.46 & 2.26 & 369\,550          & 5.70                                \\
		4 & neom & 648          & 14.46 & 2.26 & 369\,550          & 5.70                                \\
		\bottomrule
%		\multicolumn{6}{l}{\footnotesize \textit{Pozn:}
%		\textrm{Zam} - počet zamítnutí, \textrm{pAgen} - průměrný počet agentů v jeden krok na křižovatce, \\
%		\textrm{sAgen} - směrodatná odchylka počtu agentů na křižovatce, \\
%		\textrm{Zpož} - součet spoždění přes všechny agenty, \textrm{pZpož} - průměrné zpoždění agentů
%		}  TODO
	\end{tabular}
	\caption{Porovnání testů při povoleném vracení u \ref{str:a_star_ars} na \hyperref[par:data_mala]{malém} čtv. typu.}\label{tab:ars_exp_male_ctvercova_v}
\end{table}

Poslední tabulka (Tabulka \ref{tab:ars_exp_male_ctvercova_zv})této části
ukazuje výsledky pokud dovolím obojí, zastavování i vracení.
Výsledky jsou mnohem lepší oproti případů pouze se zastavováním.
Nejlépe si vedly případy, kdy agenti mohou navštívit jeden vrchol čtyřikrát.
V těchto případech se snížil počet zamítnutí oproti běhů, pouze se zastavováním.
Avšak zpoždění se pro všechny prodlevy cesty zhoršily.
Naopak výsledky pro maximálně dvě návštěvy se výrazně zhoršili v počtu zamítnutých agentů.
Zpoždění agentů lehce kleslo.
Celkově dávaly nejlepší výsledky běhy se $4$ návštěvami vrcholu a neomezenou délkou cesty.

\begin{table}[b!]
	\centering
	\begin{tabular}{c c r D{.}{,}{2.2}D{.}{,}{1.2} r D{.}{,}{1.2}}
		\toprule \\
		\pulrad{\textbf{\ref{par:ars_mnv}}} & \pulrad{\textbf{\ref{par:ars_mpc}}} & \pulrad{\textbf{Zam}} &
		\mc{\pulrad{\textbf{pAgen}}} & \mc{\pulrad{\textbf{sAgen}}} & \pulrad{\textbf{Zpož}} & \mc{\pulrad{\textbf{pZpož}}} \\
		\midrule
		2 & 4    & 230          & 13.65 & 2.47 & \textbf{260\,549} & \multicolumn{1}{B{.}{,}{1.2}}{3.99} \\
		2 & 16   & 209          & 13.70 & 2.46 & 261\,577          & 4.01                                \\
		2 & neom & 209          & 13.70 & 2.46 & 261\,577          & 4.01                                \\
		4 & 4    & 197          & 13.52 & 2.44 & 269\,268          & 4.13                                \\
		4 & 16   & \textbf{185} & 13.55 & 2.44 & 271\,026          & 4.15                                \\
		4 & neom & \textbf{185} & 13.55 & 2.45 & 271\,026          & 4.15                                \\
		\bottomrule
%		\multicolumn{6}{l}{\footnotesize \textit{Pozn:}
%		\textrm{Zam} - počet zamítnutí, \textrm{pAgen} - průměrný počet agentů v jeden krok na křižovatce, \\
%		\textrm{sAgen} - směrodatná odchylka počtu agentů na křižovatce, \\
%		\textrm{Zpož} - součet spoždění přes všechny agenty, \textrm{pZpož} - průměrné zpoždění agentů
%		}  TODO
	\end{tabular}
	\caption{Porovnání testů při povoleném zastavování a vracení u \ref{str:a_star_ars} na \hyperref[par:data_mala]{malém} čtv. typu.}\label{tab:ars_exp_male_ctvercova_zv}
\end{table}

Celkově ze všech běhů dal nejlepší výsledek běh s povoleným zastavením, nejvýše dvěma návštěvami jednoho vrcholu
a prodlevou cesty nejvýše čtyři.

Plánovací doby jsou ve všech testech podobné,
pohybují se v rozmezí od $20$ do $45$ mikro sekund (\ref{tab:ars_exp_male_ctvercova_casy}).
Není překvapivé, že nejkratší dobu plánování měl nejvíce omezený běh, a jako jediný se dostal pod 30 mikro sekund.
Překvapivě běhy s povoleným vracením si skoro ve všech případech vedly lépe než běhy s jinou kombinací povolení.
Avšak rozdíly zde nebyly příliš velké.
Zvýšení nejvyšší prodlevy cesty obecně prodlužuje dobu plánování, což je očekávané.

Rozptyl plánování má větší rozdíly mezi běhy.

\begin{table}[b!]
	\centering
	\begin{tabular}{c c c D{.}{,}{5.2}D{.}{,}{6.2}}
		\toprule \\
		\pulrad{\textbf{\ref{par:ars_mnv}}} & \pulrad{\textbf{\ref{par:ars_mpc}}} & \pulrad{\textbf{Rozšíř.}$^a$} &
		\mc{\pulrad{\textbf{pCas}(ns)}} & \mc{\pulrad{\textbf{sCas}(ns)}}\\
		\midrule
		1 & 0   & \mc{---} & \multicolumn{1}{B{.}{,}{5.2}}{21\,743.20} & 86\,125.91                                \\
		1 & 4   & \mc{---} & 34\,663.69                                & 42\,215.05                                \\
		1 & 16  & \mc{---} & 37\,149.72                                & 98\,599.58                                \\
		\hline
		2 & 4   & \mc{---} & 35\,803.93                                & 62\,769.97                                \\
		2 & 4   & v        & 32\,623.66                                & 43\,525.54                                \\
		2 & 4   & z        & 32\,255.84                                & 65\,130.27                                \\
		2 & 4   & vz       & 38\,023.27                                & 120\,757.87                               \\
		\hline
		2 & 16  & \mc{---} & 38\,647.83                                & 61\,985.38                                \\
		2 & 16  & v        & 34\,959.60                                & 78\,228.89                                \\
		2 & 16  & z        & 39\,791.32                                & 77\,174.06                                \\
		2 & 16  & vz       & 34\,565.93                                & 99\,110.14                                \\
		\hline
		2 & inf & \mc{---} & 37\,911.81                                & 77\,849.57                                \\
		2 & inf & v        & 33\,684.13                                & 57\,048.30                                \\
		2 & inf & z        & 40\,006.97                                & 73\,607.71                                \\
		2 & inf & vz       & 42\,829.24                                & 140\,512.64                               \\
		\hline
		4 & 4   & \mc{---} & 35\,936.40                                & \multicolumn{1}{B{.}{,}{6.2}}{23\,533.66} \\
		4 & 4   & v        & 34\,181.51                                & 119\,814.05                               \\
		4 & 4   & z        & 40\,526.34                                & 90\,238.40                                \\
		4 & 4   & vz       & 33\,956.34                                & 90\,075.94                                \\
		\hline
		4 & 16  & \mc{---} & 38\,708.90                                & 70\,182.88                                \\
		4 & 16  & v        & 36\,558.72                                & 125\,204.29                               \\
		4 & 16  & z        & 39\,607.51                                & 28\,566.92                                \\
		4 & 16  & vz       & 33\,937.91                                & 90\,771.55                                \\
		\hline
		4 & inf & \mc{---} & 39\,331.89                                & 72\,604.26                                \\
		4 & inf & v        & 34\,950.11                                & 100\,684.08                               \\
		4 & inf & z        & 39\,554.96                                & 74\,800.68                                \\
		4 & inf & vz       & 37\,886.43                                & 102\,252.66                               \\
		\multicolumn{5}{l}{\footnotesize \textit{Pozn:}
			$^a$ Rozšíření se zastavením mají $z$, s vracením $v$.
		}
%		\textrm{Zam} - počet zamítnutí, \textrm{pAgen} - průměrný počet agentů v jeden krok na křižovatce, \\  TODO
%		\textrm{sAgen} - směrodatná odchylka počtu agentů na křižovatce, \\
%		\textrm{Zpož} - součet spoždění přes všechny agenty, \textrm{pZpož} - průměrné zpoždění agentů
	\end{tabular}
	\caption{Porovnání časů běhu testů u \ref{str:a_star_ars} na \hyperref[par:data_mala]{malém} čtv. typu.}\label{tab:ars_exp_male_ctvercova_casy}
\end{table}

\subsubsection{\nameref{par:data_mala} \hyperref[subsec:oktagonalni_typ]{oktagonální} křižovatka}
\label{subsubsec:exp_ars_mala_oktagonalni_krizovatka}

U této křižovatky opět budu testovat se stejnými hodnotami parametrů.
Akorát už nebudu testovat variantu s vracením, jelikož varianta se zastavováním dává lepší výsledky.

V tabulce (Tabulka \ref{tab:ars_exp_male_oktagonalni}) jsou výsledky všech běhů.
Celkově nejlépe vyšla varianta s \ref{par:ars_mnv} $= 4$, \ref{par:ars_mpc} $=4$ a se zastavováním.

Varianta, kdy agenti mohou jet pouze nejkratšími cestami vyšla oproti předchozí křižovatce katastroficky.
Počet zamítnutých agentů je řádově desetkrát vyšší než u ostatních agentů.
Zpoždění je taktéž mnohem vyšší.
Oktagonální křižovatka má vyšší počet vrcholů, což zvyšuje počet nejkratších cest.
Vypadá to ale, že se agenti navzájem více omezují než u čtvercové křižovatky.

Z dat avšak vyplývá, že až na variantu bez zpoždění je výhodné omezit délku cesty agenta.
Opět mají běhy s maximální prodlevou 16 a neomezenou prodlevou stejné výsledky.
Varianty s prodlevou $4$ mají vždy lepší výsledek než neomezené varianty.

Varianty se zastavováním si vedly opět mnohem lépe.

\begin{table}[b!]
	\centering
	\begin{tabular}{c c c r D{.}{,}{2.2}D{.}{,}{1.2} r D{.}{,}{1.2}}
		\toprule \\
		\pulrad{\textbf{\ref{par:ars_mnv}}} & \pulrad{\textbf{\ref{par:ars_mpc}}} & \pulrad{\textbf{Rozšíř.}$^a$} &
		\pulrad{\textbf{Zam}} & \mc{\pulrad{\textbf{pAgen}}} & \mc{\pulrad{\textbf{sAgen}}} &
		\pulrad{\textbf{Zpož}} & \mc{\pulrad{\textbf{pZpož}}} \\
		\midrule
		1 & 0    & \mc{---} & 3\,437       & \multicolumn{1}{B{.}{,}{2.2}}{11.35} & 1.79 & 400\,861          & 6.48                                \\
		1 & 4    & \mc{---} & 322          & 13.91                                & 2.47 & \textbf{291\,287} & \multicolumn{1}{B{.}{,}{1.2}}{4.48} \\
		1 & neom & \mc{---} & 325          & 14.02                                & 2.46 & 306\,255          & 4.71                                \\
		\hline
		2 & 4    & \mc{---} & 312          & 13.95                                & 2.47 & 293\,364          & 4.51                                \\
		2 & 4    & zast     & 166          & 14.30                                & 2.64 & 292\,872          & 4.50                                \\
		2 & 16   & \mc{---} & 333          & 13.98                                & 2.46 & 297\,111          & 4.57                                \\
		2 & 16   & zast     & 200          & 14.33                                & 2.62 & 298\,657          & 4.59                                \\
		2 & neom & \mc{---} & 333          & 13.98                                & 2.46 & 297\,111          & 4.57                                \\
		\hline
		4 & 4    & \mc{---} & 312          & 13.95                                & 2.47 & 293\,364          & 4.51                                \\
		4 & 4    & zast     & \textbf{151} & 14.26                                & 2.62 & 294\,134          & 4.51                                \\
		4 & 16   & \mc{---} & 331          & 13.98                                & 2.46 & 296\,840          & 4.57                                \\
		4 & 16   & zast     & 214          & 14.31                                & 2.61 & 302\,225          & 4.64                                \\
		4 & neom & \mc{---} & 331          & 13.98                                & 2.46 & 296\,840          & 4.57                                \\
		\bottomrule
%		\multicolumn{6}{l}{\footnotesize \textit{Pozn:}
%		\textrm{Zam} - počet zamítnutí, \textrm{pAgen} - průměrný počet agentů v jeden krok na křižovatce, \\
%		\textrm{sAgen} - směrodatná odchylka počtu agentů na křižovatce, \\
%		\textrm{Zpož} - součet spoždění přes všechny agenty, \textrm{pZpož} - průměrné zpoždění agentů
%		}  TODO
	\end{tabular}
	\caption{Porovnání všech testů u \ref{str:a_star_ars} na \hyperref[par:data_mala]{malém} okt. typu.}\label{tab:ars_exp_male_oktagonalni}
\end{table}

Plánovací časy (Tabulka \ref{tab:ars_exp_male_oktagonalni_casy}) jsou obecně horší než u čtvercové křižovatky.
Jelikož má křižovatka vyšší počet vrcholů, musí algoritmus prohledat více možností.
Opět vyšla nejlépe varianta s největším omezením.
Avšak rozdíl mezi jednotlivými běhy byl malý.

\begin{table}[b!]
	\centering
	\begin{tabular}{c c c D{.}{,}{5.2}D{.}{,}{6.2}}
		\toprule \\
		\pulrad{\textbf{\ref{par:ars_mnv}}} & \pulrad{\textbf{\ref{par:ars_mpc}}} & \pulrad{\textbf{Rozšíř.}$^a$} &
		\mc{\pulrad{\textbf{pCas}(ns)}} & \mc{\pulrad{\textbf{sCas}(ns)}}\\
		\midrule
		1 & 0    & \mc{---} & \multicolumn{1}{B{.}{,}{5.2}}{28\,725.37} & 90\,733.85                                \\
		1 & neom & \mc{---} & 58\,722.69                                & 73\,367.74                                \\
		1 & 4    & \mc{---} & 56\,550.24                                & 111\,243.58                               \\
		\hline
		2 & 4    & \mc{---} & 56\,472.24                                & 81\,253.50                                \\
		2 & 4    & zast     & 73\,366.60                                & 91\,528.06                                \\
		2 & 16   & \mc{---} & 57\,265.18                                & \multicolumn{1}{B{.}{,}{5.2}}{39\,435.40} \\
		2 & 16   & zast     & 73\,494.82                                & 62\,010.80                                \\
		2 & neom & \mc{---} & 66\,864.38                                & 150\,919.54                               \\
		\hline
		4 & 4    & \mc{---} & 59\,041.47                                & 132\,689.35                               \\
		4 & 4    & zast     & 72\,064.74                                & 65\,291.17                                \\
		4 & 16   & zast     & 76\,035.25                                & 101\,553.22                               \\
		4 & 16   & \mc{---} & 59\,189.93                                & 114\,791.97                               \\
		4 & neom & \mc{---} & 60\,199.58                                & 93\,328.24                                \\
%		\multicolumn{5}{l}{\footnotesize \textit{Pozn:}
%			$^a$ Rozšíření se zastavením mají $z$, s vracením $v$.
%		}
%		\textrm{Zam} - počet zamítnutí, \textrm{pAgen} - průměrný počet agentů v jeden krok na křižovatce, \\  TODO
%		\textrm{sAgen} - směrodatná odchylka počtu agentů na křižovatce, \\
%		\textrm{Zpož} - součet spoždění přes všechny agenty, \textrm{pZpož} - průměrné zpoždění agentů
	\end{tabular}
	\caption{Porovnání časů běhu testů u \ref{str:a_star_ars} na \hyperref[par:data_mala]{malém} okt. typu.}\label{tab:ars_exp_male_oktagonalni_casy}
\end{table}

\subsubsection{\nameref{par:data_mala} \hyperref[subsec:hexagonalni_typ]{hexagonální} křižovatka}
\label{subsubsec:exp_ars_mala_hexagonalni_krizovatka}

Zde provádím testování opět se stejnými hodnotami parametrů.
Výsledky (Tabulka \ref{tab:ars_exp_male_oktagonalni}) ukazují stejný vzor jako u oktagonálního typu křižovatky.
Nejvíce omezená křižovatka dává mnohem horší výsledky než ostatní nastavení parametrů.
Opět nejlepší výsledek dává omezení vrcholů na dvě návštěvy a nejvyšší prodleva cesty rovna $4$.
Stejně tak nehraje roli omezení prodlevy cesty na $16$ a neomezení prodlevy.

Varianty se zastavováním jsou vždy lepší než varianty bez zastavování.
Avšak pokud zakážu zastavování, vyjde nejlépe varianta s nejvýše jednou návštěvou vrcholu
a prodloužením cesty nejvýše čtyři.

\begin{table}[b!]
	\centering
	\begin{tabular}{c c c r D{.}{,}{2.2}D{.}{,}{1.2} r D{.}{,}{1.2}}
		\toprule \\
		\pulrad{\textbf{\ref{par:ars_mnv}}} & \pulrad{\textbf{\ref{par:ars_mpc}}} & \pulrad{\textbf{Rozšíř.}$^a$} &
		\pulrad{\textbf{Zam}} & \mc{\pulrad{\textbf{pAgen}}} & \mc{\pulrad{\textbf{sAgen}}} &
		\pulrad{\textbf{Zpož}} & \mc{\pulrad{\textbf{pZpož}}} \\
		\midrule
		1 & 0    & \mc{---} & 22\,552         & \multicolumn{1}{B{.}{,}{2.2}}{16.54} & \multicolumn{1}{B{.}{,}{2.2}}{2.11} & 1\,405\,504          & 18.46 \\
		1 & 4    & \mc{---} & 3\,828          & 25.17                                & 2.33                                & \textbf{1\,055\,081} & \multicolumn{1}{B{.}{,}{2.2}}{11.12} \\
		1 & 16   & \mc{---} & 6\,075          & 25.93                                & 2.16                                & 1\,239\,441          & 13.38                                \\
		1 & neom & \mc{---} & 6\,075          & 25.94                                & 2.14                                & 1\,239\,441          & 13.38                                \\
		\hline
		2 & 4    & \mc{---} & 4\,287          & 25.39                                & 2.34                                & 1\,088\,430          & 11.53                                \\
		2 & 4    & zast     & \textbf{3\,459} & 25.84                                & 2.40                                & 1\,104\,524          & 11.60                                \\
		2 & 16   & \mc{---} & 5\,822          & 26.10                                & 2.22                                & 1\,232\,405          & 13.27                                \\
		2 & 16   & zast     & 3\,674          & 26.56                                & 2.39                                & 1\,168\,589          & 12.30                                \\
		2 & neom & \mc{---} & 5\,822          & 26.10                                & 2.20                                & 1\,232\,405          & 13.27                                \\
		2 & neom & zast     & 3\,674          & 26.57                                & 2.37                                & 1\,168\,589          & 12.30                                \\
		\hline
		4 & 4    & \mc{---} & 4\,287          & 25.39                                & 2.34                                & 1\,088\,430          & 11.53                                \\
		4 & 4    & zast     & 3\,561          & 25.80                                & 2.44                                & 1\,094\,422          & 11.50                                \\
		4 & 16   & \mc{---} & 5\,907          & 26.09                                & 2.17                                & 1\,245\,951          & 13.43                                \\
		4 & 16   & zast     & 3\,990          & 26.33                                & 2.44                                & 1\,157\,499          & 12.22                                \\
		4 & neom & \mc{---} & 5\,907          & 26.09                                & 2.17                                & 1\,245\,951          & 13.43                                \\
		4 & neom & zast     & 3\,990          & 26.33                                & 2.44                                & 1\,157\,499          & 12.22                                \\
		\bottomrule
%		\multicolumn{6}{l}{\footnotesize \textit{Pozn:}
%		\textrm{Zam} - počet zamítnutí, \textrm{pAgen} - průměrný počet agentů v jeden krok na křižovatce, \\
%		\textrm{sAgen} - směrodatná odchylka počtu agentů na křižovatce, \\
%		\textrm{Zpož} - součet spoždění přes všechny agenty, \textrm{pZpož} - průměrné zpoždění agentů
%		}  TODO
	\end{tabular}
	\caption{Porovnání všech testů u \ref{str:a_star_ars} na \hyperref[par:data_mala]{malém} hex. typu.}\label{tab:ars_exp_male_hexagonalni}
\end{table}

Časy plánování (Tabulka \ref{tab:ars_exp_male_oktagonalni_casy}) ukazuje doby plánování.
Výsledky opět ukazují, že více omezená křižovatka plánuje agenty rychleji.

\begin{table}[b!]
	\centering
	\begin{tabular}{c c c D{.}{,}{6.2}D{.}{,}{6.2}}
		\toprule \\
		\pulrad{\textbf{\ref{par:ars_mnv}}} & \pulrad{\textbf{\ref{par:ars_mpc}}} & \pulrad{\textbf{Rozšíř.}$^a$} &
		\mc{\pulrad{\textbf{pCas}(ns)}} & \mc{\pulrad{\textbf{sCas}(ns)}}\\
		\midrule
%		1 & 0   & \mc{---} & \multicolumn{1}{B{.}{,}{5.2}}{21\,743.20} & 86\,125.91                                \\
		1  & 0    & \mc{---} & \multicolumn{1}{B{.}{,}{5.2}}{52\,305.30} & \multicolumn{1}{B{.}{,}{6.2}}{109\,719.15} \\
		1  & 4    & \mc{---} & 141\,108.89                               & 151\,316.05                                \\
		1  & 16   & \mc{---} & 157\,540.37                               & 197\,392.66                                \\
		1  & neom & \mc{---} & 165\,476.15                               & 326\,814.49                                \\
		2  & 4    & \mc{---} & 148\,302.74                               & 184\,517.32                                \\
		2  & 4    & zast     & 188\,478.27                               & 197\,808.70                                \\
		2  & 16   & \mc{---} & 162\,271.29                               & 183\,678.94                                \\
		2  & 16   & zast     & 208\,250.98                               & 225\,798.34                                \\
		2  & neom & \mc{---} & 160\,943.84                               & 142\,601.36                                \\
		2  & neom & zast     & 209\,706.68                               & 256\,419.70                                \\
		4  & 4    & \mc{---} & 144\,167.36                               & 176\,283.75                                \\
		4  & 4    & zast     & 195\,061.53                               & 278\,502.19                                \\
		4  & 16   & \mc{---} & 162\,900.44                               & 220\,369.60                                \\
		4  & 16   & zast     & 214\,376.08                               & 237\,392.60                                \\
		4  & neom & \mc{---} & 166\,260.43                               & 184\,528.94                                \\
		4  & neom & zast     & 213\,507.34                               & 222\,956.27                                \\
%		\multicolumn{5}{l}{\footnotesize \textit{Pozn:}
%			$^a$ Rozšíření se zastavením mají $z$, s vracením $v$.
%		}
%		\textrm{Zam} - počet zamítnutí, \textrm{pAgen} - průměrný počet agentů v jeden krok na křižovatce, \\  TODO
%		\textrm{sAgen} - směrodatná odchylka počtu agentů na křižovatce, \\
%		\textrm{Zpož} - součet spoždění přes všechny agenty, \textrm{pZpož} - průměrné zpoždění agentů
	\end{tabular}
	\caption{Porovnání časů běhu testů u \ref{str:a_star_ars} na \hyperref[par:data_mala]{malém} hex. typu.}\label{tab:ars_exp_male_hexagonalni_casy}
\end{table}

\subsubsection{\nameref{par:data_stredni} \hyperref[subsec:ctvercovy_typ]{čtvercová} křižovatka s výjezdy}
\label{subsubsec:exp_ars_stredni_ctvercova_krizovatka_ex}

V této části budu testovat běhy, na středně velké křižovatce.
Agenti budou mít přesně zadaný výjezd.
Očekávám podobné výsledky jako v předchozích případech, nic nenapovídá změně v chování při zvětšení prostoru.

Nastavení parametrů ponechám.
Budu tedy testovat pro maximální počet návštěv vrcholu (\ref{par:ars_mnv}) hodnoty $1, 2, 4$
a pro nejvyšší prodlevu cesty (\ref{par:ars_mpc}) hodnoty $0, 4, 16, neom.$.
Poté opět vyzkouším varianty se zastavováním a bez zastavování.

Výsledky bez zastavování ukazují odlišné výsledky oproti malé křižovatce.
Data ukazují, že pokud nastavím maximální prodlevu na $4$, nehraje roli zvyšování počtu návštěv vrcholu nad hodnotu $2$.
Tyto výsledky dávají nejmenší počet zamítnutí.
Počet zamítnutí při jedné návštěvě vrcholu je lehce vyšší.
Hůře si dařil nejvíce omezený běh.
Nejhorší výsledky daly běhy s vysokou povolenou prodlevou.

Naopak nejméně omezený běh dal nejmenší zdržení agentů.
Dle mého názoru je toto způsobeno nižší čekací dobou agentů před křižovatkou.

\begin{table}[b!]
	\centering
	\begin{tabular}{c c r D{.}{,}{2.2}D{.}{,}{1.2} r D{.}{,}{2.2}}
		\toprule \\
		\pulrad{\textbf{\ref{par:ars_mnv}}} & \pulrad{\textbf{\ref{par:ars_mpc}}} &
		\pulrad{\textbf{Zam}} & \mc{\pulrad{\textbf{pAgen}}} & \mc{\pulrad{\textbf{sAgen}}} &
		\pulrad{\textbf{Zpož}} & \mc{\pulrad{\textbf{pZpož}}} \\
		\midrule
		1 & 0    & 58\,295          & \multicolumn{1}{B{.}{,}{2.2}}{42.26} & 3.19 & 3\,697\,296          & 26.96                                \\
		1 & 4    & 54\,652          & 50.04                                & 2.98 & 3\,690\,410          & 26.22                                \\
		1 & 16   & 63\,700          & 51.13                                & 2.97 & 3\,450\,186          & 26.19                                \\
		2 & 4    & \textbf{54\,210} & 50.71                                & 2.93 & 3\,722\,173          & 26.36                                \\
		2 & 16   & 62\,453          & 52.05                                & 3.00 & 3\,458\,916          & 26.01                                \\
		2 & neom & 62\,469          & 52.03                                & 3.00 & 3\,466\,961          & 26.08                                \\
		4 & 4    & \textbf{54\,210} & 50.71                                & 2.93 & 3\,722\,173          & 26.36                                \\
		4 & 16   & 62\,408          & 52.04                                & 2.99 & 3\,481\,487          & 26.17                                \\
		4 & neom & 62\,086          & 52.06                                & 2.96 & \textbf{3\,449\,296} & \multicolumn{1}{B{.}{,}{2.2}}{25.87} \\
		\bottomrule
%		\multicolumn{6}{l}{\footnotesize \textit{Pozn:}
%		\textrm{Zam} - počet zamítnutí, \textrm{pAgen} - průměrný počet agentů v jeden krok na křižovatce, \\
%		\textrm{sAgen} - směrodatná odchylka počtu agentů na křižovatce, \\
%		\textrm{Zpož} - součet spoždění přes všechny agenty, \textrm{pZpož} - průměrné zpoždění agentů
%		}  TODO
	\end{tabular}
	\caption{Porovnání testů bez zastavování u \ref{str:a_star_ars} na \hyperref[par:data_stredni]{středním} čtv. typu.}\label{tab:ars_exp_stredni_ctvercova}
\end{table}

Pokud dovolím agentům se zastavit, výsledky jsou opět znatelně lepší (Tabulka \ref{tab:ars_exp_stredni_ctvercova_z}).
Nejvíce omezená varianta se zastavování dala celkově nejlepší výsledek.
Má nejmenší počet zamítnutí a nejmenší průměrné zdržení agentů.
Ostatní běhy mají mírně horší výsledky.
Měření napovídají, že není žádný rozdíl mezi neomezenou prodlevou cesty a omezenou prodlevou na $16$.

\begin{table}[b!]
	\centering
	\begin{tabular}{c c r D{.}{,}{2.2}D{.}{,}{1.2} r D{.}{,}{2.2}}
		\toprule \\
		\pulrad{\textbf{\ref{par:ars_mnv}}} & \pulrad{\textbf{\ref{par:ars_mpc}}} &
		\pulrad{\textbf{Zam}} & \mc{\pulrad{\textbf{pAgen}}} & \mc{\pulrad{\textbf{sAgen}}} &
		\pulrad{\textbf{Zpož}} & \mc{\pulrad{\textbf{pZpož}}} \\
		\midrule
		2 & 4    & \textbf{46\,431} & 52.32                                & 3.13 & 3\,783\,326          & \multicolumn{1}{B{.}{,}{2.2}}{25.39} \\
		2 & 16   & 48\,506          & 53.48                                & 3.03 & \textbf{3\,793\,290} & 25.82                                \\
		2 & neom & 48\,506          & 53.49                                & 2.99 & \textbf{3\,793\,290} & 25.82                                \\
		4 & 4    & 46\,462          & \multicolumn{1}{B{.}{,}{2.2}}{52.28} & 3.12 & 3\,789\,392          & 25.44                                \\
		4 & 16   & 48\,489          & 53.34                                & 3.06 & 3\,783\,635          & 25.75                                \\
		4 & neom & 48\,489          & 53.34                                & 3.06 & 3\,783\,635          & 25.75                                \\
		\bottomrule
%		\multicolumn{6}{l}{\footnotesize \textit{Pozn:}
%		\textrm{Zam} - počet zamítnutí, \textrm{pAgen} - průměrný počet agentů v jeden krok na křižovatce, \\
%		\textrm{sAgen} - směrodatná odchylka počtu agentů na křižovatce, \\
%		\textrm{Zpož} - součet spoždění přes všechny agenty, \textrm{pZpož} - průměrné zpoždění agentů
%		}  TODO
	\end{tabular}
	\caption{Porovnání testů se zastavováním u \ref{str:a_star_ars} na \hyperref[par:data_stredni]{středním} čtv. typu.}\label{tab:ars_exp_stredni_ctvercova_z}
\end{table}

\subsubsection{\nameref{par:data_stredni} \hyperref[subsec:oktagonalni_typ]{oktagonální} křižovatka}
\label{subsubsec:exp_ars_stredni_oktagonalni_krizovatka_ex}

\begin{table}[b!]
	\centering
	\begin{tabular}{c c r D{.}{,}{2.2}D{.}{,}{1.2} r D{.}{,}{2.2}}
		\toprule \\
		\pulrad{\textbf{\ref{par:ars_mnv}}} & \pulrad{\textbf{\ref{par:ars_mpc}}} &t
		\pulrad{\textbf{Zam}} & \mc{\pulrad{\textbf{pAgen}}} & \mc{\pulrad{\textbf{sAgen}}} &
		\pulrad{\textbf{Zpož}} & \mc{\pulrad{\textbf{pZpož}}} \\
		\midrule
		1 & 0    & 56\,718          & \multicolumn{1}{B{.}{,}{2.2}}{42.24} & 3.30 & \textbf{3\,676\,058} & 26.48                                \\
		1 & 8    & 43\,764          & 58.65                                & 3.21 & 3\,895\,377          & \multicolumn{1}{B{.}{,}{2.2}}{25.67} \\
		1 & 16   & 44\,996          & 58.70                                & 3.27 & 3\,876\,707          & 25.75                                \\
		1 & 32   & 44\,847          & 58.74                                & 3.19 & 3\,874\,030          & 25.71                                \\
		1 & neom & 44\,847          & 58.74                                & 3.19 & 3\,874\,030          & 25.71                                \\
		2 & 8    & 42\,445          & 59.28                                & 3.27 & 4\,147\,171          & 27.09                                \\
		2 & 16   & 42\,938          & 59.35                                & 3.24 & 4\,152\,804          & 27.21                                \\
		2 & 32   & 42\,938          & 59.35                                & 3.25 & 4\,152\,804          & 27.21                                \\
		2 & neom & 42\,938          & 59.35                                & 3.25 & 4\,152\,804          & 27.21                                \\
		4 & 8    & \textbf{42\,364} & 59.34                                & 3.22 & 4\,182\,190          & 27.30                                \\
		4 & 16   & 43\,000          & 59.35                                & 3.24 & 4\,165\,331          & 27.31                                \\
		4 & 32   & 43\,000          & 59.35                                & 3.24 & 4\,165\,331          & 27.31                                \\
		4 & neom & 43\,000          & 59.35                                & 3.24 & 4\,165\,331          & 27.31                                \\
		\bottomrule
%		\multicolumn{6}{l}{\footnotesize \textit{Pozn:}
%		\textrm{Zam} - počet zamítnutí, \textrm{pAgen} - průměrný počet agentů v jeden krok na křižovatce, \\
%		\textrm{sAgen} - směrodatná odchylka počtu agentů na křižovatce, \\
%		\textrm{Zpož} - součet spoždění přes všechny agenty, \textrm{pZpož} - průměrné zpoždění agentů
%		}  TODO
	\end{tabular}
	\caption{Porovnání testů se zastavováním u \ref{str:a_star_ars} na \hyperref[par:data_stredni]{středním} okt. typu.}\label{tab:ars_exp_stredni_oktagonalni}
\end{table}

\begin{table}[b!]
	\centering
	\begin{tabular}{c c r D{.}{,}{2.2}D{.}{,}{1.2} r D{.}{,}{2.2}}
		\toprule \\
		\pulrad{\textbf{\ref{par:ars_mnv}}} & \pulrad{\textbf{\ref{par:ars_mpc}}} &
		\pulrad{\textbf{Zam}} & \mc{\pulrad{\textbf{pAgen}}} & \mc{\pulrad{\textbf{sAgen}}} &
		\pulrad{\textbf{Zpož}} & \mc{\pulrad{\textbf{pZpož}}} \\
		\midrule
%		2 & 4    & \textbf{46\,431} & 52.32                                & 3.13 & 3\,783\,326          & \multicolumn{1}{B{.}{,}{2.2}}{25.39} \\
		2  & 8    & \textbf{35\,048} & \multicolumn{1}{B{.}{,}{2.2}}{62.26} & 3.21 & 4\,341\,239          & 27.05                                \\
		2  & 16   & 35\,418          & 62.34                                & 3.25 & 4\,340\,047          & 27.11                                \\
		2  & 32   & 35\,418          & 62.34                                & 3.25 & 4\,340\,047          & 27.11                                \\
		2  & neom & 35\,418          & 62.34                                & 3.25 & 4\,340\,047          & 27.11                                \\
		4  & 8    & 35\,482          & 61.91                                & 3.23 & 4\,326\,442          & 27.03                                \\
		4  & 16   & 35\,240          & 62.03                                & 3.28 & \textbf{4\,326\,349} & \multicolumn{1}{B{.}{,}{2.2}}{26.99} \\
		4  & 32   & 35\,240          & 62.03                                & 3.28 & \textbf{4\,326\,349} & \multicolumn{1}{B{.}{,}{2.2}}{26.99} \\
		4  & neom & 35\,240          & 62.03                                & 3.28 & \textbf{4\,326\,349} & \multicolumn{1}{B{.}{,}{2.2}}{26.99} \\
		\bottomrule
%		\multicolumn{6}{l}{\footnotesize \textit{Pozn:}
%		\textrm{Zam} - počet zamítnutí, \textrm{pAgen} - průměrný počet agentů v jeden krok na křižovatce, \\
%		\textrm{sAgen} - směrodatná odchylka počtu agentů na křižovatce, \\
%		\textrm{Zpož} - součet spoždění přes všechny agenty, \textrm{pZpož} - průměrné zpoždění agentů
%		}  TODO
	\end{tabular}
	\caption{Porovnání testů se zastavováním u \ref{str:a_star_ars} na \hyperref[par:data_stredni]{středním} okt. typu.}\label{tab:ars_exp_stredni_oktagonalni_z}
\end{table}

\subsubsection{\nameref{par:data_stredni} \hyperref[subsec:hexagonalni_typ]{hexagonální} křižovatka s výjezdy}
\label{subsubsec:exp_ars_stredni_hexagoalni_krizovatka_ex}


\subsubsection{\nameref{par:data_stredni} \hyperref[subsec:ctvercovy_typ]{čtvercová} křižovatka bez výjezdů}
\label{subsubsec:exp_ars_stredni_ctvercova_krizovatka_noex}

\subsubsection{\nameref{par:data_stredni} \hyperref[subsec:oktagonalni_typ]{oktagonální} křižovatka bez výjezdů}
\label{subsubsec:exp_ars_stredni_oktagonalni_krizovatka_noex}

\subsubsection{\nameref{par:data_stredni} \hyperref[subsec:hexagonalni_typ]{hexagonální} křižovatka bez výjezdů}
\label{subsubsec:exp_ars_stredni_hexagoalni_krizovatka_noex}


\subsubsection{\nameref{par:data_velka} \hyperref[subsec:ctvercovy_typ]{čtvercová} křižovatka s výjezdy}
\label{subsubsec:exp_ars_velka_ctvercova_krizovatka_ex}

\subsubsection{\nameref{par:data_velka} \hyperref[subsec:oktagonalni_typ]{oktagonální} křižovatka s výjezdy}
\label{subsubsec:exp_ars_velka_oktagonalni_krizovatka_ex}

\subsubsection{\nameref{par:data_velka} \hyperref[subsec:hexagonalni_typ]{hexagonální} křižovatka s výjezdy}
\label{subsubsec:exp_ars_velka_hexagoalni_krizovatka_ex}


\subsubsection{\nameref{par:data_velka} \hyperref[subsec:ctvercovy_typ]{čtvercová} křižovatka bez výjezdů}
\label{subsubsec:exp_ars_velka_ctvercova_krizovatka_noex}

\subsubsection{\nameref{par:data_velka} \hyperref[subsec:oktagonalni_typ]{oktagonální} křižovatka bez výjezdů}
\label{subsubsec:exp_ars_velka_oktagonalni_krizovatka_noex}

\subsubsection{\nameref{par:data_velka} \hyperref[subsec:hexagonalni_typ]{hexagonální} křižovatka bez výjezdů}
\label{subsubsec:exp_ars_velka_hexagoalni_krizovatka_noex}
