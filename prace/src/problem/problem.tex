\section{Problém křižovatky}\label{sec:problem}

%Popis problému křižovatky.

%Obecnější řešení (decentralizované plánování, časování semaforů, \ldots)

Chytré křižovatky se již objevily v mnohých městech.
Jsou to světelné křižovatky, které dokážou poznat, že všechna auta v daném směru už projela.
Při~detekci takovéto situace křižovatka nastaví červenou z~příslušného směru a zároveň pustí auta z~jiného směru dříve.
Plánování pořadí a délek jednotlivých zelených všech směrů je komplexní záležitost, pokročilejší plánování popsali například \citet*{Goldstein}.
\citet*{Liang} ve svém článku popsali trénování světelných křižovatek pomocí zpětnovazebního učení
a rozšířili algoritmus i~na~tisíce propojených křižovatek.
Na~tento způsob řešení jsem se ale ve~své práci nezaměřil, protože nabízí minimální zlepšení
na~jedné křižovatce v~hustých provozech a minimálně využívá autonomie vozidel.

Další způsob je decentralizované plánování, které spočívá v~komunikaci mezi auty.
Touto technikou se zabývali například \citet*{Wu}.
V~jejich článku porovnávají kruhový objezd se světelnou křižovatkou.
Jelikož není přítomna centrální řídící jednotka, pro~převedení řešení
do~reálného světa stačí přidat určitý protokol do~vozidel v~provozu.
Avšak aplikovat komplexnější plánování je při~větším počtu aut obtížné.

Já se zaměřím pouze na postupy s~centrální jednotkou.
Při tomto postupu auta nekomunikují navzájem mezi sebou, ale na křižovatce existuje skříňka rozhodující trasy aut.
