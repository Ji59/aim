\section{Simulace}\label{sec:simulace}

%Popis běhu simulátoru - vygenerování agentů a předání řešiči.
%
%Definice sledovaných parametrů - zdržení, zamítnutí, kolize.

Pro tuto práci jsem si vytvořil vlastní simulátor křižovatky.
V simulátoru lze nastavit parametry křižovatky a agenta.
Po startu simulace se načtou agenti, naleznou se pro ně trasy a odsimuluje se průjezd agentů.
Čas běhu simulace je rozdělen na diskrétní kroky.
V~každém kroku se pokusí simulace naplánovat agenty, kteří v daném kroku požádali o vjezd, nebo byl jejich vjezd zamítnut.
Zároveň se všichni agenti na křižovatce přesunou o jednu hranu dále do následujícího vrcholu daného svou trajektorií.
Pro~naplánování jsou vybráni agenti z~každého vjezdu, kteří přijeli nejdříve.
Ostatním agentům je zamítnut vjezd v daném kroku.
Nemůže se tedy stát, že by agent předjel jiného agenta čekajícího před~ním.

Simulace nabízí různé statistiky pomocí nichž je možné vygenerované trasy porovnat.
Tyto statistiky jsou popsány níže.

\paragraph{Zdržení} \label{par:zdrzeni} jednoho agenta se~spočte jako součet
rozdílu délky cesty od~optimální cesty a doba čekání před vjezdem do~křižovatky.
Jinými slovy \hyperref[item:zdrzeni]{zdržení} značí počet kroků, o~které agent vyjel z~křižovatky později oproti situaci,
kdyby přijel na~prázdnou křižovatku (křižovatku bez~jiných agentů) a
ihned by~projel svým \hyperref[par:pruh]{pruhem} (nejkratší možnou cestou).
Celkové \hyperref[item:zdrzeni]{zdržení} všech agentů je součet \hyperref[item:zdrzeni]{zdržení} přes~všechny agenty,
co se~pokusili křižovatkou projet.

\paragraph{Počet zamítnutých agentů}\label{par:zamitnuti} udává množství agentů, kteří čekali na~křižovatce příliš dlouho a
vzdali~se čekání ve~frontě.
I~když křižovatkou neprojeli, pořád se kroky jejich čekání přičítají do~celkového \hyperref[item:zdrzeni]{zdržení}.

\paragraph{Počet kolizí}\label{par:kolize} udává množství sražených agentů.

\paragraph{Doba běhu}\label{par:doba_behu} algoritmu počítá kolik nanosekund běžel algoritmus
v~součtu přes všechny kroky na~reálném hardware.

\subsection{Generování agentů}\label{subsec:generovani_agentu}

%Popis generování nových agentů - způsob vybírání množství agentů a
%hodnot pro agenta (vjezd, výjezd, rychlost, velikost, \ldots).

Pokud uživatel nespouští simulaci s~předpřipravenými agenty ze souboru,
simulátor nabízí možnost generování agentů za~běhu.
Před~startem simulace je možné navolit~si určité vlastnosti agentů a po~kolik kroků se mají agenti generovat.
Uživatel si může určit počet vygenerovaných agentů v~každém kroku simulace nastavením parametrů $na_{\min}$ a~$na_{\max}$.
Simulace vygeneruje nový počet agentů uniformně náhodně mezi $na_{\min}$ a~$na_{\max}$.  % TODO gaus?

Dále je možné určit preference směrů odkud agenti budou přijíždět a kam budou směřovat.
Pro~každý směr je možné určit pravděpodobnost, s~jakou se zde agent objeví, či kam bude směřovat.
Tyto pravděpodobnosti se~musí sečíst na~$1$.

Poté je možné měnit samotné parametry agentů.
Hlavními parametry agenta jsou šířka a délka a \hyperref[par:odchylka]{odchylka rychlosti}.

\paragraph{Odchylka}\label{par:odchylka} určuje o~kolik procent je rychlost agenta odlišná
vůči oznámené rychlosti křižovatce, a~nabývá hodnot mezi nulou až~sto procenty.
Kromě rychlosti ovlivňuje \hyperref[par:odchylka]{odchylka} i~příjezd agenta.
Příjezd je možné zpozdit až~o~tolik procent kroku, kolik je \hyperref[par:odchylka]{odchylka}.
Tímto způsobem se snažím simulovat reálnější křižovatky, kde došlo k~určité chybě v~komunikaci či měření agenta.
Pomocí \hyperref[par:odchylka]{odchylky} se snažím sledovat odolnost plánování vůči těmto jevům.
