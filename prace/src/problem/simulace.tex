\section{Simulace}\label{sec:simulace}

Popis běhu simulátoru - vygenerování agentů a předání řešiči.

Definice sledovaných parametrů - zdržení, zamítnutí, kolize.

%
%V~každém kroku se pokusí simulace naplánovat agenty použitím vybraného algoritmu.
%Pro~naplánování jsou vybráni agenti z~každého vjezdu, kteří přijeli nejdříve.
%Nemůže se stát, že by agent předjel jiného agenta čekajícího před~ním.
%
%Simulace nabízí různé statistiky pomocí nichž je možné jednotlivé algoritmy porovnat.
%Tyto statistiky jsou popsány níže.
%
%\begin{itemize}
%	\item \emph{Zdržení} agentů. \emph{Zdržení} jednoho agenta se~spočte jako součet
%	rozdílu délky cesty od~optimální cesty a doba čekání před vjezdem do~křižovatky.
%	Jinými slovy \emph{zdržení} značí počet kroků, o~které agent vyjel z~křižovatky později oproti situaci,
%	kdyby přijel na~prázdnou křižovatku (křižovatku bez~jiných agentů) a ihned by~projel nejkratší možnou cestou.
%	\emph{Zdržení} všech agentů je součet \emph{zdržení} přes~všechny agenty, co se~pokusili křižovatkou projet.
%	\item Počet \emph{zamítnutých} agentů udává množství agentů, kteří čekali na~křižovatce příliš dlouho a
%	vzdali~se čekání ve~frontě.
%	I~když křižovatkou neprojeli, pořád se kroky jejich čekání přičítají do~celkového \emph{zdržení}.
%	\item Počet \emph{kolizí} udává množství sražených agentů.
%	\item \emph{Doba běhu} algoritmu udává kolik nanosekund běžel algoritmus v~součtu přes všechny kroky.
%\end{itemize}

\subsection{Generování agentů}\label{subsec:generovani_agentu}

Popis generování nových agentů - způsob vybírání množství agentů a
hodnot pro agenta (vjezd, výjezd, rychlost, velikost, \ldots).

%
%Před~startem simulace je možné navolit~si určité vlastnosti agentů a po~kolik kroků se mají agenti generovat.
%Uživatel si může určit počet vygenerovaných agentů v~každém kroku simulace nastavením parametrů $na_{\min}$ a~$na_{\max}$.
%Simulace vygeneruje nový počet agentů uniformně náhodně mezi $na_{\min}$ a~$na_{\max}$.
%
%Dále je možné určit preference \emph{směrů} odkud agenti budou přijíždět a~kam budou směřovat.
%Pro~každý \emph{směr} je možné určit pravděpodobnost, s~jakou se zde agent objeví, či tam bude směřovat.
%Tyto pravděpodobnosti se musí sečíst na~$1$.
%
%Poté je možné měnit samotné parametry agentů.
%Z~těchto parametrů budu používat pouze \emph{odchylku}.
%\emph{Odchylka} určuje o~kolik procent jsou parametry agenta odlišné vůči sděleným parametrům křižovatce,
%a~nabývá hodnot mezi nulou a~sto procenty.
%Ovlivněné parametry jsou šířka, délka a~rychlost agenta.
%Kromě těchto parametrů ovlivňuje odchylka i~příjezd agenta.
%Příjezd je možné zpozdit až~o~\emph{odchylku} kroku.
%Tímto způsobem se snažím simulovat reálnější křižovatky, kde došlo k~určité chybě v~měření.
