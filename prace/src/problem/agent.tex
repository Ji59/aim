\section{Agent}\label{sec:agent}

%Popis zjednodušení auta na agenta.
%Popis parametrů agenta.


\emph{Agent} značí v~simulátoru chytré auto, které má zájem projet křižovatkou.
Tvar agenta je zjednodušený na obdélník, který má určitou délku, šířku a naklonění.
Tento obdélník reprezentuje pohled na auto shora.

Jelikož se agent pohybuje po hranách grafu křižovatky, je jeho cesta složená z úseček.
Proto jsem umožnil agentovy otáčet se na místě.
Tato vlastnost by neměla mít na pohyb auta velký vliv pokud je auto dostatečně malé oproti vzdálenostem vrcholů grafu křižovatky.

Prohledávající algoritmy na grafech mají



Při~příjezdu auta na~křižovatku nahlásí auto křižovatce odkud jede a kam směřuje.
Křižovatka poté pomocí vybraného algoritmu vygeneruje \emph{cestu} skrze křižovatku.
\emph{Cesta} je tvořena posloupností vrcholů, přes které má agent jet.
Pokud algoritmus nenajde žádnou cestu skrze křižovatku, křižovatka zamítne vjezd \emph{agenta}.
V~případě kdy agent čeká na~vjezd příliš dlouho, \uv{agentu dojde trpělivost a vydá se jinou cestou}.
Znamená to, že je ze~simulace odstraňen.
Avšak stále je zaznamenáno, že se agent pokusil křižovatkou projet.

\emph{Agent} je reprezentován obdélníkem s~délkou $d=0.6~b$ a~šířkou $w=0.37~b$,
kde $b$ značí \emph{velikost bloku} dané křižovatky.
Zároveň se agent dokáže otáčet na~místě a okamžitě podle potřeby.
Agent se vždy otáčí ve~směru jízdy.
Agent se pohybuje rychlostí jedné hrany grafu křižovatky za~jeden krok simulace.

Pokud se nějaké dva obdélníky protnou, nastane srážka příslušných dvou agentů.
Agenti nemají žádný způsob jak informovat křižovatku o~srážce.
Z~toho důvodu po~kolizi sražení agenti zmizí ze~simulace, tudíž už s~nimi nemohou žádní další agenti kolidovat.

Simulace podporuje načítání agentů ze~souboru nebo generování agentů za~běhu simulace.

