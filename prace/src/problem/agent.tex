\section{Agent}\label{sec:agent}

%Popis zjednodušení auta na agenta.
%Popis parametrů agenta.


\nameref{sec:agent} je zjednodušení chytrého auta pro snadnější použití v~algoritmech.
Tvar \hyperref[sec:agent]{agenta} je zjednodušený na~obdélník, který má určitou délku, šířku a naklonění v~rovině.
Tento obdélník reprezentuje pohled na~auto shora.
Šířka a délka \hyperref[sec:agent]{agenta} je určená vůči \hyperref[par:velikost_bloku]{velikosti bloku} dané křižovatky.

Pokud se nějaké dva obdélníky protnou, nastane srážka příslušných dvou \hyperref[sec:agent]{agentů}.
\hyperref[sec:agent]{Agenti} nemají žádný způsob jak informovat křižovatku o~srážce.
Z~toho důvodu po~kolizi sražení \hyperref[sec:agent]{agenti} zmizí a žádní jiní \hyperref[sec:agent]{agenti} s~nimi nemohou kolidovat.

Jelikož se \hyperref[sec:agent]{agent} pohybuje po~hranách grafu křižovatky, je jeho cesta složená z~úseček.
Proto jsem umožnil \hyperref[sec:agent]{agentovy} otáčet se na~místě a otáčení probíhá okamžitě po~příjezdu do~vrcholu.
\hyperref[sec:agent]{Agent} tedy pořád cestuje a otáčí se vždy ve~směru jízdy.
Tato vlastnost by neměla mít na~pohyb auta velký vliv, pokud je auto dostatečně malé
oproti vzdálenostem vrcholů grafu křižovatky (\hyperref[par:velikost_bloku]{velikosti bloku}).

\hyperref[sec:agent]{Agent}při~příjezdu na~křižovatku nahlásí křižovatce odkud jede a kam směřuje.
\hyperref[sec:krizovatka]{Křižovatka} se poté pokusí najít \hyperref[par:cesta]{cestu} splňující \hyperref[sec:agent]{agentovi požadavky}.

\paragraph{Cesta}\label{par:cesta} je tvořena posloupností vrcholů, přes které má \hyperref[sec:agent]{agent} jet.
Pokud křižovatka nenajde žádnou cestu, zamítne vjezd \hyperref[sec:agent]{agenta}.
V~případě, kdy~\hyperref[sec:agent]{agent} čeká na~vjezd příliš dlouho,
\uv{dojde \hyperref[sec:agent]{agentovi} trpělivost a vydá se jinou cestou}.
Znamená to, že \hyperref[sec:agent]{agent} nahlásí křižovatce, že už~nemá zájem o~průjezd.
