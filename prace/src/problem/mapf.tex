\subsubsection{Offline~MAPF}\label{subsubsec:offline_mapf}

%Definice MAPF a s ním spojených pojmů (Sum of costs, \ldots).
%
%Analogie a rozdíly oproti problému práce.

Pro hromadné hledání cest v diskrétním prostoru existuje hromada řešení.
Problém, které hromadné plánování řeší je označován \emph{Multi-Agent Path Finding} \labeltext{MAPF}{str:mapf}
a podrobnou definici a varianty popsali například \citet*{osti_10114869}.
\ref{str:mapf} má na~vstupu dvojici $G, A$, kde $G=(V, E)$ je graf a $A = \{a_1, \dots, a_k\}$ je množina agentů.
Opět stejně jako u~\nameref{sec:a_star} je možné diskretizovat křižovatku
a převést na~graf (popsáno v sekci \ref{sec:krizovatka}).
Každý agent $a_i$ má svojí výchozí pozici $s_i \in V$ a cílovou pozici $g_i \in V$.
Popis převodu auta na agenta je popsán v sekci \nameref{sec:agent}.
Stejně jako u \nameref{sec:a_star} pracuje \ref{str:mapf} s diskrétními časovými úseky (kroky).
Během jednoho kroku může agent přejet do sousedního vrcholu, nebo počkat v aktuálním.
Plán pro agenta $a_i$ je sled $\pi_i = s_i, v_2, \dots, v_{n-1}, g_i$ na grafu $G$, čili $v_2, \dots, v_{n-1} \in V$ a
$(s_i, v_2) \in E, (v_{n-1}, g_i) \in E, \forall_{i \in 2, \dots, n-2} (v_i, v_{i+1}) \in E$.
Délka plánu je $|\pi_i| = n$, pozici agenta~$a_i$ v~kroku~$c$ značím~$\pi_i[c]$.

Dle definice~\ref{str:mapf} jsou agenti $a_i$ a $a_j$, $i \neq j$ v \emph{kolizi} v kroku $c$
právě tehdy když platí jedna z následujících podmínek.
\begin{gather}
  \pi_i[c] = \pi_j[c] \label{eq:mapf_kolize_vrchol}\\
  \pi_i[c] = \pi_j[c + 1] \land \pi_i[c + 1] = \pi_j[c] \label{eq:mapf_kolize_hrana}
\end{gather}
Slovy řečeno, agenti jsou v \emph{kolizi}, pokud jsou ve stejný čas na~stejném místě, nebo projíždí stejnou hranou.
Pro problém křižovatky je nutné kontrolu kolize rozšířit, jelikož agenti mají svojí velikost (\ref{sec:kolize}).

\ref{str:mapf} se snaží o nalezení plánu $\pi = \cup_{i=1}^{k} \pi_i$, který nemá žádné kolize.
Takovýto plán je nazýván validní.
Pro problém mohou existovat různé plány, tyto plány bývají často porovnány pomocí \emph{Sum Of Costs} \labeltext{SoC}{str:soc} metriky.
Plán $\pi$ má cenu podle metriky \ref{str:soc}: $soc(\pi) = |\pi| = \sum_{i=1}^{k} |\pi_i|$.
Alternativní způsob porovnání je objektivní funkcí $\textrm{makespan}\labeltext{makespan}{str:makespan}(\pi)$ pro plán $\pi$,
která vrací počet kroků, než všichni agenti dorazí do svého cíle.
Vzorcem $\textrm{makespan}(\pi)=\max_{i\in A} |\pi_i|$.

\subsubsection{Řešení~\nameref{subsubsec:offline_mapf}}\label{subsubsec:reseni_offline_mapf}

%Stručný popis známých algoritmů pro MAPF s citacemi (CBS, A*, SAT).


Nejjednodušší způsob řešení je využít A* algoritmus, kde stav je kartézským součinem stavů všech plánovaných agentů.
Toto řešení má často vysoký větvící faktor.
Proto se vyvinuly vylepšení, například \emph{Independence Detection}, \emph{Conflict Avoidance Table} nebo
\emph{Operator Decomposition} \citep{Standley_2010} a mnoho dalších.
Bližší popis těchto vylepšení je v sekci \nameref{sec:a_star}. % TODO hromadny A*

\citet*{Sharon} navrhli algoritmus \nameref{subsec:conflict_based_search},
který nalezne nejkratší cesty pro všechny agenty nezávisle na ostatních.
Poté hledá konflikty mezi jednotlivými plány.
Pokud algoritmus nalezne konflikt, hledání se rozdělí na dva podpřípady.
První větev výpočtu najde alternativní cestu pro prvního agenta, druhá větev pro druhého.
Takto postupně vzniká binární strom.
Pokud je nalezena nekolizní cesta pro všechny vrcholy, algoritmus skončí.
Pořadí prohledávání vrcholů ve stromu je určeno podle SOC metriky.
Toto pořadí zaručuje optimální řešení \citep{Sharon}.
Algoritmus byl nadále rozšířen a vylepšen \citep{Boyarski}.
Bližší popis algoritmu je v sekci \nameref{subsec:conflict_based_search}

\emph{MAPF} problém je možné převést na známý \emph{SAT} problém.  % TODO ref to SAT definition
Nejprve se vytvoří výrokové proměnné pro každého agenta, každý vrchol a každý čas.
Agent musí splňovat určité podmínky, například agent se nachází v čase příjezdu na vrcholu vjezdu
nebo agent může být v jeden čas maximálně na jednom vrcholu.
Následně přidáme podmínky zaručující nekolizní cesty pro agenty.
Tento způsob řešení je spíše vhodný pro optimalizování \ref{str:makespan} funkce,
avšak je možné vytvořit varianty cílené na \ref{str:soc} pomocí rozšíření na \emph{MAXSAT} \citep{bartak}.
Blíže je toto řešení popsáno v sekci \nameref{sec:sat-planner}.

Existují i jiná řešení \ref{str:mapf}, například za použití zpětnovazebního učení \citep*{Zhiyao}.
Pro svou práci jsem se rozhodl věnovat se pouze základním metodám
\nameref{sec:a_star}, \nameref{subsec:conflict_based_search} a \emph{SAT}. % TODO ref to SAT

\subsubsection{Online~MAPF}\label{subsubsec:online_mapf}

%Popis rozšíření z offline na online, popis způsobů řešení.
%
%Definice optimality (optimal vs snapshot-optimal).

\nameref{subsubsec:offline_mapf} hledá řešení pouze jednou, avšak auta přijíždějí na křižovatku neustále.
Můžeme předpokládat, že většina cestujících auty nezná přesný, často ani přibližný čas svého příjezdu ke křižovatce.
Proto by křižovatka měla být schopna plánovat nově přijíždějící auta s ohledem na auta na křižovatce.
Naštěstí existuje rozšíření \emph{offline~MAPF} problému na problém \nameref{subsubsec:online_mapf} \citet*{Svancara}.
Online varianta \ref{str:mapf} splňuje všechny potřeby naší křižovatky.

\emph{Online~MAPF} má u každého agenta $a_i = (t_i, s_i, g_i)$ kromě místa příjezdu a cíle také čas příjezdu $t_i$.
Avšak Tento čas není dopředu znám.
\emph{Online~MAPF} začíná s počátečním \emph{offline~MAPF} plánem pro agenty, kteří přijeli v čase $0$.
Tento plán budu značit $\pi^0$.
Pokaždé, když se objeví noví agenti, vytvoří se nový plán $\pi^j$.
Celkový plán je tedy $\Pi = (\pi^0, \pi^1, \dots, \pi^m)$, kde $m$ je počet unikátních kroků ($t_1, t_2, \dots, t_m$), kdy se objevili agenti.
Označím si $\pi^j[x:y]$ část plánu $\pi^j$ v krocích $x, x + 1, \dots, y - 1, y$.
Celkový plán, který budou agenti vykonávat je tedy $Ex[\Pi] = \pi^0[0:t_1] \circ \pi^1[t_1 + 1:t_2] \circ \dots \circ \pi^m[t_m + 1:\infty]$.

\citet{Svancara} zmínili problémy s~\emph{online~MAPF}.
První problém nastane, pokud agenti zůstanou na svém místě po doražení do cíle.
Zároveň pokud by se agenti okamžitě objevili v grafu, mohli by ihned způsobit kolizi, kterou algoritmy nemohli predikovat.
Žádný z těchto problémů u mě nastat nemůže, jelikož auta přijíždějí na vjezdy
a zároveň mizí z křižovatky po doražení výjezdu.

Opět zavedu cenu plánu jako součet délek plánů pro jednotlivé agenty $|Ex[\Pi]| = \sum_{i=1}^{k} |Ex[\Pi]_i| = \sum_{i=1}^{k} t_{Ex[\Pi]}[g_i] - t_i$,
kde $t_{Ex[\Pi]}[g_i]$ je krok, kdy agent $a_i = (t_i, s_i, g_i)$ naposledy dorazil do cílového vrcholu $g_i$.
Z analýzy \citet{Svancara} víme, že cena $|Ex[\Pi]|$ je ekvivalentní objektivní funkci $\sum_{t=1}^{\infty} \textrm{NotAtGoal}(t)$,
kde $\textrm{NotAtGoal}(t)$ udává počet agentů, kteří ještě nedorazili do svého cíle v čase $t$.
Také objektivní funkce $\sum_{i=1}^{k} |Ex[\Pi]_i| - o_i$, kde $o_i$ je délka nejkratší cesty mezi $s_i$ a $g_i$,
je ekvivalentní $|Ex[\Pi]|$.

Každý \emph{online~MAPF} problém je možné převést na \emph{offline~MAPF} pokud dáme dopředu algoritmu vědět, kdy se agenti objeví.
Díky tomu můžeme porovnat optimalitu online řešičů.
\citet{Svancara} dokázali, že žádný online algoritmus nemůže zajistit offline optimální řešení.
\emph{Snapshot-optimální}\labeltext{snapshot-optimální}{str:snapshot_opt} plány jsou optimální plány za předpokladu, že se žádní noví agenti neobjeví.
\citet*{Morag} provedli rozsáhlé experimenty a zjistili, že \ref{str:snapshot_opt} plány nejsou o moc horší než optimální.
Ve všech typech experimentů \citet{Morag} byly \ref{str:snapshot_opt} ceny plánů alespoň v $80\%$ běhů totožné s optimálním plánem
a ve zbylých případech se plány lišily minimálně.


%Rozšíření \emph{offline~MAPF} problému na online variantu zkoumali ve své práci \citet*{Svancara}.
%\emph{Online~MAPF} má u každého agenta $a_i = (t_i, s_i, g_i)$ kromě místa příjezdu a cíle také čas příjezdu $t_i$.
%Tento čas není dopředu znám.
%\emph{Online~MAPF} začíná s počátečním \emph{offline~MAPF} plánem pro agenty, kteří přijeli v čase $0$.
%Tento plán budu značit $\pi^0$.
%Pokaždé, když se objeví noví agenti, vytvoří se nový plán $\pi^j$.
%Celkový plán je tedy $\Pi = (\pi^0, \pi^1, \dots, \pi^m)$, kde $m$ je počet unikátních kroků ($t_1, t_2, \dots, t_m$), kdy se objevili agenti.
%Označím si $\pi^j[x:y]$ část plánu $\pi^j$ v krocích $x, x + 1, \dots, y - 1, y$.
%Celkový plán, který budou agenti vykonávat je tedy $Ex[\Pi] = \pi^0[0:t_1] \circ \pi^1[t_1 + 1:t_2] \circ \dots \circ \pi^m[t_m + 1:\infty]$.
%
%\citet{Svancara} zmínili problémy s~\emph{online~MAPF}.
%První problém nastane, pokud agenti zůstanou na svém místě po doražení do cíle.
%Zároveň pokud by se agenti okamžitě objevili v grafu, mohli by ihned způsobit kolizi, kterou algoritmy nemohli predikovat.
%Žádný z těchto problémů u mě nastat nemůže, jelikož agenti mohou být zamítnuti, pokud by došlo ke kolizi hned na vjezdu.
%Agenti taky mizí z křižovatky po doražení do výjezdu.
%
%Opět zavedu cenu plánu jako součet délek plánů pro jednotlivé agenty $|Ex[\Pi]| = \sum_{i=1}^{k} |Ex[\Pi]_i| = \sum_{i=1}^{k} t_{Ex[\Pi]}[g_i] - t_i$,
%kde $t_{Ex[\Pi]}[g_i]$ je krok, kdy agent $a_i = (t_i, s_i, g_i)$ naposledy dorazil do cílového vrcholu $g_i$.
%Z analýzy \citet{Svancara} víme, že cena $|Ex[\Pi]|$ je ekvivalentní objektivní funkci $\sum_{t=1}^{\infty} \textrm{NotAtGoal}(t)$,
%kde $\textrm{NotAtGoal}(t)$ udává počet agentů, kteří ještě nedorazili do svého cíle v čase $t$.
%Také objektivní funkce $\sum_{i=1}^{k} |Ex[\Pi]_i| - o_i$, kde $o_i$ je délka nejkratší cesty mezi $s_i$ a $g_i$,
%je ekvivalentní $|Ex[\Pi]|$.
%
%Každý \emph{online~MAPF} problém je možné převést na \emph{offline~MAPF} pokud dáme dopředu algoritmu vědět, kdy se agenti objeví.
%Díky tomu můžeme porovnat optimalitu online řešičů.
%\citet{Svancara} dokázali, že žádný online algoritmus nemůže zajistit offline optimální řešení.
%\emph{Snapshot-optimální} plány jsou optimální plány za předpokladu, že se žádní noví agenti neobjeví.
%\citet*{Morag} provedli rozsáhlé experimenty a zjistili, že \emph{snapshot-optimální} plány nejsou o moc horší než optimální.
%Ve všech typech experimentů byly \emph{snapshot-optimální} ceny plánů alespoň v $80\%$ běhů totožné s optimálním plánem
%a ve zbylých případech se plány lišily minimálně.

\subsubsection{Řešení~\nameref{subsubsec:online_mapf}}\label{subsubsec:reseni_online_mapf}


%Popis úpravy offline algoritmů pro řešení online MAPF\@.


V práci \citet{Svancara} jsou návrhy různých strategií pro řešení \emph{online~MAPF} problémů:
\begin{itemize}
  \item \textbf{Replan~Single}\labeltext{RS}{str:rs} - tento přístup je totožný s~přístupem \nameref{subsec:individualni_planovani},
  každý nový agent je naplánován s ohledem na předchozí.
  \item \textbf{Replan~Single~Grouped}\label{par:replan_single_grouped}\labeltext{RSG}{str:rsg} -
  v tomto přístupu se plánují pouze noví agenti, ale oproti \ref{str:rs} plánování probíhá pro všechny agenty najednou.
  Zde lze použít \emph{offline~MAPF} řešič, který se musí vyhnout kolizím s již naplánovanými trasami.
  \item \textbf{Replan~All}\labeltext{RA}{str:ra} - za použití této strategie se použije
  \emph{offline~MAPF} řešič na všechny agenty pokaždé, když dorazí noví agenti.
  Pokud je řešič optimální, \ref{str:ra} vrací \ref{str:snapshot_opt} řešení pro všechny agenty \citep{Svancara}.
  \item \textbf{Online~Independence~Detection}\labeltext{OID}{str:oid} - tento přístup se snaží minimalizovat množství přeplánovaných agentů.
  Nejprve najde cestu pro všechny nové agenty ignorujíce už naplánované.
  Poté zjistí kolize mezi starými a novými agenty.
  Pokud byly nalezeny kolize, přeplánují se trasy kolizních agentů.
  Pro zaručení \ref{str:snapshot_opt} plánu je nutné udělat dodatečné úpravy \citep{Svancara}.
  \item \textbf{Suboptimal~Independence~Detection}\labeltext{SubID}{str:subid} - pozměňuje \ref{str:oid} dovolováním neoptimálních cest.
  Přesněji cena plánu SubID je nejvýše $D$ krát delší než cena \emph{snapshot~optimálního} plánu.
  Avšak díky této úpravě by měl být počet přeplánování, a tedy i čas výpočtu, nižší.
\end{itemize}


%V práci \citet{Svancara} jsou návrhy různých postupů řešení \emph{online~MAPF} problémů:
%\begin{itemize}
%  \item \textbf{Replan~Single} (RS) - tento přístup je totožný s~přístupem \nameref{sec:individualni_planovani}.
%  \item \textbf{Replan~Single~Grouped}\label{par:replan-single-grouped} (RSG) - v tomto přístupu se plánují pouze noví agenti.
%  Plánování probíhá pro všechny agenty najednou.
%  Zde lze použít \emph{offline~MAPF} řešič, který se musí vyhnout kolizím s již naplánovanými trasami.
%  \item \textbf{Replan~All} (RA) - za použití této strategie se použije \emph{offline~MAPF} řešič na všechny agenty pokaždé, když dorazí noví agenti.
%  Pokud je řešič optimální, \emph{Replan~all} vrací \emph{snapshot~optimální} řešení \citep{Svancara}.
%  \item \textbf{Online~Independence~Detection} (OID)- Tento přístup se snaží minimalizovat množství přeplánovaných agentů.
%  Nejprve najde cestu pro všechny nové agenty ignorujíce už naplánované.
%  Poté zjistí kolize mezi starými a novými agenty.
%  Pokud byly nalezeny kolize, přeplánují se trasy kolizních agentů.
%  Pro zaručení \emph{snapshot~optimálního} plánu je nutné udělat dodatečné úpravy \citep{Svancara}.
%  \item \textbf{Suboptimal~Independence~Detection} (SubID) - pozměňuje OID dovolováním neoptimálních cest.
%  Přesněji cena plánu SubID je nejvýše $D$ krát delší než cena \emph{snapshot~optimálního} plánu.
%  Avšak díky této úpravě by měl být počet přeplánování, a tedy i čas výpočtu, nižší.
%\end{itemize}
%
%
%\section{Inteligentní křižovatka}\label{sec:inteligentni-krizovatka}
%Problém popsaný v této práci, přidává do \emph{online~MAPF} další podmínky.
%U křižovatky všichni agenti mají specifikovány vrcholy, kde může být jejich start a cíl.
%Agenti také nejsou pouhé body, ale mají svojí velikost.
%Díky tomu je zjišťování kolizí komplexnější.
%
%Dále je možné přiblížit se reálné křižovatce dalšími úpravami.
%\begin{itemize}
%  \item Pokud agentovi nezáleží na pruhu, kterým vyjede, může sdělit algoritmu pouze směr výjezdu.
%  Algoritmus má za cíl najít cestu na libovolný výjezd v daném směru.
%  V důsledku může agent místo jednoho koncového vrcholu mít množinu vrcholů.
%  \item Agent představuje jedoucí vozidlo.
%  Proto mohu po algoritmu vyžadovat, aby se agent nikdy nezastavil na místě.
%  Zároveň mohu vyžadovat, aby se agentovo cesta neměnila po vjezdu do křižovatky.
%  Podmínku, že křižovatka nemůže měnit individuální plány za běhu, již zmínil \citet{Dresner}.
%  Kdybychom změnu plánů dovolili, mohlo by dojít k chybě v komunikaci, díky níž by agent prováděl původní plán,
%  avšak křižovatka by počítala s novým plánem.
%\end{itemize}
%
%Všechny tyto varianty zkoumám a porovnávám v experimentech.
