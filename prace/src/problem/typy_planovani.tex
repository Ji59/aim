\section{Typy plánování}\label{sec:typy_planovani}

%Stručný popis rozdílu mezi ind. a hrom. plánováním.

Rozlišuji dva případy centrálního plánování.
\textrm{\nameref{subsec:individualni_planovani}} plánuje trasu pro každé auto zvlášť.
Každé následné plánování potom hledá trasu pro nové auto takovým způsobem, aby nedošlo ke kolizi s už naplánovanými auty.

S tímto přístupem by se mohlo stát, že naplánovaná trasa auta přerušuje kratší trasy následných aut.
Mohlo by být výhodnější nenaplánovat nejkratší trasu předchozímu autu a celkově dojít lepšímu řešení.
Tento problém se snaží vyřešit \nameref{subsec:hromadne_planovani}.

\textrm{\nameref{subsec:hromadne_planovani}} shlukuje auta do skupin a hledá trasy pro auta společně.
Díky tomuto přístupu je možné najít trasy, ve kterých jsou celkově pasažéři v autech nejvíce spokojení.

\subsection{Individuální plánování}\label{subsec:individualni_planovani}

%Přesnější definice.
%
%Popsání práce \citet{Dresner}.
%
%Stručný popis BFS, A*.




Jak bylo popsáno výše, auta jsou naplánováni jeden po druhém a
každý nový plán je rozvržen tak, aby nekolidoval s žádným už naplánovaným autem.

Při plánování se používá \emph{First Come First Served} \labeltext{FCFS}{str:fcfs} strategie pro~určení pořadí plánování.
\ref{str:fcfs} strategie přiřazuje nejvyšší prioritu v~plánování autu, které dorazilo ke~křižovatce nejdříve.
Tímto způsobem jsou minimalizovány čekací doby jednotlivých aut a maximalizována spokojenost aut díky férovému přístupu.

Model chytré křižovatky s touto strategií už nasimulovali \citet*{Dresner}.
Jejich přístup odpovídal reálnému provozu, křižovatka neměla centrální jednotku.
Auta byla schopná vzájemné komunikace.
Auta v jejich práci jsou schopna jezdit pouze v předem daných pruzích.
V těchto pruzích následně mohou auta zrychlovat či zpomalovat.
Algoritmus nejprve přiřadí autu maximální rychlost a zjistí, zda by na jeho cestě došlo ke kolizi.
Pokud ano, zkusí nižší rychlost před místem kolize.
Postup se opakuje, dokud nedojde k nalezení nekolizní cesty skrze křižovatku.
Agenti jsou plánováni pomocí \ref{str:fcfs} strategie.
Tímto algoritmem jsem se inspiroval u~algoritmu \nameref{sec:safe_lanes}.

U řešení \citet{Dresner} se mi nelíbilo omezení aut na~pruhy.
\nameref{sec:a_star} je známý algoritmus na hledání nejkratší trasy v obecném prostoru.
Algoritmus byl vytvořen \citet*{Hart1968} pro robotické účely.
Pro použití \nameref{sec:a_star} je avšak nejdříve nutné diskretizovat křižovatku a čas, a upravit model aut.
Tato úprava je popsána v samostatných sekcích (\ref{sec:krizovatka}, \ref{sec:agent}).
\nameref{sec:a_star} dovoluje autům plně využít celou plochu křižovatky a algoritmus nám zaručuje optimální cestu pro každé auto při zafixování ostatních aut.
\nameref{sec:a_star} je založen na algoritmu \emph{Breath First Search} s chytřejším procházením stavů.
Podrobný popis algoritmu je popsán v samostatné kapitole (\ref{sec:a_star}).

%Jako první se problematikou zabývali \citet*{Dresner}.
%Jejich přístup více odpovídal reálnému provozu.
%Agenti měli plynulé zatáčení a uměli zrychlovat a zpomalovat.
%Dále \citet{Dresner} řešili problémy komunikace mezi agenty a křižovatkou, řešení situace při kolizi a podporu pro vozidla s lidským řidičem.
%Agenti v jejich práci jsou schopni jezdit pouze v předem daných pruzích.
%V těchto pruzích následně mohou auta zrychlovat či zpomalovat.
%Algoritmus nejprve přiřadí autu maximální rychlost a zjistí, zda by na jeho cestě došlo ke kolizi.
%Pokud ano, zkusí nižší rychlost před místem kolize.
%Postup se opakuje, dokud nedojde k nalezení nekolizní cesty skrze křižovatku.
%Agenti jsou plánováni pomocí \ref{str:fcfs} strategie.
%Tímto algoritmem jsem se inspiroval u~algoritmu \nameref{sec:safe_lanes}.
%V mé implementaci jezdí auta stejnou rychlostí v předem daných pruzích.
%Pokud by u agenta došlo ke kolizi, je jeho příjezd odložen.
%
%Řešení je možné rozšířit pomocí libovolného prohledávacího algoritmu.
%Tento přístup jsem vyzkoušel použitím známého algoritmu \nameref{sec:a_star},
%kdy prvního agenta naplánuji nejkratší cestou, poté naplánuji druhého tak, aby nekolidoval s prvním.
%Takto pokračuji pro všechny přijíždějící agenty.

\subsection{Hromadné plánování}\label{subsec:hromadne_planovani}

%Popis plánovače, výhody a nevýhody oproti individuálnímu plánování
%(optimalita řešení, porovnání velikosti prohledávaných prostorů).

Výše popsaná situace, kdy auto blokuje výhodnější trasy ostatních, ukazuje nevýhodu postupného plánování.
Může být výhodné rozdělit si auta do časových intervalů, kdy přijedou na křižovatku.
Následně lze použít algoritmus pro naplánování co nejlepší trasy pro všechna auta v rámci jednoho intervalu najednou.
Pokud auta informují křižovatku o svém příjezdu s předstihem, můžeme i přeplánovat trasu podle nově nahlášených aut.

%
%Řešení tohoto typu jsou složitější, avšak teoreticky by měla být schopna tvořit celkově lepší plány.
%
%Chytrá křižovatka se dá převézt na online \emph{MAPF} (Multi-Agent Path Finding) problém.
%

\subsubsection{Offline~MAPF}\label{subsubsec:offline_mapf}

Definice MAPF a s ním spojených pojmů (Sum of costs, \ldots).

Analogie a rozdíly oproti problému práce.

%\nameref{subsec:offline_mapf} má na~vstupu dvojici $G, A$, kde $G=(V, E)$ je graf a $A = \{a_1, \dots, a_k\}$ je množina agentů.
%Každý agent $a_i$ má svojí výchozí pozici $s_i \in V$ a cílovou pozici $g_i \in V$.
%Čas je rozdělen na diskrétní úseky (kroky).
%Během jednoho kroku může agent přejet do sousedního vrcholu, nebo počkat v aktuálním.
%Plán pro agenta $a_i$ je sled $\pi_i = s_i, v_2, \dots, v_{n-1}, g_i$ na grafu $G$, čili $v_2, \dots, v_{n-1} \in V$ a
%$(s_i, v_2) \in E, (v_{n-1}, g_i) \in E, \forall_{i \in 2, \dots, n-2} (v_i, v_{i+1}) \in E$.
%Délka plánu je $|\pi_i| = n$, pozice agenta $a_i$ v kroku $c$ je $\pi_i[c]$.
%
%Agenti $a_i$ a $a_j$, $i \neq j$ jsou v \emph{kolizi} v kroku $c$ právě tehdy když $\pi_i[c] = \pi_j[c]$ nebo
%$\pi_i[c] = \pi_j[c + 1] \land \pi_i[c + 1] = \pi_j[c]$.
%Slovy řečeno, agenti jsou v \emph{kolizi}, pokud jsou na stejném místě, nebo projíždí stejnou hranou.
%
%Cílem \emph{offline MAPF} je nalezení plánu $\pi = \cup_{i=1}^{k} \pi_i$, který nemá žádné kolize.
%Takovýto plán nazýváme validní.
%Pro problém mohou existovat různé plány, tyto plány bývají často porovnány pomocí \emph{SOC} (Sum Of Costs).
%Plán $\pi$ má cenu $|\pi| = \sum_{i=1}^{k} |\pi_i|$.
%Alternativní způsob porovnání je objektivní funkcí pro plán $\pi$ funkce $\textrm{makespan}\labeltext{makespan}{str:makespan}(\pi)$,
%která značí počet kroků, než všichni agenti dorazí do svého cíle.

\paragraph{Řešení~\nameref{subsubsec:offline_mapf}}\label{par:reseni_offline_mapf}

.

Stručný popis známých algoritmů pro MAPF s citacemi (CBS, A*, SAT).

%
%Nejjednodušší způsob řešení je využít A* algoritmus, kde následníci stavu jsou kartézským součinem přes všechny možné tahy všech agentů.
%Toto řešení má často vysoký větvící faktor.
%Proto se vyvinuly vylepšení, například \emph{Independence Detection}, \emph{Conflict Avoidance Table} nebo
%\emph{Operator Decomposition} \citep{Standley_2010} a mnoho dalších.
%
%\citet*{Sharon} navrhli algoritmus \emph{CBS} (Conflict-Based Search), který nalezne nejkratší cesty pro všechny agenty.
%Poté hledá konflikty mezi jednotlivými plány.
%Pokud nalezne konflikt, vznikne omezující podmínka pro jednoho agenta v kolizi.
%Tato podmínka znemožní agentovi být na konfliktním vrcholu.
%Poté se s novou podmínkou spustí nové prohledávání pro tohoto agenta.
%Zároveň vznikne druhá větev výpočtu, ve které má tuto podmínku druhý agent z kolize.
%Takto postupně vzniká binární strom, kde synové vrcholu mají vždy podmínky z rodiče plus pro prvního / druhého agenta novou podmínku.
%Pokud je nalezena nekolizní cesta pro všechny vrcholy, algoritmus skončí.
%Pořadí prohledávání listů ve stromu je určeno podle SOC listů.
%Toto pořadí zaručuje optimální řešení \citep{Sharon}.
%Algoritmus byl nadále rozšířen a vylepšen \citep{Boyarski}.
%
%\emph{MAPF} problém je možné převést na SAT problém.
%Nejprve se vytvoří výrokové proměnné pro každého agenta, každý vrchol a každý čas.
%Následně přidáme podmínky, aby agenti nebyli v kolizi.
%Tento způsob řešení je spíše vhodný pro optimalizování \ref{str:makespan} funkce,
%avšak je možné vytvořit varianty cílené na SOC \citep{bartak}.
%Blíže je toto řešení popsáno v kapitole \nameref{sec:sat-planner}.
%
%Další způsob řešení je za použití zpětnovazebního učení \citep*{Zhiyao}.

\subsubsection{Online~MAPF}\label{subsubsec:online_mapf}

Popis rozšíření z offline na online, popis způsobů řešení.

Definice optimality (optimal vs snapshot-optimal).

%
%Rozšíření \emph{offline~MAPF} problému na online variantu zkoumali ve své práci \citet*{Svancara}.
%\emph{Online~MAPF} má u každého agenta $a_i = (t_i, s_i, g_i)$ kromě místa příjezdu a cíle také čas příjezdu $t_i$.
%Tento čas není dopředu znám.
%\emph{Online~MAPF} začíná s počátečním \emph{offline~MAPF} plánem pro agenty, kteří přijeli v čase $0$.
%Tento plán budu značit $\pi^0$.
%Pokaždé, když se objeví noví agenti, vytvoří se nový plán $\pi^j$.
%Celkový plán je tedy $\Pi = (\pi^0, \pi^1, \dots, \pi^m)$, kde $m$ je počet unikátních kroků ($t_1, t_2, \dots, t_m$), kdy se objevili agenti.
%Označím si $\pi^j[x:y]$ část plánu $\pi^j$ v krocích $x, x + 1, \dots, y - 1, y$.
%Celkový plán, který budou agenti vykonávat je tedy $Ex[\Pi] = \pi^0[0:t_1] \circ \pi^1[t_1 + 1:t_2] \circ \dots \circ \pi^m[t_m + 1:\infty]$.
%
%\citet{Svancara} zmínili problémy s~\emph{online~MAPF}.
%První problém nastane, pokud agenti zůstanou na svém místě po doražení do cíle.
%Zároveň pokud by se agenti okamžitě objevili v grafu, mohli by ihned způsobit kolizi, kterou algoritmy nemohli predikovat.
%Žádný z těchto problémů u mě nastat nemůže, jelikož agenti mohou být zamítnuti, pokud by došlo ke kolizi hned na vjezdu.
%Agenti taky mizí z křižovatky po doražení do výjezdu.
%
%Opět zavedu cenu plánu jako součet délek plánů pro jednotlivé agenty $|Ex[\Pi]| = \sum_{i=1}^{k} |Ex[\Pi]_i| = \sum_{i=1}^{k} t_{Ex[\Pi]}[g_i] - t_i$,
%kde $t_{Ex[\Pi]}[g_i]$ je krok, kdy agent $a_i = (t_i, s_i, g_i)$ naposledy dorazil do cílového vrcholu $g_i$.
%Z analýzy \citet{Svancara} víme, že cena $|Ex[\Pi]|$ je ekvivalentní objektivní funkci $\sum_{t=1}^{\infty} \textrm{NotAtGoal}(t)$,
%kde $\textrm{NotAtGoal}(t)$ udává počet agentů, kteří ještě nedorazili do svého cíle v čase $t$.
%Také objektivní funkce $\sum_{i=1}^{k} |Ex[\Pi]_i| - o_i$, kde $o_i$ je délka nejkratší cesty mezi $s_i$ a $g_i$,
%je ekvivalentní $|Ex[\Pi]|$.
%
%Každý \emph{online~MAPF} problém je možné převést na \emph{offline~MAPF} pokud dáme dopředu algoritmu vědět, kdy se agenti objeví.
%Díky tomu můžeme porovnat optimalitu online řešičů.
%\citet{Svancara} dokázali, že žádný online algoritmus nemůže zajistit offline optimální řešení.
%\emph{Snapshot-optimální} plány jsou optimální plány za předpokladu, že se žádní noví agenti neobjeví.
%\citet*{Morag} provedli rozsáhlé experimenty a zjistili, že \emph{snapshot-optimální} plány nejsou o moc horší než optimální.
%Ve všech typech experimentů byly \emph{snapshot-optimální} ceny plánů alespoň v $80\%$ běhů totožné s optimálním plánem
%a ve zbylých případech se plány lišily minimálně.

\paragraph{Řešení~\nameref{subsubsec:online_mapf}}\label{par:reseni_online_mapf}



Popis úpravy offline algoritmů pro řešení online MAPF\@.

%
%V práci \citet{Svancara} jsou návrhy různých postupů řešení \emph{online~MAPF} problémů:
%\begin{itemize}
%  \item \textbf{Replan~Single} (RS) - tento přístup je totožný s~přístupem \nameref{sec:individualni_planovani}.
%  \item \textbf{Replan~Single~Grouped}\label{par:replan-single-grouped} (RSG) - v tomto přístupu se plánují pouze noví agenti.
%  Plánování probíhá pro všechny agenty najednou.
%  Zde lze použít \emph{offline~MAPF} řešič, který se musí vyhnout kolizím s již naplánovanými trasami.
%  \item \textbf{Replan~All} (RA) - za použití této strategie se použije \emph{offline~MAPF} řešič na všechny agenty pokaždé, když dorazí noví agenti.
%  Pokud je řešič optimální, \emph{Replan~all} vrací \emph{snapshot~optimální} řešení \citep{Svancara}.
%  \item \textbf{Online~Independence~Detection} (OID)- Tento přístup se snaží minimalizovat množství přeplánovaných agentů.
%  Nejprve najde cestu pro všechny nové agenty ignorujíce už naplánované.
%  Poté zjistí kolize mezi starými a novými agenty.
%  Pokud byly nalezeny kolize, přeplánují se trasy kolizních agentů.
%  Pro zaručení \emph{snapshot~optimálního} plánu je nutné udělat dodatečné úpravy \citep{Svancara}.
%  \item \textbf{Suboptimal~Independence~Detection} (SubID) - pozměňuje OID dovolováním neoptimálních cest.
%  Přesněji cena plánu SubID je nejvýše $D$ krát delší než cena \emph{snapshot~optimálního} plánu.
%  Avšak díky této úpravě by měl být počet přeplánování, a tedy i čas výpočtu, nižší.
%\end{itemize}
%
%
%\section{Inteligentní křižovatka}\label{sec:inteligentni-krizovatka}
%Problém popsaný v této práci, přidává do \emph{online~MAPF} další podmínky.
%U křižovatky všichni agenti mají specifikovány vrcholy, kde může být jejich start a cíl.
%Agenti také nejsou pouhé body, ale mají svojí velikost.
%Díky tomu je zjišťování kolizí komplexnější.
%
%Dále je možné přiblížit se reálné křižovatce dalšími úpravami.
%\begin{itemize}
%  \item Pokud agentovi nezáleží na pruhu, kterým vyjede, může sdělit algoritmu pouze směr výjezdu.
%  Algoritmus má za cíl najít cestu na libovolný výjezd v daném směru.
%  V důsledku může agent místo jednoho koncového vrcholu mít množinu vrcholů.
%  \item Agent představuje jedoucí vozidlo.
%  Proto mohu po algoritmu vyžadovat, aby se agent nikdy nezastavil na místě.
%  Zároveň mohu vyžadovat, aby se agentovo cesta neměnila po vjezdu do křižovatky.
%  Podmínku, že křižovatka nemůže měnit individuální plány za běhu, již zmínil \citet{Dresner}.
%  Kdybychom změnu plánů dovolili, mohlo by dojít k chybě v komunikaci, díky níž by agent prováděl původní plán,
%  avšak křižovatka by počítala s novým plánem.
%\end{itemize}
%
%Všechny tyto varianty zkoumám a porovnávám v experimentech.

