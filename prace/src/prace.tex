%%% Hlavní soubor. Zde se definují základní parametry a odkazuje se na ostatní části. %%%

%% Verze pro jednostranný tisk:
% Okraje: levý 40mm, pravý 25mm, horní a dolní 25mm
% (ale pozor, LaTeX si sám přidává 1in)
\documentclass[12pt,a4paper]{report}
\setlength\textwidth{145mm}
\setlength\textheight{247mm}
\setlength\oddsidemargin{15mm}
\setlength\evensidemargin{15mm}
\setlength\topmargin{0mm}
\setlength\headsep{0mm}
\setlength\headheight{0mm}
% \openright zařídí, aby následující text začínal na pravé straně knihy
\let\openright=\clearpage

%% Pokud tiskneme oboustranně:
% \documentclass[12pt,a4paper,twoside,openright]{report}
% \setlength\textwidth{145mm}
% \setlength\textheight{247mm}
% \setlength\oddsidemargin{14.2mm}
% \setlength\evensidemargin{0mm}
% \setlength\topmargin{0mm}
% \setlength\headsep{0mm}
% \setlength\headheight{0mm}
% \let\openright=\cleardoublepage

%% Vytváříme PDF/A-2u
\usepackage[a-2u]{pdfx}

%% Přepneme na českou sazbu a fonty Latin Modern
\usepackage[czech]{babel}
\usepackage{lmodern}
\usepackage[T1]{fontenc}
\usepackage{textcomp}

%% Použité kódování znaků: obvykle latin2, cp1250 nebo utf8:
\usepackage[utf8]{inputenc}

%%% Další užitečné balíčky (jsou součástí běžných distribucí LaTeXu)
\usepackage{amsmath}        % rozšíření pro sazbu matematiky
\usepackage{amsfonts}       % matematické fonty
\usepackage{amsthm}         % sazba vět, definic apod.
\usepackage{bbding}         % balíček s nejrůznějšími symboly
% (čtverečky, hvězdičky, tužtičky, nůžtičky, ...)
\usepackage{bm}             % tučné symboly (příkaz \bm)
\usepackage{graphicx}       % vkládání obrázků
\usepackage{fancyvrb}       % vylepšené prostředí pro strojové písmo
\usepackage{indentfirst}    % zavede odsazení 1. odstavce kapitoly
\usepackage{natbib}         % zajištuje možnost odkazovat na literaturu
% stylem AUTOR (ROK), resp. AUTOR [ČÍSLO]
\usepackage[nottoc]{tocbibind} % zajistí přidání seznamu literatury,
% obrázků a tabulek do obsahu
\usepackage{icomma}         % inteligetní čárka v matematickém módu
\usepackage{dcolumn}        % lepší zarovnání sloupců v tabulkách
\usepackage{booktabs}       % lepší vodorovné linky v tabulkách
\usepackage{paralist}       % lepší enumerate a itemize
\usepackage{xcolor}
\usepackage{hyperref}
\usepackage{mathtools}         % barevná sazba

\usepackage{changepage}
\usepackage{wasysym}

%%% Údaje o práci

% Název práce v jazyce práce (přesně podle zadání)
\newcommand{\NazevPrace}{Autonomní křižovatka}

% Název práce v angličtině
\newcommand{\NazevPraceEN}{Autonomous traffic junction}

% Jméno autora
\newcommand{\AutorPrace}{Jiří Kotal}

% Rok odevzdání
\newcommand{\RokOdevzdani}{2023}

% Název katedry nebo ústavu, kde byla práce oficiálně zadána
% (dle Organizační struktury MFF UK, případně plný název pracoviště mimo MFF)
\newcommand{\Katedra}{Katedra teoretické informatiky a matematické logiky}
\newcommand{\KatedraEN}{Department of Theoretical Computer Science and Mathematical Logic}

% Jedná se o katedru (department) nebo o ústav (institute)?
\newcommand{\TypPracoviste}{Katedra}
\newcommand{\TypPracovisteEN}{Department}

% Vedoucí práce: Jméno a příjmení s~tituly
\newcommand{\Vedouci}{prof. RNDr. Roman Barták, Ph.D.}

% Pracoviště vedoucího (opět dle Organizační struktury MFF)
\newcommand{\KatedraVedouciho}{Katedra teoretické informatiky a matematické logiky}
\newcommand{\KatedraVedoucihoEN}{Department of Theoretical Computer Science and Mathematical Logic}

% Studijní program a obor
\newcommand{\StudijniProgram}{Informatika}
\newcommand{\StudijniObor}{Obecná informatika}

% Nepovinné poděkování (vedoucímu práce, konzultantovi, tomu, kdo
% zapůjčil software, literaturu apod.)
\newcommand{\Podekovani}{%
	Poděkování.
%	Vážený vedoucí [jméno],
%
%	Chtěl bych Vám a rodičům poděkovat za podporu a pomoc při psaní mé bakalářské práce. Je mi ctí moci pracovat pod Vaším vedením a bylo mi velkou radostí, že jste byli ochotni mi věnovat svůj čas a zkušenosti.
%
%	Díky Vašemu vedení a radám jsem byl schopen sestavit kvalitní a důkladnou práci, která splňuje všechny požadavky na úspěšné zakončení mého studia. Navíc jsem si mohl během psaní práce rozšířit své znalosti a získat cenné zkušenosti, které mi pomohou v budoucí kariéře.
%
%	Rodičům také děkuji za podporu a pomoc během celého mého studia. Jejich rady a podpora byly pro mě nesmírně cenné a bez nich bych se asi neobešel. Jsem Vám vděčný za vše, co pro mě udělali a doufám, že budu moci v budoucnu splatit alespoň část toho, co jste pro mě udělali.
%
%	S pozdravem,
%	[jméno]
}

% Abstrakt (doporučený rozsah cca 80-200 slov; nejedná se o zadání práce)
\newcommand{\Abstrakt}{
	Práce se zabývá problematikou průjezdů vozů plně automatizovanou křižovatkou a~to z~pohledu multi-agentního hledání cest (MAPF, multi-agent path finding).
	Cílem je prozkoumat různé abstrakce křižovatky a různé přístupy pro~hledání nekolizního průjezdu křižovatkou.
	Zkoumané techniky jsou implementovány a empiricky porovnány v~simulovaném prostředí.
}

\newcommand{\AbstraktEN}{
	This~work deals with the~issue of passing cars through autonomous traffic junction from~perspective of~multi-agent path finding (MAPF).
	The~aim is to~explore different abstractions of~the~junction and different approaches to~finding non-collisional passage through the~junction.
	Various techniques are implemented and empirically compared in~a~simulated environment.
}


% 3 až 5 klíčových slov (doporučeno), každé uzavřeno ve složených závorkách
\newcommand{\KlicovaSlova}{%
		{plánování cest, }
		{koordinace, }
		{multi-agentní systémy, }
		{křižovatka, }
		{doprava}
}
\newcommand{\KlicovaSlovaEN}{%
		{path planning, }
		{coordination, }
		{multi-agent systems, }
		{traffic junction, }
		{transport}
}

%% Balíček hyperref, kterým jdou vyrábět klikací odkazy v PDF,
%% ale hlavně ho používáme k uložení metadat do PDF (včetně obsahu).
%% Většinu nastavítek přednastaví balíček pdfx.
\hypersetup{unicode}
\hypersetup{breaklinks=true}

%% Definice různých užitečných maker (viz popis uvnitř souboru)
\include{makra}

%% Titulní strana a různé povinné informační strany
\begin{document}
	\include{titulka}

%%% Strana s automaticky generovaným obsahem bakalářské práce

	\tableofcontents

%%% Jednotlivé kapitoly práce jsou pro přehlednost uloženy v samostatných souborech
	\chapter*{Úvod}\label{ch:uvod}
\addcontentsline{toc}{chapter}{Úvod}

Vývoj aut jde velice rychle kupředu.
Každým rokem jsme blíže k světu s~autonomními auty, které budou řídit za~nás.
Samořiditelná auta už~jezdí mezi námi, avšak stále jsou značně omezena manuálně řízenými vozy a~zákony.
Tato auta mají téměř okamžitou odezvu na~vnější vlivy, dokáží se navigovat skrze dopravu s~vysokou přesností
a~také mají výhodu jednodušší vzájemné komunikace.
Těchto vlastností se dá využít pro~zvýšení efektivity dopravní sítě.
Zlepšení na~běžných silnicích nemusí být značné, avšak u~křižovatek by~mohl být rozdíl velmi velký.

Zaměřit se na~křižovatky je důležité i~z~pohledu nehodovosti.
Ačkoliv většina silniční komunikace probíhá na~nekřížících se úsecích, na~našem území dochází kolem $21\%$ všech nehod
právě na~křižovatkách (\href{https://www.czso.cz/documents/10180/20534694/32025414a06.pdf}
{Český Statistický Úřad - https://www.czso.cz/documents/10180/20534694/32025414a06.pdf}).

Tato práce se~zabývá problémem projetí co~největšího množství autonomních aut skrze křižovatku.
K~řešení tohoto problému používá křižovatka určitou centrální skříňku, se kterou auta komunikují.
Po~příjezdu auta ke křižovatce se auto nahlásí centrální skříňce odkud přijíždí a kam by chtělo jet.
Křižovatka poté autu naplánuje nekolizní trasu skrze křižovatku.

Pro~tuto práci jsem si vytvořil simulátor prostředí, kde~je možné napodobit průjezdy aut různými křižovatkami.
Simulovaný svět obsahuje značná zjednodušení reálného světa, avšak dle~mého názoru jsou nápady určitou formou přenositelné.

Práce obsahuje různá řešení tohoto problému pomocí simulovaného prostředí na~různých křižovatkách a~množství aut.
Tato~řešení jsou následně porovnána dle~rozličných parametrů.

	\chapter{Zavedení pojmů}\label{ch:zavedeni_pojmu}

\section{Problém křižovatky}\label{sec:problem}

%Popis problému křižovatky.

%Obecnější řešení (decentralizované plánování, časování semaforů, \ldots)

Chytré křižovatky se již objevily v mnohých městech.
Jsou to světelné křižovatky, které dokážou poznat, že všechna auta v daném směru už projela.
Při~detekci takovéto situace křižovatka nastaví červenou z~příslušného směru a zároveň pustí auta z~jiného směru dříve.
Plánování pořadí a délek jednotlivých zelených všech směrů je komplexní záležitost, pokročilejší plánování popsali například \citet*{Goldstein}.
\citet*{Liang} ve svém článku popsali trénování světelných křižovatek pomocí zpětnovazebního učení
a rozšířili algoritmus i~na~tisíce propojených křižovatek.
Na~tento způsob řešení jsem se ale ve~své práci nezaměřil, protože nabízí minimální zlepšení
na~jedné křižovatce v~hustých provozech a minimálně využívá autonomie vozidel.

Další způsob je decentralizované plánování, které spočívá v~komunikaci mezi auty.
Touto technikou se zabývali například \citet*{Wu}.
V~jejich článku porovnávají kruhový objezd se světelnou křižovatkou.
Jelikož není přítomna centrální řídící jednotka, pro~převedení řešení
do~reálného světa stačí přidat určitý protokol do~vozidel v~provozu.
Avšak aplikovat komplexnější plánování je při~větším počtu aut obtížné.

Já se zaměřím pouze na postupy s~centrální jednotkou.
Při tomto postupu auta nekomunikují navzájem mezi sebou, ale na křižovatce existuje skříňka rozhodující trasy aut.


\section{Typy plánování}\label{sec:typy_planovani}

%Stručný popis rozdílu mezi ind. a hrom. plánováním.

Rozlišuji dva případy centrálního plánování.
\textit{\nameref{subsec:individualni_planovani}} plánuje trasu pro~každé auto zvlášť.
Každé následné plánování potom hledá trasu pro~nové auto takovým způsobem, aby nedošlo ke~kolizi s~už naplánovanými auty.

S~tímto přístupem by~se mohlo stát, že naplánovaná trasa auta přerušuje kratší trasy následných aut.
Mohlo by být výhodnější nenaplánovat nejkratší trasu předchozímu autu a celkově dojít lepšímu řešení.
Tento problém se snaží vyřešit \nameref{subsec:hromadne_planovani}.

\textit{\nameref{subsec:hromadne_planovani}} shlukuje auta do~skupin a hledá trasy pro~auta společně.
Díky tomuto přístupu je možné najít trasy, ve~kterých stráví auta v~průměru méně času na~křižovatce.

\subsection{Individuální plánování}\label{subsec:individualni_planovani}

%Přesnější definice.
%
%Popsání práce \citet{Dresner}.
%
%Stručný popis BFS, A*.

Jak bylo popsáno výše, auta jsou naplánována jedno po~druhém a
každý nový plán je rozvržen tak, aby nekolidoval s~žádným už~naplánovaným autem.

Při~plánování se používá \emph{First Come First Served} \labeltext{FCFS}{str:fcfs} strategie pro~určení pořadí plánování.
\ref{str:fcfs} strategie přiřazuje nejvyšší prioritu v~plánování autu, které dorazilo ke~křižovatce nejdříve.
Tímto způsobem jsou minimalizovány čekací doby jednotlivých aut
a maximalizována \uv{spokojenost} aut díky férovému přístupu (Spokojenost aut se váže k živím pasažérům v~autě).

Model chytré křižovatky s~touto strategií už nasimulovali \citet*{Dresner}.
Jejich přístup odpovídal reálnému provozu, křižovatka neměla centrální jednotku.
Auta byla schopná vzájemné komunikace.
Auta v~jejich práci jsou schopna jezdit pouze v~předem daných pruzích.
V~těchto pruzích následně mohou auta zrychlovat či zpomalovat.
Algoritmus nejprve přiřadí autu maximální rychlost a zjistí, zda by na~jeho cestě došlo ke~kolizi.
Pokud ano, zkusí nižší rychlost před~místem kolize.
Postup se opakuje, dokud nedojde k~nalezení nekolizní cesty skrze křižovatku.
Agenti jsou plánováni pomocí \ref{str:fcfs} strategie.
Tímto algoritmem jsem se inspiroval u~algoritmu \nameref{sec:safe_lanes}.

U~řešení \citet{Dresner} se mi nelíbilo omezení aut na~pruhy, rád bych umožnil autům využívat celou plochu křižovatky.
Proto bych využil známý \nameref{sec:a_star} algoritmus na~hledání nejkratší trasy v~obecném prostoru.
Algoritmus byl vytvořen \citet*{Hart1968} pro~robotické účely.
Pro~použití \nameref{sec:a_star} je avšak nejdříve nutné diskretizovat křižovatku a čas, a upravit model aut.
Tato úprava je popsána v~samostatných sekcích (\ref{sec:krizovatka}, \ref{sec:agent}).
\nameref{sec:a_star} dovoluje autům plně využít celou plochu křižovatky a algoritmus nám zaručuje optimální cestu pro~každé auto při~zafixování ostatních aut.
\nameref{sec:a_star} je založen na~algoritmu \emph{Breath First Search} s~chytřejším procházením stavů.
Podrobný popis algoritmu je v~samostatné kapitole (\ref{sec:a_star}).

\subsection{Hromadné plánování}\label{subsec:hromadne_planovani}

%Popis plánovače, výhody a nevýhody oproti individuálnímu plánování
%(optimalita řešení, porovnání velikosti prohledávaných prostorů).

Výše popsaná situace, kdy auto blokuje výhodnější trasy ostatních, ukazuje nevýhodu postupného plánování.
Může být výhodné rozdělit si auta do~časových intervalů, kdy přijedou na~křižovatku.
Následně lze použít algoritmus pro~naplánování co nejlepší trasy pro~všechna auta v~rámci jednoho intervalu najednou.
Pokud auta informují křižovatku o~svém příjezdu s~předstihem, můžeme i~přeplánovat trasu podle nově nahlášených aut.

%
%Řešení tohoto typu jsou složitější, avšak teoreticky by měla být schopna tvořit celkově lepší plány.
%
%Chytrá křižovatka se dá převézt na online \emph{MAPF} (Multi-Agent Path Finding) problém.
%

\subsubsection{Offline~MAPF}\label{subsubsec:offline_mapf}

%Definice MAPF a s ním spojených pojmů (Sum of costs, \ldots).
%
%Analogie a rozdíly oproti problému práce.

Pro hromadné hledání cest v diskrétním prostoru existuje hromada řešení.
Problém, které hromadné plánování řeší je označován \emph{Multi-Agent Path Finding} \labeltext{MAPF}{str:mapf}
a podrobnou definici a varianty popsali například \citet*{osti_10114869}.
\ref{str:mapf} má na~vstupu dvojici $G, A$, kde $G=(V, E)$ je graf a $A = \{a_1, \dots, a_k\}$ je množina agentů.
Opět stejně jako u~\nameref{sec:a_star} je možné diskretizovat křižovatku
a převést na~graf (popsáno v sekci \ref{sec:krizovatka}).
Každý agent $a_i$ má svojí výchozí pozici $s_i \in V$ a cílovou pozici $g_i \in V$.
Popis převodu auta na agenta je popsán v sekci \nameref{sec:agent}.
Stejně jako u \nameref{sec:a_star} pracuje \ref{str:mapf} s diskrétními časovými úseky (kroky).
Během jednoho kroku může agent přejet do sousedního vrcholu, nebo počkat v aktuálním.
Plán pro agenta $a_i$ je sled $\pi_i = s_i, v_2, \dots, v_{n-1}, g_i$ na grafu $G$, čili $v_2, \dots, v_{n-1} \in V$ a
$(s_i, v_2) \in E, (v_{n-1}, g_i) \in E, \forall_{i \in 2, \dots, n-2} (v_i, v_{i+1}) \in E$.
Délka plánu je $|\pi_i| = n$, pozici agenta~$a_i$ v~kroku~$c$ značím~$\pi_i[c]$.

Dle definice~\ref{str:mapf} jsou agenti $a_i$ a $a_j$, $i \neq j$ v \emph{kolizi} v kroku $c$
právě tehdy když platí jedna z následujících podmínek.
\begin{gather}
	\pi_i[c] = \pi_j[c] \label{eq:mapf_kolize_vrchol}\\
	\pi_i[c] = \pi_j[c + 1] \land \pi_i[c + 1] = \pi_j[c] \label{eq:mapf_kolize_hrana}
\end{gather}
Slovy řečeno, agenti jsou v \emph{kolizi}, pokud jsou ve stejný čas na~stejném místě, nebo projíždí stejnou hranou.
Pro problém křižovatky je nutné kontrolu kolize rozšířit, jelikož agenti mají svojí velikost (\ref{sec:kolize}).

\ref{str:mapf} se snaží o nalezení plánu $\pi = \cup_{i=1}^{k} \pi_i$, který nemá žádné kolize.
Takovýto plán je nazýván validní.
Pro problém mohou existovat různé plány, tyto plány bývají často porovnány pomocí \emph{Sum Of Costs} \labeltext{SoC}{str:soc} metriky.
Plán $\pi$ má cenu podle metriky \ref{str:soc}: $soc(\pi) = |\pi| = \sum_{i=1}^{k} |\pi_i|$.
Alternativní způsob porovnání je objektivní funkcí $\textrm{makespan}\labeltext{makespan}{str:makespan}(\pi)$ pro plán $\pi$,
která vrací počet kroků, než všichni agenti dorazí do svého cíle.
Vzorcem $\textrm{makespan}(\pi)=\max_{i\in A} |\pi_i|$.

\subsubsection{Řešení~\nameref{subsubsec:offline_mapf}}\label{subsubsec:reseni_offline_mapf}

%Stručný popis známých algoritmů pro MAPF s citacemi (CBS, A*, SAT).


Nejjednodušší způsob řešení je využít A* algoritmus, kde stav je kartézským součinem stavů všech plánovaných agentů.
Toto řešení má často vysoký větvící faktor.
Proto se vyvinuly vylepšení, například \emph{Independence Detection}, \emph{Conflict Avoidance Table} nebo
\emph{Operator Decomposition} \citep{Standley_2010} a mnoho dalších.
Bližší popis těchto vylepšení je v sekci \nameref{sec:a_star}. % TODO hromadny A*

\citet*{Sharon} navrhli algoritmus \nameref{subsec:conflict_based_search},
který nalezne nejkratší cesty pro všechny agenty nezávisle na ostatních.
Poté hledá konflikty mezi jednotlivými plány.
Pokud algoritmus nalezne konflikt, hledání se rozdělí na dva podpřípady.
První větev výpočtu najde alternativní cestu pro prvního agenta, druhá větev pro druhého.
Takto postupně vzniká binární strom.
Pokud je nalezena nekolizní cesta pro všechny vrcholy, algoritmus skončí.
Pořadí prohledávání vrcholů ve stromu je určeno podle SOC metriky.
Toto pořadí zaručuje optimální řešení \citep{Sharon}.
Algoritmus byl nadále rozšířen a vylepšen \citep{Boyarski}.
Bližší popis algoritmu je v sekci \nameref{subsec:conflict_based_search}

\emph{MAPF} problém je možné převést na známý \emph{SAT} problém.  % TODO ref to SAT definition
Nejprve se vytvoří výrokové proměnné pro každého agenta, každý vrchol a každý čas.
Agent musí splňovat určité podmínky, například agent se nachází v čase příjezdu na vrcholu vjezdu
nebo agent může být v jeden čas maximálně na jednom vrcholu.
Následně přidáme podmínky zaručující nekolizní cesty pro agenty.
Tento způsob řešení je spíše vhodný pro optimalizování \ref{str:makespan} funkce,
avšak je možné vytvořit varianty cílené na \ref{str:soc} pomocí rozšíření na \emph{MAXSAT} \citep{bartak}.
Blíže je toto řešení popsáno v sekci \nameref{sec:sat-planner}.

Existují i jiná řešení \ref{str:mapf}, například za použití zpětnovazebního učení \citep*{Zhiyao}.
Pro svou práci jsem se rozhodl věnovat se pouze základním metodám
\nameref{sec:a_star}, \nameref{subsec:conflict_based_search} a \emph{SAT}. % TODO ref to SAT

\subsubsection{Online~MAPF}\label{subsubsec:online_mapf}

%Popis rozšíření z offline na online, popis způsobů řešení.
%
%Definice optimality (optimal vs snapshot-optimal).

\nameref{subsubsec:offline_mapf} hledá řešení pouze jednou, avšak auta přijíždějí na křižovatku neustále.
Můžeme předpokládat, že většina cestujících auty nezná přesný, často ani přibližný čas svého příjezdu ke křižovatce.
Proto by křižovatka měla být schopna plánovat nově přijíždějící auta s ohledem na auta na křižovatce.
Naštěstí existuje rozšíření \emph{offline~MAPF} problému na problém \nameref{subsubsec:online_mapf} \citet*{Svancara}.
Online varianta \ref{str:mapf} splňuje všechny potřeby naší křižovatky.

\emph{Online~MAPF} má u každého agenta $a_i = (t_i, s_i, g_i)$ kromě místa příjezdu a cíle také čas příjezdu $t_i$.
Avšak Tento čas není dopředu znám.
\emph{Online~MAPF} začíná s počátečním \emph{offline~MAPF} plánem pro agenty, kteří přijeli v čase $0$.
Tento plán budu značit $\pi^0$.
Pokaždé, když se objeví noví agenti, vytvoří se nový plán $\pi^j$.
Celkový plán je tedy $\Pi = (\pi^0, \pi^1, \dots, \pi^m)$, kde $m$ je počet unikátních kroků ($t_1, t_2, \dots, t_m$), kdy se objevili agenti.
Označím si $\pi^j[x:y]$ část plánu $\pi^j$ v krocích $x, x + 1, \dots, y - 1, y$.
Celkový plán, který budou agenti vykonávat je tedy $Ex[\Pi] = \pi^0[0:t_1] \circ \pi^1[t_1 + 1:t_2] \circ \dots \circ \pi^m[t_m + 1:\infty]$.

\citet{Svancara} zmínili problémy s~\emph{online~MAPF}.
První problém nastane, pokud agenti zůstanou na svém místě po doražení do cíle.
Zároveň pokud by se agenti okamžitě objevili v grafu, mohli by ihned způsobit kolizi, kterou algoritmy nemohli predikovat.
Žádný z těchto problémů u mě nastat nemůže, jelikož auta přijíždějí na vjezdy
a zároveň mizí z křižovatky po doražení výjezdu.

Opět zavedu cenu plánu jako součet délek plánů pro jednotlivé agenty $|Ex[\Pi]| = \sum_{i=1}^{k} |Ex[\Pi]_i| = \sum_{i=1}^{k} t_{Ex[\Pi]}[g_i] - t_i$,
kde $t_{Ex[\Pi]}[g_i]$ je krok, kdy agent $a_i = (t_i, s_i, g_i)$ naposledy dorazil do cílového vrcholu $g_i$.
Z analýzy \citet{Svancara} víme, že cena $|Ex[\Pi]|$ je ekvivalentní objektivní funkci $\sum_{t=1}^{\infty} \textrm{NotAtGoal}(t)$,
kde $\textrm{NotAtGoal}(t)$ udává počet agentů, kteří ještě nedorazili do svého cíle v čase $t$.
Také objektivní funkce $\sum_{i=1}^{k} |Ex[\Pi]_i| - o_i$, kde $o_i$ je délka nejkratší cesty mezi $s_i$ a $g_i$,
je ekvivalentní $|Ex[\Pi]|$.

Každý \emph{online~MAPF} problém je možné převést na \emph{offline~MAPF} pokud dáme dopředu algoritmu vědět, kdy se agenti objeví.
Díky tomu můžeme porovnat optimalitu online řešičů.
\citet{Svancara} dokázali, že žádný online algoritmus nemůže zajistit offline optimální řešení.
\emph{Snapshot-optimální}\labeltext{snapshot-optimální}{str:snapshot_opt} plány jsou optimální plány za předpokladu, že se žádní noví agenti neobjeví.
\citet*{Morag} provedli rozsáhlé experimenty a zjistili, že \ref{str:snapshot_opt} plány nejsou o moc horší než optimální.
Ve všech typech experimentů \citet{Morag} byly \ref{str:snapshot_opt} ceny plánů alespoň v $80\%$ běhů totožné s optimálním plánem
a ve zbylých případech se plány lišily minimálně.


%Rozšíření \emph{offline~MAPF} problému na online variantu zkoumali ve své práci \citet*{Svancara}.
%\emph{Online~MAPF} má u každého agenta $a_i = (t_i, s_i, g_i)$ kromě místa příjezdu a cíle také čas příjezdu $t_i$.
%Tento čas není dopředu znám.
%\emph{Online~MAPF} začíná s počátečním \emph{offline~MAPF} plánem pro agenty, kteří přijeli v čase $0$.
%Tento plán budu značit $\pi^0$.
%Pokaždé, když se objeví noví agenti, vytvoří se nový plán $\pi^j$.
%Celkový plán je tedy $\Pi = (\pi^0, \pi^1, \dots, \pi^m)$, kde $m$ je počet unikátních kroků ($t_1, t_2, \dots, t_m$), kdy se objevili agenti.
%Označím si $\pi^j[x:y]$ část plánu $\pi^j$ v krocích $x, x + 1, \dots, y - 1, y$.
%Celkový plán, který budou agenti vykonávat je tedy $Ex[\Pi] = \pi^0[0:t_1] \circ \pi^1[t_1 + 1:t_2] \circ \dots \circ \pi^m[t_m + 1:\infty]$.
%
%\citet{Svancara} zmínili problémy s~\emph{online~MAPF}.
%První problém nastane, pokud agenti zůstanou na svém místě po doražení do cíle.
%Zároveň pokud by se agenti okamžitě objevili v grafu, mohli by ihned způsobit kolizi, kterou algoritmy nemohli predikovat.
%Žádný z těchto problémů u mě nastat nemůže, jelikož agenti mohou být zamítnuti, pokud by došlo ke kolizi hned na vjezdu.
%Agenti taky mizí z křižovatky po doražení do výjezdu.
%
%Opět zavedu cenu plánu jako součet délek plánů pro jednotlivé agenty $|Ex[\Pi]| = \sum_{i=1}^{k} |Ex[\Pi]_i| = \sum_{i=1}^{k} t_{Ex[\Pi]}[g_i] - t_i$,
%kde $t_{Ex[\Pi]}[g_i]$ je krok, kdy agent $a_i = (t_i, s_i, g_i)$ naposledy dorazil do cílového vrcholu $g_i$.
%Z analýzy \citet{Svancara} víme, že cena $|Ex[\Pi]|$ je ekvivalentní objektivní funkci $\sum_{t=1}^{\infty} \textrm{NotAtGoal}(t)$,
%kde $\textrm{NotAtGoal}(t)$ udává počet agentů, kteří ještě nedorazili do svého cíle v čase $t$.
%Také objektivní funkce $\sum_{i=1}^{k} |Ex[\Pi]_i| - o_i$, kde $o_i$ je délka nejkratší cesty mezi $s_i$ a $g_i$,
%je ekvivalentní $|Ex[\Pi]|$.
%
%Každý \emph{online~MAPF} problém je možné převést na \emph{offline~MAPF} pokud dáme dopředu algoritmu vědět, kdy se agenti objeví.
%Díky tomu můžeme porovnat optimalitu online řešičů.
%\citet{Svancara} dokázali, že žádný online algoritmus nemůže zajistit offline optimální řešení.
%\emph{Snapshot-optimální} plány jsou optimální plány za předpokladu, že se žádní noví agenti neobjeví.
%\citet*{Morag} provedli rozsáhlé experimenty a zjistili, že \emph{snapshot-optimální} plány nejsou o moc horší než optimální.
%Ve všech typech experimentů byly \emph{snapshot-optimální} ceny plánů alespoň v $80\%$ běhů totožné s optimálním plánem
%a ve zbylých případech se plány lišily minimálně.

\subsubsection{Řešení~\nameref{subsubsec:online_mapf}}\label{subsubsec:reseni_online_mapf}


%Popis úpravy offline algoritmů pro řešení online MAPF\@.


V práci \citet{Svancara} jsou návrhy různých strategií pro řešení \emph{online~MAPF} problémů:
\begin{itemize}
	\item \textbf{Replan~Single}\labeltext{RS}{str:rs} - tento přístup je totožný s~přístupem \nameref{subsec:individualni_planovani},
	každý nový agent je naplánován s ohledem na předchozí.
	\item \textbf{Replan~Single~Grouped}\label{par:replan_single_grouped}\labeltext{RSG}{str:rsg} -
	v tomto přístupu se plánují pouze noví agenti, ale oproti \ref{str:rs} plánování probíhá pro všechny agenty najednou.
	Zde lze použít \emph{offline~MAPF} řešič, který se musí vyhnout kolizím s již naplánovanými trasami.
	\item \textbf{Replan~All}\labeltext{RA}{str:ra} - za použití této strategie se použije
	\emph{offline~MAPF} řešič na všechny agenty pokaždé, když dorazí noví agenti.
	Pokud je řešič optimální, \ref{str:ra} vrací \ref{str:snapshot_opt} řešení pro všechny agenty \citep{Svancara}.
	\item \textbf{Online~Independence~Detection}\labeltext{OID}{str:oid} - tento přístup se snaží minimalizovat množství přeplánovaných agentů.
	Nejprve najde cestu pro všechny nové agenty ignorujíce už naplánované.
	Poté zjistí kolize mezi starými a novými agenty.
	Pokud byly nalezeny kolize, přeplánují se trasy kolizních agentů.
	Pro zaručení \ref{str:snapshot_opt} plánu je nutné udělat dodatečné úpravy \citep{Svancara}.
	\item \textbf{Suboptimal~Independence~Detection}\labeltext{SubID}{str:subid} - pozměňuje \ref{str:oid} dovolováním neoptimálních cest.
	Přesněji cena plánu SubID je nejvýše $D$ krát delší než cena \emph{snapshot~optimálního} plánu.
	Avšak díky této úpravě by měl být počet přeplánování, a tedy i čas výpočtu, nižší.
\end{itemize}


%V práci \citet{Svancara} jsou návrhy různých postupů řešení \emph{online~MAPF} problémů:
%\begin{itemize}
%  \item \textbf{Replan~Single} (RS) - tento přístup je totožný s~přístupem \nameref{sec:individualni_planovani}.
%  \item \textbf{Replan~Single~Grouped}\label{par:replan-single-grouped} (RSG) - v tomto přístupu se plánují pouze noví agenti.
%  Plánování probíhá pro všechny agenty najednou.
%  Zde lze použít \emph{offline~MAPF} řešič, který se musí vyhnout kolizím s již naplánovanými trasami.
%  \item \textbf{Replan~All} (RA) - za použití této strategie se použije \emph{offline~MAPF} řešič na všechny agenty pokaždé, když dorazí noví agenti.
%  Pokud je řešič optimální, \emph{Replan~all} vrací \emph{snapshot~optimální} řešení \citep{Svancara}.
%  \item \textbf{Online~Independence~Detection} (OID)- Tento přístup se snaží minimalizovat množství přeplánovaných agentů.
%  Nejprve najde cestu pro všechny nové agenty ignorujíce už naplánované.
%  Poté zjistí kolize mezi starými a novými agenty.
%  Pokud byly nalezeny kolize, přeplánují se trasy kolizních agentů.
%  Pro zaručení \emph{snapshot~optimálního} plánu je nutné udělat dodatečné úpravy \citep{Svancara}.
%  \item \textbf{Suboptimal~Independence~Detection} (SubID) - pozměňuje OID dovolováním neoptimálních cest.
%  Přesněji cena plánu SubID je nejvýše $D$ krát delší než cena \emph{snapshot~optimálního} plánu.
%  Avšak díky této úpravě by měl být počet přeplánování, a tedy i čas výpočtu, nižší.
%\end{itemize}
%
%
%\section{Inteligentní křižovatka}\label{sec:inteligentni-krizovatka}
%Problém popsaný v této práci, přidává do \emph{online~MAPF} další podmínky.
%U křižovatky všichni agenti mají specifikovány vrcholy, kde může být jejich start a cíl.
%Agenti také nejsou pouhé body, ale mají svojí velikost.
%Díky tomu je zjišťování kolizí komplexnější.
%
%Dále je možné přiblížit se reálné křižovatce dalšími úpravami.
%\begin{itemize}
%  \item Pokud agentovi nezáleží na pruhu, kterým vyjede, může sdělit algoritmu pouze směr výjezdu.
%  Algoritmus má za cíl najít cestu na libovolný výjezd v daném směru.
%  V důsledku může agent místo jednoho koncového vrcholu mít množinu vrcholů.
%  \item Agent představuje jedoucí vozidlo.
%  Proto mohu po algoritmu vyžadovat, aby se agent nikdy nezastavil na místě.
%  Zároveň mohu vyžadovat, aby se agentovo cesta neměnila po vjezdu do křižovatky.
%  Podmínku, že křižovatka nemůže měnit individuální plány za běhu, již zmínil \citet{Dresner}.
%  Kdybychom změnu plánů dovolili, mohlo by dojít k chybě v komunikaci, díky níž by agent prováděl původní plán,
%  avšak křižovatka by počítala s novým plánem.
%\end{itemize}
%
%Všechny tyto varianty zkoumám a porovnávám v experimentech.



\section{Abstraktní model křižovatky}\label{sec:krizovatka}

%Shrnutí hledání na grafu, definice grafu.
%
%Popis různých vzhledů křižovatky.
%Převod křižovatky na graf a parametry převodu.


Všechny mnou používané algoritmy jsou založeny na~prohledávání orientovaného grafu.
Z~tohoto důvodu je nutné převést křižovatku na~graf.
Převod je proveden rozdělením křižovatky na~diskrétní nepřekrývající~se bloky zaplňující celou plochu křižovatky.
Mezi sousedními bloky mohou auta přejíždět.

Formálně je křižovatka převedena na~orientovaný graf,
kde~každý blok je reprezentován jedním vrcholem a hrany vedou mezi bloky se společnou stěnou.
%Vzdálenost mezi sousedními bloky (čas přejezdu) je u~všech bloků stejný a trvá právě jeden krok simulace.  TODO

Bloky jsou rozděleny do~třech skupin.
První skupina jsou vrcholy reprezentující vnitřní plochu křižovatky.
Zbylé dvě skupiny reprezentují \hyperref[par:vjezdy]{vjezdy do~křižovatky} a \hyperref[par:vyjezdy]{výjezdy z~křižovatky}.

Rozhodl jsem se pro~$3$~různé typy křižovatek určené rozdělením plochy na~bloky.
Tyto~typy budu nazývat \hyperref[subsec:ctvercovy_typ]{čtvercový}, \hyperref[subsec:oktagonalni_typ]{oktagonální}
a~\hyperref[subsec:hexagonalni_typ]{hexagonální}.

Rozdělení křižovatky má určité parametry ovlivňující výsledný graf.
Těmito parametry jsou \hyperref[par:granularita]{granularita},
\hyperref[par:vjezdy]{počet vjezdů} a~\hyperref[par:vyjezdy]{počet výjezdů}.

\paragraph{Granularita}\label{par:granularita} značí na~jak velké množství bloků je křižovatka rozdělena.
Přesný význam granularity se liší u~každého typu a je popsán zvlášť v~jednotlivých kapitolách.
Na~\hyperref[par:granularita]{granularitě} závisí tzv.~\hyperref[par:velikost_bloku]{velikost bloku}.

\paragraph{Velikost bloku}\label{par:velikost_bloku} značí velikost základních bloků rozdělení
(rovna vzdálenosti mezi vrcholy), opět se ale u~různých typů liší.
Tuto~veličinu zavádím kvůli jednoduchému porovnání velikosti auta vůči bloku.

Každý typ křižovatky má předurčený počet \emph{směrů} ze~kterých auta přijíždějí.

\paragraph{Vjezdy}\label{par:vjezdy} značí množství vrcholů reprezentujících pruhy vjezdů z~každého \emph{směru} křižovatky.
Tyto vrcholy mají pouze jednu výstupní hranu a žádná hrana do nich nevede.

\paragraph{Výjezdy}\label{par:vyjezdy} mají podobný význam, určují počet výjezdních pruhů z~každého \emph{směru}.
Vrcholy výjezdů mají oproti \hyperref[par:vjezdy]{vjezdům} jednu vstupní hranu a žádnou výstupní.

Jelikož se na~silnicích auta pohybují po~pravé straně silnice,
jsou vjezdy a výjezdy z~jednoho směru seřazeny vůči hraně křižovatky zleva doprava v~pořadí
\ref{str:mezera_l}, \hyperref[par:vyjezdy]{výjezdy}, \ref{str:mezera_s}, \hyperref[par:vyjezdy]{vjezdy}, \ref{str:mezera_p}.
Kde~\emph{mezeraL}\labeltext{mezeraL}{str:mezera_l}, \emph{mezeraS}\labeltext{mezeraS}{str:mezera_s}
a~\emph{mezeraP}\labeltext{mezeraP}{str:mezera_p} značí posloupnost vrcholů na~hraně křižovatky
nesousedících s~vjezdem ani~výjezdem.
Všechny \hyperref[par:vjezdy]{vjezdy} jsou vždy přímo vedle sebe a \hyperref[par:vyjezdy]{výjezdy} taktéž.
Většina křižovatek je symetrická proto jsem určil, že velikost části \ref{str:mezera_l} bude shodná s~velikostí části \ref{str:mezera_p}.
Opět kvůli symetričnosti jsem zvolil velikost části \ref{str:mezera_p} co~nejblíže velikostem
krajových mezer \ref{str:mezera_l} a \ref{str:mezera_p}.
Platí tedy $\ref{str:mezera_l}=\ref{str:mezera_p}\wedge(\ref{str:mezera_s}=\ref{str:mezera_l}\vee\ref{str:mezera_s}+1=\ref{str:mezera_l})$.

\hyperref[par:pruh]{Pruhy} jsou známý koncept u křižovatek a v dopravě obecně.
Rozhodl jsem se proto formálně definovat \hyperref[par:pruh]{pruh}.
\paragraph{Pruh}\label{par:pruh} mezi určitým vjezdem a výjezdem definuji jako nejkratší cestu mezi těmito vrcholy.
Délka cesty je určena počtem vrcholů, přes které vede.
Pokud vedou dvě různé cesty přes stejný počet vrcholů, je upřednostněna ta, na~které musí auto méně zatáčet.
Čili cesty jsou následně porovnány podle úhlu, o~který se musí auta během jízdy otočit.
Pokud je i~tato hodnota stejná, porovnávají se auta podle počtu vrcholů, kterými neprojíždí rovně.
Při~posledním porovnání je preferována cesta s~více vrcholy.
Takto jsem se rozhodl, protože díky tomu budou mít auta méně ostré zatáčky.

\subsection{Čtvercový typ}\label{subsec:ctvercovy_typ}

\nameref{subsec:ctvercovy_typ} křižovatky rozděluje plochu do~čtvercových bloků.
Celková hlavní plocha křižovatky (tedy veškerá plocha mimo
\hyperref[par:vjezdy]{vjezdy} a \hyperref[par:vyjezdy]{výjezdy}) tvoří čtverec.

\nameref{par:granularita} značí počet bloků na~jedné straně plochy křižovatky.
Celkový počet bloků hlavní plochy křižovatky tudíž činí $g^2$, kde $g$ je \hyperref[par:granularita]{granularita} křižovatky.
Pozice vrcholu je přesně uprostřed jemu odpovídajícího bloku.
Mezi každými bloky, který spolu sdílí stěnu, vede ve~výsledným grafu hrana.
Tento typ křižovatky má čtyři směry, jeden směr na~každé straně čtverce hlavní plochy.
Křižovatka je převedena na~graf s~$g^2 + 4i + 4o$ vrcholy,
kde $g$ značí \hyperref[par:granularita]{granularitu}, $i$~\hyperref[par:vjezdy]{vjezdy} a $o$~\hyperref[par:vyjezdy]{výjezdy}.

Ukázka křižovatky s~\hyperref[par:granularita]{granularitou}~$4$, jedním \hyperref[par:vjezdy]{vjezdem} a
jedním \hyperref[par:vyjezdy]{výjezdem} je vidět na~obrázku~\ref{fig:square_type_graph}.
Na~obrázku jsou šedou barvou označeny vrcholy značící běžnou plochu křižovatky,
červeno-šedou barvou označeny vrcholy reprezentující \hyperref[par:vjezdy]{vjezdy} a
modro-šedé vrcholy reprezentují \hyperref[par:vyjezdy]{výjezdy}.

\begin{figure}[h]
	\centering
	\includegraphics[width=\textwidth]{../img/Square_grid}
	\caption{Ukázka čtvercového typu křižovatky.}
	\label{fig:square_type_graph}
\end{figure}

\subsection{Oktagonální typ}\label{subsec:oktagonalni_typ}

\nameref{subsec:oktagonalni_typ} křižovatky vychází z~typu \hyperref[subsec:ctvercovy_typ]{čtvercového},
avšak rozšiřuje tento typ o~možnost diagonální jízdy.
Tohoto efektu docílí křižovatka přidáním dodatečných vrcholů mezi každé čtvercové bloky dotýkající se rohem.
Z~vizuálního hlediska se ze~čtvercových bloků stanou oktagonální (osmiúhelníkové),
odtud plyne samostatný název tohoto typu.
Tyto bloky mohou mít až~$8$~sousedů.
Mezi těmito oktagonálními bloky vzniknou nové bloky reprezentující diagonální přejezdy.
Nově vytvořené bloky mají nanejvýše $4$~sousedy, a tvoří pomyslný čtverec.
Z~vizuálních důvodů byly odebrány rohové bloky.

\hyperref[par:granularita]{Granularita}, \hyperref[par:vjezdy]{vjezdy} a \hyperref[par:vyjezdy]{výjezdy}
mají stejný význam jako u~\hyperref[subsec:ctvercovy_typ]{čtvercového typu}.
Počet vrcholů u~této křižovatky činí $g^2 - 4 + (g-1)^2 + 4i + 4o = 2g^2 - 2g - 3 + 4(i + o)$,
kde $g$ je \hyperref[par:granularita]{granularita},
$i$ počet \hyperref[par:vjezdy]{vjezdů} a $o$ počet \hyperref[par:vyjezdy]{výjezdů}.
\nameref{par:velikost_bloku} je vzdálenost mezi dvěma sousedními vrcholy reprezentujícími oktagonální blok,
opět totožná s~velikostí bloku u~čtvercového typu.

Na~obrázku (Obrázek~\ref{fig:octagonal_type_graph}) je znázorněn příklad
\hyperref[subsec:oktagonali_typ]{oktagonálního typu} křižovatky s~\hyperref[par:granularita]{granularitou}~$4$,
jedním \hyperref[par:vjezdy]{vjezdem} a jedním \hyperref[par:vyjezdy]{výjezdem}.

Barvy vrcholů a bloků jsou stejné jako u~\hyperref[subsec:ctvercovy_typ]{čtvercového typu},
šedou barvou jsou označeny vrcholy značící běžnou plochu křižovatky,
červeno-šedou barvou označeny vrcholy reprezentující \hyperref[par:vjezdy]{vjezdy} a
modro-šedé vrcholy reprezentují \hyperref[par:vyjezdy]{výjezdy}.

\begin{figure}[h]
	\centering
	\includegraphics[width=\textwidth]{../img/Octagonal_grid}
	\caption{Ukázka oktagonálního typu křižovatky.}
	\label{fig:octagonal_type_graph}
\end{figure}

\subsection{Hexagonální typ}\label{subsec:hexagonalni_typ}

\nameref{subsec:hexagonalni_typ} je tvořen bloky s~tvary hexagonu (šestiúhelníku).
Při~převodu křižovatky na~graf je každý hexagonální blok nahrazen jedním vrcholem ležícím uprostřed původního bloku.
Opět hrany grafu vedou mezi bloky se společnou stěnou.
Každý blok má tedy až~$6$~sousedů.
Tyto bloky tvoří dohromady plochu tvaru velkého hexagonu.
Díky této reprezentaci má křižovatka tohoto typu $6$~směrů odkud mohou auta přijíždět.

Toto rozdělení sebou nese jednu velkou nevýhodu.
Pokud chce auto jet rovně skrze křižovatku (do~protilehlého směru), křižovatka mu nemůže nabídnout rovnou cestu.

\hyperref[par:granularita]{Granularita} u~tohoto typu značí počet bloků na~jedné straně celkové plochy.
Hodnota je taktéž rovna počtu vrcholů z~kraje křižovatky do~středu.
Počet vrcholů grafu tedy činí $6 \times g \times (g-1) + 6i + 6o$,
kde $g$ je \hyperref[par:granularita]{granularita},
$i$ počet \hyperref[par:vjezdy]{vjezdů} a $o$ počet \hyperref[par:vyjezdy]{výjezdů}.
\nameref{par:velikost_bloku} je opět vzdálenost mezi dvěma sousedními vrcholy.

Na~obrázku (Obrázek~\ref{fig:hexagonal_type_graph}) je znázorněn příklad
\hyperref[subsec:hexagonalni_typ]{hexagonálního typu} křižovatky s~\hyperref[par:granularita]{granularitou}~$4$,
jedním \hyperref[par:vjezdy]{vjezdem} a jedním \hyperref[par:vyjezdy]{výjezdem}.

Barvy vrcholů a bloků jsou opět stejné.
Šedou barvou jsou označeny vrcholy značící běžnou plochu křižovatky,
červeno-šedou barvou označeny vrcholy reprezentující \hyperref[par:vjezdy]{vjezdy} a
modro-šedé vrcholy reprezentují \hyperref[par:vyjezdy]{výjezdy}.

\begin{figure}[h]
	\centering
	\includegraphics[width=\textwidth]{../img/Hexagonal_grid}
	\caption{Ukázka hexagonálního typu křižovatky.}
	\label{fig:hexagonal_type_graph}
\end{figure}


\section{Agent}\label{sec:agent}

%Popis zjednodušení auta na agenta.
%Popis parametrů agenta.


\nameref{sec:agent} je zjednodušení chytrého auta pro snadnější použití v~algoritmech.
Tvar \hyperref[sec:agent]{agenta} je zjednodušený na~obdélník, který má určitou délku, šířku a naklonění v~rovině.
Tento obdélník reprezentuje pohled na~auto shora.
Šířka a délka \hyperref[sec:agent]{agenta} je určená vůči \hyperref[par:velikost_bloku]{velikosti bloku} dané křižovatky.

Pokud se nějaké dva obdélníky protnou, nastane srážka příslušných dvou \hyperref[sec:agent]{agentů}.
\hyperref[sec:agent]{Agenti} nemají žádný způsob jak informovat křižovatku o~srážce.
Z~toho důvodu po~kolizi sražení \hyperref[sec:agent]{agenti} zmizí a žádní jiní \hyperref[sec:agent]{agenti} s~nimi nemohou kolidovat.

Jelikož se \hyperref[sec:agent]{agent} pohybuje po~hranách grafu křižovatky, je jeho cesta složená z~úseček.
Proto jsem umožnil \hyperref[sec:agent]{agentovy} otáčet se na~místě a otáčení probíhá okamžitě po~příjezdu do~vrcholu.
\hyperref[sec:agent]{Agent} tedy pořád cestuje a otáčí se vždy ve~směru jízdy.
Tato vlastnost by neměla mít na~pohyb auta velký vliv, pokud je auto dostatečně malé
oproti vzdálenostem vrcholů grafu křižovatky (\hyperref[par:velikost_bloku]{velikosti bloku}).

\hyperref[sec:agent]{Agent} při~příjezdu na~křižovatku nahlásí křižovatce odkud jede a kam směřuje.
\hyperref[sec:krizovatka]{Křižovatka} se poté pokusí najít \hyperref[par:cesta]{cestu} splňující \hyperref[sec:agent]{agentovy požadavky}.

\paragraph{Cesta}\label{par:cesta} je tvořena posloupností vrcholů, přes které má \hyperref[sec:agent]{agent} jet.
Pokud křižovatka nenajde žádnou cestu, zamítne vjezd \hyperref[sec:agent]{agenta}.
V~případě, kdy~\hyperref[sec:agent]{agent} čeká na~vjezd příliš dlouho,
\uv{dojde \hyperref[sec:agent]{agentovi} trpělivost a vydá se jinou cestou}.
Znamená to, že \hyperref[sec:agent]{agent} nahlásí křižovatce, že už~nemá zájem o~průjezd.


\section{Simulace}\label{sec:simulace}

%Popis běhu simulátoru - vygenerování agentů a předání řešiči.
%
%Definice sledovaných parametrů - zdržení, zamítnutí, kolize.

Pro tuto práci jsem si vytvořil vlastní simulátor křižovatky.
V simulátoru lze nastavit parametry křižovatky a agenta.
Po startu simulace se načtou agenti, naleznou se pro ně trasy a odsimuluje se průjezd agentů.
Čas běhu simulace je rozdělen na diskrétní kroky.
V~každém kroku se pokusí simulace naplánovat agenty, kteří v daném kroku požádali o vjezd, nebo byl jejich vjezd zamítnut.
Zároveň se všichni agenti na křižovatce přesunou o jednu hranu dále do následujícího vrcholu daného svou trajektorií.
Pro~naplánování jsou vybráni agenti z~každého vjezdu, kteří přijeli nejdříve.
Ostatním agentům je zamítnut vjezd v daném kroku.
Nemůže se tedy stát, že by agent předjel jiného agenta čekajícího před~ním.

Simulace nabízí různé statistiky pomocí nichž je možné vygenerované trasy porovnat.
Tyto statistiky jsou popsány níže.

\paragraph{Zdržení} \label{par:zdrzeni} jednoho agenta se~spočte jako součet
rozdílu délky cesty od~optimální cesty a doba čekání před vjezdem do~křižovatky.
Jinými slovy \hyperref[item:zdrzeni]{zdržení} značí počet kroků, o~které agent vyjel z~křižovatky později oproti situaci,
kdyby přijel na~prázdnou křižovatku (křižovatku bez~jiných agentů) a ihned by~projel nejkratší možnou cestou.
Celkové \hyperref[item:zdrzeni]{zdržení} všech agentů je součet \hyperref[item:zdrzeni]{zdržení} přes~všechny agenty,
co se~pokusili křižovatkou projet.

\paragraph{Počet zamítnutých agentů}\label{par:zamitnuti} udává množství agentů, kteří čekali na~křižovatce příliš dlouho a
vzdali~se čekání ve~frontě.
I~když křižovatkou neprojeli, pořád se kroky jejich čekání přičítají do~celkového \hyperref[item:zdrzeni]{zdržení}.

\paragraph{Počet kolizí}\label{par:kolize} udává množství sražených agentů.

\paragraph{Doba běhu}\label{par:doba_behu} algoritmu počítá kolik nanosekund běžel algoritmus
v~součtu přes všechny kroky na~reálném hardware.

\subsection{Generování agentů}\label{subsec:generovani_agentu}

%Popis generování nových agentů - způsob vybírání množství agentů a
%hodnot pro agenta (vjezd, výjezd, rychlost, velikost, \ldots).

Pokud uživatel nespouští simulaci s~předpřipravenými agenty ze souboru,
simulátor nabízí možnost generování agentů za~běhu.
Před~startem simulace je možné navolit~si určité vlastnosti agentů a po~kolik kroků se mají agenti generovat.
Uživatel si může určit počet vygenerovaných agentů v~každém kroku simulace nastavením parametrů $na_{\min}$ a~$na_{\max}$.
Simulace vygeneruje nový počet agentů uniformně náhodně mezi $na_{\min}$ a~$na_{\max}$.  % TODO

Dále je možné určit preference směrů odkud agenti budou přijíždět a kam budou směřovat.
Pro~každý směr je možné určit pravděpodobnost, s~jakou se zde agent objeví, či kam bude směřovat.
Tyto pravděpodobnosti se~musí sečíst na~$1$.

Poté je možné měnit samotné parametry agentů.
Hlavními parametry agenta jsou šířka a délka a \hyperref[par:odchylka]{odchylka rychlosti}.

\paragraph{Odchylka}\label{par:odchylka} určuje o~kolik procent je rychlost agenta odlišná
vůči oznámené rychlosti křižovatce, a~nabývá hodnot mezi nulou až~sto procenty.
Kromě rychlosti ovlivňuje \hyperref[par:odchylka]{odchylka} i~příjezd agenta.
Příjezd je možné zpozdit až~o~tolik procent kroku, kolik je \hyperref[par:odchylka]{odchylka}.
Tímto způsobem se snažím simulovat reálnější křižovatky, kde došlo k~určité chybě v~komunikaci či měření agenta.
Pomocí \hyperref[par:odchylka]{odchylky} se snažím sledovat odolnost plánování vůči těmto jevům.


	\chapter{Algoritmy}\label{ch:algoritmy}

Popis společné části algoritmů (vstup, výstup).

Stručný text o parametrech algoritmů.

%Algoritmy plánují cesty pro~všechny agenty přijíždějící v~daném kroku.
%Každému úspěšně naplánovanému agentovi se přidělí nalezená cesta.
%Po~skončení algoritmus vrátí množinu naplánovaných agentů.
%Plánování je prováděno pomocí funkce \emph{plan\_agents}.
%Níže je zobrazen pseudokód výchozího chování funkce \emph{plan\_agents}.
%Toto volání využívá pomocné funkce popsané později.
%% @formatter:off
%\begin{code}[xrightmargin=6em]
%// vstup plánovaný krok step, množina agentů agents
%// výstup množina naplánovaných agentů planned_agents
%plan_agents(step, agents)
%    agents_mapped <- empty_set
%    for agent in agents
%      agents_mapped.add([agent, agent.entry, agent.exits])
%    return plan_agents(agents_mapped, step)
%\end{code}
%% @formatter:on
%
%Algoritmy implementují další pomocné funkce pro~zjednodušení a propojení algoritmů.
%První pomocná funkce se~nazývá \emph{plan\_agent}.
%Tato funkce dostává na~vstup krok, ve~kterém má být agent naplánován.
%Další parametry jsou agent, který má být naplánován, vrchol,
%ze~kterého agent vyjíždí a množina vrcholů, na~kterých může agent skončit.
%Tato funkce nemá žádnou výchozí implementaci.
%Funkce vrací agenta pokud byl úspěšně naplánován, jinak $NULL$.
%% @formatter:off
%\begin{code}[xrightmargin=14em]
%// vstup plánovaný krok step, agent agent,
%// vjezd entry a výjezdy exits
%// výstup agent nebo NULL
%plan_agent(step, agent, entry, exits)
%  // implementace algoritmu
%  if plánování úspěšné
%    agent.path <- planned_path
%    return agent
%  else
%    return NULL
%\end{code}
%% @formatter:on
%
%Další pomocná funkce je přetížení funkce \emph{plan\_agents}, která má na~vstupu krok plánování.
%Další vstup je množina trojic.
%První prvek trojice je agent, který má~být naplánován.
%Další prvek je vrchol ze~kterého agent vyjíždí.
%Poslední prvek je množina cílových vrcholů, což jsou možné vrcholy, na~kterých má agent skončit.
%Funkce opět vrací množinu naplánovaných agentů.
%Pseudokód výchozího chování je následovný.
%% @formatter:off
%\begin{code}[xrightmargin=6em]
%// vstup plánovaný krok step, množina trojic agentů,
%// vjezdů a výjezdů agents_entries_exits
%// výstup množina naplánovaných agentů planned_agents
%plan_agents(step, agents_entries_exits)
%  planned_agents <- empty_set
%  for agent_entry_exits in agents_entries_exits
%    agent = agent_entry_exits[0]
%    entry = agent_entry_exits[1]
%    exists = agent_entry_exits[2]
%    planned_agent <- plan_agent(step, agent, entry, exists)
%    if planned_agent is not NULL
%      planned_agents.add(planned_agent)
%  return planned_agents
%\end{code}
%% @formatter:on

\section{Kontrola kolize}\label{sec:kolize}

%Rozbor případů, kdy může nastat mezi agenty kolize (základ v MAPF).
%Rozšíření problému na agenty s nenulovou velikostí.
%Popis pomocných datových struktur.

Při~hledání cest musí algoritmus brát v~potaz již naplánované agenty.
Pro~tyto účely jsem si vytvořil následující pomocnou datovou strukturu, do~které ukládám potřebné informace.

\paragraph{Tabulka~obsazených~pozic}\label{par:obsazene_pozice} si pamatuje pro~každý krok množinu dvojic vrcholu a agenta.
Díky této struktuře můžu jednoduše a rychle zjistit,
zda~se v~daný krok vyskytuje již naplánovaný agent na~určeném vrcholu, a popřípadě o~kterého agent se~jedná.
Po~každém naplánování agenta je postupně přidána dvojice do~každého kroku, kdy~se agent vyskytuje na~křižovatce,
aby byla \nameref{par:obsazene_pozice} aktuální.

Kontrola kolize probíhá třemi fázemi, kontroluje se~\nameref{subsec:bezpecnost_vrcholu},
\nameref{subsec:cesta_do_vrcholu} a \nameref{subsec:cesta_z_vrcholu}.

\paragraph{Safe distance}\label{par:safe_distance} (značený~$d$) určuje minimální povolenou vzdálenost mezi dvěma agenty.
\nameref{par:safe_distance} je společný parametr všech kontrol a lze nastavit před~spuštěním simulace.
Nenulová hodnota \nameref{par:safe_distance} má šanci snížit kolize,
pokud jsou zavedené nepřesnosti parametrem \nameref{par:odchylka}.

Jelikož se můžou agenti v~libovolný okamžik jakkoliv natočit, pracuji ve~výpočtech se zjednodušeným modelem agentů.
Namísto počítání složitého aktuálního natočení agenta a následné převedení na~obdélník,
je agent nahrazen pomyslným kruhem.

\paragraph{Poloměr agenta}\label{par:polomer_agenta} určuje poloměr kruhu zjednodušeného modelu
a spočítá se z~agentovo délky~$l$ a šířky~$w$ jako $\frac{\sqrt {l^2 + w^2}}{2}$.
Kontroly poté zjišťují, jestli jsou kruhy tvořené pozicí agenta a jeho poloměrem disjunktní.
Jinými slovy agenti jsou kontrolami vyhodnoceni v~kolizních trasách,
pokud se během cesty středy agentů přiblíží na~vzdálenost menší nebo rovnu součtu jejich poloměrů a \emph{safe distance}.
Toto zjednodušení nemůže způsobit kolizi, jelikož je celý agent umístěn uvnitř \hyperref[par:polomer_agenta]{poloměru agenta}.
Zároveň počítání s~kruhem značně zrychluje samotný výpočet.

\subsection{Bezpečnost vrcholu}\label{subsec:bezpecnost_vrcholu}

%Popis postupu kontroly, pseudokód, náčrtek.

\hyperref[subsec:bezpecnost_vrcholu]{Kontrola bezpečnosti vrcholu} zjišťuje,
zda~je bezpečný výskyt agenta v~určitém kroku na~určeném vrcholu.
Kontrola je rozšíření první \ref{str:mapf} podmínky
\uv{žádní dva agenti se nesmí nacházet na~stejném vrcholu v~jednom kroku} \eqref{eq:mapf_kolize_vrchol}.

Tato kontrola pracuje s~vrcholem~$v$, krokem~$s$ a poloměrem agenta~$r$, pro~kterého je kontrola určená.
Dále algoritmus zná maximální povolenou velikost agenta.
Z~maximální délky a šířky je předpočítán maximální poloměr agenta~$m$.

Na~obrázku (Obrázek~\ref{fig:kolize_na_vrcholu}) je ukázka situací, ve~kterých tato kontrola selže.
Vlevo dochází ke~kolizi, jelikož jsou dva agenti na~stejném vrcholu.
Vpravo nastává situace, kdy~jsou dva vrcholy a agenti na~nich příliš blízko.
Situace je kontrolou vyhodnocena jako kolizní i~když se agenti nepřekrývají.

\begin{figure}[h]
	\centering
	\includegraphics[width=\textwidth]{../img/kolize_vrchol}
	\caption{
		Ukázka kolizních situací, které jsou detekovány \hyperref[subsec:bezpecnost_vrcholu]{kontrolou bezpečnosti vrcholu}.
		Na~obrázcích jsou černě zobrazeny vrcholy.
		Obdélníky reprezentují agenty a kruhy tvoří bezpečnou zónu příslušných agentů.
	}
	\label{fig:kolize_na_vrcholu}
\end{figure}

Kontrola pro~vrchol $v$ začne procházet všechny vrcholy grafu od~nejbližšího $v$ podle eukleidovské vzdálenosti.
Pro~každý vrchol~$u$, který je blíže než~$r + d + m$, se nejprve zjistí,
jestli je na~vrcholu~$u$ v~kroku~$s$ nějaký agent.
Pokud není, pokračuje kontrola dalším vrcholem.
Jinak se spočítá poloměr agenta~$r'$ nacházejícího se v~kroku~$s$ na~vrcholu~$u$.
Pokud je eukleidovská vzdálenost vrcholů $u$ a $v$ menší nebo rovna~$r + r' + d$, kontrola selže.
V~opačném případě přejde kontrola na~další vrchol.
Pro~vrcholy vzdálené více než~$r + d + m$ uspěje kontrola triviálně,
neexistuje možnost že by agenti na~těchto vrcholech byly v~kolizi.

Níže je popsaný algoritmus na~kontrolu bezpečnosti vrcholu.
% @formatter:off
\begin{code}
// konstanty tabulka obsazených pozic t, minimální vzdálenost agentů d,
// maximální poloměr agenta m

// vstup krok s, vrchol v, poloměr agenta r
// výstup true pokud může agent být na v, jinak false
safe_vertex(s, v, r)
for u in sorted(V, x -> dist(x, v))
if dist(u, v) > r + m + d return true
else
n <- t[s][v]
r' <- diameter(n)
if dist(u, v) <= r + r' + d return false
return true
\end{code}
% @formatter:on

\subsection{Cesta do~vrcholu}\label{subsec:cesta_do_vrcholu}

%Popis postupu kontroly, pseudokód, náčrtek.

\hyperref[subsec:cesta_do_vrcholu]{Kontrola cesty do~vrcholu} zjišťuje,
zda~může plánovaný agent bezpečně přejet do~určeného vrcholu v~určitém kroku,
aniž by došlo ke~kolizi s~nějakým agentem opouštějícím daný vrchol.
Kontrola zahrnuje druhou \ref{str:mapf} podmínku
\uv{žádní dva agenti nesmí projíždět stejnou hranou v~jednom kroku} \eqref{eq:mapf_kolize_hrana}.
Avšak jelikož mají agenti nenulovou velikost, je nutné kontrolovat i~případy, kdy agenti neprojíždí stejnou hranou.
Tyto situace jsou častější čím menší úhel je mezi sousedy vrcholu.
Například pro~čtvercový typ, kde je úhel mezi sousedy $90^\circ$, nastává tato kolize pouze pro~velké agenty.
U~oktagonálního typu křižovatky dochází ke~kolizi mnohem častěji, protože úhel mezi sousedy činí $45^\circ$.
Ukázka kolizního stavu je zobrazena na~obrázku (Obrázek~\ref{fig:kolize_cesta_do}).

\begin{figure}[h]
	\centering
	\includegraphics[width=\textwidth]{../img/kolize_cesta_do}
	\caption{
		Ukázka kolizní situace, která je detekována \hyperref[subsec:cesta_do_vrcholu]{kontrolou cesty do~vrcholu}.
		Na~obrázcích jsou černě zobrazeny vrcholy.
		Obdélníky reprezentují agenty a kruhy tvoří bezpečnou zónu příslušných agentů.
		Čárkované prázdné obdélníky značí cílové stavy agentů po~jednom kroku.
		Zelený agent je již naplánovaný, černevý je aktuálně plánován.
	}
	\label{fig:kolize_cesta_do}
\end{figure}


Tato~kontrola probíhá, pokud agent~$a$ opouští vrchol~$u$ v~kroku~$s$ a přijíždí do~vrcholu~$v$ v~následujícím kroku~$s + 1$.
Algoritmus nejprve zjistí, zda existuje naplánovaný agent~$b$, který je v~kroku~$s$ na~$v$.
Pokud žádný takový agent neexistuje, kontrola uspěje a agetova cesta z~$u$ do~$v$ v~kroku $s$ je bezpečná.
Jinak se zjistí vrchol~$w$, na~kterém se nachází agent~$b$ v~následujícím kroku $s + 1$.
Pokud nastane speciální případ $u=w$, odpovídá stav zmiňované \ref{str:mapf} podmínce \eqref{eq:mapf_kolize_hrana}.

Jelikož kontrola zná pozice vrcholů $u$, $v$ a $w$, je schopna dopočítat si čas, ve~kterém se agenti nejvíce přiblíží.
Pokud se agenti srazí, musí být v~kolizní poloze i~v~čase, kdy~jsou sobě nejblíže.
Naopak pokud jsou dostatečně daleko i~když jsou si nejblíže, při~cestě nemůže dojít ke~kolizi.
Proto stačí zkontrolovat vzdálenost v~čase, kdy~jsou agenti nejblíž.

Označím polohu agenta~$a$ jako $[x_a, y_a]$ a polohu agenta~$b$ jako $[x_b, y_b]$.
Stejným způsobem si označím pozice vrcholů $u$, $v$ a $w$ jako $[x_u, y_u]$, $[x_v, y_v]$ resp. $[x_w, y_w]$.
Dále si označím čas mezi kroky~$s$ a $s + 1$ jako~$t\in[0, 1]$.
Pro~$t = 0$ je poloha agenta~$a$ shodná s~pozicí vrcholu~$u$ a poloha agenta~$b$ shodná s~pozicí vrcholu~$v$.
Analogicky v~čase $t = 1$ se nachází agent~$a$ na~vrcholu~$v$ a agent~$b$ na~vrcholu~$w$.

Jelikož se agenti pohybují po~úsečce mezi vrcholy,
pro~$t\in[0, 1]$ se agent $a$ nachází na~$x_a = tx_u + (1 - t)x_v$, $y_a = ty_u + (1 - t)y_v$.
Poloha agenta~$b$ je analogicky $x_b = tx_v + (1 - t)x_w$ a $y_b = ty_v + (1 - t)y_w$.
Vzdálenost agentů v~závislosti na~čase~$t$ je $\sqrt{(x_a - x_b)^2 + (y_a - y_b)^2}$.
Dále upravím vzorec pod~odmocninou.
\begin{align*}
	((x_a - x_b)^2 &+ (y_a - y_b)^2) = \\
	((tx_u + (1 - t)x_v - tx_v - (1 - t)x_w)^2 &+ (ty_u + (1 - t)y_v - ty_v - (1 - t)y_w)^2) = \\
	((tx_u + x_v - tx_v - tx_v - x_w + tx_w)^2 &+ (ty_u + y_v - ty_v - ty_v - y_w + ty_w)^2) = \\
	((t(x_u - x_v - x_v + x_w) + x_v - x_w)^2 &+ (t(y_u - y_v - y_v + y_w) + y_v - y_w)^2) = \\
	(t(x_u - x_v - x_v + x_w) + x_v - x_w)^2 &+ (t(y_u - y_v - y_v + y_w) + y_v - y_w)^2 = \\
	(t(x_u - 2x_v + x_w) + x_v - x_w)^2 &+ (t(y_u - 2y_v + y_w) + y_v - y_w)^2 \\
\end{align*}

Pro~zjednodušení si označím
\begin{align*}
	x_0 &= x_u - 2x_v + x_w &\qquad
	x_1 &= x_w - x_v \\
	y_0 &= y_u - 2y_v + y_w &\qquad
	y_1 &= y_w - y_v
\end{align*}

Dosazením do~předchozího vzorce dostávám
\begin{align*}
	((x_a - x_b)^2 &+ (y_a - y_b)^2) = \\
	(t(x_u - 2x_v + x_w) + x_v - x_w)^2 &+ (t(y_u - 2y_v + y_w) + y_v - y_w)^2 = \\
	(tx_0 - x_1)^2 &+ (ty_0 - y_1)^2
\end{align*}

Pro~nalezení nejmenší vzdálenosti zjistím čas~$t$, ve~kterém se agenti nacházejí nejblíže.
K~tomu spočítám derivaci vzdálenosti a zjistím, kdy je rovna nule.
Nejprve využiji faktu, že $\min\left(\sqrt{x}\right) = \min(x) \Rightarrow \frac{d}{dx} \sqrt {x} = 0 \leftrightarrow \frac{d}{dx} x=0$.
Následně
\begin{align}
	\frac{d}{dt} \sqrt{(x_a - x_b)^2 + (y_a - y_b)^2} &= 0 \nonumber \\
	\frac{d}{dt} ((x_a - x_b)^2 + (y_a - y_b)^2) &= 0 \nonumber \\
	\frac{d}{dt} ((tx_0 - x_1)^2 + (ty_0 - y_1)^2) &= 0 \nonumber \\
	\frac{d}{dt} (tx_0 - x_1)^2 + \frac{d}{dt} (ty_0 - y_1)^2 &= 0 \nonumber \\
	2(tx_0 - x_1)x_0 + 2(ty_0 - y_1)y_0 &= 0 \nonumber \\
	tx_0^2 - x_0 x_1 + ty_0 - y_0 y_1 &= 0 \nonumber \\
	t(x_0^2 + y_0^2) &= x_0 x_1 + y_0 + y_1 \label{eq:kol_d_dt}
\end{align}

Rozeberu dva případy podle podmínky
\begin{gather}
	x_0^2 + y_0^2 = 0\label{kol:nulova_podminka}
\end{gather}

Pokud je splněna podmínka~\ref{kol:nulova_podminka}, platí $x_0^2 = 0$ a $y_0^2 = 0$.
Odtud
\begin{align*}
	x_u - 2 x_v + x_w &= 0 \\
	x_u + x_w &= 2 x_v \\
	\frac{x_u + x_w}{2} &= x_v
\end{align*}
Předchozí rovnosti jsou splněné pouze když se~$x_v$ nachází přesně uprostřed $x_u$ a $x_w$.
Analogicky $y_0^2 = 0 \Leftrightarrow \frac{y_u + y_w}{2} = y_v$, tedy $y_v$ je přesně uprostřed $y_u$ a $y_w$.
Obě tyto podmínky jsou splněny jenom když se vrchol~$v$ nachází uprostřed úsečky z~$u$ do~$w$.

V~tom případě se agenti nepovoleně přiblíží pokud vzdálenost vrcholů $v$ a $w$ je menší než
součet poloměru agentů $d_a$ a $d_b$ a dovolené vzdálenosti mezi~agenty \hyperref[par:safe_distance]{safe distance}~$d$.
Dostávám tedy nerovnost $(x_w - x_v)^2 + (y_w - y_v)^2 = x_1^2 + y_1^2 > (d_a + d_b + d)^2$.


Pokud podmínka~\ref{kol:nulova_podminka} neplatí, je možné spočítat čas~$t'$,
kdy~jsou agenti nejblíže vyjádřením z~\ref{eq:kol_d_dt}.
\begin{equation}
	\label{eq:kol_t}
	t' = \frac{x_0 x_1 + y_0 y_1}{x_0^2 + y_0^2}
\end{equation}

Po~dopočítání času spočítám vzdálenost agentů $a$ a $b$ v~čase~$t'$.
Rozdíl $x$-ové souřadnice agentů je roven
\begin{gather*}
	t' x_u + (1 - t')x_v - (t' x_v + (1 - t')x_w) =
	t' x_u + x_v - t' x_v - t' x_v - x_w + t' x_w = \\
	t'(x_u - 2x_v + x_w) + x_v - x_w =
	t' x_0 - x_1
\end{gather*}

Obdobně rozdíl $y$-ové souřadnice činí $t' y_0 - y_1$.
Pro~tento případ kontrola projde jenom pokud $(t' x_0 - x_1)^2 + (t' y_0 - y_1)^2 > (r_a + r_b + d)^2$.

Pro~výsledný algoritmus si nejdříve nadefinuji
pomocnou funkci \textrm{safe\_neighbour}\labeltext{\textrm{safe\_neighbour}}{str:safe_neighbour}.
Tato~funkce pomocí postupu výše zkontroluje, zda-li dojde ke~kolizi mezi agenty $a$ a $b$,
pokud už~známe vrcholy $u$, $v$ a $w$.
% @formatter:off
\begin{code}
// konstanty bezpečná vzdálenost d

// agent a s poloměrem da cestuje z u do v,
// agent b s poloměrem db cestuje z v do w
safe_neighbour(v, u, w, da, db)
if u is w return false

x0 <- u.x - 2*v.x + w.x
x1 <- w.x - v.x
y0 <- y.u - 2*v.y + w.y
y1 <- w.y - v.y

// vzdálenost mezi středy agentů na druhou
dist <- (da + db + d) ** 2

if x0 is 0 and y0 is 0
return x1 ** 2 + y1 ** 2 > dist

else
t' <- (x0 * x1 + y0 * y1 ) / (x0 ** 2 + y0 ** 2)
x_diff <- (t' * x0 - x1) ** 2
y_diff <- (t' * y0 - y1) ** 2
return x_diff + y_diff > dist
\end{code}
\label{alg:check_neighbour}
% @formatter:on

Výsledný algoritmus kolize vypadá následovně.
% @formatter:off
\begin{code}
// konstanty tabulka obsazených pozic t

// agent a s poloměrem da cestuje z vrcholu u do v v kroku s
safe_step_to(s, u, v, da)
b <- t[s][v]
if b is Null return true

w <- b.path[s + 1]
db <- diameter(b)
return safe_neighbour(v, u, w, da, db)
\end{code}
% @formatter:on

\subsection{Cesta z~vrcholu}\label{subsec:cesta_z_vrcholu}

%Popis postupu kontroly, pseudokód.


Poslední kontrola ověřuje opačný případ předchozí kontroly.
Zjišťuje se, zda plánovaný agent~$a$ může bezpečně odjet z~vrcholu~$v$
aniž by~se srazil s~jiným již naplánovaným agentem~$b$ cestujícím do~$v$.
Ukázka kolizního stavu je zobrazena na~obrázku (Obrázek~\ref{fig:kolize_cesta_z}).

\begin{figure}[h]
	\centering
	\includegraphics[width=\textwidth]{../img/kolize_cesta_z}
	\caption{
		Ukázka kolizní situace, která je detekována \hyperref[subsec:cesta_z_vrcholu]{kontrolou cesty z~vrcholu}.
		Na~obrázcích jsou černě zobrazeny vrcholy.
		Obdélníky reprezentují agenty a kruhy tvoří bezpečnou zónu příslušných agentů.
		Čárkované prázdné obdélníky značí cílové stavy agentů po~jednom kroku.
		Zelený agent je již naplánovaný, černevý je aktuálně plánován.
	}
	\label{fig:kolize_cesta_z}
\end{figure}

Formálně agent~$a$ cestuje z~$v$ do~$w$ v~kroku~$s$ a agent~$b$ cestuje z~$u$ do~$v$ opět v~kroku~$s$.
Pokud se na~situaci podívám z~pohledu druhého agenta (prohodím agenta $a$ za $b$), dostanu předchozí případ.
Z~tohoto důvodu můžu pro~kontrolu opět použít funkci \ref{str:safe_neighbour}, akorát prohodím parametry.
Výsledný algoritmus je následovný.
% @formatter:off
\begin{code}
// konstanty tabulka obsazených pozic t

// agent a s poloměrem da cestuje z vrcholu v do w v kroku s
safe_step_from(s, v, w, da)
b <- t[s + 1][v]
if b is Null return true

u <- b.path[s]
db <- diameter(b)
return safe_neighbour(v, u, w, db, da)
\end{code}
% @formatter:on


\section{Safe lanes}\label{sec:safe_lanes}

%Převedení řešení \citet{Dresner} na graf.

%Parametry a pseudokód.

Algoritmus~\nameref{sec:safe_lanes} je založen na~křižovatkách s předem definovanými pruhy pro~auta.
Tímto způsobem řešení jsem se inspiroval u~práce \citet{Dresner}.
V~jejich práci používali jednu křižovatku s~danými pruhy.
Agentům dovolovali pouze měnit rychlost.
Já použiji jejich koncept jízdy v~pruzích, avšak moji agenti rychlost měnit nemůžou.

\citet{Dresner} plánování spadá pod \nameref{subsec:individualni_planovani}.
Plánují tedy agenty postupně jednoho po~druhém.
\nameref{sec:safe_lanes} algoritmus používá stejný přístup.
Prochází všechny přijíždějící agenty v~neurčitém pořadí a zkusí každému agentovi přiřadit nekolizní cestu.

Algoritmus~\nameref{sec:safe_lanes} se podívá na~pruh, popřípadě pruhy, podle vjezdu a výjezdu či~výjezdů daných agentem.
Pořadí procházení pruhů je dáno jeho délkou.
Délky jednotlivých pruhů si může algoritmus předem spočítat.
Pro~každý vrchol na~cestě dané pruhem provede algoritmus kontrolu popsanou v~předchozí kapitole~\ref{sec:kolize}.
Agentovi je přiřazena první nalezená nekolizní cesta.
Pokud taková cesta neexistuje, vjezd agenta je zamítnut.

Následující kód ukazuje plánování jednoho agenta.
% @formatter:off
\begin{code}
// konstanty tabulka obsazených pozic t, množina pruhů p

plan_agent(step, agent)
  r <- agent.diameter
  for exit in sorted(agent.exits, x -> dist(entry, x))
    path <- p[agent.entry, exit]
    last <- path[0]
    for i in 1, ..., path.length - 1
      s <- step + i - 1
      vertex = path[i]
      safe_transfer <- safe_transfer_set(s, last, vertex, r, t)
      if not safe_transfer
        continue
    agent.path <- path
    add_planned_agent(t, agent, s)  // přidám agenta do t
    return agent
  return NULL
\end{code}
% @formatter:on


\section{A*}\label{sec:a_star}

%Přesný popis A* algoritmu.

\nameref{sec:a_star} je známý prohledávací algoritmus.
Algoritmus potřebuje znát prohledávací prostor určený možnými stavy.
Dále je nutné uvést určitou heuristiku, která pro určitý vrchol vrátí
spodní odhad na cenu zbylé cesty z aktuálního stavu do cílového stavu.
Algoritmus si zároveň pro každý navštívený stav pamatuje cenu cesty z počátečního stavu do aktuálního.
\nameref{sec:a_star} postupně prochází navštívené stavy a pro každý stav přidá sousední stavy.
Pořadí procházených stavů je určené součtem ceny cesty do aktuálního stavu z počátečního a heuristiky v aktuálním stavu.
Tento součet je spodní odhad na minimální cenu cesty z počátku do cíle vedoucí přes navštívené stavy,
přes které byl aktuální vrchol dosažen.
\nameref{sec:a_star} zaručuje při tomto postupu optimalitu nalezené cesty.

V následujících kapitolách popíšu dvě různé implementace \nameref{sec:a_star} algoritmu pro řešení problému křižovatky.

\subsection{Individuální A* (A*RS)}\label{subsec:individualni_a_star}

%Popis úpravy A* algoritmu pro řešený problém, parametry a pseudokód.

\labeltext{A*RS}{str:individualni_a_star} patří do kategorie \ref{str:rs} algoritmů.
Plánuje totiž stejně jako \nameref{sec:safe_lanes} jednoho agenta po druhém.
Akorát algoritmus dovoluje agentům \uv{opustit} svoje pruhy.

Cena cesty se počítá podobně jako při hledání \hyperref[par:pruh]{pruhu} v křižovatce.
Cena mí více kritérií, a to \hyperref[par:ars_vzdalenost]{vzdálenost},
\hyperref[par:ars_uhel_zataceni]{úhel zatáčení} a \hyperref[par:ars_pocet_zataceni]{počet zatáčení}.

\paragraph{Vzdálenost}\label{par:ars_vzdalenost} je počet hran grafu, přes které cesta vede.
Duplicitní hrany se započítávají vícekrát.

\paragraph{Úhel zatáčení}\label{par:ars_uhel_zataceni} určuje úhel, o který se musí agent za cesty otočit.
Pro každý prostřední vrchol na cestě se dopočítá úhel mezi hranami,
přes kterou se agent na vrchol dostal a kterou odjel.
\nameref{par:ars_uhel_zataceni} je součet absolutních hodnot těchto úhlů.

\paragraph{Počet zatáčení}\label{par:ars_pocet_zataceni} udává počet
nenulových \hyperref[par:ars_uhel_zataceni]{úhlů zatáčení}.

Cesty jsou nejprve porovnávány podle \hyperref[par:ars_vzdalenost]{vzdálenosti},
poté \hyperref[par:ars_uhel_zataceni]{úhlu zatáčení} a nakonec podle \hyperref[par:ars_pocet_zataceni]{počtu zatáček}.

\paragraph{Heuristika}\label{par:ars_heuristika} v tomto případě je minimální délka cesty
z aktuálního vrcholu do nejbližšího z cílových vrcholů.
Pokud žádná taková cesta neexistuje, je hodnota \hyperref[par:ars_heuristika]{heuristiky} $\infty$.

Prohledávací prostor jsou rozšířené vrcholy křižovatky.
Pro jednodušší výpočet si u stavu mimo vrcholu pamatuji též krok, ve kterém by agent na daný vrchol přijel.
Dále si ukládám předchozí stav, cenu cesty z počátku a odhad ceny zbylé cesty dané heuristikou.
Datová struktura stavu vypadá následovně:
% @formatter:off
\begin{code}[frame=none]
stav {
	vrchol
	krok
	rodic       // předchozí stav
	vzdalenost  // vzdálenost z počátečního stavu
	uhel        // úhel zatáčení na cestě z počátečního stavu
	zatacky     // počet zatáčení na cestě z počátečního stavu
	heuristika  // hodnota heuristiky v aktuálním stavu
}
\end{code}
% @formatter:on

Následující stavy daného stavu jsou všechny validní stavy dané sousedy vrcholu aktuálního stavu.
Formálně pro vrchol $u$ jsou jeho sousedi vrcholy ${v \in V | (u,v)\in E}$, 
kde $V$ je množina vrcholů a $E$ množina hran.

\subsection{Parametry}\label{subsec:parametry}
Pro reálnější pohyby agentů po křižovatce je vhodné omezit množinu sousedů vrcholu.
Avšak určení vhodného omezení je komplikované.
Proto jsem se rozhodl umožnit omezení měnit následujícími parametry.

MAXIMUM_VERTEX_VISITS_DEF = 2;

ALLOW_AGENT_STOP_DEF = false;

MAXIMUM_PATH_DELAY_DEF = Integer.MAX_VALUE;

ALLOW_AGENT_RETURN_DEF = false;


\subsection{Hromadný A*}\label{subsec:hromadny_a_star}

Rozšíření A* pro více agentů, popis vylepšení.
Parametry, pseudokód.

\subsection{Conflict-Based Search (\ref{str:cbs})}\label{subsec:conflict_based_search}\labeltext{CBS}{str:cbs}

%Popis algoritmu, úprava pro můj problém.
%Parametry, pseudokód.

\nameref{subsec:conflict_based_search} algoritmus \citep*{Sharon} rozšiřuje jakýkoliv \ref{str:rs} algoritmus
na multiagentní plánování.
V mém případě budu rozšiřovat \ref{str:a_star_ars}.

\ref{str:cbs} začíná individuálním naplánováním všech agentů nezávisle na~sobě.
Čili agenti nesmějí mít kolize s~již cestujícími agenty,
avšak mohou mí kolizní trajektorii s~jinými aktuálně plánovanými agenty.
Poté se zkontroluje, zda-li nemají nějací agenti kolizní trajektorie.
Pokud ne, plánování úspěšně končí.
Jinak se prohledávání rozdělí na dva případy.
V obou případech je přeplánován jeden agent s podmínkou, že se musí vyhnout koliznímu místu.
Poté se opakuje opětovné hledání kolizí a rozdělování na případy.
Aby nedošlo k zacyklení, je nutné při plánování agenta vyhnout se nejen aktuální kolizi, ale také všem předchozím.
Výpočet postupně vytváří strom,
kde každý vrchol obsahuje cesty agentů (mohou být navzájem kolizní) a tabulku zakázaných pozic.
Algoritmus skončí v~prvním nalezeném vrcholu neobsahujícím kolizní trasy.

Mohlo by se stát, že plánování jednoho agenta selže.
V~tom případě je agent zcela odstraněn z~vrcholu.
Následně jsou nalezeni agenti, kteří byli v~historii přeplánováni kvůli odstraněnému agentovi.
Pro~tyto agenty jsou nalezeny nové cesty, jelikož pro~ně může existovat lepší cesta.

Algoritmus postupně prochází listy stromu výpočtu.
Pořadí průchodu je určeno počtem agentů.
Pokud je počet agentů u~více listů shodný, vybere se vrchol s nejmenší vzdáleností podobně jako u \ref{str:a_star_arsg}.
Algoritmus naplánuje pouze agenty, kteří mají cesty ve~vybraném listu, vjezd zbylých agentů je zamítnut.

\ref{str:cbs} najde optimální cestu pro všechny agenty \citep{Sharon}.
Avšak velikost stromu může být obrovská.
Proto jsem se rozhodl obětovat optimalitu s zjednodušit práci algoritmu.
Ve zjednodušeným režimu algoritmus přeplánuje takovým způsobem, aby neměl žádné kolize s ostatními plánovanými agenty.

\subsubsection{Parametry}\label{subsubsec:cbs_parametry}

\nameref{subsubsec:cbs_parametry} algoritmu jsou stejné jako u \ref{str:varsg} a mají podobný význam.
Hodnoty Maximum návštěv vrcholu (\ref{par:ars_mnv}), Povolené zastavování (\ref{par:ars_pz}),
Maximální prodleva při~cestě (\ref{par:ars_mpc}) a Povolené vracení (\ref{par:ars_pv})
algoritmus používá při~plánování jednoho agenta.
Tyto \hyperref[subsubsec:ars_parametry]{parametry} ovlivňují plánování stejně jako u \ref{str:a_star_ars}.
Hodnota parametru \ref{par:arsg_zvp} opět určuje po jak dlouhé prodlevě má algoritmus přejít na zjednodušené plánování.

\subsection{CBS-OID}\label{subsec:cbsoid}

\ref{str:cbs} lze podobně jako \ref{str:varsg} rozšířit na \ref{str:oid} variantu.
K plánovaným agentů se přidají agenti z předchozích kroků.
Jako počáteční cesty těchto agentů se použijí jejich již naplánované trasy, tudíž se znova nepočítají.
Výpočet je poté shodný, až na~případy, kdy pro~ně nebyla nalezena cesta.
Pokud k~takové situaci dojde, namísto odstranění agenta se odstraní celý list ze~stromu výpočtu.

Parametry jsou rozšířené stejně jako u \nameref{subsubsec:a_star_aoid} o~Maximální počet agentů (\ref{par:aoid_mpa})
a Počet přeplánovaných kroků (\ref{par:aoid_ppk}).
Význam těchto parametrů je shodný.



\section{SAT planner}\label{sec:sat-planner}

%Definice SAT a MAXSAT, popis řešiče.
%Rozdíl mezi optimálním ohodnocením a splňujícím ohodnocením.
%
%Popis převodu problému na SAT.
%Popis parametrů a odhad na počet proměnných a počet klauzulí.
%
%Pseudokód.

SAT je známý a prozkoumaný problém, na který existují vysoce optimalizované řešiče.
Proto není úplně zcestné pokusit se problém křižovatky převést na SAT problém.

SAT\labeltext{SAT}{str:sat} je problém určení, zda-li existuje splňující ohodnocení výrokových proměnných logické formule.
Vstupní hodnotou je tedy výroková formule a výstupem ohodnocení proměnných takové, že daná formule je splněná.
Zadaná formule většinou bývá v konjunktivní normální formě (\ref{str:sat_cnf})\labeltext{CNF}{str:sat_cnf}, což je konjunkce klauzulí.
Klauzule jsou disjunkce literálů a literál je výroková proměnná, nebo její negace.
Například formule v \ref{str:sat_cnf} pro proměnné $p_1, \dots, p_{10}$ může být
\[
	\bigwedge_{i=1}^{7}(p_i \vee p_{i+1} \vee p_{i + 3}).
\]

MAXSAT je rozšíření \ref{str:sat}.
Klauzule jsou rozdělené na dvě skupiny,
\ref{str:sat_hard}\labeltext{\emph{hard}}{str:sat_hard} a \ref{str:sat_soft}\labeltext{\emph{soft}}{str:sat_soft}.
Aby bylo ohodnocení splňující, musí být splněny všechny \ref{str:sat_hard} klauzule.
Úkolem řešiče je nalézt splňující ohodnocení, které maximalizuje počet splněných \ref{str:sat_soft} klauzulí.
Tento problém je očividně těžší, jelikož nestačí najít libovolné řešení, ale to nejlepší.

Vážený MAXSAT přidává navíc možnost přiřadit \ref{str:sat_soft} klauzulím váhy.
Řešič se nesnaží maximalizovat počet splněných klauzulí, ale součet jejich vah.

Pokud chceme naplánovat agenty pomocí váženého MAXSAT, stačí převést plánování do \ref{str:sat_cnf}.
Převedení \ref{str:mapf} problému na \ref{str:sat_cnf} bylo mnohokrát popsáno \citep{bartak}.
Z tohoto postupu budu vycházet.

\subsection{Převod do \ref{str:sat_cnf}}\label{subsec:sat_prevod_do_cnf}

Vhodný začátek převodu je nadefinování výrokových proměnných.
Poté popíšu tvorbu klauzulí.

\subsubsection{Výrokové proměnné}\label{subsubsec:sat_vyrokove_promenne}

Vytvořím si výrokové proměnné pro každého agenta, pro každý krok a pro každý vrchol grafu.
Pokud se agent $a$ vyskytuje v čase $t$ na vrcholu $v$, je výroková proměnná $p_{t,a,v}$ pravdivá, jinak je nepravdivá.
Avšak abych nedostal nekonečnou \ref{str:sat_cnf}, určím si maximální dobu cesty \ref{par:sat_mpk}.
Čas v proměnné je počet kroků od plánovaného kroku a mí hodnotu ${0, \dots, mpk}$
Počet výrokových proměnných je celkem $(mpk + 1) * |A| * |V|$, kde $A$ je množina agentů a $V$ množina vrcholů.

\subsubsection{\ref{str:sat_hard} Podmínky}\label{subsubsec:sat_hard_podminky}

V této kapitole popíšu způsob, jak vytvořit odpovídající \ref{str:sat_cnf}.
Avšak nebudu popisovat jednotlivé klauzule.
Namísto toho popíšu tvorbu klauzulí jednoduššími výrazy (např.\ pro každé $p_i$, maximálně jeden z $p_i$, \ldots).
Převod těchto výrazů do validní \ref{str:sat_cnf} je triviální.
Pokud formule obsahuje některou funkci, je možné funkci vyhodnotit předem a daný výraz přidat pokud to má smysl.

Agent přijede na vrchol \hyperref[par:vjezdy]{vjezdu} v kroku příjezdu, pokud bude úspěšně naplánován.
Zároveň musí být v jednom z časů v íli, aby byla cesta kompletní.
Z toho vyplývá první podmínka pro každého agenta, která značí, že daný agent není v počáteční čas na vjezdu,
nebo je v jenom jeden čas na právě jednom výjezdu.
Matematicky:
\[
	(\forall_{a \in A}) \left(\neg p_{0,a,a_e} + \sum_{t=1}^{mpk} \sum_{f \in a_f} p_{t, a, f} = 1\right),
\]
kde $a_e$ je vrchol vjezdu agenta $a$ a $a_f$ jeho výjezdy.

Agent nemůže nacházet na více vrcholech najednou.
Jinými slovy může být pro jednoho agenta a jeden čas maximálně jedna proměnná pravdivá:
\[
	(\forall a \in A)(\forall t \in {0, \dots, mpk})\left(\sum_{v=1}^{|V|} p_{t,a,v} \leq 1\right).
\]

Pokud je agent v určitý krok na vrcholu $v$, musí být v dalším kroku na některým vrcholu z jeho sousedů $N(v)$.
Množina sousedů může obsahovat i samotný vrchol $v$,
pokud \hyperref[par:sat_povolene_zastavovani]{povolíme zastavování}.
Toto platí až na vrcholy výjezdu, které opět pro agenta $a$ označím $a_f$, a také to neplatí pro poslední krok.
Matematickým zápisem tomu odpovídá podmínka
\[
	(\forall a \in A)(\forall t \in {0, \dots, mpk - 1})
	(\forall v \in V\\a_f)(p_{t,a,v} \rightarrow \vee_{n \in N(v)} (p_{t+1,a,n})).
\]

Počítání lze zrychlit zakázáním neplatných kombinací času a vrcholu.
Pro tyto účely si označím $d(u, v)$ jako délku nejkratší cesty mezi vrcholy $u$ a $v$.
Pokud mezi nimi cesta neexistuje, je hodnota $\infty$.
Agent nemůže být v čase $t$ na vrcholu vzdáleném více než $t$, jelikož se tam nemá jak dostat.
Stejně tak nemůže být v čase $t$ na vrcholu, který má vzdálenost k nejbližšímu cíli větší než $t$,
protože potom neexistuje způsob, jak se dostat do cíle včas.
Odtud plynou podmínka
\begin{gather*}
(\forall a \in A)(\forall t \in {0, \dots, mpk})(\forall v \in V)
	\\
	((d(a_e, v) > t \vee (\min_{f \in a_f} d(v, f)) > mpk - t) \rightarrow \neg p_{t, a, v}).
\end{gather*}

Nadále je nutné vyhnout se cestujícím agentům.
K tomu opět využiji funkce na kontrolu kolizí.
Projdu všechny vrcholy a pro každého agenta zjistím,
na kterých vrcholech se nesmí nacházet pomocí funkce \ref{alg:kol_safe_vertex}:
\[
	(\forall a \in A)(\forall t \in {0, \dots, mpk})(\forall v \in V)
	(\neg \ref{alg:kol_safe_vertex}(a_p + t, v, a_d) \rightarrow \neg p_{t, a, v}),
\]
kde $a_p$ je kro příjezdu agenta a $a_d$ je jeho \hyperref[par:polomer_agenta]{poloměr}.

Kontrola bezpečné \hyperref[subsec:cesta_do_vrcholu]{cesty do vrcholu}
a poté \hyperref[subsec:cesta_z_vrcholu]{z vrcholu} probíhá podobně.
Jediný rozdíl je, že se musí kontrolovat dvojice sousedních vrcholů.
\begin{gather*}
(\forall a \in A)(\forall t \in {0, \dots, mpk - 1})(\forall v \in V)
	(\forall n \in N(v)) \\
	(\neg(\ref{alg:kol_safe_step_to}(a_p + t, v, n, da) \wedge \ref{alg:kol_safe_step_from}(a_p + t, v, n, da))
	\rightarrow \neg p_{t, a, v}).
\end{gather*}

Poslední nutná podmínka je zamezení kolizím mezi plánovanými agenty.
Podmínky vypadají podobně předchozím klauzulím.
Je nutné projít všechny dvojice agentů a poté všechny kombinace vrcholů.
Pokud se vrcholy nacházejí moc blízko, nemůžou se agenti vyskytovat na patřičných vrcholech v jeden čas.
Zápisem:
\begin{gather*}
(\forall a, b \in {A \choose 2})(\forall v \in V)(\forall u \in \ref{alg:sat_close_vertices}(v, da, db))
	\\
	(\forall t \in {0, \dots, mpk})
	(\neg ((p_{t, a, u} \wedge p_{t, b, v}) \vee (p_{t, a, v} \wedge p_{t, b, u})))),
\end{gather*}
kde $da$ a $db$ jsou poloměry agentů $a$ resp. $b$.
Ve vzorci používám funkci \ref{alg:sat_close_vertices}, která pro daný vrchol $v$ a dvojici agentů
vrací všechny vrcholy, na kterých nesmí být některý agent, jestliže je druhý agent na $v$.
\labeltext{\textrm{close\_vertices}}{alg:sat_close_vertices}
% @formatter:off
\begin{code}[fontsize=\footnotesize]
// minimální vzdálenost agentů d

// vrchol, poloměr prvního agenta, poloměr druhého agenta
// výstup množina vrcholů nebezpečně blízká vstupnímu vrcholu
close_vertices(u, v, m, n)
	vertices <- empty
	for u in V
		if dist(u, v) <= m + n + d
		vertices.add(u)
	return vertices
\end{code}
% @formatter:on

Zároveň se agenti nesmějí srazit při cestách mezi vrcholu.
K tomu opět využiji funkci \ref{alg:kol_safe_neighbour} podobně jako při kontrole vjezdu do vrcholu.
Zkontroluji pro všechny $u$, $v$ a $w$ takové, že $v \in N(u) \wedge w \in N(v)$,
že agent $a$ může přejet mezi $u$ a $v$, a agent $b$ může přejet z $v$ do $w$.
Toto vyzkouším pro všechny dvojice agentů, včetně prohozeného pořadí agentů.
Po prohození agentů totiž podmínka odpovídá \hyperref[subsec:cesta_z_vrcholu]{kontrole cesty z vrcholu}.
Pokud není přejezd možný, nesmí se z žádných po sobě jdoucích krocích tato situace stát.
Zápisem:
\begin{gather*}
(\forall a \in A)(\forall b \neq a \in A)(\forall u \in V)
	(\forall v \in N(u))(\forall w \in N(v)) \\
	(\neg \ref{alg:kol_safe_neighbour}(v, u, w, da, db) \rightarrow
	(\forall t \in {0, \dots, mpk - 1}) \\
	(\neg (p_{t, a, u} \wedge p_{t + 1, a, v} \wedge p_{t, b, v} \wedge p_{t + 1, b, w}))
	)
\end{gather*}

\subsubsection{\ref{str:sat_soft} Podmínky}\label{subsubsec:sat_soft_podminky}

Podobně jako u všech předešlých algoritmech budu optimalizovat \ref{str:soc} metriku.
U každému agenta tedy budu chtít co nejdřívější příjezd do cíle.
Proto vytvořím jednoprvkové klauzule pro každého agenta a pro každý vrchol s cenou určenou časem.
Klauzuli v čase $t$ ($p_{t, a, v}$) přidělím váhu $mpk - t + 1$.
Tím bude mít příjezd v $t = 1$ váhu $mpk$ a v čase $t = mpk$ váhu $1$.

Abych maximalizoval počet naplánovaných agentů, vytvořím ještě pro každého agenta $a$
klauzuli $p_{0, a, a_e}$ s vahou alespoň $(mpk + 2) * (|A| - 1)$, kde $a_e$ je vrchol vjezdu agenta $a$.

\subsection{Parametry}\label{subsec:sat_parametry}

Aby byl algoritmus porovnatelný s ostatními algoritmy, přidal jsem podobné parametry použité v předešlých algoritmech.

\paragraph{Maximální počet kroků (\ref{par:sat_mpk})}\labeltext{MPK}{par:sat_mpk}
udává maximální délku plánu pro všechny agenty.

\paragraph{Maximum návštěv vrcholu (\ref{par:ars_mnv})} má stejný význam jako
parametr \ref{par:ars_mnv} u \hyperref[subsubsec:ars_parametry]{parametrů \ref{str:a_star_ars}}.
Hodnota udává maximální počet výskytů jednoho vrcholu na cestě.
Vzorcem $(\forall a \in A)(\forall v \in V)(\sum_{t=0}^{mpk} p_{t, a, v} \leq 1)$.

\paragraph{Povolené zastavování (\ref{par:ars_pz})}\label{par:sat_povolene_zastavovani} je taktéž vzatý
z \hyperref[subsubsec:ars_parametry]{parametrů \ref{str:a_star_ars}}.
Pokud je tento parametr nastaven, agent může stát na~místě.
Znamená to přidání vrcholu do množiny sousedů daného vrcholu.

\paragraph{Maximalizace} určuje, zda-li má řešič hledat libovolné splňující ohodnocení,
nebo maximalizovat váhu klauzulí.
Vypnutí optimalizace značně zrychluje výpočet, avšak může vést k mnohem horším výsledkům.
Jelikož jsem dovolil zamítnout agentovi vjezd, je možné všechny \ref{str:sat_hard} podmínky splnit
nastavením všech proměnných na $false$.

\subsection{SAT-RS}\label{subsec:sat_rs}

Nejjednodušší případ plánování je pomocí \ref{str:rs} strategie.
Algoritmus plánuje agenty sekvenčně jednoho za druhým.
Tudíž nemusí kontrolovat vzájemné kolize mezi plánovanými agenty.
Počet výrokových proměnných činí $|V| * (mpk + 1)$.

Po nalezení splňujícího ohodnocení se algoritmus podívá na proměnnou $p_{0, a, a_e}$.
Pokud je $false$, vjezd agenta je zamítnut.
Jinak se projdou všechny ostatní proměnné, a vyberou se pravdivé.
Z těch se podle $t$ sestaví cesta do prvního cíle.
Mohlo by se stát, že některé proměnné jsou nastaveny na $true$ i po čase příjezdu do cíle.
Tyto proměnné algoritmus ignoruje.

\subsection{SAT-RSG}\label{subsec:sat_rsg}
Jak název napovídá, algoritmus plánuje všechny nové agenty v daném kroku.
Počet proměnných vzroste na $|A| * |V| * (mpk + 1)$.
Jelikož počet proměnných vzroste lineárně, počet všech možných ohodnocení vzroste exponenciálně.
Po nalezení splňujícího ohodnocení se pro agenty poskládají cesty stejným postupem jako u \nameref{subsec:sat_rs}.

\subsection{SAT-RA}\label{subsec:sat_ra}

Plánování může proběhnout i pro již naplánované agenty pro nalezení lepších tras strategií \ref{str:ra}.
Avšak je nutné změnit určité podmínky.
Symbol $a_e$ u dříve naplánovaných agentů nese význam vrcholu, na kterém se v plánovaném kroku nachází agent.
Zároveň agent již vjel na křižovatku.
Proto je nutné zaručit, že bude naplánován.
To lze provést odstraněním $\neg p_{0, a, a_e}$ z první \ref{str:sat_hard} podmínky.
Dostanu tedy zjednodušenou podmínku:
\[
	(\forall_{a \in A}) \left(\sum_{t=1}^{mpk} \sum_{f \in a_f} p_{t, a, f} = 1\right).
\]

Zbytek algoritmu je stejný s \nameref{str:rsg}.


	\chapter{Experimenty}\label{ch:experimenty}

%Podrobnější popis sledovaných dat.

Po křižovatce požadujeme, aby zaručovala bezpečnou cestu autům.
Nyní budu předpokládat, že agenti dávají přesnou informaci křižovatce a plně dodržují plán.
Za tohoto předpokladu nám algoritmus zaručuje nekolizní trasy.

Následujícím faktorem je zpoždění aut.
Budu sledovat \hyperref[par:zamitnuti]{počet zamítnutých agentů} a \hyperref[par:zdrzeni]{zdržení} naplánovaných agentů.
V tabulkách budu porovnávat celkový \hyperref[par:zamitnuti]{počet zamítnutých agentů} (Zam).
U \hyperref[par:zdrzeni]{zdržení} budu počítat celkový součet (Zpož)
a průměrné \hyperref[par:zdrzeni]{zdržení} na agenta (pZp) vypočítané ze všech agentů.

Další zajímavý faktor je obsazenost křižovatky,
který udává v kolika krocích ze všech byl některý agent na určitém bloku křižovatky.
Ve výsledkách budu počítat průměrný počet agentů na křižovatce (pAg) přes všechny kroky.
%Budu také sledovat průměrnou délku naplánovaných cest. TODO

Důležitá je také doba běhu algoritmu.
Budu sledovat pro každý krok, kolik uběhlo času od spuštění plánování daného kroku.
Odtud zpočtu průměrný čas na jeden krok v milisekundách (Čas).

Nejprve provedu testy pro porovnání parametrů jednotlivých agentů.
Poté srovnám nejlepší nastavení algoritmů proti sobě.
%V poslední sekci zavedu k agentům určité nepřesnosti, které můžou způsobit kolize.
%Poté budu pozorovat vliv těchto nepřesností na počet kolizí u jednotlivých algoritmů.


\section{Testovací data}\label{sec:testovaci_data}

%Popis experimentálních dat - velikost křižovatky, počet agentů a kroků, safe distance.

\subsection{Délka simulace}\label{subsec:delka_simulace}

Simulace bude přidávat agenty po $32768$ kroků.
Pokud bude plánování trvat příliš dlouhou dobu, bude simulace předčasně ukončena.
Maximální dobu běhu simulace jsem omezil na 2 hodiny.

\subsubsection{Křižovatka}

Algoritmy budu testovat na každém typu křižovatky o dvou velikostech.
\paragraph{Malá}\label{par:data_mala} křižovatka bude mít \hyperref[par:velikost_krizovatky]{velikost} rovnou~$4$.
Tato křižovatka bude mít pouze jeden \hyperref[par:vjezdy]{vjezd} a jeden \hyperref[par:vyjezdy]{výjezd}.
%\paragraph{Střední}\label{par:data_stredni} křižovatka bude mít \hyperref[par:velikost_krizovatky]{velikost}~$8$.
%\hyperref[par:vjezdy]{Počet vjezdů} a \hyperref[par:vyjezdy]{výjezdů} bude činit $3$.
\paragraph{Velká}\label{par:data_velka} křižovatka bude mít \hyperref[par:velikost_krizovatky]{velikost}~$16$.
Počet \hyperref[par:vjezdy]{vjezdů} a \hyperref[par:vyjezdy]{výjezdů} je zvýšen na $4$.

\subsubsection{Agenti}

Agenti budou nejdříve náhodně vygenerovaní zvlášť pro každou velikost.
Generování proběhne dvakrát pro každou velikost,
jednou pro hexagonální typ a jednou společně pro čtvercový a oktagonální typ.
Poté se ti samí agenti použijí k porovnání jednotlivých algoritmů, abych snížil vliv náhody na výsledky.
V každém kroku přibude mezi $0 - en$ agentů, kde $en$ je celkový počet vjezdů do křižovatky ze všech stran.
Přesný počet je náhodně vygenerován každý krok.
Popis generování agentů je blíže popsán v sekci \ref{subsec:generovani_agentu}.

Délka agentů bude $0.56$ \hyperref[par:velikost_bloku]{velikosti bloku} křižovatky
a šířka $0.35$ \hyperref[par:velikost_bloku]{velikosti bloku}.
Tyto hodnoty jsem zvolil, jelikož umožňují nekolizní pozice agentů na sousedních vrcholech u všech typů křižovatek.
Zároveň ale agenti nejsou příliš malí na to, aby jejich velikost nehrála žádnou roli.

Dále budu porovnávat případy, kdy agenti mají daný přesný výjezd, nebo pouze směr výjezdu.
Tyto případy má cenu porovnávat pouze u
%\hyperref[par:data_stredni]{střední} a
\hyperref[par:data_velka]{velké} křižovatky, jelikož obsahuje více výjezdů.



\section{Parametry algoritmů}\label{sec:parametry_algoritmu}

%Podrobnější výsledky pro každý algoritmus zvlášť, sledování vlivu parametrů algoritmů na výsledky.
%
%Vyhodnocení nejlepších parametrů.

V této kapitole budu porovnávat vliv parametrů algoritmů na běh algoritmů.
V obecnosti budu testovat algoritmy s hodně omezenými parametry a s málo omezujícím nastavením parametrů.

\begin{table}[h]
	\centering
%	\begin{adjustwidth}{-1.5cm}{}
	\begin{tabular}{c c c | r r D{.}{,}{3.2} D{.}{,}{2.2} D{.}{,}{4.2}}
		\toprule \\
		\pulrad{\B{Typ}} & \pulrad{\B{Vel}} & \pulrad{\B{Výj}} &
		\pulrad{\B{Krok}} & \pulrad{\B{Zam}} & \mc{\pulrad{\B{pAg}}} &
		\mc{\pulrad{\B{pZp}}} & \mc{\pulrad{\B{Čas}}} \\
		\midrule
		S & m & - & 32782 & \B{4056}   & \multicolumn{1}{B{.}{,}{3.2}}{11.19}  & \multicolumn{1}{B{.}{,}{2.2}}{6.89}  & 16.75   \\
		O & m & - & 32785 & 13488      & 9.34                                  & 10.36                                & \multicolumn{1}{B{.}{,}{4.2}}{11.30}   \\
		H & m & - & 32796 & 32273      & 14.15                                 & 18.18                                & 87.44                                  \\
		\hline
		S & v & e & 32850 & 111987     & 77.02                                 & 60.70                                & \multicolumn{1}{B{.}{,}{4.2}}{1244.95} \\
		S & v & n & 32849 & \B{75983}  & \multicolumn{1}{B{.}{,}{3.2}}{86.91}  & \multicolumn{1}{B{.}{,}{2.2}}{44.18} & 1449.67 \\
		\hline
		O & v & e & 32849 & 138867     & 61.62                                 & 60.30                                & \multicolumn{1}{B{.}{,}{4.2}}{937.51}  \\
		O & v & n & 32846 & \B{107830} & \multicolumn{1}{B{.}{,}{3.2}}{71.98}  & \multicolumn{1}{B{.}{,}{2.2}}{58.92} & 1914.34 \\
		\hline
		H & v & e & 32893 & 234142     & 99.14                                 & 92.83                                & \multicolumn{1}{B{.}{,}{4.2}}{2399.98} \\
		H & v & n & 32894 & \B{212628} & \multicolumn{1}{B{.}{,}{3.2}}{109.63} & \multicolumn{1}{B{.}{,}{2.2}}{92.37} & 4221.17 \\
		\bottomrule
		\multicolumn{8}{l}{\footnotesize
		\textrm{Typ} - Typ křižovatky (\textrm{S}~\nameref{subsec:ctvercovy_typ}, \textrm{O}~\nameref{subsec:oktagonalni_typ}, \textrm{H}~\nameref{subsec:hexagonalni_typ})
		} \\
		\multicolumn{8}{l}{\footnotesize
		\textrm{Vel} - velikost křižovatky (\textrm{m}~\nameref{par:data_mala}, \textrm{v}~\nameref{par:data_velka})
		} \\
		\multicolumn{8}{l}{\footnotesize
		\textrm{Výj} - Výjezdy u~velké křižovatky (\textrm{e}~jediný daný výjezd, \textrm{n}~libovolný výjezd)
		} \\
		\multicolumn{8}{l}{\footnotesize
		\textrm{Krok} - počet kroků simulace, \textrm{Zam} - počet zamítnutí
		} \\
		\multicolumn{8}{l}{\footnotesize
		\textrm{pAg} - průměrný počet agentů v jeden krok na křižovatce
		} \\
		\multicolumn{8}{l}{\footnotesize
		\textrm{pZp} - průměrné zpoždění agentů
		} \\
		\multicolumn{8}{l}{\footnotesize
		\textrm{Čas} - průměrný počet mikrosekund na plánování jednoho kroku
		}
	\end{tabular}
	\caption{Porovnání vlivu parametrů u \nameref{sec:safe_lanes} na různých typech křižovatkek.}\label{tab:safe_lanes_exp}
%	\end{adjustwidth}
\end{table}

\subsection{\nameref{sec:a_star} porovnání parametrů}\label{subsec:a_star_porovnani_parametru}

V této kapitole porovnám vliv parametrů u algoritmů \ref{str:a_star_ars}, \ref{str:varsg} a \nameref{subsubsec:a_star_aoid}.

Všechny společné parametry těchto algoritmů (\hyperref[par:ars_mnv]{maximum návštěv vrcholu},
\hyperref[par:ars_pz]{povolené zastavování},\hyperref[par:ars_mpc]{maximální prodleva při cestě}
i~\hyperref[par:ars_pv]{povolené vracení}) omezují prohledávací prostor při plánování.
Proto bych čekal s větším omezením kratší dobu plánování, avšak za cenu horších výsledků.

Pro všechny typy křižovatky vyzkouším omezit \hyperref[par:ars_mnv]{maximum návštěv vrcholu (\ref{str:ars_mnv})}
na $1$ či $2$.
Pokud bude hodnota \ref{str:ars_mnv} nastavena na $1$, omezím
\hyperref[par:ars_mpc]{maximální prodlevu cesty (\ref{str:ars_mpc})} podle typu a velikosti křižovatky.
Pro čtvercovou a oktagonální na hodnotu $8$, a pro hexagonální většinou na $16$.
Zároveň nedovolím agentům \hyperref[par:ars_pz]{zastavování} (\ref{str:ars_pz})
ani \hyperref[par:ars_pv]{vracení} (\ref{str:ars_pv}).

Při nastavení \ref{str:ars_mnv} na $2$, \hyperref[par:ars_mpc]{maximální prodlevu cesty}
neomezím většinou vůbec.
Vyzkouším možnosti, kdy dovolím agentům pouze \hyperref[par:ars_pz]{zastavování},
nebo \hyperref[par:ars_pz]{zastavování} a zároveň \hyperref[par:ars_pv]{vracení} (\ref{str:ars_pv}).
V tabulkách s výsledky bude toto nastavení zobrazeno ve sloupci \textrm{Omez}.
Pokud bude povolené zastavování, bude sloupec obsahovat hodnotu~$s$.
Pokud dovolím agentům vracení, bude ve~sloupci napsáno~$r$.

\subsubsection{\ref{str:a_star_ars} na \hyperref[par:data_mala]{malé} křižovatce}
\label{subsubsec:exp_ars_mala_krizovatka}

Pokud bude nastaven \ref{str:ars_mnv} na $1$, omezím \ref{str:ars_mpc}
pro čtvercový a oktagonální typ na hodnotu $8$, a pro hexagonální na $16$.
Pro běhy s \ref{str:ars_mnv} $2$ bude \hyperref[par:ars_mpc]{prodleva cesty} neomezená, značená hodnotou $inf$.
Jelikož má křižovatka $16$ vrcholů kromě vjezdů a výjezdů, není rozdíl mezi neomezenými cestami a
cestami omezenými na $34$ kroků pro výpočet s \ref{str:ars_mnv} nastavené na $2$.

V tabulce (Tabulka \ref{tab:ars_exp_mala}) jsou vidět výsledky na všech typech křižovatky
velikosti 4, jedním vjezdem a výjezdem.


Z výsledků je znatelné, že více omezený prohledávací prostor vede ke značně horším výsledkům,
avšak průměrný čas plánování jednoho kroku je nejnižší.

Dovolení či zakázání vracení má různý vliv na výsledky u různých typů křižovatek.
U čtvercového typu byly rozdíly mezi variantami minimální.
U oktagonálního si vedla lépe varianta s povoleným vracením, avšak u hexagonálního typu si vedla hůře.

Překvapilo mě, že oktagonální typ měl mnohem více zamítnutí než čtvercový typ.
Dle mého názoru je tento jev způsoben menším manipulativním prostorem pro auta.
Oktagonální typ oproti čtvercovému nemá čtyři rohové vrcholy.
Zároveň pokud stojí auto na čtvercovém vrcholu reprezentujícím diagonální přejezd mezi dvěma oktagonálními vrcholy,
blokuje jiné přejezdy na těchto sousedních vrcholech. % TODO obrázek
Toto je dle mého názoru i důvod, proč si varianta s vracením vedla nejlépe na této křižovatce.
Umožňuje agentovi větší pohyblivost, tudíž i více možností vyhnout se jiným agentům.

Povolené vracení vedlo k nejnižšímu zpoždění ve všech třech typech křižovatky,
ačkoliv počet zamítnutí nebyl vždy nejmenší.
Mohlo by to být způsobeno tím, že zpoždění je určeno pouze nezamítnutými agenty.
Jelikož tato varianta mimo oktagonální typ křižovatky naplánovala méně agentů,
dochází zde k menšímu ovlivňování cesty jinými agenty.
Zároveň ale tato varianta rozšiřuje množinu možných cest a tyto nové cesty jsou asi kratší,
než nalezené cesty bez vracení.

Dalším zajímavým rozdílem mezi čtvercovou a oktagonální křižovatkou je rozdíl v průměrném počtu agentů na křižovatce.
Při nejvíce omezených cestách přechod z čtvercové na oktagonální křižovatku tento průměr snížil.
Ve zbylých dvou případech průměr vzrostl, ačkoliv celkový počet cestujících agentů klesl.
Z~toho lze usoudit, že na čtvercové křižovatce byly naplánované trasy značně kratší.

\input{experimenty/ars_small_table}

\subsubsection{\ref{str:a_star_ars} na \hyperref[par:data_velka]{velké} křižovatce bez výjezdů}
\label{subsubsec:exp_ars_velka_krizovatka_bez_vyjezdu}

Pro tyto testy jsem přirozeně zvýšil omezení \ref{str:ars_mpc} pro čtvercový a oktagonální typ na hodnotu $32$,
a pro hexagonální na $64$, pokud je nastavený \ref{str:ars_mnv} na $1$.
Pro běhy s \ref{str:ars_mnv} $2$ jsou délky cest opět neomezené.

V tabulce (Tabulka \ref{tab:ars_exp_velka_bez_vyjezdu}) jsou vidět výsledky na všech typech křižovatky
velikosti $16$ se $4$ vjezdy a $4$ výjezdy.

Z výsledků je dobře vidět souvislost mezi nejmenším počtem zamítnutých agentů, nejvyšším počtem agentů na křižovatce a
nejmenším průměrným zpožděním agenta.

Avšak zbytek výsledků je dosti překvapivý.
Na čtvercovém a hexagonálním typu křižovatky byl nejlepší nejméně omezený algoritmus,
zatímco na oktagonální křižovatce vyšla značně lépe varianta s povoleným zastavováním a $2$ návštěvami vrcholu.
Přidání diagonálních přejezdů tentokrát výrazně pomohlo u všech tří běhů.

Nejpřekvapivější pro mě byly časy plánování.
Nejrychlejší u čtvercového typu byl nejvíce omezený algoritmus,
avšak u zbylých dvou typů běžel nejrychleji nejméně omezený běh.
Běh s povoleným zastavováním byl u všech tří typů nejpomalejší a pro hexagonální křižovatky ani nestihl doběhnout.
Dle mého názoru je to způsobené vysokým počtem plánování agentů, která nejsou úspěšná.
Ale aby tento fakt algoritmus zjistil musí vyzkoušet vysoký počet možností.
Pokud ale povolím vracení agenta, dle mého názoru mnoha těmto agentům umožním jet relativně krátkou trasou.

\input{experimenty/ars_big_table_no_exits}

\subsubsection{\ref{str:a_star_ars} na \hyperref[par:data_velka]{velké} křižovatce s výjezdem}
\label{subsubsec:exp_ars_velka_krizovatka_s_vyjezdem}

Zde jsem použil totožné nastavení parametrů jako u běhů bez specifikovaných výjezdů.

V tabulce \ref{tab:ars_exp_velka_s_vyjezdy} jsou zobrazeny výsledky.

Algoritmus se zde chová většinově podle očekávání.
Nejméně omezené varianty dávají nejlepší výsledky na všech typech křižovatky.
Pokud všechny varianty doběhly, největší omezení vede k největšímu počtu zamítnutých agentů a
největšímu průměrnému zpoždění.

Jediné překvapení je ve sloupci s dobou plánování, avšak pořadí je stejné jako při nespecifikovaných výjezdech.
Na všech typech křižovatky plánovala prostřední varianta v průměru nejpomaleji.
Na čtvercové byla nejrychlejší první, nejvíce omezená varianta.
Na zbylých typech křižovatky nejméně omezené běhy.
Důvody jsou podle mě stejné jako u situace bez výjezdů.


\input{experimenty/ars_big_table_exits}


\subsubsection{\ref{str:a_star_arsg} na \hyperref[par:data_mala]{malé} křižovatce}
\label{subsubsec:exp_arsg_mala_krizovatka}

Jelikož se jedná o rozšíření \ref{str:a_star_ars}, použil jsem stejná nastavení parametrů.
Tedy pro čtvercový a oktagonální typ jsem nastavil \ref{par:ars_mpc} na hodnotu $8$, a pro hexagonální na $16$.
Při bězích s \ref{par:ars_mnv} nastaveno na $2$ bude \hyperref[par:ars_mpc]{prodleva cesty} opět neomezená.

Algoritmus má jediný parametr navíc, a to \nameref{par:arsg_zvp}, který udává,
po jak dlouhé prodlevě má výpočet přejít na zjednodušený režim.
U všech experimentů je tento parametr nastaven na jednu sekundu.

V tabulce (Tabulka \ref{tab:arsg_exp_mala}) jsou vidět výsledky na všech typech křižovatky
velikostí $4$ a jedním vjezdem a výjezdem.

Překvapivě si v této simulaci na všech křižovatkách vedla nejlépe nejvíce omezená varianta.
Měla výrazně menší počet zamítnutých agentů a průměrné zpoždění.
Avšak měla nejmenší počet agentů na křižovatce.
To napovídá možnosti, kdy méně omezené varianty dokáží naplánovat složitější a delší cesty,
zatímco omezenější běhy by tyto agenty ve stejném kroku zamítly a naplánovaly jim optimálnější trasy následující krok.

Běhy s povoleným zastavováním si vždy vedly hůře než běhy s povoleným vracením.
Avšak měly rychlejší plánování srovnatelné s nejvíce omezenou variantou.

\input{experimenty/arsg_small_table}

\subsubsection{\ref{str:a_star_arsg} na \hyperref[par:data_velka]{velké} křižovatce bez výjezdů}
\label{subsubsec:exp_arsg_velka_krizovatka_bez_vyjezdu}

\ref{par:ars_mpc} je opět nastaveno pro čtvercový a oktagonální typ na hodnotu $32$,
a pro hexagonální na $64$, pokud je nastavený \ref{par:ars_mnv} na $1$.
Por běhy s \ref{par:ars_mnv} $2$ jsou délky cest opět neomezené.

V tabulce (Tabulka \ref{tab:arsg_exp_velka_bez_vyjezdu}) jsou vidět výsledky na všech typech křižovatky
velikostí $16$ se $4$ vjezdy a $4$ výjezdy.

V této tabulce mě většina hodnot překvapila, výsledky mezi jednotlivými typy křižovatek jsou navzájem nesrovnatelné.
Na čtvercovém typu křižovatky měl nejmenší počet zamítnutých agentů běh s nejméně omezenými agenty.
Zároveň měl nejvíce agentů na křižovatce.
Nejméně omezená varianta měla mírně víc zamítnutých agentů a mírně menší počet agentů na křižovatce.
Avšak dosáhla výrazně nižšího průměrného zpoždění.
Zároveň zvládla plánovat agenty nejrychleji.
Prostředního běh měl nejhorší výsledky až na dobu plánování.

U oktagonálního typu jednoznačně vyhrála první varianta,
dosáhla nejmenšího počtu zamítnutí a průměrného zpoždění.
Ačkoliv se plánovací časy oproti čtvercové křižovatce zpětinásobily, pořád byly prot tento typ nejmenší.
Nejméně omezený běh měl výrazně horší počet zamítnutých agentů, průměrné zpoždění a plánovací čas.
Avšak má vyšší počet agentů na křižovatce.
To by nasvědčovalo mé hypotéze, že se snížením omezení pohybu dokáže algoritmus na plánovat dříve nenaplánované agenty,
avšak za cenu delších cest a blokování budoucích agentů.
Dalším překvapením je výsledek prostředního běhu.
Ačkoliv si oproti čtvercovému typu pomohl, celkově vyšel nejhůře, včetně plánovací doby.
Je možné, že algoritmus často přechází na zjednodušený režim, ve kterém plánuje méně agentů, než by jinak zvládl.
Dále je možné, že opět dochází ke vzájemnému překážení agentů, jako u předchozí varianty.
Je totiž vidět, že i tento běh má vyšší zaplněnost křižovatky než běh první.

Na hexagonální křižovatce žádná varianta nedoběhla do konce, takže data tím mohou být zkreslená.
Výsledky jsou zde opačné oproti předchozímu případu.
Prostřední varianta zvládla vypočítat nejvíce kroků a nejspíše díky tomu dosáhnout nejmenšího počtu zamítnutí.
Zaplněnost křižovatky byla přibližně stejná, avšak průměrné zpoždění měl nejméně omezený běh.
To může být způsobeno výrazně menším počtem kroků, a tedy i menším počtem naplánovaných agentů.
Plány agentů ze začátku simulace jsou mnohem kratší,
jelikož na křižovatce nejsou žádní agenti, kteří by blokovali tyto trasy.
Intuitivně měl tento běh nejvyšší čas plánování, více než o třetinu vyšší než zbylé dva testy.

Nejsem si jistý, proč ta dříve nejrychlejší varianta byla tak pomalá.
Dle mého názoru existuje optimální omezení agentů, které se liší pro každý typ a velikost křižovatky.
Pokud agenta omezím více, rychleji zjistím, že pro něj neexistuje cesta.
Avšak zároveň můžou nastat situace, ve kterých agentovi chyběl jediný krok do cíle
a agent by byl úspěšně naplánován.

\input{experimenty/arsg_big_table_no_exits}

\subsubsection{\ref{str:a_star_arsg} na \hyperref[par:data_velka]{velké} křižovatce s výjezdem}
\label{subsubsec:exp_arsg_velka_krizovatka_s_vyjezdem}

Testoval jsem opět totožné nastavení parametrů, výsledky jsou v tabulce \ref{tab:arsg_exp_velka_s_vyjezdy}.

Na čtvercové křižovatce se algoritmus choval podle očekávání, nejvíce omezený běh měl nejvíce zamítnutých agentů,
ale nejkratší plánovací čas.
Tento běh měl nejnižší počet zamítnutých agentů,
avšak to je podle mého způsobeno menším počtem celkově naplánovaných agentů,
což vede k menšímu počtu jedoucích agentů, kterým se musí plánovaní agenti vyhnout.
Nejméně omezená varianta měla opačné hodnoty, nejméně zamítnutých agentů za cenu vyšší časové náročnosti.

Na oktagonální i hexagonální křižovatce si vedla nejhůře nejvíce omezená varianta,
jelikož měla nejvyšší časový běh, což vedlo k nejméně naplánovaným agentům.
Naopak prostřední varianta má nejnižší čas plánování, což jí učinilo favoritem na hexagonálním typu.
Myslím si, že tato varianta dokáže úspěšně naplánovat vyšší počet agentů i v situacích,
kdy by je první varianta odmítla.
V takových případů musí první varianta zkusit pro agenta všechny možné trasy,
zatímco druhé variantě může stačit mnohem menší podprostor.
Vypadá to, že zde opravdu tato situace nastává.
Avšak rozšíření prohledávaného prostoru už negativně ovlivnilo časové nároky.

Nejméně omezená varianta si udržela nejmenší počet zamítnutí na oktagonálním typu křižovatky,
avšak nejspíše díky dlouhému času běhu měla vyšší počet zamítnutých agentů
než druhá varianta na hexagonálním typu křižovatky.


\input{experimenty/arsg_big_table_exits}


\subsubsection{\nameref{subsubsec:a_star_aoid} na \hyperref[par:data_mala]{malé} křižovatce}
\label{subsubsec:exp_aoid_mala_krizovatka}

Tento algoritmus rozšiřuje \ref{str:varsg}, proto jsem se snažil použít stejné nastavení parametrů.
Bohužel to nebylo vždy možné, jelikož přidaní agenti značně zvýšili časovou náročnost.

Algoritmus má pár parametrů navíc oproti \ref{str:varsg}.
Těmi jsou \ref{str:aoid_mpa} udávající maximální počet plánovaných agentů
a \ref{str:aoid_ppk}, který značí, kolik kroků může být agent na cestě, aby byl zvážen pro přeplánování.
\ref{str:aoid_ppk} bude vždy nastaveno na $8$ kroků,
\ref{str:aoid_mpa} je proměnlivé, vybraná hodnota je vidět v tabulce.

Algoritmus opět přejde na zjednodušený výpočet po jedné sekundě.

V tabulce (Tabulka \ref{tab:aoid_exp_mala}) jsou vidět výsledky na všech typech křižovatky
velikosti 4, jedním vjezdem a výjezdem.

Na čtvercové křižovatce všechny varianty doběhly,
avšak rozdíl v době plánování je mezi nimi mnohem větší než v předchozích případech.
Delší časy mohly způsobit častější přechod na zjednodušený režim, což může vést k horším výsledkům.
Zároveň je vidět zvyšování zamítnutí a průměrného zpoždění se zvyšujícím se zaplněním křižovatky.
Opět by to nasvědčovalo možnosti, kdy vyšší omezení naplánuje obecně lepší trasu o krok později.

Stejné nastavení parametrů nebylo úspěšné na oktagonální křižovatce, ani jedno nastavení nedoběhlo do konce.
Nárůst složitosti může být způsoben vyšším počtem vrcholů grafu křižovatky
nebo faktem, že přidané vrcholy jsou mnohem blíže sebe
a agenti si tak navzájem omezují pohyb více než na čtvercovém typu.

Na hexagonálním typu křižovatky jsem zkusil stejné nastavení parametrů pro nejvíce omezenou variantu.
Algoritmus spočítal mnohem méně kroků než na oktagonálním typu, což by nasvědčovalo hypotéze,
že za přidanou složitost může vyšší počet vrcholů křižovatky.
U zbylých dvou experimentů jsem na této křižovatce snížil maximální počet plánovaných agentů.
Je vidět, že snížení tohoto počtu z $16$ na $14$ zaručí, že algoritmus bez problémů doběhne do konce.
Zároveň počet zamítnutých agentů a průměrná prodleva se příliš nemění pro počet plánovaných agentů $12$ a $14$.
Mohlo by to být způsobeno jiným nastavením nejvyšší prodlevy cesty,
avšak dle mého názoru povolení přeplánování více agentům nemá na výsledky tak velký vliv.

\input{experimenty/aoid_small_table}

\subsubsection{\nameref{subsubsec:a_star_aoid} na \hyperref[par:data_velka]{velké} křižovatce bez výjezdů}
\label{subsubsec:exp_aoid_velka_krizovatka_bez_vyjezdu}

Tabulky \ref{tab:aoid_exp_velka_bez_vyjezdu} a \ref{tab:aoid_exp_velka_s_vyjezdy} obsahují pouze omezené výsledky,
jelikož zbylé varianty zaplnily paměť a selhaly.

Z toho usuzuji, že počet přeplánovaných agentů musí být velice malý pro velké křižovatky.
To činí tento algoritmus značně nepoužitelný.

\input{experimenty/aoid_big_table_no_exits}
\input{experimenty/aoid_big_table_exits}







\section{Hromadné výsledky}\label{sec:hromadne_vysledky}

Porovnání algoritmů mezi sebou s nejlepšími parametry.

Porovnání čtvercové a oktagonální křižovatky.


%\section{Neoptimální agenti}\label{sec:neoptimalni_agenti}

%Vzájemné porovnání algoritmů při datech, kdy křižovatka má nepřesná data o agentech.

%	\include{kap01}
%	\include{kap02}
%	\include{kap03}
%	\include{kap04}

	\chapter*{Závěr}

%Stručné shrnutí výsledků, problémy řešení, nápady na vylepšení.

Zjistil jsem, že vzhled křižovatky a její velikost ovlivňuje chování algoritmů.
Proto neexistuje jeden nejlepší algoritmus a ani jedno nejlepší nastavení algoritmů.

Pokud není křižovatka rozdělená na mnoho bloků, \ref{str:cbs} se choval velmi dobře.
\ref{str:a_star_ars} je dost spolehlivý a dává taktéž dobré výsledky.

Naopak varianty přeplánovávající agenty měly vysoké časy plánování, díky čemu často nedoběhly.
A pokud doběhly, patřily k těm horším algoritmům.

\ref{str:sat} algoritmy byly velmi pomalé.
Myslím si, že to je způsobeno mým rozhodnutím používat MAX-SAT\@.
Zároveň mohly být výsledky ovlivněny výběrem \ref{str:sat} řešiče.

\ref{str:varsg} a \ref{subsubsec:a_star_aoid} měly problémy s pamětí díky vysokému počtu možných kombinací
pozic agentů, pokud se jich plánuje více najednou.
Zjednodušený režim zaručoval alespoň nějaké řešení v rozumném čase, avšak podle mého způsoboval horší výsledky,
než měl \ref{str:a_star_ars}.

\ref{str:cbs} vyřešil problém s pamětí u \ref{str:varsg} algoritmu.
Avšak pro vyšší množství agentů se tvořily velké stromy, které značně zvyšovaly čas plánování.
Toto bylo obzvláště vidět u \nameref{subsec:cbsoid} algoritmu.
Podle mého názoru zde značně zapůsobil zjednodušený režim, jelikož pokud některý agent nešel naplánovat,
bez tohoto režimu musel algoritmus vyzkoušet všechny možné kombinace vrcholů a hran.
S použitím zjednodušení se větve výpočtu uzavíraly, a musely se nejvýše projít pouze aktuálně existující vrcholy.
Toto je podle mého názoru hlavní důvod, proč si tento algoritmus vedl tak dobře.

Řekl bych, že na menších křižovatkách se vyplatí agenty více omezovat, jelikož na těchto křižovatkách není tolik
místa pro manévrování agentů.
Zdá se mi, že se zde více vyplatí počkat s agentem před křižovatkou a poté projet kratší cestou,
než naplánovat dlouho cestu po křižovatce v aktuálním kroku.
Naopak pro větší křižovatky se obecně vyplatilo omezit agenty méně.
Myslím si, že to je kvůli většímu místu.
Zároveň menší omezení agentů dovoluje naplánování agenta, který by byl jinak zamítnut.
Díky tomu se pro tohoto agenta nemusí zdlouhavě procházet všechny možné cesty, což navyšuje časovou náročnost.

Umožnění agentům diagonální jízdy při přechodu ze čtvercové křižovatky na oktagonální na malé křižovatce značně uškodilo.
Avšak při velké velikosti křižovatky to naopak pomohlo.
Myslím si, že to má podobný důvod jako u parametrů.
Na menší křižovatce agenti v nových vrcholech spíše překáží ostatním, což prodlužuje cesty všech.
Naopak na velké křižovatce umožňují vyšší počet způsobů, jak se navzájem vyhnout.

\addcontentsline{toc}{chapter}{Závěr}


%%% Seznam použité literatury
	\include{literatura}

%%% Obrázky v bakalářské práci
%%% (pokud jich je malé množství, obvykle není třeba seznam uvádět)
	\listoffigures

%%% Tabulky v bakalářské práci (opět nemusí být nutné uvádět)
%%% U matematických prací může být lepší přemístit seznam tabulek na začátek práce.
	\listoftables

%%% Použité zkratky v bakalářské práci (opět nemusí být nutné uvádět)
%%% U matematických prací může být lepší přemístit seznam zkratek na začátek práce.
	\chapwithtoc{Seznam použitých zkratek}
	\begin{table}[h!]
	\centering
	\begin{tabular}{l l}
%		\toprule
		\midrule \\
		\ref{str:a_star_ars}    & Individuální A*                     \\
		\ref{str:a_star_arsg}   & Hromadný A*                         \\
		\ref{str:cbs}           & Conflict-Based Search               \\
		\ref{str:sat_cnf}       & Konjunktivní Normální Forma         \\
		\ref{str:fcfs}          & First Come First Served             \\
		\ref{str:mapf}          & Multi-Agent Path Finding            \\
		\ref{str:maxsat}        & Maximální Splnitelnost              \\
		\ref{str:ars_mkc}       & Maximální Krok Cesty                \\
		\ref{str:ars_mnv}       & Maximum Návštěv Vrcholu             \\
		\ref{str:aoid_mpa}      & Maximální Počet Agentů              \\
		\ref{str:ars_mpc}       & Maximální Prodleva při~Cestě        \\
		\ref{str:sat_mpk}       & Maximální Počet Kroků               \\
		\ref{str:oid}           & Online Independence Detection       \\
		pAg                     & průměrný počet agentů na křižovatce \\
		\ref{str:aoid_ppk}      & Počet Přeplánovaných Kroků          \\
		\ref{str:ars_pv}        & Povolené Vracení                    \\
		\ref{str:ars_pz}        & Povolené Zastavování                \\
		pZp                     & průměrné zpoždění na agenta         \\
		\ref{str:rs}            & Replan Single                       \\
		\ref{str:rsg}           & Replan Single Grouped               \\
		\ref{str:sat}           & Splnitelnost                        \\
		\nameref{subsec:sat_ra} & Replan All                          \\
		\ref{str:soc}           & Sum Of Costs                        \\
		\ref{str:suboid}        & Suboptimal Independence Detection   \\
		\ref{str:varsg}         & Vylepšený Hromadný A*               \\
		Zam                     & Počet Zamítnutých Agentů            \\
		\ref{str:arsg_zvp}      & Zjednodušený Výpočet Po             \\
%		\bottomrule
%		\multicolumn{8}{l}{\footnotesize
%		}
	\end{tabular}\label{tab:seznam_zkratek}
\end{table}

%%% Přílohy k bakalářské práci, existují-li. Každá příloha musí být alespoň jednou
%%% odkazována z vlastního textu práce. Přílohy se číslují.
%%%
%%% Do tištěné verze se spíše hodí přílohy, které lze číst a prohlížet (dodatečné
%%% tabulky a grafy, různé textové doplňky, ukázky výstupů z počítačových programů,
%%% apod.). Do elektronické verze se hodí přílohy, které budou spíše používány
%%% v elektronické podobě než čteny (zdrojové kódy programů, datové soubory,
%%% interaktivní grafy apod.). Elektronické přílohy se nahrávají do SISu a lze
%%% je také do práce vložit na CD/DVD. Povolené formáty souborů specifikuje
%%% opatření rektora č. 72/2017.
	\appendix


%	\chapter{Přílohy}
%
%
%	\section{První příloha}

	\openright
\end{document}
